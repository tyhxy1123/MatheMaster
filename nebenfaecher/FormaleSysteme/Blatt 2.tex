% This work is licensed under the Creative Commons
% Attribution-NonCommercial-ShareAlike 4.0 International License. To view a copy
% of this license, visit http://creativecommons.org/licenses/by-nc-sa/4.0/ or
% send a letter to Creative Commons, PO Box 1866, Mountain View, CA 94042, USA.

\documentclass[12pt,a4paper]{article} 

% This work is licensed under the Creative Commons
% Attribution-NonCommercial-ShareAlike 4.0 International License. To view a copy
% of this license, visit http://creativecommons.org/licenses/by-nc-sa/4.0/ or
% send a letter to Creative Commons, PO Box 1866, Mountain View, CA 94042, USA.

% PACKAGES
\usepackage[english, ngerman]{babel}	% Paket für Sprachselektion, in diesem Fall für deutsches Datum etc
\usepackage[utf8]{inputenc}	% Paket für Umlaute; verwende utf8 Kodierung in TexWorks 
\usepackage[T1]{fontenc} % ö,ü,ä werden richtig kodiert
\usepackage{amsmath} % wichtig für align-Umgebung
\usepackage{amssymb} % wichtig für \mathbb{} usw.
\usepackage{amsthm} % damit kann man eigene Theorem-Umgebungen definieren, proof-Umgebungen, etc.
\usepackage{mathrsfs} % für \mathscr
\usepackage[backref]{hyperref} % Inhaltsverzeichnis und \ref-Befehle werden in der PDF-klickbar
\usepackage{graphicx}
\usepackage{grffile}
\usepackage{setspace} % wichtig für Lesbarkeit. Schöne Zeilenabstände
\usepackage{enumitem} % für custom Liste mit default Buchstaben
\usepackage{ulem} % für bessere Unterstreichung
\usepackage{contour} % für bessere Unterstreichung
\usepackage{epigraph} % für das coole Zitat
\usepackage{float}            % figure-Umgebungen besser positionieren
\usepackage{xfrac}
\usepackage{xr} % man Referenzen aus anderen teX-files importieren und darauf verlinken
\usepackage{bbm} %sorgt für Symbol für Indikatorfunktion
\usepackage{color} % bringt Farbe ins Spiel
\usepackage{pdflscape} % damit kann man einzelne Seiten ins Querformat drehen
\usepackage{aligned-overset} % besseres Einrücken, siehe: https://tex.stackexchange.com/questions/257529/overset-and-align-environment-how-to-get-correct-alignment
\usepackage{pgfplots}
	\pgfplotsset{compat=newest}

\usepackage[
    type={CC},
    modifier={by-nc-sa},
    version={4.0},
]{doclicense} % für CC Lizenz-Vermerk

\usepackage{tikz}
	\usepackage{tikz-qtree}
	\usetikzlibrary{arrows}
	\usetikzlibrary{automata}
  \usetikzlibrary{babel}
	\usetikzlibrary{calc}
	\usetikzlibrary{cd}
	\usetikzlibrary{fit}
	\usetikzlibrary{matrix}
	\usetikzlibrary{positioning}
	\usetikzlibrary{shapes.geometric}
	
\usepackage{csquotes}
	\MakeOuterQuote{"}

\usepackage{xargs} % for multiple optional args in newcommand
\usepackage{lmodern} % provides a bigger set of font sizes
\usepackage{anyfontsize} % supports fallback scaling for non-existing font size
\usepackage{scrhack} % provides a hack for deprecated float environments used by some libs

% Ich habe gelesen, dass man folgendes Package zuletzt einbinden soll:
\usepackage[english, ngerman, capitalise]{cleveref} % bessere Verweise

% THEOREM-ENVIRONMENTS

\newtheoremstyle{mystyle}
  {20pt}   % ABOVESPACE \topsep is default, 20pt looks nice
  {20pt}   % BELOWSPACE \topsep is default, 20pt looks nice
  {\normalfont} % BODYFONT
  {0pt}       % INDENT (empty value is the same as 0pt)
  {\bfseries} % HEADFONT
  {}          % HEADPUNCT (if needed)
  {5pt plus 1pt minus 1pt} % HEADSPACE
	{}          % CUSTOM-HEAD-SPEC
\theoremstyle{mystyle}

% Definitionen der Satz, Lemma... - Umgebungen. Der Zähler von "satz" ist dem "section"-Zähler untergeordnet, alle weiteren Umgebungen bedienen sich des satz-Zählers.
\newtheorem{satz}{Satz}[section]
\newtheorem{lemma}[satz]{Lemma}
\newtheorem{korollar}[satz]{Korollar}
\newtheorem{proposition}[satz]{Proposition}
\newtheorem{beispiel}[satz]{Beispiel}
\newtheorem{definition}[satz]{Definition}
<<<<<<< HEAD
\newtheorem{bemerkungnr}[satz]{Bemerkung}
=======
\newtheorem{theorem}[satz]{Theorem}
>>>>>>> 31c1b8aef8833910045abfa6196d99bf06dfe6b5

% Bemerkungen, Erinnerungen und Notationshinweise werden ohne Numerierungen dargestellt.
\newtheorem*{bemerkung}{Bemerkung.}
\newtheorem*{erinnerung}{Erinnerung.}
\newtheorem*{notation}{Notation.}
\newtheorem*{beisp}{Beispiel.} %Beispiel ohne Nummerierung
<<<<<<< HEAD

=======
\newtheorem*{defi}{Definition.} %Definition ohne Nummerierung
>>>>>>> 31c1b8aef8833910045abfa6196d99bf06dfe6b5

% SHORTCUTS
\newcommand{\R}{\mathbb{R}}				 % reelle Zahlen
\newcommand{\Rn}{\R^n}						 % der R^n
\newcommand{\N}{\mathbb{N}}				 % natürliche Zahlen
\newcommand{\Z}{\mathbb{Z}}				 % ganze Zahlen
\newcommand{\C}{\mathbb{C}}			   % komplexe Zahlen
\renewcommand{\mit}{\text{ mit }}   % mit
\newcommand{\falls}{\text{falls }} % falls
\renewcommand{\d}{\text{ d}}        % Differential d

% ETWAS SPEZIELLERE ZEICHEN
% disjunkte Vereinigung
\newcommand{\bigcupdot}{
	\mathop{\vphantom{\bigcup}\mathpalette\setbigcupdot\cdot}\displaylimits
}
\newcommand{\setbigcupdot}[2]{\ooalign{\hfil$#1\bigcup$\hfil\cr\hfil$#2$\hfil\cr\cr}}
% großes Kreuz
\newcommand*{\bigtimes}{\mathop{\raisebox{-.5ex}{\hbox{\huge{$\times$}}}}} 

% WHITESPACE COMMANDS
% nicht restriktiver newline command
\newcommand{\enter}{$ $\newline} 
% praktischer Tabulator
\newcommand\tab[1][1cm]{\hspace*{#1}}

% TEXT ÜBER ZEICHEN
\newcommand{\stackeq}[1]{\stackrel{#1}{=}} 

% UNDERLINE
% besseres underline 
\renewcommand{\ULdepth}{1pt}
\contourlength{0.5pt}
\newcommand{\ul}[1]{
	\uline{\phantom{#1}}\llap{\contour{white}{#1}}
}


% This work is licensed under the Creative Commons
% Attribution-NonCommercial-ShareAlike 4.0 International License. To view a copy
% of this license, visit http://creativecommons.org/licenses/by-nc-sa/4.0/ or
% send a letter to Creative Commons, PO Box 1866, Mountain View, CA 94042, USA.

% hier noch ein paar Commands die nur ich nutze, weil ich sie mir im Laufe der Jahre angewöhnt habe und sie mir jetzt nicht abgewöhnen will:

\newcommand{\gdw}{\Leftrightarrow}             % genau dann, wenn
\DeclareMathOperator{\id}{id}                  % identische Abbildung
\DeclareMathOperator{\dom}{dom}                % Domain, Definitionsbereich
\DeclareMathOperator{\rg}{rg}                  % Rang, Bild einer Funktion, Rang einer Matrix
\newcommand{\K}{\mathbb{K}}                    % Körper
\newcommand{\limn}{\lim\limits_{n\to\infty}}   % genau dann, wenn
\newcommand{\sonst}{\text{sonst }}             % sonst
\renewcommand{\O}{\mathcal{O}}                 % Landau-O
\DeclareMathOperator{\Div}{div}                % Divergenz
\renewcommand{\div}{\Div}                      % Divergenz
\DeclareMathOperator{\spann}{span}             % Span
\DeclareMathOperator{\diam}{diam}       	   % Diameter, Durchmesser einer Menge
\DeclareMathOperator{\inner}{int}              % Das innere einer Menge





\author{Willi Sontopski}

\parindent0cm %Ist wichtig, um führende Leerzeichen zu entfernen

\usepackage{pdflscape}
\usepackage{rotating}
\usepackage{scrpage2}
\pagestyle{scrheadings}
\clearscrheadfoot

\ihead{Willi Sontopski}
\chead{Formale Systeme WiSe 18 19}
\ohead{}
\ifoot{Blatt 2}
\cfoot{Version: \today}
\ofoot{Seite \pagemark}

\newcommand{\depth}{\text{depth}}
\newcommand{\length}{\text{length}}

\begin{document}
%\setcounter{section}{1}

\section*{Aufgabe 2.1}
\subsection*{Aufgabe 2.1 (a)}
\begin{align*}
\depth\Big(\big(\neg p\to(\neg p\wedge q)\big)\Big)
&=\max\Big(\depth(\neg p),\depth\big((\neg p\wedge q)\big)\Big)+1\\
&=\max\Big(\depth(p)+1,\max\big(\depth(\neg p),\depth(q)\big)+1\Big)+1\\
&=\max\Big(0+1,\max\big(\depth(p)+1,0\big)+1\Big)+1\\
&=\max\Big(1,\max\big(0+1,0\big)+1\Big)+1\\
&=3
\end{align*}

\subsection*{Aufgabe 2.1 (b)}
\begin{align*}
&\length:\mathcal{L}(\mathcal{R})\to\N_0,\qquad\\
&\varphi\mapsto\left\lbrace\begin{array}{cl}
0, & \falls \varphi=\Lambda\\
1, & \falls \varphi=p\in\mathcal{R}\\
\length(x) + 1, & \falls \varphi=\neg x\mit x\in\mathcal{L}(\mathcal{R})\\
\length(x_1) + \length(x_2) + 3, & \falls \varphi=(x_1\circ x_2)\mit x_1,x_2\in\mathcal{L}(\mathcal{R})%\text{ und }\circ\in\lbrace\wedge,\vee,\to,\leftrightarrow\rbrace
\end{array}\right.
\end{align*}

\subsection*{Aufgabe 2.1 (c)}
Induktionsanfang: Sei $F\in\mathcal{R}$. Dann gilt:
\begin{align*}
\length(F)\stackeq{\text{Def}}1>0\stackeq{\text{Def}}\depth(F)
\end{align*}
Induktionsvoraussetzung: Seien $F_1,F_2\in\mathcal{L}(\mathcal{R})$ beliebig aber fest mit
\begin{align*}
\length(F_1)>\depth(F_1)\qquad\text{und}\qquad\length(F_2)>\depth(F_2).
\end{align*}
Induktionsschritt: Sei $\circ\in\lbrace\wedge,\vee,\to,\leftrightarrow\rbrace$. Dann gilt:
\begin{align*}
\length(\neg F_1)
&\stackeq{\text{Def}}
\length(F_1)+1\\
&\stackrel{\text{IV}}{>}
\depth(F_1)+1\\
&\stackeq{\text{Def}}
\depth(\neg F_1)\\
%%%%%%%%%%%%%%%%%%%%
\length((F_1\circ F_2))
&\stackeq{\text{Def}}
\length(F_1)+\length(F_2)+3\\
&\stackrel{\text{IV}}{>}
\depth(F_1)+\depth(F_2)+3\\
&\stackrel{\text{Math}}{>}
\max\big(\depth(F_1),\depth(F_2)\big)+1\\
&\stackeq{\text{Def}}
\depth((F_1\circ F_2))
\end{align*}

\section*{Aufgabe 2.2}
\subsection*{Aufgabe 2.2 (a)}
\begin{enumerate}[label=(\arabic*)]
\item $\begin{aligned}
\big\lbrace (\neg p\wedge(q\to r)),\neg p, p, (q\to r),q,r\big\rbrace
\end{aligned}$
\item $\begin{aligned}
\big\lbrace p\big\rbrace
\end{aligned}$
\end{enumerate}

\subsection*{Aufgabe 2.2 (b)}
\begin{enumerate}[label=(\arabic*)]
\item $\begin{aligned}\big\lbrace
(p\wedge q)
\big\rbrace\end{aligned}$, da $F$ enthalten und keine Negationen drin
\item $\begin{aligned}\big\lbrace
(p\wedge q), p, q
\big\rbrace\end{aligned}$
\item $\emptyset$ Was soll denn sonst drin sein?
\end{enumerate}

\section*{Aufgabe 2.3}
Induktionsanfang: Sei $F=p\in\mathcal{R}$. Dann gilt:\\

Induktionsvoraussetzung:

Induktionsschritt:

\section*{Aufgabe 2.4}
Beachte $\top:=$ wahr und $\bot:=$ falsch.
\subsection*{Aufgabe 2.4 (a)}
\begin{tabular}{c|c||c|c|c}
$p$ & $q$ & $(p\vee q)$ & $(p\vee q)\to q$ & $(((p\vee q)\to q)\to q)$\\ \hline
$\top$ & $\top$ & $\top$ & $\top$ & $\top$\\
$\bot$ & $\top$ & $\top$ & $\top$ & $\top$\\
$\top$ & $\bot$ & $\top$ & $\bot$ & $\top$\\
$\bot$ & $\bot$ & $\bot$ & $\top$ & $\bot$
\end{tabular}

\subsection*{Aufgabe 2.4 (b)}
Die Formel aus Teil (a) ist
\begin{itemize}
\item erfüllbar, denn sie kann wahr liefern.
\item \underline{nicht} allgemeingültig, weil sie auch falsch liefern kann.
\item widerlegbar, denn sie kann auch falsch liefern.
\item \underline{nicht} unerfüllbar, denn sie kann auch wahr liefern.
\end{itemize}

\subsection*{Aufgabe 2.4 (c)}
\begin{tabular}{c}
$p$\\ \hline
0\\
1
\end{tabular}

\section*{Aufgabe 2.5}
\subsection*{Aufgabe 2.5 (a)}
\begin{tabular}{c|c||c|c|c}
$p$ & $q$ & $(p\to q)$ & $(p\to q)\to q$ & $(((p\to q)\to q)\to q)$\\ \hline
$\top$ & $\top$ & $\top$ & $\top$ & $\top$ \\
$\bot$ & $\top$ & $\top$ & $\top$ & $\top$ \\
$\top$ & $\bot$ & $\bot$ & $\top$ & $\bot$\\
$\bot$ & $\bot$ & $\top$ & $\bot$ & $\top$
\end{tabular}\\

Die Formel ist \underline{nicht} allgemeingültig, erfüllbar, \underline{nicht} unerfüllbar und widerlegbar.
\subsection*{Aufgabe 2.5 (b)}
%\begin{sidewaystable} %Alternative zu Landscape. Ist geschmackssache was besser ist. Kann man hier umstellen, wenn man möchte.
\begin{landscape}
\begin{tabular}{c|c|c||c|c|c|c|c|c|c}
$p$ & $q$ & $r$ & $(p\to q)$ & $(q\to r)$ & $((p\to q)\wedge(q\to r))$ & $(r\to q)$ & $(q\to p)$ & $((r\to q)\wedge(q\to p))$ & $F$ \\ \hline
$\top$ & $\top$ & $\top$ & $\top$ & $\top$ & $\top$ & $\top$ & $\top$ & $\top$ & $\top$\\
$\bot$ & $\top$ & $\top$ & $\top$ & $\top$ & $\top$ & $\top$ & $\bot$ & $\bot$ & $\top$\\
$\top$ & $\bot$ & $\top$ & $\bot$ & $\top$ & $\bot$ & $\bot$ & $\top$ & $\bot$ & $\bot$\\
$\bot$ & $\bot$ & $\top$ & $\top$ & $\top$ & $\top$ & $\bot$ & $\top$ & $\bot$ & $\top$\\ \hline
$\top$ & $\top$ & $\bot$ & $\top$ & $\bot$ & $\bot$ & $\top$ & $\top$ & $\top$ & $\top$\\
$\bot$ & $\top$ & $\bot$ & $\top$ & $\bot$ & $\bot$ & $\top$ & $\bot$ & $\bot$ & $\bot$\\
$\top$ & $\bot$ & $\bot$ & $\bot$ & $\top$ & $\bot$ & $\top$ & $\top$ & $\top$ & $\top$\\
$\bot$ & $\bot$ & $\bot$ & $\top$ & $\top$ & $\top$ & $\top$ & $\top$ & $\top$ & $\top$
\end{tabular}
\end{landscape}
%\end{sidewaystable}

Die Formel ist \underline{nicht} allgemeingültig, erfüllbar, \underline{nicht} unerfüllbar und widerlegbar.

\subsection*{Aufgabe 2.5 (c)}
\begin{tabular}{c|c||c|c|c|c}
$p$ & $q$ & $(p\to q)$ & $(p\to q)\to q$ & $(((p\to q)\to q)\to q)$ & $\neg(((p\to q)\to q)\to q)$\\ \hline
$\top$ & $\top$ & $\top$ & $\top$ & $\top$ & $\bot$\\
$\bot$ & $\top$ & $\top$ & $\top$ & $\top$ & $\bot$\\
$\top$ & $\bot$ & $\bot$ & $\top$ & $\bot$ & $\top$\\
$\bot$ & $\bot$ & $\top$ & $\bot$ & $\top$ & $\bot$
\end{tabular}\\

Die Formel ist \underline{nicht} allgemeingültig, erfüllbar, \underline{nicht} unerfüllbar und widerlegbar.

\subsection*{Aufgabe 2.5 (d)}
To Do

\end{document}