\documentclass[12pt,a4paper]{article} %Schriftgröße = 12, Pappier = A4, Stil = Artikel
\usepackage[utf8]{inputenc} % Zeichensatzkodierung = UTF8, 
\usepackage[german]{babel} %Babel-Paket 7 Sprache = Deutsch
\usepackage[T1]{fontenc}
\usepackage{amsmath} % Irgendein Mathe-Paket
\usepackage{amsfonts}
\usepackage{amssymb}
\usepackage{graphicx}
\usepackage{tikz}
\usepackage[left=2cm,right=2cm,top=2cm,bottom=2cm]{geometry} % Randeinstellungen
\usepackage{hyperref}
\author{Willi Sontopski}

\parindent0cm %Ist wichtig, um führende Leerzeichen zu entfernen

\usepackage{scrpage2}
\pagestyle{scrheadings}
\clearscrheadfoot

\ihead{Willi Sontopski}
\chead{Formale System WiSe 18 19}
\ohead{Version: \today}
\ifoot{Blatt 1}
\cfoot{willi\_sontopski@arcor.de}
\ofoot{Seite \pagemark}

\newcommand{\B}{\mathbb{B}}
\newcommand{\R}{\mathbb{R}}
\newcommand{\Q}{\mathbb{Q}}
\newcommand{\Z}{\mathbb{Z}}
\newcommand{\C}{\mathbb{C}}
\newcommand{\N}{\mathbb{N}}
\newcommand{\gdw}{\Leftrightarrow}
\newcommand{\limn}{\lim\limits_{n\to\infty}}
\renewcommand{\d}{\text{ d}}
\newcommand{\falls}{\text{falls }}
\renewcommand{\mit}{\text{ mit }}
\newcommand{\sonst}{\text{sonst }}

\newcommand{\myeq}[1]{\mathrel{\stackrel{\makebox[0pt]{\mbox{\normalfont\tiny #1}}}{=}}}
\newcommand{\myrel}[2]{\mathrel{\stackrel{\makebox[0pt]{\mbox{\normalfont\tiny #1}}}{#2}}}
\newcommand\tab[1][1cm]{\hspace*{#1}}

\usepackage{enumerate}
\newcommand{\ok}{\text{öK}}
\newcommand{\sk}{\text{sK}}
\renewcommand{\k}{\text{K}}
\newcommand{\bj}{\text{bJ}}

\begin{document}
%\setcounter{section}{1}

\section*{Aufgabe 1.1}
\subsection*{Aufgabe 1.1 (a)}
\begin{enumerate}[{(1)}]
\item  Nein, da, Klammern fehlen. Richtig wäre: $(3-2)+(1-(3\times 4))$ oder $((3-2)+1)-(3\times 4)$
\item Nein, da rationale Zahlen nicht geklammert werden dürfen. Und ein unäres Minus ist auch nicht definiert.
\item Nein, da das Gleichheitszeichen nicht Teil des Alphabets ist.
\item Nein, da es kein unäres Minuszeichen gibt.
\item Nein. Nur falls $p,q\in\Q$.
\end{enumerate}

\subsection*{Aufgabe 1.1 (b)}
Das ist Abhängig von der Wahl des Alphabets $(\mathcal{R},\mathcal{J},\mathcal{S})$.
\begin{enumerate}[{(1)}]
\item Ja.
\item Ja.
\item Ja.
\item Nein, weil Klammern fehlen. Richtig wäre $((p\wedge p)\wedge(p\wedge p))$
\item Nein, weil Klammern zu viel sind und fehlen.
\item Ja.
\end{enumerate}

\section*{Aufgabe 1.2}
Induktionsanfang: $n=0:~0\leq0=0^2$
Induktionsvoraussetzung: Gelte $n\leq n^2$ für beliebiges aber festes $n\in\N$.
Induktionsschritt: 
$n+1\leq
n+\underbrace{2\cdot n}_{\geq0}+1
\myrel{IV}{\leq} n^2+2\cdot n+1=(n+1)^2\qquad\square$

\section*{Aufgabe 1.3}
\subsection*{Aufgabe 1.3 (a)}
Beweis durch Induktion, Induktionsanfang: $n=1:$ $p$ ist Formel (wenn aussagenlogische Variable)\\
Induktionsvoraussetzung: Für ein beliebiges aber festes $n\in\N$ gibt es eine aussagenlogische Formel der Länge $n$.\\
Induktionsschritt: Nach IV ist $x$ aussagenlogische Formel der Länge $n$. Dann ist $\neg x$ nach Def. 3.5.2 auch aussagenlogische Formel. $\square$

\subsection*{Aufgabe 1.3 (b)}
k.A.

\section*{Aufgabe 1.4}
\subsection*{Aufgabe 1.4 (a)}
Betrachte die Funktionen, die aufgehende bzw. schließende Klammern zählen:
\begin{align*}
sk\equiv\ok:\mathcal{L}(\mathcal{R})\to\N,\qquad
x\mapsto\left\lbrace\begin{array}{cl}
0, & \falls x\in\mathcal{R}\text{ (ist Atom)}\\
\ok(y), & \falls x=\neg y\\
1+\ok(y)+\ok(z), & \falls x=(y\circ z)
\end{array}\right.
\end{align*}
Man sieht schon, dass beide Funktionen für alle $x\in\mathcal{L}(\mathcal{R})$ übereinstimmen. Aber man kann das Offensichtliche natürlich noch induktiv zeigen:\\

Induktionsanfang: Sei $a\in\mathcal{R}$ Atom. Dann gilt: $\ok(a)=0=\sk(a)$.\\
Induktionsvoraussetzung: Gelte für beliebiges aber festes $x\in\mathcal{L}(\mathcal{R})$ $\ok(x)=\sk(x)$.\\ 
Induktionsschritt:
\begin{align*}
\ok(\neg x)\myeq{Def}\ok(x)\myeq{IV}\sk(x)\myeq{Def}\sk(\neg x)\\
\ok((a\circ b))\myeq{Def}1+\ok(a)+\ok(b)\myeq{IV}1+\sk(a)+\sk(b)\myeq{Def}\sk((a\circ b))
\end{align*}

\subsection*{Aufgabe 1.4 (b)}
Betrachte die Funktion, die Klammern bzw. binäre Junktoren zählt:
\begin{align*}
\k:\mathcal{L}(\mathcal{R})\to\N,\qquad
x\mapsto\left\lbrace\begin{array}{cl}
0, & \falls x\in\mathcal{R}\text{ (ist Atom)}\\
\k(y), & \falls x=\neg y\\
2+\k(y)+\k(z), & \falls x=(y\circ z)
\end{array}\right.\qquad \bj:\equiv ok\equiv\sk
\end{align*}

Induktionsanfang: Sei $a\in\mathcal{R}$ Atom. Dann gilt: $\k(a)=0=2\cdot 0=\bj(a)$.\\
Induktionsvoraussetzung: Gelte für beliebiges aber festes $x\in\mathcal{L}(\mathcal{R})$ $\k(x)=2\cdot\bj(x)$.\\ 
Induktionsschritt:
\begin{align*}
\k(\neg x)\myeq{Def}\k(x)\myeq{IV}2\cdot\bj(x)\myeq{Def}\bj(\neg x)\\
\k((a\circ b))\myeq{Def}2+\k(a)+\k(b)\myeq{IV}2+2\cdot\bj(a)+2\cdot\bj(b)=2\cdot(1+\bj(a)+\bj(b))\myeq{Def}2\cdot\bj((a\circ b))
\end{align*}

\subsection*{Aufgabe 1.4 (c)}
Nein, Gegenbeispiel: $\neg\neg\neg p$ wobei  $p$ aussagenlogische Variable ist und $\neg$ unärer Junktor.

\section*{Aufgabe 1.5}
strukturelle Rekursion?
\subsection*{Aufgabe 1.5 (a)}
Sei $A\in\mathcal{R}$ aussagenlogische Variable. Setze
\begin{align*}
h:\mathcal{L}(\mathcal{R})\to\N_0,\qquad x\mapsto \left\lbrace\begin{array}{cl}
0, & \falls x\in\mathcal{R}\wedge x\neq A\\
1, & \falls x\in\mathcal{R}\wedge x=A\\
h(y), & \falls x=\neg y\\
h(y)+h(z), & \falls x=(y\circ z)
\end{array}\right.
\end{align*}

\subsection*{Aufgabe 1.5 (b)}
\begin{align*}
\text{laenge}:\mathcal{L}(\mathcal{R})\to\N_0,\qquad x\mapsto \left\lbrace\begin{array}{cl}
1, & \falls x\in\mathcal{R}\\
1+\text{laenge}(y), & \falls x=\neg y\\
3+\text{laenge}(y)+\text{laenge}(z), & \falls x=(y\circ z)
\end{array}\right.
\end{align*}
Und damit:
\begin{align*}
\text{laenge}((ü\vee(q\wedge\neg p))) 
&=3+\text{laenge}(p)+\text{laenge}((q\wedge\neg p))\
&=3+1+3+\text{laenge}(q)+\text{laenge}(\neg p)\\
&=7 + 1 + 1+ \text{laenge}(p)\\
&=9+1=10
\end{align*}

\subsection*{Aufgabe 1.5 (c)}
\begin{align*}
\text{du}:\mathcal{L}(\mathcal{R})\to\N_0,\qquad x\mapsto \left\lbrace\begin{array}{cl}
\neg x, & \falls x\in\mathcal{R}\\
\text{du}(G), & \falls x=\neg G\\
(\text{du}(G)\vee\text{du}(H)), & \falls x=(G\wedge H)\\
(\text{du}(G)\wedge\text{du}(H)), & \falls x=(G\vee H)\\
(\text{du}(H)\wedge\neg\text{du}(G)), & \falls x=(H\to G)\\
((\text{du}(H)\wedge\neg \text{du}(G))\vee(\text{du}(G)\wedge\neg\text{du}(H))), & \falls x=(H\leftrightarrow G)\\
\end{array}\right.
\end{align*}
Und damit:
\begin{align*}
\text{du}((p\to(p\wedge\neg q)))
&=(\text{du}(p)\wedge\neg\text{du}((p\wedge\neg q)))\\
&=\neg p\wedge\neg(\text{du}(p)\vee\text{du}(\neg q))\\
&=\neg p\wedge\neg\neg p\vee\neg\text{du}(q)\\
&=\neg p\wedge\neg\neg p\vee\neg\neg q\\
\end{align*}
Vielleicht sollte die Funktion noch ein paar Klammern setzen :D

\end{document}