% This work is licensed under the Creative Commons
% Attribution-NonCommercial-ShareAlike 4.0 International License. To view a copy
% of this license, visit http://creativecommons.org/licenses/by-nc-sa/4.0/ or
% send a letter to Creative Commons, PO Box 1866, Mountain View, CA 94042, USA.

\documentclass[12pt,a4paper]{article} 

% This work is licensed under the Creative Commons
% Attribution-NonCommercial-ShareAlike 4.0 International License. To view a copy
% of this license, visit http://creativecommons.org/licenses/by-nc-sa/4.0/ or
% send a letter to Creative Commons, PO Box 1866, Mountain View, CA 94042, USA.

% PACKAGES
\usepackage[english, ngerman]{babel}	% Paket für Sprachselektion, in diesem Fall für deutsches Datum etc
\usepackage[utf8]{inputenc}	% Paket für Umlaute; verwende utf8 Kodierung in TexWorks 
\usepackage[T1]{fontenc} % ö,ü,ä werden richtig kodiert
\usepackage{amsmath} % wichtig für align-Umgebung
\usepackage{amssymb} % wichtig für \mathbb{} usw.
\usepackage{amsthm} % damit kann man eigene Theorem-Umgebungen definieren, proof-Umgebungen, etc.
\usepackage{mathrsfs} % für \mathscr
\usepackage[backref]{hyperref} % Inhaltsverzeichnis und \ref-Befehle werden in der PDF-klickbar
\usepackage{graphicx}
\usepackage{grffile}
\usepackage{setspace} % wichtig für Lesbarkeit. Schöne Zeilenabstände
\usepackage{enumitem} % für custom Liste mit default Buchstaben
\usepackage{ulem} % für bessere Unterstreichung
\usepackage{contour} % für bessere Unterstreichung
\usepackage{epigraph} % für das coole Zitat
\usepackage{float}            % figure-Umgebungen besser positionieren
\usepackage{xfrac}
\usepackage{xr} % man Referenzen aus anderen teX-files importieren und darauf verlinken
\usepackage{bbm} %sorgt für Symbol für Indikatorfunktion
\usepackage{color} % bringt Farbe ins Spiel
\usepackage{pdflscape} % damit kann man einzelne Seiten ins Querformat drehen
\usepackage{aligned-overset} % besseres Einrücken, siehe: https://tex.stackexchange.com/questions/257529/overset-and-align-environment-how-to-get-correct-alignment
\usepackage{pgfplots}
	\pgfplotsset{compat=newest}

\usepackage[
    type={CC},
    modifier={by-nc-sa},
    version={4.0},
]{doclicense} % für CC Lizenz-Vermerk

\usepackage{tikz}
	\usepackage{tikz-qtree}
	\usetikzlibrary{arrows}
	\usetikzlibrary{automata}
  \usetikzlibrary{babel}
	\usetikzlibrary{calc}
	\usetikzlibrary{cd}
	\usetikzlibrary{fit}
	\usetikzlibrary{matrix}
	\usetikzlibrary{positioning}
	\usetikzlibrary{shapes.geometric}
	
\usepackage{csquotes}
	\MakeOuterQuote{"}

\usepackage{xargs} % for multiple optional args in newcommand
\usepackage{lmodern} % provides a bigger set of font sizes
\usepackage{anyfontsize} % supports fallback scaling for non-existing font size
\usepackage{scrhack} % provides a hack for deprecated float environments used by some libs

% Ich habe gelesen, dass man folgendes Package zuletzt einbinden soll:
\usepackage[english, ngerman, capitalise]{cleveref} % bessere Verweise

% THEOREM-ENVIRONMENTS

\newtheoremstyle{mystyle}
  {20pt}   % ABOVESPACE \topsep is default, 20pt looks nice
  {20pt}   % BELOWSPACE \topsep is default, 20pt looks nice
  {\normalfont} % BODYFONT
  {0pt}       % INDENT (empty value is the same as 0pt)
  {\bfseries} % HEADFONT
  {}          % HEADPUNCT (if needed)
  {5pt plus 1pt minus 1pt} % HEADSPACE
	{}          % CUSTOM-HEAD-SPEC
\theoremstyle{mystyle}

% Definitionen der Satz, Lemma... - Umgebungen. Der Zähler von "satz" ist dem "section"-Zähler untergeordnet, alle weiteren Umgebungen bedienen sich des satz-Zählers.
\newtheorem{satz}{Satz}[section]
\newtheorem{lemma}[satz]{Lemma}
\newtheorem{korollar}[satz]{Korollar}
\newtheorem{proposition}[satz]{Proposition}
\newtheorem{beispiel}[satz]{Beispiel}
\newtheorem{definition}[satz]{Definition}
<<<<<<< HEAD
\newtheorem{bemerkungnr}[satz]{Bemerkung}
=======
\newtheorem{theorem}[satz]{Theorem}
>>>>>>> 31c1b8aef8833910045abfa6196d99bf06dfe6b5

% Bemerkungen, Erinnerungen und Notationshinweise werden ohne Numerierungen dargestellt.
\newtheorem*{bemerkung}{Bemerkung.}
\newtheorem*{erinnerung}{Erinnerung.}
\newtheorem*{notation}{Notation.}
\newtheorem*{beisp}{Beispiel.} %Beispiel ohne Nummerierung
<<<<<<< HEAD

=======
\newtheorem*{defi}{Definition.} %Definition ohne Nummerierung
>>>>>>> 31c1b8aef8833910045abfa6196d99bf06dfe6b5

% SHORTCUTS
\newcommand{\R}{\mathbb{R}}				 % reelle Zahlen
\newcommand{\Rn}{\R^n}						 % der R^n
\newcommand{\N}{\mathbb{N}}				 % natürliche Zahlen
\newcommand{\Z}{\mathbb{Z}}				 % ganze Zahlen
\newcommand{\C}{\mathbb{C}}			   % komplexe Zahlen
\renewcommand{\mit}{\text{ mit }}   % mit
\newcommand{\falls}{\text{falls }} % falls
\renewcommand{\d}{\text{ d}}        % Differential d

% ETWAS SPEZIELLERE ZEICHEN
% disjunkte Vereinigung
\newcommand{\bigcupdot}{
	\mathop{\vphantom{\bigcup}\mathpalette\setbigcupdot\cdot}\displaylimits
}
\newcommand{\setbigcupdot}[2]{\ooalign{\hfil$#1\bigcup$\hfil\cr\hfil$#2$\hfil\cr\cr}}
% großes Kreuz
\newcommand*{\bigtimes}{\mathop{\raisebox{-.5ex}{\hbox{\huge{$\times$}}}}} 

% WHITESPACE COMMANDS
% nicht restriktiver newline command
\newcommand{\enter}{$ $\newline} 
% praktischer Tabulator
\newcommand\tab[1][1cm]{\hspace*{#1}}

% TEXT ÜBER ZEICHEN
\newcommand{\stackeq}[1]{\stackrel{#1}{=}} 

% UNDERLINE
% besseres underline 
\renewcommand{\ULdepth}{1pt}
\contourlength{0.5pt}
\newcommand{\ul}[1]{
	\uline{\phantom{#1}}\llap{\contour{white}{#1}}
}


% This work is licensed under the Creative Commons
% Attribution-NonCommercial-ShareAlike 4.0 International License. To view a copy
% of this license, visit http://creativecommons.org/licenses/by-nc-sa/4.0/ or
% send a letter to Creative Commons, PO Box 1866, Mountain View, CA 94042, USA.

% hier noch ein paar Commands die nur ich nutze, weil ich sie mir im Laufe der Jahre angewöhnt habe und sie mir jetzt nicht abgewöhnen will:

\newcommand{\gdw}{\Leftrightarrow}             % genau dann, wenn
\DeclareMathOperator{\id}{id}                  % identische Abbildung
\DeclareMathOperator{\dom}{dom}                % Domain, Definitionsbereich
\DeclareMathOperator{\rg}{rg}                  % Rang, Bild einer Funktion, Rang einer Matrix
\newcommand{\K}{\mathbb{K}}                    % Körper
\newcommand{\limn}{\lim\limits_{n\to\infty}}   % genau dann, wenn
\newcommand{\sonst}{\text{sonst }}             % sonst
\renewcommand{\O}{\mathcal{O}}                 % Landau-O
\DeclareMathOperator{\Div}{div}                % Divergenz
\renewcommand{\div}{\Div}                      % Divergenz
\DeclareMathOperator{\spann}{span}             % Span
\DeclareMathOperator{\diam}{diam}       	   % Diameter, Durchmesser einer Menge
\DeclareMathOperator{\inner}{int}              % Das innere einer Menge





\author{Willi Sontopski}

\parindent0cm %Ist wichtig, um führende Leerzeichen zu entfernen

\usepackage{pdflscape}
\usepackage{rotating}
\usepackage{scrpage2}
\pagestyle{scrheadings}
\clearscrheadfoot

\ihead{Willi Sontopski}
\chead{Formale Systeme WiSe 18 19}
\ohead{}
\ifoot{Probeklausur}
\cfoot{Version: \today}
\ofoot{Seite \pagemark}

\renewcommand{\L}{\mathcal{L}} %sonst nicht nichtbar
\newcommand{\RR}{\mathcal{R}} %\R sind reelle Zahlen
\newcommand{\PP}{\mathcal{P}} %\P ist W-Maß oder Sonderzeichen
\renewcommand{\S}{\mathcal{S}} %sonst Paragraph
\newcommand{\F}{\mathcal{F}}
\newcommand{\G}{\mathcal{G}}
\newcommand{\W}{\mathcal{W}} 

\begin{document}
%\setcounter{section}{1}

\section*{Aufgabe 1 (14 Punkte)}
Geben Sie die Definitionen für folgende Begriffe:
\begin{enumerate}
\item (aussagenlogische) Formel (3 Punkte)
\item (aussagenlogische) Interpretation (4 Punkte)
\item Allgemeingültige Formel (Tautologie) (1 Punkt)
\item Eine (aussagenlogische Konsequenz (4 Punkte)
\item Konjunktive Normalform (CNF) (2 Punkte)
\end{enumerate}
\begin{lösung}
\begin{enumerate}
\item Siehe Definiton 3.5.\\
Eine (aussagenlogische) Formel ist ein Element der Menge aller (aussagenlogischen) Formeln. Diese ist die kleinste Menge $\L(\RR)$ von Zeichenreihen über $\RR$ (Menge der aussagenlogischen Variablen), den Junktoren und den Sonderzeichen, die die folgenden Eigenschaften erfüllen:
\begin{enumerate}
\item $F\in\RR\implies F\in\L(\RR)$
\item $F\in\L(\RR)\implies\neg F\in\L(\RR)$
\item $F,G\in\L(\RR)\wedge\circ/2\text{ Junktor }\implies(F\circ G)\in\L(\RR)$
\end{enumerate}
\item Siehe Definiton 3.9.\\
Eine (aussagenlogische) Interpretation $I=(\W,\cdot^I)$ besteht aus der Menge $\W:=\lbrace\bot,\top\rbrace$ der Wahrheitswerte und einer Abbildung $\cdot^I:\L(\RR)\to\W$, welche die folgende Bedingung erfüllt:
\begin{align*}
[F]^I=\left\lbrace\begin{array}{cl}
\neg^\ast[G]^I & \falls F \text{ von der Form }\neg G\\
\big([G_1]^I\circ^\ast[G_2]^I\big), &\falls F\text{ von der Form }(G_1\circ G_2)
\end{array}\right.
\end{align*}
\item Siehe Definition 3.12.\\
Eine Formel $F\in\L(\RR)$ heißt allgemeingültig / Tautologie, wenn für alle Interpretationen $I=(\W,\cdot^I)$ gilt: $F^I=\top$.
\item Siehe Definition 3.16.\\
Eine aussagenlogische Formel $F\in\L(\RR)$ ist genau dann eine (aussagen-)logische Konsequenz einer Menge von Formeln $\G$, i.Z. $\G\models F$, wenn für jede Interpretation $I=(\W,\cdot^I)$ gilt:\\
Wenn $I$ Modell für $\G$ ist, dann ist $I$ auch Modell für $F$.
\item Siehe Definition 3.27.\\
Eine Formel ist in konjunktiver Normalform gdw. sie von der Form $\langle C_1,\ldots,C_m\rangle,m\geq0$ ist und jedes $C_j,~1\leq j\leq m$ eine Klausel ist.\\
Eine Klausel ist eine verallgemeinerte Diskunktion  $[L_1,\ldots,L_n],~n\geq0$, wobei jedes $L_i,~1\leq i\leq n$ ein Literal (aussagenlogische Variable oder dessen Negat) ist.
\end{enumerate}
\end{lösung}

\section*{Aufgabe 2 (8 Punkte)}
Seien $p,q,r,s$ aussagenlogische Variablen. Geben Sie für jede der genannten Aussagen an, ob diese wahr oder falsch sind.\\

Hinweis: Falsche Antworten führen zu Punktabzug, wobei die gesamte Aufgabe mit mindestens 0 Punkten bewertet wird.

\begin{enumerate}[label=\alph*)]
\item $(p\to\neg p)\equiv p$
\item $(p\to p)\equiv(q\to p)$
\item $(\neg p\vee\neg q)\equiv \neg(p\vee q)$
\item $(\neg p\wedge\neg q)\equiv \neg(p\vee q)$
\item $\neg(p\wedge(\neg p\wedge q))$ ist allgemeingültig.
\item $(p\wedge(\neg p\wedge q))$ ist erfüllbar.
\item $\big\lbrace[p,\neg q,r],[\neg r,\neg s]\big\rbrace\models[p,\neg q,\neg s]$
\item $\big\langle[p,q,\neg p],[r,\neg s]\big\rangle\equiv\big\langle[r,\neg s]\big\rangle$
\end{enumerate}
\begin{lösung}\
\begin{enumerate}[label=\alph*)]
\item falsch: Falls $p^I=\bot$ ist $(p\to\neg p)\equiv\top$ und $p\equiv\bot$
\item falsch (rechte Seite Tautologie, aber links Seite nicht allgemeingültig)
\item falsch (offensichtlich)
\item richtig (De Morgan)
\item richtig (offensichtlich)
\item falsch (offensichtlich)
\item
\item
\end{enumerate}
\end{lösung}

\section*{Aufgabe 3 (6 Punkte)}
Eine \textbf{definite Klausel} ist eine Klausel mit einer beliebigen Anzahl $\geq1$ von Literalen von denen genau eines positiv ist.\\
Beweisen Sie, dass eine Formel in Klauselform, die nur definite Klauseln enthält, erfülbar ist.

\begin{proof}
Sei also $F$ eine Formel in Klauselform, die nur definite Klauseln enthält, also
\begin{align*}
F=\langle G_1,\ldots,G_n\rangle
\end{align*}


\end{proof}

\section*{Aufgabe 4 (4 Punkte)}
Sei $F\in\L(\RR)$ eine aussagenlogische Formel. Sei $\S(F)$ die  \textbf{Menge aller Teilformeln} von $F$. $G$ ist eine \textbf{Teilformel} von $F:\gdw G\in \S(F)$. Sei $\mathcal{P}_F$ die \textbf{Menge aller Positionen} in $F$.
\begin{enumerate}
\item Bestimmen Sie, ob folgende Aussage gilt:
\begin{align*}
\big|\S(F)\big|=\big|\PP_F\big|
\end{align*}
\item Beweisen oder widerlegen Sie die Aussage schrittweise.\\
(Zur Lösung der Aufgabe dürfen Sie alle in der Vorlesung und/oder Übung bewiesenen Resultate (Theoreme, Sätze, etc.) benutzt werden.)
\end{enumerate}
\begin{lösung}
Ja, die Aussage stimmt.
\begin{proof}
Strukturelle Induktion über $F\in\L(\RR)$:\\
\underline{Induktionsanfang:} Sei $F=A\in\RR$ ein Atom. Dann ist $|\S(F)|=|F|=1=|\lbrace\Lambda\rbrace|=|\PP_F|$\\

\underline{Induktionsvoraussetzung:} Seien $F_1,F_2\in\L(\RR)$    so, dass $\big|\S(F)\big|=\big|\PP_F\big|$ gilt.\\

\underline{Induktionsschritte:}
Negation:
\begin{align*}
a
\end{align*}
Sei $\circ$ ein beliebiger zweistelliger Junktor. Dann gilt:
\begin{align*}
a
\end{align*}
\end{proof}
\end{lösung}

\section*{Aufgabe 5 (9 Punkte)}
Im Folgenden (wie in der Vorlesung)
\begin{itemize}
\item sei $\RR=\big\lbrace p_1,p_2,p_3,\ldots\big\rbrace$ die Menge der aussagenlogischen Variablen und
\item wir definieren $\L(\RR,n)\subseteq\L(\RR)$ als die Menge der aussagenlogischen Formeln, in denen höchstens die aussagenlogischen Variablen $p_1,\ldots,p_n$ vorkommen.
\item Für eine aussagenlogische Formelmenge $\F$ sei $\G_n=\F\cap\L(\RR,n)$
\end{itemize}
Zeigen Sie ähnlich wie im Beweis des \textbf{Endlichkeitssatzes} in der Vorlesung, dass jede der Mengen $\G_n$ erfüllbar ist, falls jede endliche Teilmenge von $\F$ erfüllbar ist.

\begin{proof}
Sei also $\F\in\L(\RR)$ beliebig und $n\in\N$ beliebig. Da nach Voraussetzung jede endliche Teilmenge von $\F$ erfüllbar ist, ist $\G_n$ erfüllbar, weil $\G_n$ endlich ist, denn:
\end{proof}


\end{document}