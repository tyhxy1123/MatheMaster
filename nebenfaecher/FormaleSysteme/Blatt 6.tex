% This work is licensed under the Creative Commons
% Attribution-NonCommercial-ShareAlike 4.0 International License. To view a copy
% of this license, visit http://creativecommons.org/licenses/by-nc-sa/4.0/ or
% send a letter to Creative Commons, PO Box 1866, Mountain View, CA 94042, USA.

\documentclass[12pt,a4paper]{article} 

% This work is licensed under the Creative Commons
% Attribution-NonCommercial-ShareAlike 4.0 International License. To view a copy
% of this license, visit http://creativecommons.org/licenses/by-nc-sa/4.0/ or
% send a letter to Creative Commons, PO Box 1866, Mountain View, CA 94042, USA.

% PACKAGES
\usepackage[english, ngerman]{babel}	% Paket für Sprachselektion, in diesem Fall für deutsches Datum etc
\usepackage[utf8]{inputenc}	% Paket für Umlaute; verwende utf8 Kodierung in TexWorks 
\usepackage[T1]{fontenc} % ö,ü,ä werden richtig kodiert
\usepackage{amsmath} % wichtig für align-Umgebung
\usepackage{amssymb} % wichtig für \mathbb{} usw.
\usepackage{amsthm} % damit kann man eigene Theorem-Umgebungen definieren, proof-Umgebungen, etc.
\usepackage{mathrsfs} % für \mathscr
\usepackage[backref]{hyperref} % Inhaltsverzeichnis und \ref-Befehle werden in der PDF-klickbar
\usepackage{graphicx}
\usepackage{grffile}
\usepackage{setspace} % wichtig für Lesbarkeit. Schöne Zeilenabstände
\usepackage{enumitem} % für custom Liste mit default Buchstaben
\usepackage{ulem} % für bessere Unterstreichung
\usepackage{contour} % für bessere Unterstreichung
\usepackage{epigraph} % für das coole Zitat
\usepackage{float}            % figure-Umgebungen besser positionieren
\usepackage{xfrac}
\usepackage{xr} % man Referenzen aus anderen teX-files importieren und darauf verlinken
\usepackage{bbm} %sorgt für Symbol für Indikatorfunktion
\usepackage{color} % bringt Farbe ins Spiel
\usepackage{pdflscape} % damit kann man einzelne Seiten ins Querformat drehen
\usepackage{aligned-overset} % besseres Einrücken, siehe: https://tex.stackexchange.com/questions/257529/overset-and-align-environment-how-to-get-correct-alignment
\usepackage{pgfplots}
	\pgfplotsset{compat=newest}

\usepackage[
    type={CC},
    modifier={by-nc-sa},
    version={4.0},
]{doclicense} % für CC Lizenz-Vermerk

\usepackage{tikz}
	\usepackage{tikz-qtree}
	\usetikzlibrary{arrows}
	\usetikzlibrary{automata}
  \usetikzlibrary{babel}
	\usetikzlibrary{calc}
	\usetikzlibrary{cd}
	\usetikzlibrary{fit}
	\usetikzlibrary{matrix}
	\usetikzlibrary{positioning}
	\usetikzlibrary{shapes.geometric}
	
\usepackage{csquotes}
	\MakeOuterQuote{"}

\usepackage{xargs} % for multiple optional args in newcommand
\usepackage{lmodern} % provides a bigger set of font sizes
\usepackage{anyfontsize} % supports fallback scaling for non-existing font size
\usepackage{scrhack} % provides a hack for deprecated float environments used by some libs

% Ich habe gelesen, dass man folgendes Package zuletzt einbinden soll:
\usepackage[english, ngerman, capitalise]{cleveref} % bessere Verweise

% THEOREM-ENVIRONMENTS

\newtheoremstyle{mystyle}
  {20pt}   % ABOVESPACE \topsep is default, 20pt looks nice
  {20pt}   % BELOWSPACE \topsep is default, 20pt looks nice
  {\normalfont} % BODYFONT
  {0pt}       % INDENT (empty value is the same as 0pt)
  {\bfseries} % HEADFONT
  {}          % HEADPUNCT (if needed)
  {5pt plus 1pt minus 1pt} % HEADSPACE
	{}          % CUSTOM-HEAD-SPEC
\theoremstyle{mystyle}

% Definitionen der Satz, Lemma... - Umgebungen. Der Zähler von "satz" ist dem "section"-Zähler untergeordnet, alle weiteren Umgebungen bedienen sich des satz-Zählers.
\newtheorem{satz}{Satz}[section]
\newtheorem{lemma}[satz]{Lemma}
\newtheorem{korollar}[satz]{Korollar}
\newtheorem{proposition}[satz]{Proposition}
\newtheorem{beispiel}[satz]{Beispiel}
\newtheorem{definition}[satz]{Definition}
<<<<<<< HEAD
\newtheorem{bemerkungnr}[satz]{Bemerkung}
=======
\newtheorem{theorem}[satz]{Theorem}
>>>>>>> 31c1b8aef8833910045abfa6196d99bf06dfe6b5

% Bemerkungen, Erinnerungen und Notationshinweise werden ohne Numerierungen dargestellt.
\newtheorem*{bemerkung}{Bemerkung.}
\newtheorem*{erinnerung}{Erinnerung.}
\newtheorem*{notation}{Notation.}
\newtheorem*{beisp}{Beispiel.} %Beispiel ohne Nummerierung
<<<<<<< HEAD

=======
\newtheorem*{defi}{Definition.} %Definition ohne Nummerierung
>>>>>>> 31c1b8aef8833910045abfa6196d99bf06dfe6b5

% SHORTCUTS
\newcommand{\R}{\mathbb{R}}				 % reelle Zahlen
\newcommand{\Rn}{\R^n}						 % der R^n
\newcommand{\N}{\mathbb{N}}				 % natürliche Zahlen
\newcommand{\Z}{\mathbb{Z}}				 % ganze Zahlen
\newcommand{\C}{\mathbb{C}}			   % komplexe Zahlen
\renewcommand{\mit}{\text{ mit }}   % mit
\newcommand{\falls}{\text{falls }} % falls
\renewcommand{\d}{\text{ d}}        % Differential d

% ETWAS SPEZIELLERE ZEICHEN
% disjunkte Vereinigung
\newcommand{\bigcupdot}{
	\mathop{\vphantom{\bigcup}\mathpalette\setbigcupdot\cdot}\displaylimits
}
\newcommand{\setbigcupdot}[2]{\ooalign{\hfil$#1\bigcup$\hfil\cr\hfil$#2$\hfil\cr\cr}}
% großes Kreuz
\newcommand*{\bigtimes}{\mathop{\raisebox{-.5ex}{\hbox{\huge{$\times$}}}}} 

% WHITESPACE COMMANDS
% nicht restriktiver newline command
\newcommand{\enter}{$ $\newline} 
% praktischer Tabulator
\newcommand\tab[1][1cm]{\hspace*{#1}}

% TEXT ÜBER ZEICHEN
\newcommand{\stackeq}[1]{\stackrel{#1}{=}} 

% UNDERLINE
% besseres underline 
\renewcommand{\ULdepth}{1pt}
\contourlength{0.5pt}
\newcommand{\ul}[1]{
	\uline{\phantom{#1}}\llap{\contour{white}{#1}}
}


% This work is licensed under the Creative Commons
% Attribution-NonCommercial-ShareAlike 4.0 International License. To view a copy
% of this license, visit http://creativecommons.org/licenses/by-nc-sa/4.0/ or
% send a letter to Creative Commons, PO Box 1866, Mountain View, CA 94042, USA.

% hier noch ein paar Commands die nur ich nutze, weil ich sie mir im Laufe der Jahre angewöhnt habe und sie mir jetzt nicht abgewöhnen will:

\newcommand{\gdw}{\Leftrightarrow}             % genau dann, wenn
\DeclareMathOperator{\id}{id}                  % identische Abbildung
\DeclareMathOperator{\dom}{dom}                % Domain, Definitionsbereich
\DeclareMathOperator{\rg}{rg}                  % Rang, Bild einer Funktion, Rang einer Matrix
\newcommand{\K}{\mathbb{K}}                    % Körper
\newcommand{\limn}{\lim\limits_{n\to\infty}}   % genau dann, wenn
\newcommand{\sonst}{\text{sonst }}             % sonst
\renewcommand{\O}{\mathcal{O}}                 % Landau-O
\DeclareMathOperator{\Div}{div}                % Divergenz
\renewcommand{\div}{\Div}                      % Divergenz
\DeclareMathOperator{\spann}{span}             % Span
\DeclareMathOperator{\diam}{diam}       	   % Diameter, Durchmesser einer Menge
\DeclareMathOperator{\inner}{int}              % Das innere einer Menge





\author{Willi Sontopski}

\parindent0cm %Ist wichtig, um führende Leerzeichen zu entfernen

\usepackage{pdflscape}
\usepackage{rotating}
\usepackage{scrpage2}
\pagestyle{scrheadings}
\clearscrheadfoot

\ihead{Willi Sontopski}
\chead{Formale Systeme WiSe 18 19}
\ohead{}
\ifoot{Blatt 6}
\cfoot{Version: \today}
\ofoot{Seite \pagemark}

\newcommand{\F}{\mathcal{F}}
\newcommand{\pos}{\text{pos}}
\newcommand{\RR}{\mathcal{R}}
\newcommand{\PP}{\mathcal{P}}
\renewcommand{\L}{\mathcal{L}}
\newcommand{\Res}{\text{Res}}
\newcommand{\Reso}{\stackrel{\Res}{\implies}}

\begin{document}
%\setcounter{section}{1}

\section*{Aufgabe 6.1 (Resulotionsableitungen)}
Resolventenbildung: Entferne $A$ aus $C_1$ und $\neg A$ aus $C_2$ und verknüpfe den Rest disjunktiv.\\
Die Auswahl der beiden Klausel für die Resolventenbildung erfolgt mit der \textit{Methode des scharfen Hinsehens}. Es ist sinnvoll rückwärts zu denken:
\begin{align*}
[A,B],[A,\neg B],[\neg A]\Reso[A,A],[\neg A]\Reso[~]
\end{align*}
Eine andere Strategie ist, zu Beginn alle einstelligen Klauseln auszunutzen. In Aufgabe 1 (b) kann man in allen Klauseln schon $\neq s$ und $r$ rausstreichen, wegen den Klauseln 4 und 5.

\subsection*{Aufgabe 6.1}
\begin{align*}
\begin{array}{rll}
9 & \Res(2,6,t) & [q,r]\\
10 & \Res(9, 7,r) & [q]\\
11 & \Res(10,8,q) & [\neg t]\\
12 & \Res(11,2, t) & [~]
\end{array}
\end{align*}

\subsection*{Aufgabe 6.2}
\begin{align*}
\begin{array}{rll}
9 & \Res(2,4,s) & [p,q]\\
10 & \Res(3,4,s) & [q,\neq p]\\
11 & \Res(9,10,p) & [q,q]\\
12 & \Res(4,8) & [\neg q,\neg q]\\
13 & \Res(11,13) & [~]
\end{array}
\end{align*}

\section*{Aufgabe 6.2 (Anwendungen des Resolutionsverfahrens}
Strategie:
\begin{enumerate}
\item Ersetze $A\to B$ durch $\neg A\vee B$.
\item Schreibe Formel so um, dass Unerfüllbarkeit zu zeigen ist.
\item Bringe Formel in KNF mit den bekannten Ersetzungsregeln.
\item Wende Resolutionsverfahren an (wie Aufgabe 1).
\end{enumerate}

\subsection*{Aufgabe 6.2 (a)}
Hier schreiben betrachten wir direkt die Negation, da wir die Tautologie-Eigenschaft zeigen wollen, indem wir zeigen, dass die Negation unerfüllbar ist.
\begin{align*}
&\neg((((p\wedge q)\to r)\wedge\neg r)\to(p\to(q\to r)))\\
&\equiv
\neg(\neg((\neg(p\wedge q)\vee r)\wedge\neg r)\vee(\neg p\vee(\neg q\vee r)))\\
&\equiv
\langle[\neg(\neg((\neg(p\wedge q)\vee r)\wedge\neg r)\vee(\neg p\vee(\neg q\vee r)))]\rangle\\
&\equiv
\langle[\neg\neg((\neg(p\wedge q)\vee r)\wedge\neg r)],[\neg(\neg p\vee(\neg q\vee r))]\rangle\\
&\equiv
\langle[((\neg(p\wedge q)\vee r)\wedge\neg r)],[\neg(\neg p\vee(\neg q\vee r))]\rangle\\
&\equiv
\langle[(\neg(p\wedge q)\vee r)],[\neg r],[\neg(\neg p\vee(\neg q\vee r))]\rangle\\
&\equiv
\langle[\neg(p\wedge q), r],[\neg r],[\neg(\neg p\vee(\neg q\vee r))]\rangle\\
&\equiv
\langle[\neg p,\neg q, r],[\neg r],[\neg(\neg p\vee(\neg q\vee r))]\rangle\\
&\equiv
\langle[\neg p,\neg q, r],[\neg r],[\neg\neg p],[\neg(\neg q\vee r)]\rangle\\
&\equiv
\langle[\neg p,\neg q, r],[\neg r],[p],[\neg(\neg q\vee r)]\rangle\\
&\equiv
\langle[\neg p,\neg q, r],[\neg r],[p],[\neg\neg q],[\neg r)]\rangle\\
&\equiv
\langle[\neg p,\neg q, r],[\neg r],[p],[q],[\neg r)]\rangle\\,
&\begin{array}{rll}
1 &&[\neg p,\neg q,r]\\
2 &&[\neg r]\\
3 &&[p]\\
4 &&[q]\\
5 &&[\neg r]\\
6 & \Res(1,2,r) & [\neg p,\neg q]\\
7 & \Res(6,3,p) & [\neg q]\\
8 & \Res(7,4,q) & []
\end{array}
\end{align*}

\subsection*{Aufgabe 6.2 (b)}
\begin{align*}
&((\neg r\vee(p\wedge q))\wedge\neg((r\to p)\wedge(r\to q)))\\
&\equiv
((\neg r\vee(p\wedge q))\wedge\neg((\neg r\vee p)\wedge(\neg r\vee q)))\\
&\equiv
\langle[((\neg r\vee(p\wedge q))\wedge\neg((\neg r\vee p)\wedge(\neg r\vee q)))]\rangle\\
&\equiv
\langle[(\neg r\vee(p\wedge q))],[\neg((\neg r\vee p)\wedge(\neg r\vee q))]\rangle\\
&\equiv
\langle[\neg r,p],[\neg r, q],[\neg((\neg r\vee p)\wedge(\neg r\vee q))]\rangle\\
&\equiv
\langle[\neg r,p],[\neg r, q],[\neg(\neg r\vee p),\neg(\neg r\vee q)]\rangle\\
&\equiv
\langle[\neg r,p],[\neg r, q],[\neg\neg r,\neg(\neg r\vee q)],[\neg p,\neg(\neg r\vee q)]\rangle\\
&\equiv
\langle[\neg r,p],[\neg r, q],[r,\neg(\neg r\vee q)],[\neg p,\neg(\neg r\vee q)]\rangle\\
&\equiv
\langle[\neg r,p],[\neg r, q],[r,\neg\neg r],[r,\neg q],[\neg p,\neg\neg r],[p,\neg q]\rangle\\
&\equiv
\langle[\neg r,p],[\neg r, q],[r,r],[r,\neg q],[\neg p,r],[p,\neg q]\rangle\\
&\begin{array}{rll}
1 &&[\neg r,p]\\
2 &&[\neg r,q]\\
3 &&[r,r]\\
4 &&[r,\neg q]\\
5 &&[\neg p,r]\\
6 &&[p,\neg q]\\
%7 & \Res(1,2,r) & [\neg p,\neg q]\\
%8 & \Res(6,3,p) & [\neg q]\\
\end{array}
\end{align*}
Verrechnet, denn das ist erfüllbar mit $p^I=q^I=r^I=\top$.

\section*{Aufgabe 6.2 (c)}
Wieder von Beginn an negieren, da wir Unerfüllbarkeit zeigen wollen.
\begin{align*}
&\neg(((p\to q)\to p)\to p)\\
&\equiv
\neg(\neg(\neg(\neg p\vee q)\vee p)\vee p)\\
&\equiv
\langle[\neg(\neg(\neg(\neg p\vee q)\vee p)\vee p)]\rangle\\
&\equiv
\langle[\neg\neg(\neg(\neg p\vee q)\vee p)], [\neg p]\rangle\\
&\equiv
\langle[(\neg(\neg p\vee q)\vee p)], [\neg p]\rangle\\
&\equiv
\langle[\neg(\neg p\vee q), p], [\neg p]\rangle\\
&\equiv
\langle[\neg\neg p, p],[\neg q, p], [\neg p]\rangle\\
&\equiv
\langle[p, p],[\neg q, p], [\neg p]\rangle\\
&\Reso[~]\mit\Res([p,p],[\neg p])
\end{align*}

\section*{Aufgabe 6.2 (d)}
Wieder von Beginn an negieren, da wir Unerfüllbarkeit zeigen wollen.
\begin{align*}
&\neg(((p\to q)\wedge(q\to r))\to\neg(\neg r\wedge p))\\
&\equiv
\neg(\neg((\neg p\vee q)\wedge(\neg q\vee r))\vee\neg(\neg r\wedge p))\\
&\equiv
\langle[\neg(\neg((\neg p\vee q)\wedge(\neg q\vee r))\vee\neg(\neg r\wedge p))]\rangle\\
&\equiv
\langle[\neg\neg((\neg p\vee q)\wedge(\neg q\vee r))],[\neg\neg(\neg r\wedge p)]\rangle\\
&\equiv
\langle[((\neg p\vee q)\wedge(\neg q\vee r))],[(\neg r\wedge p)]\rangle\\
&\equiv
\langle[(\neg p\vee q)],[(\neg q\vee r)],[(\neg r\wedge p)]\rangle\\
&\equiv
\langle[\neg p, q],[\neg q,r],[(\neg r\wedge p)]\rangle\\
&\equiv
\langle[\neg p, q],[\neg q,r],[\neg r],[p]\rangle\\
&\begin{array}{rll}
1 &&[\neg p,q]\\
2 &&[\neg q,r]\\
3 &&[\neg r]\\
4 &&[p]\\
5 & \Res(2,3,r) & [\neg q]\\
6 & \Res(1,4) & [q]\\
7 & \Res(5,6) &[~]
\end{array}
\end{align*}

\section{Aufgabe 6.3 (Positive/negative Klauseln und Erfüllbarkeit}
Eine Klausel heißt \textbf{positiv} gdw. sie nur positive Literale (= aussagenlogische Variable) enthält, und eine Klausel heißt \textbf{negativ} gdw. sie nur negative Literale (= negierte aussagenlogische Variable) enthält.\nl
Eine Klauselmenge (= Menge von Klauseln) heißt \textbf{erfüllbar} gdw. es eine Interpretation $I$ gibt, die jede Klausel aus der Klauselmenge erfüllt.

\section*{Aufgabe 6.3 (a)}
Eine Klauselmenge ist erfüllbar, wenn sie keine positive Klausel enthält.

\begin{proof}
Zeige dies indirekt, zeige also: Wenn eine Klauselmenge unerfüllbar ist, dann enthält sie eine positive Klausel.\\
Sei also $M=\lbrace C_1,\ldots,C_n\rbrace$ eine unerfüllbare Klauselmenge. Nach Definition gibt es dann keine Interpretation, die alle $C_i,i=1,\ldots,n$ erfüllt.\\ Anders ausgedrückt: Für alle Interpretationen $I$ gibt es ein $j\in\lbrace1,\ldots n\rbrace$ so, dass $C_j^I=\bot$. (Beachte, dass $j$ von $I$ abhängt.) Sei also $I$ beliebig aber fest. Benenne $C_j=[L_{1,j},\ldots, L_{m,j}]$.
\begin{align*}
C_j^I\Longleftrightarrow\forall k\in\lbrace 1,\ldots,m\rbrace: L_{k,j}^I=\bot
\end{align*}
TODO: Beweis ist Mist.
\end{proof}

\section*{Aufgabe 6.3 (b)}
Eine Klauselmenge ist erfüllbar, wenn sie keine negative Klausel enthält.

\begin{proof}

\end{proof}

\section*{Aufgabe 6.3 (c)}
Die Aussage ist falsch, denn die leere Klausel $[~]$ ist sowohl positive Klausel als auch negative Klausel.

\section*{Aufgabe 6.4 (Tautologieelimination)}
Sei $\langle D_1,\ldots,D_n\rangle$ eine verallgemeinerte Konjunktion mit verallgemeinerten Disjunktionen $D_1,\ldots,D_n$. Dann gilt:\\
Wenn in einer verallgemeinerten Disjunktion $D_j~(j\in\lbrace1,\ldots,n\rbrace)$ sowohl $F$ als auch $\neg F$ vorkommt (wobei $F$ beliebige aussagenlogische Formel ist), dann gilt:
\begin{align*}
\big\langle D_1,\ldots,D_{j-1},D_j,D_{j+1},\ldots, D_n\big\rangle\equiv\big\langle D_1,\ldots,D_{j-1},D_{j+1},\ldots,D_n\big\rangle
\end{align*}
\begin{proof}
Sei $I$ beliebige Interpretation und $fj\in\lbrace1,\ldots,n\rbrace$. Da $F,\neg F\in D_j$ ist $_jD$ Tautologie, d.h. $D_j^I=\top$. Da $\top$ das neutrale Element der Disjunktion ist und $\langle\cdot\rangle$ die verallgemeinerte Disjunktion ist folgt die Behauptung aus der Tatsache, dass $I$ beliebig war.
\end{proof}

\section*{Aufgabe 6.5 (Subsumtion)}
Eine Klausel $C$ \textbf{subsumiert} eine Klausel $C'$,i.Z. $C\subseteq C'$ gdw. jedes Literal aus $C$ auch in $C'$ vorkommt.\nl
Sei $F=\langle C_1,\ldots,C_n\rangle$ eine Formel in Klauselform und seien $C_i$ und  $C_j$ Klauseln aus $F$ mit $i\neq j,i,j\in\lbrace1,\ldots,n\rbrace$.\\
Wenn die Klausel $C_i$ die Klausel $C_j$ subsumiert, dann gilt:
\begin{align*}
\big\langle C_1,\ldots,C_n\big\rangle\text{ ist erfüllbar gdw. }\big\langle C_1,\ldots C_{j-1},C_{j+1},\ldots,C_n\big\rangle\text{ ist erfüllbar.}
\end{align*}
\begin{proof}
To Do
\end{proof}

\section*{Aufgabe 6.6 (Folgerungen und endlich viele Prämisse)}
\begin{align*}
\F\models G\Longleftrightarrow \exists\F'\subseteq\F\text{ endlich }:\F'\models G
\end{align*}
\begin{proof}
To Do
\end{proof}

\section*{Aufgabe 6.7 (Korrektheit und Vollständigkeit)}
\subsection*{Aufgabe 6.7 (a)}
Geben Sie ein Verfahren zur Ermittlung der aussagenlogischen Allgemeingültigkeit
an, das korrekt aber nicht vollständig ist.

\begin{lösung}
To Do 
\end{lösung}

\subsection*{Aufgabe 6.7 (b)}
Geben Sie ein Verfahren zur Ermittlung der aussagenlogischen Allgemeingültigkeit
an, das vollständig aber nicht korrekt ist.

\begin{lösung}
To Do
\end{lösung}

\end{document}