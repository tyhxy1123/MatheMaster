% This work is licensed under the Creative Commons
% Attribution-NonCommercial-ShareAlike 4.0 International License. To view a copy
% of this license, visit http://creativecommons.org/licenses/by-nc-sa/4.0/ or
% send a letter to Creative Commons, PO Box 1866, Mountain View, CA 94042, USA.

\section{Aufgabenblatt 12}
\subsection*{Aufgabe $\ast$)}
Die dritte Bedingung in $L$ sagt, dass die Wörter in $L$ nicht mit $a$ beginnen dürfen.
Kurz geschrieben (Achtung, keine offizielle Notation!)
\begin{align*}
	L=\Big\lbrace w=u_1 babc u_2,w=u_3 ccc u_4,w\neq a u_5:u_1,\ldots,u_5\in\Sigma^\ast\Big\rbrace
\end{align*}
Somit erhält man den Regulären Ausdruck 
\begin{align*}
	r=(b+c)^\ast\cdot(a+b+c)^\ast\cdot(b\cdot a\cdot b\cdot c+c\cdot c\cdot c)\cdot(a+b+c)^\ast
\end{align*}

\subsection*{Aufgabe $\ast\ast$)}
\begin{enumerate}[label=(\alph*)]
	\item $\begin{aligned}
		L(r_1)=\Big\lbrace b^m a^n:m\in\N_{\geq0},n\in\lbrace0,1\rbrace\Big\rbrace
	\end{aligned}$
	\item $\begin{aligned}
		L(r_1)=\Big\lbrace b^m a^n:m\in\N_{\geq0},n\in\lbrace0,1\rbrace\Big\rbrace
	\end{aligned}$
	\item $\begin{aligned}
		L(r_1)=\Big\lbrace b^m a^n:m\in\N_{\geq0},n\in\lbrace0,1\rbrace\Big\rbrace
	\end{aligned}$
\end{enumerate}

\subsection{Aufgabe 1}

\begin{lösung}
	\underline{Zeige a):}
	
	\underline{Zeige b):}
	
	
	\underline{Zeige c):}
	
	
	\underline{Zeige e):}
		
	\underline{Zeige f):}
	
\end{lösung}

\subsection{Aufgabe 2}

\begin{lösung}
	
\end{lösung} 

\subsection{Aufgabe 3}
Betrachte die Grammatik
\begin{align*}
	G_0&=\Big(\lbrace S,T,U,V,R\rbrace,\lbrace a,b\rbrace,P_0,S\Big)\\
	P_0&=\left\lbrace
		\begin{array}{c}
			 S\to\varepsilon,S\to aSb,S\to T,S\to R,\\
			 T\to bbT, T\to U\\
			 U\to aa U,U\to bbT\\
			 V\to bSa\\
			 R\to\varepsilon\\
			 R\to bSa
		\end{array}\right\rbrace		
\end{align*}
	
\subsubsection{Aufgabe 3 a)}
Geben Sie zu $G_0$ alle nicht-terminierenden Symbole und nicht-erreichbaren Symbole an und geben Sie eine zu $G_0$ äquivalente reduzierte Grammatik $G_1$ an.

\begin{lösung}
	%TODO
\end{lösung}

\subsubsection{Aufgabe 3 b)}
Konstruieren Sie eine Grammatik $G_2$ mit $L(G_2)=L(G_1)\setminus\lbrace\varepsilon\rbrace$, die keine Regeln der Form $A\to\varepsilon$ für $A\in N$ enthält.

\begin{lösung}
	%TODO
\end{lösung}

\subsubsection{Aufgabe 3 c)}
Geben Sie ein zu $G_1$ äquivalente $\varepsilon$-freie Grammatik $G_3$ an.
Erweitern Sie dazu, wenn nötig, die Grammatik $G_2$ um ein neues Startsymbol $S_3$ und entsprechende Regeln.

\begin{lösung}
	%TODO
\end{lösung}

\subsubsection{Aufgabe 3 d)}
Geben Sie eine zu $G_3$ äquivalente Grammatik $G_4$ an, die keine Produktionen der Form $A\to B$ mit Nichtterminalsymbolen $A,B$ enthält.

\begin{lösung}
	%TODO
\end{lösung}

\subsubsection{Aufgabe 3 e)}
Geben Sie eine zu $G_4$ äquivalente Grammatik $G_5$ in Chomsky-Normalform an.

\begin{lösung}
	%TODO
\end{lösung}