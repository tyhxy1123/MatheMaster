\section{Lineare Optimierung}
ganzzahlige Lineare Programme:
\begin{equation*}
	\min_{x} c^Tx \text{ subject to } \quad Ax = b, x \geq 0, x \in \Z^n
\end{equation*} 
\subsection{Optimaltiätsbedingungen für kontinuierliche LPs}
\begin{itemize}
	\item LP in Normalform 
		\begin{equation*}
			\min_{x} c^Tx \text{ subject to } Ax = b, x\geq 0
		\end{equation*} 
	\item zulässige Menge: Polyeder
		\begin{equation*}
			P = \left\{ x | Ax = b ,x \geq 0 \right\}
		\end{equation*} 
	\item Problem konvex $\to$	lokale Minima = globale Minima	
	\item duales Problem
		\begin{equation}\label{D}
			\max_{y}b^Ty \text{ subject to } A^Ty \leq c, y \in \R^n
		\end{equation} 
	\item schwache Dualität\\
		$\forall x$ zulässig für $(P)$\\% todo mach das ordentlich
		$\forall y$ zulässig für $(D)$:% todo mach das ordentlich
		\begin{equation*}
			c^Tx \geq b^Ty
		\end{equation*} 
		bei Gleichheit: $x$ löst $(P)$, $y$ löst $(D)$
	\item Starke Dualität\\
		Es sind äquivalent 
		\begin{enumerate}
			\item ($P$) hat eine Lösung $x^*$
			\item ($D$) hat eine Lösung $y^*$ 
			\item mk
		\end{enumerate} 
	\item KKT-Bedingungen für LPs
		\begin{itemize}
			\item $x^*$ löst $(P)$ 
			\item $y^*$ löst $(D)$ 
			\item $(x^*,y^*)$ lösen das KKT-System 
				\begin{align*}
					Ax^* &= b , \ x^* \geq 0 \quad \text{(primale Zulässigkeit)}\\
					A^{T} y^* &\leq c\quad \text{(duale Zulässigkeit)}\\
					\forall i = 1,\dots , n: x_{i}&= 0 \text{ oder } \left(A^{T}y^*  \right)_{i} = c_{i} \text{ (Komplementaritätsbedingungen)}
				\end{align*} 
				\begin{equation*}
					b^{T}\underbrace{y^* = (\underbrace{x^*}_{\geq 0})^{T}}_{\text{primal Zul.}} \underbrace{\underbrace{A^{T} y^*}_{\leq c }}_{\text{dual Z.}} \stackrel{\text{Kompbed.}}= \left(x^*\right)^{T} c
				\end{equation*} 
		\end{itemize} 
		
	\item Ecken und Lösungen\\
		$ \hat{x}$ Ecke von $P \iff$  $ \hat{x}$ kann nicht als echte Konvexkombination von Punkten aus $P$ geschrieben werden
		$\iff$ Sei 
		\begin{equation*}
			I = \{i = 1 ,\dots, n | \hat{x}_{i} \geq 0 \} = \supp(\hat{x})
		\end{equation*} 
		Dann gilt $\rang(A_{\cdot,I})=|I|$
	\item Hat das LP $(P)$ eine Lösung, so gibt es auch eine Ecklösung.
	\item Es gibt nur endlich viele Ecken
	\item Idee des Simplex-Verfahrens: Ecken von $P$ in sinnvolle Reihenfolge absuchen.\\
		\underline{Annahme:} $A \in \R^{m \times n }$ hat \underline{vollen Zeilenrang} m. \\
		Sei $\hat{x}$ eine Ecke von $P$ und $I = \{i | \hat{x}_{i}>0\}$ \\
		$\implies$ rang($A_{\cdot,I}   )= |I| \leq m$ \\
		$\implies$ Es gibt $B \subset \{1,\dots,n\}$ mit $I \subset B, |B| = m $ und $A_{B}= A_{\cdot,B}$ ist regulär
	\item Basen\\
		Sei $B \subset \{1,\dots,n\}$ mit $|B|=m$ und $N = \{1,\dots,n\}\textbackslash B$\\
		$\to B$ heißt \underline{Basis}, falls $A_{B}$ regulär\\
		$\to$ $B$ heißt \underline{primal zulässige Basis}, falls
		\begin{equation*}
			x_{B}=A_{B}^{-1}b \geq 0 
		\end{equation*} $\implies$ Mit $x_{N} = 0$ ist $(x_{B},x_{N})$ zulässig für $(P)$ und eine Ecke von $P$ \\
		$\to$ B heißt \underline{dual zulässige Basis}, falls 
		\begin{equation*}
			A_{N}^TA_{B}^{-T}c_{b} \leq c_{N} \implies y \text{ ist zulässig für }(D)
		\end{equation*} 
		$\implies$ $\left(x_{B},x_{N} \right)$ mit $x_{B} = A^{-1}_{B},\ x_{N}= 0$ heißt \underline{Basislösung}. Ist $x_{B} > 0 $ so heißt die Basislösung \underline{nichtdegeneriert}.
\end{itemize} 

\subsection{Wiederholung Simplex-Verfahren}
Annahme: $A \in R^{m \times n}$ hat Rang $m$.\\
Sei $B$ eine primal zulässige Basis mit Basislösung 
\begin{align*}
	\hat{x}_{N}&=0,\\
	\hat{x}_{B}&= A_{B}^{-1}b
\end{align*} 
Für alle $x \in P$ gilt:
\begin{align*}
	Ax&=(A_{B},A_{N})(x_{B},x_{N})^{T}\\
	  & A_{B}x_{B} + A_{N}x_{N}= b
\end{align*} 
$\implies x_{B} = A_{B}^{-1}b - A_{B}^{-1}A_{N}x_{N} \to$ Zielfunktion vereinfachen
\begin{align*}
	c^Tx&= c_{B}^Tx_{B} + c_{N}^Tx_{N} = c_{B}^T \underbrace{A_{B}^{-1}b}_{= \hat{x}_{B}} - c_{B}^TA_{B}^{-1}A_{N}x_{N} + c_{N}^Tx_{N}\\
		&= c^{T}_{B} \hat{x}_{B} +( \underbrace{ c_{N} - A_{N}^{T} \underbrace{A_{B}^{-T}c_{B}}_{=y} }_{=: z_{N}, \text{ reduzierte Kosten}} )^{T} x_{N} 
\end{align*} 

\begin{itemize}
	\item gilt $z_{N} \geq 0$: $\hat{x}$ löst LP ( mit dualer Variable y)
	\item Andernfalls: Wähle $j\in \N$ mit $z_{j}<0$ \\
		Frage: Wie groß kann $x_{j}$ gewählt werden, ohne $P$ zu verlassen?\\
		neues $x_{N}= \gamma e_{j}$ \\
		$\implies$ neues $x_{B} = \hat{x}_{B}-A_{B}^{-1}A_{N}\gamma e_{j}= \hat{x}_{B}-A_{B}^{-1}A_{N}\gamma A_{B}^{-1}A_{\cdot,I} \overset{!}{\geq} 0$
		\begin{itemize}
			\item Wähle $\gamma = \min \{ \frac{\hat{x}_{i}}{w_{i}}|w_{i} > 0 \}$
			\item Andernfalls wähle $i \in B$ mit $\gamma = \frac{\hat{x}_{i}}{w_{i}}$
		\end{itemize} 
	\item neue Ecke
		\begin{align*}
			\hat{x}_{B}-\gamma_{w}&= x_{B}^+\\
			x_{j}^+& = \gamma\\
			x_{k}^+ &= 0 \quad \forall \text{ vorherigen } k\\
			B^+ &= (B \ \{i\})\cup \{j\}\\
			N^+ &= (N \ \{j\})\cup \{i\}\\
			B& = \{1,3,5\} \\
			B& = (3,1,5)
		\end{align*}
\end{itemize} 
gesucht: primal zulässige Basis für 
\begin{equation*}
	\min_{x} c^Tx \text{ subject to } Ax = b , x \geq 0
\end{equation*} 
Simplex-Phase 1: Löse das Hilfsproblem
\begin{align*}
	\min_{x,s} \sum_{i=1}^{m} s_{i} \text{ subject to } (A\ I) (x , s)^T & = b ,\\
	(x,s)& \geq 0
\end{align*} 
Gilt in einer Lösung ($x^0$), dass $s^* =0$, so ist das $x^*$ zulässig für $(P)$ und die Basis des Hilfsproblems \glqq ist\grqq\ eine primal zulässige Basis für $(P)$. 
Primal zulässige Startbasis für Hilfsproblem
\begin{equation*}
	x = 0, s = b \geq 0, \text{ Basismatrix } I
\end{equation*} 
Tableau-Methode für 
\begin{equation*}
	\min_{x}c^Tx \text{ subject to } Ax \leq b , x \geq 0
\end{equation*} 
