\section{Lineare Optimierung}
ganzzahlige Lineare Programme:
\begin{equation*}
	\min_{x} c^Tx \text{ subject to } \quad Ax = b, x \geq 0, x \in \Z^n
\end{equation*}
Zusätzliche obere Schranken für die Variablen sind auch möglich.
Beispielsweise wäre durch $0 \leq x \leq 1$, $x \in \Z^n$ ein binäres Problem gegeben.
\subsection{Optimalitätsbedingungen für kontinuierliche LPs}
\begin{itemize}
	\item LP in Normalform
		\begin{equation}\label{nf}
			\min_{x} c^Tx \text{ subject to } Ax = b, x\geq 0\tag{P}
		\end{equation}
	\item Die zulässige Menge ist ein Polyeder:
		\begin{equation*}
			P = \{ x | Ax = b ,x \geq 0 \}
		\end{equation*}
  \item Problem konvex (d.\,h\ Zielfunktion und zulässige Menge sind konvex) $\to$	lokale Minima = globale Minima
  \item Das dazu \underline{duale Problem} ist gegeben durch
		\begin{equation*}\label{dual_cont}
			\max_{y}b^Ty \text{ subject to } A^Ty \leq c, y \in \R^n \tag{D}
		\end{equation*}
\end{itemize}
\begin{lemma}[Schwache Dualität] \label{thm:schwache_dualitat}
		Es gilt für alle $x$ zulässig für \eqref{nf} und alle $y$ zulässig für \eqref{dual_cont}:
		\begin{equation*}
			c^Tx \geq b^Ty
		\end{equation*}
		Bei Gleichheit gilt außerdem: $x$ löst \eqref{nf}, $y$ löst \eqref{dual_cont}.
\end{lemma} % end lemma Schwache Dualität
\begin{lemma}[Starke Dualität] \label{thm:starke_dualitat}
  Es sind äquivalent
  \begin{enumerate}
    \item \eqref{nf} hat eine Lösung $x^*$.
    \item \eqref{dual_cont} hat eine Lösung $y^*$.
    \item $(x^*,y^*)$ lösen das \underline{KKT-System}
      \begin{align*}
        Ax^* &= b , \ x^* \geq 0 & \text{(primale Zulässigkeit)}\\
        A^{T} y^* &\leq c & \text{(duale Zulässigkeit)}\\
        \forall i = 1,\dots , n: x_{i}&= 0 \text{ oder } \left(A^{T}y^*  \right)_{i} = c_{i} & \text{ (Komplementaritätsbedingungen)}
      \end{align*}
      Dann gilt auch, dass die \underline{Dualitätslücke} gleich $0$ ist:
      \begin{equation*}
        b^{T}\underbrace{y^* = (\underbrace{x^*}_{\geq 0})^{T}}_{\text{primal Zul.}} \underbrace{\underbrace{A^{T} y^*}_{\leq c }}_{\text{dual Z.}} \stackrel{\text{Kompbed.}}= \left(x^*\right)^{T} c
      \end{equation*}
    \end{enumerate}
\end{lemma} % end lemma Starke Dualität
\paragraph{Lösungsansätze für lineare Programme}
\begin{itemize}
  \item Innere-Punkte-Verfahren. (polynomielle Laufzeit)
  \item Ellipsoid-Verfahren. (polynomielle Laufzeit)
  \item Symplex-Verfahren (theoretisch exponentielle Laufzeit, aber in der Praxis gut)
\end{itemize}
\paragraph{Ecken und Lösungen}
\begin{itemize}
  \item $\hat{x}$ ist Ecke von $P$ $\iff$ $ \hat{x}$ kann nicht als echte Konvexkombination von Punkten aus $P$ geschrieben werden
		$\iff$ Sei
		\begin{equation*}
			I = \{i = 1 ,\dots, n | \hat{x}_{i} \geq 0 \} = \supp(\hat{x})
		\end{equation*}
		Dann gilt $\rang(A_{\cdot,I})=|I|$.
	\item Hat das LP \eqref{nf} eine Lösung, so gibt es auch eine Ecklösung.
	\item Es gibt nur endlich viele Ecken.
	\item Idee des Simplex-Verfahrens: Ecken von $P$ in sinnvoller Reihenfolge absuchen.\\
		\underline{Annahme:} $A \in \R^{m \times n }$ hat \underline{vollen Zeilenrang} $m$. \\
		Sei $\hat{x}$ eine Ecke von $P$ und $I = \{i \mid \hat{x}_{i}>0\}$ \\
		$\implies$ rang($A_{\cdot,I}   )= |I| \leq m$ \\
    $\implies$ Es gibt $B \subseteq \{1,\dots,n\}$ mit $I \subseteq B, |B| = m $ und $A_{B}:= A_{\cdot,B}$ ist regulär. (Finde $B$ durch Basisergänzung, da $A$ $m$ linear unabhängige Spalten hat.)
	\item Basen\\
		Sei $B \subset \{1,\dots,n\}$ mit $|B|=m$ und $N = \{1,\dots,n\}\setminus B$
    \begin{itemize}[label=$\to$] % warum eigentlich ein Pfeil?
      \item $B$ heißt \underline{Basis}, falls $A_{B}$ regulär.
      \item $B$ heißt \underline{primal zulässige Basis}, falls
        \begin{equation*}
          x_{B}=A_{B}^{-1}b \geq 0
        \end{equation*} $\implies$ Mit $x_{N} = 0$ ist $(x_{B},x_{N})$ zulässig für \eqref{nf} und eine Ecke von $P$.
      \item $B$ heißt \underline{dual zulässige Basis}, falls
        \begin{equation*}
          A_{N}^TA_{B}^{-T}c_{b} \leq c_{N} \implies y \text{ ist zulässig für \eqref{dual_cont}}
        \end{equation*}
      \item $\left(x_{B},x_{N} \right)$ mit $x_{B} = A^{-1}_{B},\ x_{N}= 0$ heißt \underline{Basislösung}. Ist $x_{B} > 0 $ so heißt die Basislösung \underline{nichtdegeneriert}.
        (Im degenerierten Fall kann $B$ unterbestimmt sein, man bleibt u.\,U.\ in der gleichen Ecke.)
    \end{itemize}
\end{itemize}
%
\subsection{Wiederholung Simplex-Verfahren}
Annahme: $A \in R^{m \times n}$ hat Rang $m$. (Ansonsten können so lange linear von den anderen Zeilen abhängige Zeilen gelöscht werden, bis $A$ vollen Zeilenrang $m$ hat.)

Sei $B$ eine primal zulässige Basis mit Basislösung
\begin{align*}
	\hat{x}_{N}&=0,\\
	\hat{x}_{B}&= A_{B}^{-1}b
\end{align*}
Für alle $x \in P$ gilt:
\begin{align*}
	Ax&=
	\begin{pmatrix}
		A_{B} & A_{N}
	\end{pmatrix}
	\begin{pmatrix}
		x_{B} \\ x_{N}
	\end{pmatrix}
	\\
	  & A_{B}x_{B} + A_{N}x_{N}= b
\end{align*}
$\implies x_{B} = A_{B}^{-1}b - A_{B}^{-1}A_{N}x_{N} \to$ Zielfunktion vereinfachen
\begin{align*}
	c^Tx&= c_{B}^Tx_{B} + c_{N}^Tx_{N} = c_{B}^T \underbrace{A_{B}^{-1}b}_{= \hat{x}_{B}} - c_{B}^TA_{B}^{-1}A_{N}x_{N} + c_{N}^Tx_{N}\\
		&= c^{T}_{B} \hat{x}_{B} +( \underbrace{ c_{N} - A_{N}^{T} \underbrace{A_{B}^{-T}c_{B}}_{=y} }_{=: z_{N}, \text{ reduzierte Kosten}} )^{T} x_{N}
\end{align*}

\begin{itemize}
	\item Gilt $z_{N} \geq 0$: $\hat{x}$ löst LP (mit dualer Variable $y$)
	\item Andernfalls: Wähle $j\in \N$ mit $z_{j}<0$.

		Frage: Wie groß kann $x_{j}$ gewählt werden, ohne $P$ zu verlassen?

		Wir wählen das neue $x_{N}$ als $\gamma e_{j}$ mit zu bestimmenden $\gamma$.
		\begin{itemize}
      \item[$\implies$] neues $x_{B}
        = \hat{x}_{B} - A_{B}^{-1}A_{N}\gamma e_{j}
        = \hat{x}_{B} - \gamma \underbrace{A_{B}^{-1}A_{\cdot,j}}_{=: w}
        \overset{!}{\geq} 0$
			\item Wähle $\gamma = \min \left\{ \frac{\hat{x}_{i}}{w_{i}} \mid w_{i} > 0 \right\}$.
      \item Wenn $\gamma = + \infty$, so ist \eqref{nf} unbeschränkt.
			\item Andernfalls wähle $i \in B$ mit $\gamma = \frac{\hat{x}_{i}}{w_{i}}$
		\end{itemize}
  \item Dann ist die neue Ecke $x_B^+$
		\begin{align*}
			\hat{x}_{B}-\gamma_{w}&= x_{B}^+\\
			x_{j}^+& = \gamma\\
			x_{k}^+ &= 0 \quad \text{ für alle anderen } k\\
			B^+ &= (B \setminus \{i\})\cup \{j\}\\
			N^+ &= (N \setminus \{j\})\cup \{i\}
			% B& = \{1,3,5\} \\
			% B& = (3,1,5)
		\end{align*}
    Pro Iteration müssen zwei Gleichungssysteme gelöst werden. Das ist sehr teuer. Da sich immer nur $1$ Spalte ändert, kann dies allerdings stark vereinfacht werden durch ein Update der LR-Zerlegung. Das geht, indem in der Implementierung die Elemente von $B$, $B^+$ eine Reihenfolge haben und man ersetzt immer $i$ an seiner Stelle durch $j$.
\end{itemize}
\paragraph{Startbasis}
Gesucht als Startbasis ist eine primal zulässige Basis für
\begin{equation*}
	\min_{x} c^Tx \text{ subject to } Ax = b , x \geq 0
\end{equation*}
Dafür gibt es die Simplex-Phase 1: Löse das Hilfsproblem
\begin{align*}
	\min_{x,s} \sum_{i=1}^{m} s_{i} \text{ subject to }
	\begin{pmatrix}
		A& I
	\end{pmatrix}
	\begin{pmatrix}
		x \\ s
	\end{pmatrix}
	, (x,s)& \geq 0
\end{align*}
Gilt in einer Lösung ($x^0$), dass $s^* =0$, so ist das $x^*$ zulässig für \eqref{nf} und die Basis des Hilfsproblems \enquote{ist} eine primal zulässige Basis für \eqref{nf}.

Eine primal zulässige Startbasis für das Hilfsproblem ist (trivialerweise) gegeben durch
\begin{equation*}
	x = 0, s = b \geq 0, \text{ Basismatrix } I
\end{equation*}
Tableau-Methode für %TODO: was sollen uns diese Zeilen sagen?
\begin{equation*}
	\min_{x}c^Tx \text{ subject to } Ax \leq b , x \geq 0
\end{equation*}
\section{Duales Simplex-Verfahren}
\begin{align}
  \text{primales LP} && \min_{x}c^{T} x \text{ subject to }Ax = b, x \geq 0 \tag{P} \label{primalLP} \\
  \text{duales LP} && \max_{y}b^{T} y \text{ subject to }A^{T} y \leq c, y \geq 0 \tag{D}\label{dualLP}
\end{align}
Wir nehmen wieder an, dass $A \in \R^{m \times n}$ Rang $m$ hat.

Für eine Basis $B$ ist wie bisher $x_{B}= A^{-1}_{B}b, x_{N}=0$.

\underline{Idee:} Wende Simplex-Verfahren auf \eqref{dualLP} an.
Dafür führe \eqref{dualLP} in die duale Normalform:
\begin{equation}\label{dualnf}
	\max_{y} b^{T} y \text{ subject to }
	\begin{pmatrix}
		A^{T} & I
	\end{pmatrix}
	\begin{pmatrix}
	y\\z
	\end{pmatrix}
	= c, \quad z \geq 0, \quad y \in \R^m \tag{D'}
\end{equation}
Eine Basis von \eqref{dualnf}: $n \{ \klammern[\Big]{ \underbrace{A^T}_m \underbrace I_n} =: D
% \begin{pmatrix}
% 	\underbrace{A^{T}}_{m} \underbrace{I}_{n}
% \end{pmatrix}
\in R^{n \times(m +n)}$
sei $H \subset \{1,\dots ,n+m \}$ mit $|H|=n$ und $D_{\cdot,H}\in \R^{n \times n}$ regulär.
\begin{theorem}
	Entweder \eqref{dualLP} ist unzulässig oder unbeschränkt oder es gibt eine optimale Basislösung $(y^*,z^*)$ so dass die zugehörige basis von der Form ist:
	\begin{equation*}
    H_{\text{opt}}=\underbrace{ \{1,..,m \}}_{\text{ganzes $A$ in Basismatrix}}\cup \, (N+\underbrace{m}_{\text{offset}})
	\end{equation*}
	mit $N \subset \{1,\dots ,n \}, |N|= n-m$ ($N$ zeigt an, welche Spalten von $I$ gebraucht werden)
	$\implies$ Basen für \eqref{dualLP} werden i.A. durch $N$ beschrieben, nicht durch $H$.
\end{theorem}
Die Dualität sagt uns, dass die Indices der Nullvariablen in \eqref{nf} gerade die Elemente von $N$ sind, nämlich die Nicht-Null-Schlupfvariablen.

Welche Basis $H$ bzw. $N$ liefert eine zulässige Basislösung für \eqref{dualLP}?
% irgendwas kompiliert hier nicht
\begin{align*}
  \begin{pmatrix}
    A^{T}  & I
  \end{pmatrix}
  \begin{pmatrix}
    y\\z
  \end{pmatrix}
  &=c, \quad z \geq 0\\
  \iff \begin{pmatrix}
      A_N^T & I_N & 0 \\
      A_B^T & 0 & I \\
    \end{pmatrix}
    \begin{pmatrix}
      y \\ z_N \\ z_B
    \end{pmatrix}
    &= \begin{pmatrix}
    c_N \\ c_B
  \end{pmatrix}
  , \quad z \geq 0 \quad (\text{nach Zeilenumsortierung})\\
  \iff A_{B}^{T} y + z_{B} &= c_{B}, \quad z_{B} \geq 0\\
    A_{N}^{T} y + z_{N} &= c_{N}, \quad z_{N} \geq 0 \\
    \iff y &= A_{B}^{-T}(c_{B}-z_{B})\\
      z_{N}&= c_{N} - A_{N}^{T}y = c_{N}-A_{N}^{T} A_{B}^{-T}(c_{B}-z_{B}) ,\quad z_{B}\geq 0, z_{N} \geq 0
\end{align*}
Wenn $(y,z_{B},z_{N})$ eine Basislösung ist, so sind die Nichtbasisvariablen $z_{B} =0$.
Also
\begin{equation*}
	y = A_{B}^{-T}c_{B}, \quad z_{N}= c_{N}- A_{N}^{T} \underbrace{A_{B}^{-T}c_{B}}_{{=y}}\geq 0
\end{equation*}
\subsection{Duales Simplex-Verfahren}
Initialisierung: Dual zulässige Basis $B$, d.\,h.\ $B \subset \{1,\dots ,n \}$ mit $|B|=m$, $A_{B}$ regulär und $z_{B}=0$, $y=A_{B}^{-T}c_{B}$, $z_{N}= c_{N}-A_{N}^{T}y \geq 0$
\begin{enumerate}%[label = Schritt \arabic)]
	\item Bestimme zugehörige komplementäre duale Variable
		\begin{align*}
			x_{N}&=0 \quad (\to \text{komplementär zu }z_{N})\\
      x_{B}&= A_{B}^{-1} b\quad \left( \implies A \begin{pmatrix}x_{B} \\ x_{N}\end{pmatrix} =b\right)
		\end{align*}
	\item Gilt $x_{B} \geq 0 $: \textbf{Stop}.\\
    $(y,z_{B},z_{N}),(x_{B},x_{N})$ KKT-Punkt $\implies$ Lösung gefunden.\\
		Andernfalls wähle $ i \in B$ mit $x_{i}<0$ ($\to z_{i}>0$ verkleinert dualen Zielfunktionswert)
	\item Löse $A_{B}^{T} w = e_{i}$, $\alpha_{N} = -A_{N}^{T} w$
	\item Schrittweitenbestimmung:\\
    Gilt $\alpha_{N} \leq 0 $: \textbf{Stop}, \eqref{dualLP} ist unbeschränkt ($\to$ \eqref{nf} unzulässig).\\
		Andernfalls berechne $\gamma = \min \left\{ \frac{z_{i}}{\alpha_{i}} \mid j \in N , \alpha_{j} >0 \right\}$ und wähle $j \in N$ mit $\gamma = \frac{z_{j}}{\alpha_{j}}$
	\item Update:
		\begin{align*}
			z_{N}^+ &= z_{N} - \gamma \alpha_{N} \quad (\to z_{N} \geq 0, z_{j}^+ =0)\\
			z_{i}^+ &= \gamma\\
			y^+ &= y -\gamma w\\
			N^+ &= (N \backslash \{ j \}) \cup \{ i \}\\
			B^+ &= (B \backslash \{ i \}) \cup \{ j \}
		\end{align*}
\end{enumerate}

\subsection{(primales) Simplex-Verfahren für LPs mit Schranken}
Betrachte
\begin{equation*}
	\min_{x} c^{T} x \text{ subject to } Ax = b ,\quad l \leq x \leq u
\end{equation*}
mit $l \in \left(\R \cup \{ - \infty \} \right)^n$, $u \in \left(\R \cup \{ +\infty \} \right)^n $.

% Mit bis zu $4n$ Schlupfvariablen kann das System auf Normalform gebracht werden. Allerdings bedeutet das ein Vielfaches an Rechenaufwand. Daher suchen wir andere Wege.
%
\underline{Ziel:} möglichst nahe an die Form
\begin{equation*}
	\min_{x} c^{T} x \text{ subject to } Ax = b ,\quad x \geq 0
\end{equation*}
kommen.\\
\underline{mögliche Modifikationen:}
\begin{itemize}
	\item Schlupfvariablen + Variablensplit $x = x^+ -x^-, x^+,x^- \geq0 $ (Nachteil: erhöht die Anzahl der Variablen und linearen Gleichungen)
	\item Fall: $x_{i} \geq l_{i}$ mit $l_{i}\in \R$ : Substituiere $\hat{x}_{i} := x_{i}- l_{i}\geq 0$
	\item Fall: $x_{i} \leq u_{i}$ mit $u_{i}\in \R$ : Substituiere  $\hat{x}_{i} := u_{i}- x_{i}\geq 0$
	\item Fall: $l_{i} \leq x_{i} \leq u_{i}$ mit $l_{i},u_{i} \in \R$ : Substituiere $0 \leq \hat{x}_{i} := x_{i}- l_{i} \leq u_{i}-l_{i}$
	\item Fall $-\infty \leq x_{i} \leq \infty$ : Freie Variable: entweder Variablensplit $x_{i}= x_{i}^+ - x_{i}^-$ oder $x_{i}$ als freie Variable im Simplex-Verfahren berücksichtigen (wie $y$ im dualen Simplex)
\end{itemize}
Angenommen, es gibt keine freien Variablen $\implies$ oBdA LP der Form
\begin{equation*}
	\min_{x} c^{T} x \text{ subject to } Ax = b, \quad 0 \leq x \leq u
\end{equation*}
mit $u \in [0, + \infty]^n$.\nl
Bisher im primalen Simplex-Verfahren:\\
Basis $B$, Basisvariablen $x_{B}= A_{B}^{-1} b \geq 0$, Nichtbasisvariablen $x_{N}=0$ \\
\underline{Idee:} Teile die Nichtbasisvariablen in
\begin{align*}
	N_{l} &= \left\{ i \in N \ | \ x_{i} = 0  \right\} \implies x_{N_{l}}=0\\
	N_{u} &= \left\{ i \in N \ | \ x_{i} = u_{i}  \right\} \implies x_{N_{u}}=u_{N_{u}}
\end{align*}
Wie im Simplex-Verfahren:
\begin{align*}
	A_{B}x_{B} + A_{N_{l}}x_{N_{l}} + A_{N_{u}} x_{N_{u}} = b \implies x_{B} = A_{B}^{-1} (b - A_{N_{l}} x_{N_{l}} - A_{N_{l}}x_{N_{l}})\\
	c^{T} x = c_{B}^{T} x_{B} + c_{N_{l}}^{T} x_{N_{l}}+c_{N_{u}}^{T} x_{N_{u}} = c_{B}^{T} A_{B}^{-1} b + z_{N_{l}}^{T} x_{N_{l}} + z_{N_{u}}^{T} x_{N_{u}}
\end{align*}
Identifiziere $i \in N_{l}$ mit $z_{i} <0 $ oder $ i\in N_{u}$ mit $z_{i}> 0$.\\
Bei Schrittweiten beachten: $x_{B} = \gamma w \in [0,u_{B}]$ und
\begin{equation*}
	x_{i}^+ = \begin{cases}
	 \gamma \in [0,u_{i}]\\
	 u_{i}-\gamma
\end{cases}
\end{equation*}
Beim Basisupdate gibt es den neuen Fall $B^+ = B$, $i$ wechselt von $N_{l}$ nach $N_{u}$ oder umgekehrt.
\begin{beispiel}[Zuschnittsoptimierung]~\nl
  	\underline{Problemstellung} Eine vorgegebene Anzahl von Objekten soll möglichst materialsparend aus einem Grundstoff zugeschnitten werden.\\
  	\underline{konkretes Beispiel}: Eine Firma produziert Metallstangen mit Länge $100\,$cm. Werden kürzere Stangen bestellt muss die Firma $100\,$cm lange Stangen zerschneiden. Ziel der Firma ist es, die Bestellungen zu erfüllen und dafür möglichst wenige Stangen zerschneiden zu müssen.\\
  	Bestellung:
	\begin{table}[H]
		\centering
		\begin{tabular}{c|c}
			Länge in cm & Anzahl bestellt\\
      		\hline
      		$25$ & $80$ \\
      		$30$ & $75$ \\
      		$35$ & $105$ \\
  		\end{tabular}
	\end{table}
	\underline{Modellierung}
  \begin{enumerate}[label={Variante \arabic*.}]
    \item \label{item:Zuschnittsillymodel} Für jeden Stab $i$ beschreiben $x_{i1},x_{i2},x_{i3} \in \N_{0}$, wieviele Stäbe der Länge $25,30,35\,$cm daraus geschnitten werden sollen.
      \begin{align}
        \implies \forall i~ & 25x_{i1} + 30x_{i2} + 35x_{i3}\leq 100 \notag\\
                            &
        \begin{rcases}
          \sum\limits_{i}^{} x_{i1} \geq 80\\
          \sum\limits_{i}^{} x_{i2} \geq 75\\
          \sum\limits_{i}^{} x_{i3} \geq 105
        \end{rcases} \tag{$\Delta$}\label{model_cutting_1}
      \end{align}
      Zielfunktion: minimiere $K = \text{ Anzahl der Stäbe}$, wobei $i = 1,\dots ,K$.
      Um $K$ zu berechnen, nutze folgende Idee: Für jeden Stab $i$ führe eine Variable $y_{i} \in \left\{0,1 \right\}$ ein, die entscheidet, ob der Stab verwendet wird oder nicht.

      Die maximale Anzahl benötigter Stäbe ist $\leq 80 + 75 +105 = 260$.
      \begin{align*}
        \implies \min_{x,y}\sum_{i=1}^{260} y_{i} &\text{ subject to } x_{i1},x_{i2},x_{i3} \in \N_{0}, y_{i}\in \left\{0,1 \right\} , \eqref{model_cutting_1},\\
                                                  & \forall i = 1,\dots ,260 : 25x_{i1} + 30x_{i2} + 35x_{i3} \leq 100y_{i}
      \end{align*}
      \underline{Problem:} gigantisch viele Variablen und Restriktionen und Redundanz in der Darstellung. Das ist ein schlechtes Zeichen für das Modell, aber kein prinzipielles Problem für das Symplex-Verfahren. (Redundanz bedeutet in diesem Fall u.\,a., dass es sehr viele äquivalente Situationen gibt.)
    \item \label{item:Zuschnittbettermodel} Schnittmuster\\
     	Führe eine Variable $x_{i} \in \N_{0}$ ein für jedes Schnittmuster $i$, nach dem eine Stange zerschnitten werden kann:
      \begin{table}[H]
        \centering
        \begin{tabular}{c|c|c|c}
          Schnittmuster & \# $25\,$cm Stäbe & \# $30\,$cm Stäbe & \# $35\,$cm Stäbe\\
          \hline
          1 & 4 & 0 & 0\\
            & 3 & 0 & 0\\
          2 & 2 & 1 & 0\\
          3 & 2 & 0 & 1\\
          4 & 1 & 2 & 0\\
          5 & 1 & 1 & 1\\
          6 & 1 & 0 & 2\\
          7 & 0 & 3 & 0\\
          8 & 0 & 2 & 1\\
          9 & 0 & 1 & 2\\
        \end{tabular}
		\end{table}

    Optimierungsproblem:
    \begin{align*}
      \min_{x} \sum_{i=1}^{9} x_{i}& \text{ subject to } x \in \Z_{0}, x\geq 0
    \end{align*}
    \begin{align*}
      \text{mindestens } &80 \text{ Stäbe mit $25$\,cm}& 4x_{1} + 2x_{2} + 2x_{3} + x_{4} + x_{5} + x_{6} &\geq 80\\
                         &75\text{ Stäbe mit $30$\,cm}& x_{2} + 2x_{4} + x_{5} + 3x_{7} + 2x_{8} + x_{9} &\geq 75\\
                         &105\text{ Stäbe mit $35$\,cm}& x_{3} + x_{5} + 2x_{6} + x_{8} + 2x_{9} &\geq 105
    	\end{align*}
		In Matrixform $Hx \leq b$.

		Vorteil:
		\begin{itemize}
			\item Anzahl Restriktionen $=$ Anzahl verschiedener Stablängen
      \item Anzahl Variablen $=$ Anzahl (sinnvoller) Schnittmuster (wächst schnell, aber langsamer als \ref{item:Zuschnittsillymodel})
		\end{itemize}

 \end{enumerate}
 \underline{Lösung für \ref{item:Zuschnittbettermodel}}
\begin{enumerate}
	\item Statt des ganzzahligen LPs
		\begin{equation*}
			\min_{x} c^{T} x \text{ subject to } Mx \geq b, x \geq 0 , x \in \Z^{n}.
		\end{equation*}
		Löse das \underline{relaxierte LP}
	\begin{equation*}
		\min_{x} c^{T} x \text{ subject to } Hx \geq b, x \geq 0
	\end{equation*}
	\item Simplex-Verfahren für das relaxierte Problem:
		\begin{itemize}
			\item Wir brauhchen eine primal zulässige Startbasis:\\
			$\to$ Für jede Stablänge $l = 25,30,35$ verwende das \glqq Einheitsschnittmuster\grqq\ bei dem die maximal mögliche Anzahl $\lfloor\frac{100}{l}\rfloor$ von Stäben der Länge $l$ und keine anderen Stablängen geschnitten werden.


			Dann sind die zugehörigen Spalten in der Basismatrix linear unabhängige Vielfache von Einheitsvektoren.

		Die zugehörigen Basisvariablen sind:
		\begin{equation*}
			x_{l} = \frac{b_{l}}{\lfloor\frac{100}{l}\rfloor}
		\end{equation*}

		\item $H$ kann sehr viele Spalten haben.

			\underline{Idee:} Immer nur die Spalten von $H$ bestimmen, die man aktuell braucht.\\
			$\to$ Startbasis und zugehörige Basismatrix $A_{B} := \begin{pmatrix}
				H,&-I
			\end{pmatrix}_{B}
      $ wie oben bestimmen ($-I$ kommt von den Schlupfvariablen, die man braucht um aus $\geq$ ein $=$ zu machen).

			$\to$ zugehörige duale Variable:
			\begin{equation*}
				y = A_{B}^{-T}c_{B}
			\end{equation*}
			$\to $ reduzierte Kosten
			\begin{equation*}
				z_{N} = c_{N} - A_{N}^{T} y
			\end{equation*}
      Hierbei ist $c$ von der Form $c = (\underbrace{1 \, \ldots \, 1}_{x-\text{Teil}} \, \underbrace{0 \, \ldots \, 0}_{s- \text{Teil}})^T$.

      Für $j \in N$, die zu den Schlupfvariablen $s$ gehören, gilt
			\begin{equation*}
				z_{j} = c_{j} - A_{\cdot j}^{T} y = 0 + e_{j}^{T} y = y_{j}
			\end{equation*}
			($y_{j}$ ist bekannt). Für $j \in N$, die zu den Variablen $x$ gehören, gilt
			\begin{equation*}\label{dantzigRegel}
				e_{j} = c_{j} - A_{\cdot j}^{T} y = 1 - \underbrace{H_{\cdot j}^{T}}_{\text{unbekannt}} y \to \min \tag{Dantzig Regel}
			\end{equation*}
      Wir müssen also die \glqq beste\grqq{} Spalte $H_{\cdot j}$ als Lösung von
			\begin{equation*}
				\max_{h} y^{T} h \text{ subject to } 25h_{1} + 30 h_{2} + 35h_{3} \leq 100, h\geq 0 , h \in \Z^3
			\end{equation*}
      bestimmen. Damit erhalten wir ein Rucksackproblem mit Gewichten $=$ Stablängen und Werte $=$ duale Variable $y$. Dieses Rucksackproblem ist ganzzahlig, aber recht klein.
		\item Sei $x_{\alpha} \in \R^n$ eine Lösung des relaxierten Problems.
			\begin{enumerate}[label = \arabic*. Fall:]
				\item $x^* \in \Z^n \implies x^*$ löst auch das ganzzahlige LP.
				\item $x^* \not \in \Z^n$
          \begin{description}
            \item [Branch \& Bound Verfahren]
              $\implies$ zwei neue LPs
              \begin{align*}
                \min_{x} c^{T} &x \text{ subject to } Ax = b , x\geq 0 , x\leq \lfloor x_{i}^*\rfloor\\
                \max_{x} c^{T} &x \text{ subject to } Ax = b , x\geq 0 , x\geq \lceil x_{i}^*\rceil
              \end{align*}
              Für Branch \& Bound Verfahren muss man Techniken entwickeln um nicht den gesamten Baum an erstellten LPs durchsuchen zu müssen.
            \item [Schnittebenen-Verfahren]
              Füge neue Ungleichung
              \begin{equation*}
                a^{T} x \leq \beta
              \end{equation*}
              mit $a^{T} x \leq \beta$ für alle zulässigen Punkte $x \in \Z^n$ und $a^{T} x^* > \beta $ zu den Restriktionen hinzu.
              Im Idealfall werden dabei alle Ecken des zulässigen Polyeders ganzzahlig damit das Simplex-Verfahren eine ganzzahlige Lösung findet.
            \item [(Runden)] Ergibt in der Regel eine gute, aber keine optimale Lösung. Man muss aufpassen, zulässig zu bleiben. Beim Zuschnittsproblem kann man immer aufrunden.
          \end{description}
			\end{enumerate}
		\end{itemize}
\end{enumerate}
\end{beispiel}
\subsection{Polyeder}
\begin{definition}
	\begin{itemize}
    \item Eine Menge $P\subset \R^n$ heißt Polyeder, wenn man sie durch endlich viele lineare Ungleichung beschreiben kann.
      (Die Endlichkeitseinschränkung ist wichtig um Polyeder von beliebigen konvexen Mengen zu unterscheiden, denn jede konvexe Menge lässt sich durch (unendlich) viele lineare Ungleichungen beschreiben.)
		\item \underline{natürliche Form} eines Polyeders:
			\begin{align*}
				P(A,b) &= \left\{ x \in \R^n \ \middle| \ a^{T}_{i} x \leq b_{i} \ \forall i =1,\dots ,n \right\}\\
					   &= \left\{ x \in \R^n \ |\ Ax \leq b \right\} \text{ mit } A \in\R^{m\times n}, b\in \R^m
			\end{align*}
		\item \underline{Normalform} eines Polyeders:
			\begin{equation*}
				P(A,b) = \left\{ x \in \R^n \ | \ Ax = b, x\geq 0 \right\} \text{ mit } A \in\R^{m\times n}, b\in \R^m
			\end{equation*}
    \item \underline{Polytop} $:=$ beschränktes Polyeder ($\iff P$ ist konvexe Hülle von endlich vielen Punkten) (diese Äquivalenz ist ein nicht-triviales Resultat!)
		\item \underline{polyhedrischer kegel} $:=$ Polyeder und Kegel ($\iff P = \left\{x \ \middle| \ Ax \leq 0 \right\}$)
		\item P heißt \underline{rationales Polyeder}$\iff$ Es gibt $A\in\Q^{m \times n}, b \in \Q^m$ mit
			\begin{equation*}
				P = P(A,b) = \left\{ x \in \R^n\ | \ Ax \leq b \right\}
			\end{equation*}
			Achtung! Nicht alle Darstellungen rationaler Polyeder sind rational:
			\begin{align*}
				P &= P(A,b) = \left\{ x \in \R\ | \ \sqrt{2}x \leq 0 \right\} \text{ ist ein rationales Polyeder}\\
				  &= \left\{x \in \R \ | \ x \leq 0  \right\}
			\end{align*}
	\end{itemize}
\end{definition}
\underline{Annahme:} Wir betrachten (meistens) nur rationale Polyeder und gehen davon aus, dass die gegebene Darstellung $A,b$ rational ist. (Andere könnten wir nicht im Computer darstellen.)

\underline{Notation:}
\begin{align*}
  && B_{r}(x^*) &:= \left\{ x \in \R^n \ | \ \|x - x^*\|<r \right\}\\
  && \overline{B_{r}(x^*)} &:= \left\{ x \in \R^n \ | \ \|x - x^*\|\leq r \right\} \\
  % \end{align*}
  % \begin{align*}
  x,y \in \R^n, \alpha \in \R:&& \alpha X &:= \left\{ \alpha \cdot x \ |\ x \in X \right\}\\
                              && X + Y &:=\left\{ x + y  \ |\ x \in X, y \in Y \right\}\\
                              && X - Y &:= X + (-Y)
\end{align*}
Dimension von Polyedern?
\begin{itemize}
	\item Ist $U \subset  \R^n$ ein Unterraum, so ist $\dim(U)=$ maximale Anzahl linear unabhängiger Vektoren in $U$.
	\item Ist $X \subset \R^n$, so ist die \underline{lineare Hülle} lin$(X)$ von $X$ der kleinste (bezüglich Inklusion) Unterraum $U$ von $\R^n$ mit $X \leq U$.
	\item Eine Menge $A \subset  \R^n$ heißt \underline{affiner Unterraum}, wenn gilt $U:=A-a^*$ (mit $a^* \in A$ beliebig) ist ein Unterraum.
		Wir definieren $\dim(A) := \dim(U)$.
  \item Konvention $\dim(\emptyset)=-1$
  \item Ist $X\subset \R^n$, so ist die \underline{affine Hülle} $\aff(X)$ von $X$ ist der kleinste (bzgl. Inklusion) affine Unterraum $A$ mit $X\subset A$.
    Wir definieren $\dim(X):=\dim(A)$ (funktioniert gut für konvexe Mengen)\\
    Seien $x_{1},\dots ,x_{m}\in\R^n$:
	\item \underline{lineare Hülle}: $\lin(x_{1},\dots ,x_{m})= \left\{ \sum\limits_{i=1}^{m} c_{i}x_{i}\ \middle| \ c_{i}\in \R \right\}$
	\item \underline{affine Hülle}: $\aff(x_{1},\dots ,x_{m})= \left\{ \sum\limits_{i=1}^{m} c_{i}x_{i}\ \middle| \ c_{i}\in \R, \sum_{i=1}^{m} c_{i}= 1 \right\}$
	\item \underline{konvexe Hülle}: $\conv(x_{1},\dots ,x_{m})= \left\{ \sum\limits_{i=1}^{m} c_{i}x_{i}\ \middle| \ c_{i}\geq 0, \sum_{i=1}^{m} c_{i}= 1 \right\}$
	\item (konvexe) \underline{konische Hülle}: $\cone(x_{1},\dots ,x_{m})= \left\{ \sum\limits_{i=1}^{m} c_{i}x_{i}\ \middle| \ c_{i}\geq 0 \right\}$
	\item Sei $X \subset \R^n$ nichtleer. Dann ist das \underline{relative Innere} von $X$ definiert als
		\begin{equation*}
			\rint(X):= \left\{ x^* \in X \ |\ \exists r >0 \text{ mit } B_{r}(x^*)\cap\aff(X) \subset X \right\}
		\end{equation*}
		der \underline{relative Rand} von $X$ ist definiert als
		\begin{equation*}
			\rbd(X) := \overline{X}\setminus \rint(X)
		\end{equation*}
	\begin{bemerkung}
		\begin{itemize}\
			\item Es gilt immer $\interior(X)\subset \rint(X) \subset X \subset \overline{X}$
			\item Ist $X$ \underline{konvex}, so ist $\rint(X)\neq \emptyset$ und $\dim X= \dim(\rint(X))$
			\item Ist  $P = P(A,b) \subset \R^n$ ein Polyeder, so gilt $\dim P=n \iff \int P \neq \emptyset \iff \exists x^* \in \R^n: Ax^*\lneq b$
		\end{itemize}
	\end{bemerkung}
\end{itemize}
\subsection{Rand von Polyedern}
\begin{definition}
	Sei $P \subset \R^n$ ein Polyeder.
	\begin{itemize}
		\item Dann heißt $a^{T} x \leq \beta$ mit $a \in \R^n, \beta \in \R$ heißt eine \underline{gültige Ungleichung} für $P$, wenn gilt
			 \begin{equation*}
				\forall x \in P: a^{T} x \leq \beta
			\end{equation*}
	\end{itemize}
\end{definition}
\begin{beispiel}
  $0^{T} x \leq 0$ ist eine gültige Ungleichung für jedes Polyeder $P$.
\end{beispiel}
\begin{itemize}
  \item Ist $a^{T} x \leq \beta$ eine gültige Ungleichung für $P$, so heißt
    \begin{equation*}
      F = P \cap \left\{ x \in \R^n\ |\ a^{T} x = \beta \right\}
    \end{equation*}
    eine \underline{Seitenfläche} (Face) von $P$
  \item Eine Seitenfläche $F$ heißt \underline{echt} (proper), wenn gilt $\emptyset \neq F \neq P$
\end{itemize}
\begin{beispiel}
  $0^{T} x \leq 0 \implies F = P, 0^{T} x \leq 1 \implies F = \emptyset$
\end{beispiel}
\begin{itemize}
  \item Eine echte Seitenfläche $F$ von $P$ heißt \underline{Facette}, wenn sie in keiner anderen \underline{echten} Seitenfläche von  $P$ enthalten ist.
\end{itemize} % TODO: schöne Bilder
\begin{definition}
  Sei $P = P(A,b) \subset \R^n$ ein Polyeder.
  \begin{itemize}
    \item Sei $F$ eine Seitenfläche von $P$. Dann heißt
      \begin{equation*}
        \eq(F):=\left\{ i = 1,\dots ,m\ |\ A_{i,\cdot}x = b_{i}\, \forall x \in F \right\}
      \end{equation*}
      die \underline{Menge der aktiven/bindenden Restriktionen} von $F$.
    \item Die Menge der inaktiven  Restriktionen
      \begin{equation*}
        \text{ineq}:= \left\{i=1,\dots ,m \right\}\backslash \eq(F)
      \end{equation*}
    \item Die Menge
      \begin{equation*}
        \eq(P) = \left\{ i =1,\dots ,m \ |\ A_{i,\cdot}x =b_{i} \ \forall x \in P \right\}
      \end{equation*}
      heißt die \underline{Menge aller impliziten Gleichungen}.
    \item Für $I \subset  \left\{ 1,\dots ,m \right\}$ heißt
      \begin{equation*}
        \fa(I):=\left\{x \in P \ |\ A_{i,\cdot}x = b_{i} \ \forall i \in I \right\}
      \end{equation*}
      die \underline{von $I$ induzierten Seitenfläche}.
      $\fa(I)$ ist eine Seitenfläche mit den gültigen Ungleichungen
      \begin{equation*}
        a^{T} x \leq \beta
      \end{equation*}
      mit $a^{T} = \sum\limits_{i \in I}^{} A_{i,\cdot}$, $\beta = \sum\limits_{i \in I}^{} b_{i}$.

      Zu zeigen $F = \fa(\eq(F))$
  \end{itemize}
\end{definition}
\begin{lemma}
	Sei $P\subset \R^n$ ein Polyeder. Dann ist $x^* \in P$ genau dann ein relativ innerer Punkt, wenn gilt: die einzige Seitenfläche $F$ mit $x^* \in F$ ist $F=P$.
\end{lemma}
\begin{proof}
	$\to$ Übung?
\end{proof}

\begin{satz}
Sei $P = P(A,b) \subset  \R^n$ ein nichtleeres Polyeder mit $A\in \R^{m \times n}, b\in \R^m$. Dann gelten:
\begin{enumerate}[label = (\alph*)]
	\item Es gibt mindestens einen Punkt $x^* \in P$ mit 
		\begin{equation*}
			A_{\eq(P),\cdot}x^* = b_{\eq(P)}, A_{\text{ineq}(P),\cdot}x^* < b_{\text{ineq}(P)}
		\end{equation*}
	\item Jeder solcher Punkte $x^*$ ist ein relativ innerer Punkt von $P$. Insbesondere ist $\rint(P)=\neq \emptyset$. 
	\item $\aff(P) = x^* + \text{kern}(A_{\eq(P),\cdot})$ \\
		$\dim(P) = \dim(\aff(P)) = \dim (\text{kern}(A_{\eq(P),\cdot}))=n - \text{Rang}(A_{\eq(P)},\cdot)$
\end{enumerate}
\end{satz}
\begin{proof}
	\begin{enumerate}[label = (\alph*)]
	\item Falls $\text{ineq}(P) = \emptyset$, so gilt $\eq(P) = \left\{ 1,\dots ,m \right\}$ und $P=\left\{ x \ |\ Ax=b \right\}$ ist ein affiner Raum. Dann hat jeder Punkt $x^* \in P$ die gewünschte Eigenschaft.\\
		Falls  $\text{ineq}(P)\neq\emptyset$, so gibt es für alle $i\in \text{ineq}(P)$ einen Punkt  $x^i \in P$ mit $A_{i,\cdot}x^i <b_{i}$. Definiere $x^* = \frac{1}{|\ineq(P)|}\sum\limits_{i \in I}^{} x^i \in P$. 
		Dann gelten 
		\begin{align*}
			A_{\eq(P),\cdot}x^* = b_{\eq(P)}
		\end{align*}
		und für alle $i \in I$ gilt
		\begin{align*}
			A_{i,\cdot} x^* = \frac{1}{|\ineq(P)|}\left[ \underbrace{A_{i,\cdot}x^i}_{< b_{i}} + \sum\limits_{l \neq i\substack{l \in I}}^{} \underbrace{A_{i,\cdot}x^l}_{\leq b_{i}} \right] < b.
		\end{align*}
	\item Sei $F$ eine beliebige Seitenfläche, induziert von $a^{T} x \leq \beta$, mit $x^*\in F$.
		Wir müssen zeigen, dass $F=P$ folgt.
		Nach Wahl von $x^*$ gibt es ein $\varepsilon > 0$, so dass gilt: für alle $y\in \text{kern}(A_{\eq(P),\cdot})$ mit $ \|y\| \leq \varepsilon$ :
		\begin{align*}
			A_{\eq(P),\cdot}(x^*+y) = b_{\eq(P)}, A_{\ineq(P),\cdot}(x^*+y) \leq b_{\ineq(P)}
		\end{align*}
		$\implies$ $\forall y \in \text{kern}(A_{\eq(P),\cdot})$ mit  $\|y\| < \varepsilon$.
		 \begin{align*}
			 x^*+y \in P &\implies a^{T} (x^* +y) \leq \beta\\
						 &\implies \underbrace{a^{T} x^*}_{=\beta} + a^{T} y \leq \beta\\
						 &\implies a^{T} y \leq 0
		\end{align*}
		$\implies$ $\forall y \in \text{kern}(A_{\eq(P),\cdot})$: $a^{T} y =0$ \\
		Daher gilt:
		\begin{align*}
			P &= \left\{ x \ |\ Ax \leq \beta \right\} \\
			  &= \left\{x \in P \ |\ A_{\eq(P),\cdot}x= b_{\eq(P)} \right\}\\
			  &= \left\{x^* + y \in P \ | \  \in \text{kern}(A_{\eq(P),\cdot}) \right\}\\
			  &\subset \left\{ x \in P \ |\ a^{T} x = \beta  \right\} = F 
		\end{align*}
		$\implies P =F\implies$ $x^*$ liegt im relativ Inneren.
	\item Wie in (b) gezeigt gilt:
		\begin{align*}
			\left\{ x^* + y \ | \ y \in \text{kern}(A_{\eq(P),\cdot}), \|y\|\leq \varepsilon \right\} 
			&\subset P\\
			&\subset  \left\{ x^* +y \ |\ y \in \text{kern}(A_{\eq(P),\cdot}) \right\}
		\end{align*}
	$\implies$
	\begin{align*}
		\underbrace{\aff \left\{ x^* + y \ | \ y \in \text{kern}(A_{\eq(P),\cdot}), \|y\|\leq \varepsilon  \right\}}_{= x^* + \text{kern}(A_{\eq(P),\cdot})}
		&\subset\aff P\\
		&\subset  \underbrace{\left\{ x^* +y \ |\ y \in \text{kern}(A_{\eq(P),\cdot}) \right\}}_{= x^* + \text{kern}(A_{\eq(P),\cdot}) }	
	\end{align*}
	$\implies \aff P = x^* + \text{kern}A_{\eq(P),\cdot}$.
	\end{enumerate}
\end{proof}
%TODO add (a) tags
\begin{satz}
	Sei $P=P(A,b) \subset  \R^n$ ein nichtleeres Polyeder mit $A \in \R^{m \times n},b \in R^m$. 
	Sei $F \neq \emptyset$ eine Seitenfläche. Dann gelten:
	\begin{enumerate}
		\item $\dim F = \dim \left( \text{kern}A_{\eq(F),\cdot} = n - \text{Rang}A_{\eq(F),\cdot} \right)$ 
		\item $F = \fa(\eq(F)) = \left\{ x \in \R^n \ | \ A_{\eq(F),\cdot} x = b_{\eq(F)}, A_{\ineq(F),\cdot} x = b_{\ineq(F)} \right\}$ 
		\item $\aff F = \left\{ x \in \R^n \ |\ A_{\eq(F),\cdot} x = b_{\eq(F)} \right\}$
	\end{enumerate}
\end{satz}
\begin{proof}
	Sei $a^{T} x \leq \beta$ eine gültige Ungleichung, die $F$ induziert, d.h.
	\begin{align*}
		F &= \left\{ x \in P \ |\ a^{T} x= \beta \right\}\\
		  &= \left\{x \in \R^n \ |\ A_{\eq(F),\cdot} x = b_{\eq(F)}, a^{T} x = \beta, A_{\ineq(F),\cdot} x \leq b_{\ineq(F)} \right\}
	\end{align*}
	$\implies \dim F ) \dim(\text{kern}(A_{\eq(F),\cdot},a^{T} )^{T} )$.
	Wir müssen zeigen, dass gilt  $\text{kern}(A_{\eq(F),\cdot}) \subset  \text{kern}(A_{\eq(F),\cdot},a^{T} )^{T}$.
	Nach dem vorigen Satz gibt es einen relativ inneren Punkt $x^*$ von $F$ mit $A_{\eq(F),\cdot}x^* = b_{\eq(F)}, a^{T} x^* =\beta, A_{\ineq(F),\cdot}x^* < b_{\ineq(F)}$.
	Angenommen, es gäbe $y^* \subset \text{kern} A_{\eq(F),\cdot}$ mit $a^{T} y^* \neq 0$. Da $\text{kern}(A_{\eq(F),\cdot})$ ein Unterraum ist, können wir oBdA annehmen, dass $a^{T} y^* > 0 $ gilt.
	Dann gilt für alle $\varepsilon > 0$ klein $x^* + \varepsilon y^* \in P$ aber 
	\begin{align*}
		a^{T} (x^* + \varepsilon y^*) = \underbrace{a^{T} x^*}_{= \beta} + \underbrace{\varepsilon}_{>0} \underbrace{a^{T} y^*}_{>0} > \beta \lightning
	\end{align*}
	$a^{T} x \leq \beta$ gültige Ungleichung \\
	$\implies \text{kern}(A_{\eq(F),\cdot}) = \text{kern}( A_{\eq(F),\cdot}, a^{T} )^{T} $\\
	$\implies \dim F = \dim (\text{kern}A_{\eq(F),\cdot}  ) = n - \text{Rang}A_{\eq(F),\cdot} $ 
\begin{equation*}
	F = \left\{ x \in \R^n \ |\ A_{\eq(F),\cdot} x = b_{\eq(F)}, A_{\ineq(F),\cdot} x \leq b_{\ineq(F)} \right\} = \fa(\eq(F))
\end{equation*}
$\implies \aff F \subset  \left\{ x \in R^n \ |\ A_{\eq(F),\cdot}x = b_{\eq(F)} \right\}$ und $\dim F = \dim ( \aff F) = \dim (\text{kern}A_{\eq(F),\cdot})$.\\
$\implies \aff F = \left\{ x \in R^n \ |\ A_{\eq(F),\cdot}x = b_{\eq(F)} \right\}$
\end{proof}

