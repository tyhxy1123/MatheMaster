% This work is licensed under the Creative Commons
% Attribution-NonCommercial-ShareAlike 4.0 International License. To view a copy
% of this license, visit http://creativecommons.org/licenses/by-nc-sa/4.0/ or
% send a letter to Creative Commons, PO Box 1866, Mountain View, CA 94042, USA.

Email: \url{schwartz@gsc.tu-darmstadt.de}\\
Übung: Dr. Martinovic\\
Unterlagen: \url{www.github.com/alexandrabschwartz/DiskreteOptimierung2019}

\section{Einführung}
\subsection{Beispiele}
\begin{beispiel}[Rucksackproblem]\enter
	Gegeben seien $m$ Objekte $j=1,\dots,m$, jeweils mit Gewicht $w_{j}>0$ und Wert $c_{j}>0$. Das maximale tragbare Gewicht sei $w_{\max}$.

	Ziel: maximiere den Wert der ausgewählten Gegenstände
\end{beispiel}

\begin{definition}[allgemeines Optimierungsproblem] \label{def:allgemeines_optimierungsproblem}
  \begin{equation*}
    \min_{x} f(x) \text{ subject to (s.t.) } x \in X
  \end{equation*}
  Dabei ist
  \begin{itemize}
    \item $x$ die Variable $x$,
    \item $f$ die Zielfunktion und
    \item $X$ die zulässige Menge
  \end{itemize}
\end{definition}
%s.t. = subject to\\
Das Rucksackproblem modellieren wir so:
\begin{align*}
\text{Variablen}:\quad & x_{j} \in \{0,1\}\, \forall j =1, \dots, m\\
				 & x_{j}=1 \iff \text{Objekt $j$ wird ausgewählt}\\
\text{Zielfunktion}:\quad& f(x)= \sum_{j=1}^{m} c_j x_{j}\rightarrow \max\\
\text{zulässige Menge}:\quad & \sum_{rj}^{m} w_j x_j \leq w_{\max}\quad \text{maximal zulässiges Gewicht}
\end{align*}
% $\implies$
\begin{equation*}
	\implies \quad \max_{x}c^Tx \text{ subject to } w^Tx \leq w_{\max},\ x \in \{0,1\}^m, x \text{ binär}
\end{equation*}
Verallgemeinerung:
\begin{itemize}
	\item mehrdimensionales Rucksackproblem mit zusätzlichen (linearen) Restriktionen, z.B.\ Volumen
  \item allgemeines Rucksackproblem mit $x\in D\subset \R^m$ z.B. $D=\{0,1\}^m$, $D=\N^m$, $D = \R_{+}^m$
\end{itemize}

\begin{beispiel}[Facility Location Problem]\
  \begin{itemize}[nosep]
    \item Eine Versandfirma hat eine Menge von Kunden $k_{1},\dots,k_{m}$.
	  \item Es gibt eine Menge von möglichen Standorten $S_{1},\dots, S_{n}$ für mögliche neue Lager.
    \item Jeder Standort $S_j$ hat einmalige Baukosten $c_{j}$ für ein neues Lager.
    \item Die Lieferkosten betragen $d_{ij}$ von Standort $S_{j}$ zu Kunde $k_{i}$.
    \item An dieser Stelle wollen wir die Kapazitätsgrenzen von Lagern (noch) ignorieren.
	  \item Frage: An welchen Standorten $S_{j}$ sollen Lager gebaut werden und welche Kunden $k_{i}$ sollen von welchem Standort $S_{j}$ beliefert werden?
  \end{itemize}
  %TODO Karte mit Kunden und Lagerstandorten zeichnen
	%TODO wtf ging hier ab?
	Die Variablen sind:
	\begin{align*}
    x_{ij}&\in \{0,1\} \text{ ob von $S_j$ an $k_j$ geliefert wird}\\
    \text{\emph{oder} } x_{ij} & \in [0,b_j] \text{ wieviel wird von $S_{j}$ an $k_{i}$ geliefert}\\
		y_{j} &\in \{0,1\}\text{ mit }  \\
		y_{j}& = 1 \iff \text{ Lager am Standort $S_{j}$ wird gebaut}
	\end{align*}

	Zielfunktion:
	\begin{equation*}
		\sum_{i=1}^{m} \sum_{j=1}^{n}  d_{ij}x_{ij} + \sum_{j=1}^{n} c_{j}y_{j} \rightarrow \min
	\end{equation*}
	zulässige Menge:
	\begin{align*}
		&\sum_{j=1}^{n} x_{ij} =1 \quad \forall i=1,\dots, m\quad \text{ Kunde k erhält gesamte Bestellung}\\
		&\left(\sum_{j=1}^{n} x_{ij} \leq c_{j} \quad \forall i=1,\dots, m \quad \text{Lagerkapazität}\right)\\
		&x_{ij} \leq b_{i}y_{j} \quad \forall i,j \text{ nur gebaute Lager $S_{j}$ verwenden}
	\end{align*}
	Im Fall $x_{ij} \in [0,b_{i}]$ sprechen wir von einem gemischt ganzzahligen Problem.
\end{beispiel}

\begin{beispiel}[Travelling Salesperson Problem, TSP]\enter
	Es sind Städte $S_{1},\dots, S_{n}$ bekannt.

	Gesucht: Rundreise, die alle Städte genau einmal besucht.

  Die möglichen Streckenabschnitte sind gegeben durch
	\begin{equation*}
		E = \left\{ (i,j) \mid \text{es gibt einen direkten Weg von $S_{i}$ nach $S_{j}$} \right\}
	\end{equation*}
	Die Kosten/ Fahrzeiten sind gegeben durch $c_{ij}$ für alle $(i,j) \in E$.

	Die Variablen: $x_{ij} \in \{0,1\}$ mit $x_{ij}=1$, wenn $S_{i}$ nach $S_j$ gefahren wird\\
	Zielfunktion:
	\begin{equation*}
		\sum_{(i,j)\in E}  c_{ij}x_{ij} \rightarrow \min
	\end{equation*}
	Restiktionen:
	\begin{align*}
		&\sum_{j:(i,j)\in E} x_{ij}=1 \quad \forall i = 1,\dots,n\quad \text{Stadt $S_{i}$ wird genau einmal verlassen}\\
		&\sum_{i:(i,j)\in E} x_{ij}=1 \quad \forall j = 1,\dots,n \quad\text{Stadt $S_{j}$ wird genau einmal besucht}\\
	\end{align*}
	Subtouren ausschließen:
	% wie war der befehl für untereinander in Sum?
	\begin{align*}
		\sum_{i\in Um j \not\in U (i,j) \in E} x_{ij} \geq 1 \quad \forall U \subset \{1,\dots,n\} \text{ mit }2\leq |U|\leq n-2
	\end{align*}
\end{beispiel}

\begin{beispiel}[lineares Zuordnungsproblem]\enter
	Tutor-innen $T_{1},\dots,T_{n}$ \\
	Übungsgruppen $G_{1},\dots, G_{n}$ \\
	kompetenzen $c_{ij}$ von Tutor $T_{i}$ für Gruppe $j$\\
	Ziel: Zuordnung von Tutor-innen zu Übungsgruppen mit hoher Gesamtkompetenz\\
	Variablen: $x_{ij}\in \{0,1\}$ mit $x_{ij}=1 \iff$ Tutor-innen $T_{i}$ betreut Gruppe $G_{i}$ \\
	Zielfunktion:
	\begin{equation*}
		\sum_{i,j=1}^{n}  c_{ij}x_{ij} \rightarrow \max
	\end{equation*}
	Restiktionen:
	\begin{align*}
		\sum_{j=1}^{n} x_{ij} &= 1\quad \forall i\ (\text{jeder nur eine Gruppe})\\
		\sum_{i=1}^{n} x_{ij} &= 1\quad \forall j\ (\text{jede Gruppe wird betreut})
	\end{align*}
\end{beispiel}

\begin{beispiel}[quadratisches Zuordnungsproblem]\enter
	Beamte $B_{1},\dots,B_{n}$ mit \\
	Gesprächshäufigkeiten $c_{ij}$ zwischen $B_{i},B_{j}$ \\
	Büroräume $R_{1}, \dots, R_{n}$ mit \\
	Abständen $d_{kl}$ zwischen $R_{k}$ und $R_{l}$ \\
	Ziel: Raumzuteilung, die die Gesamtwegstrecke minimiert\\
	Variable:
	\begin{equation*}
		x_{ik} \in \{0,1\}\qquad \forall i,=1,\dots,n
	\end{equation*}
	mit $x_{ik}$ genau dann, wenn $B_{i}$ Raum $R_{k}$ bekommt\\
	Restriktionen:
	\begin{align*}
		\sum_{k=1}^{n} x_{ik} &=1 \ \forall i\\
		\sum_{i=1}^{n} x_{ik} &=1 \ \forall k\\
	\end{align*}
	Zielfunktion:
	\begin{equation*}
		\sum_{i,j=1}^{n} c_{ij}\sum_{k,l=1}^{n} d_{kl}x_{ik}x_{jl}
	\end{equation*}
	quadratische Zielfunktion \\
	alternative Variante: Permutation $\pi \in S_{n}$
\end{beispiel}

\subsection{Grundbegriffe}
\begin{itemize}
	\item allgemeines Optimierungsproblem:
		\begin{equation*}
			\min_{x} f(x) \text{ subject to } x\in X
		\end{equation*}
		mit Variable $x \in \R^n$,\\
		Zielfunktion $f:\R^n \to \R$,\\
		zulässige Menge $X \subset \R^n (X = \R^n$ unrestringierte Optimierung)
	\item globales Minimum:
		\begin{equation*}
			x^* \in X \mit f(x^*)\leq f(x)\quad \forall x \in X
		\end{equation*}
		$\to$ gut zu lösen für konvexe Probleme, sonst schwer
	\item lokales Minimum:
		\begin{equation*}
			x^{'} \in X \mit f(x^{'}) \leq f(x)\quad \forall x \in X \cap B_{\varepsilon}(x^*)
		\end{equation*}
		mit $\varepsilon > 0$\\
		bei diskreten Problemen: zulässige Punkte liegen isoliert, sind also alles lokale Minima\\
		$\to$  wir wollen globale Minima
\end{itemize}
