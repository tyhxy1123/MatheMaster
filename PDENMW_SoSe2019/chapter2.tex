\chapter{Refinements of triangulations/conformity}

\begin{definition}
	Let $\T_{0}$ be a conforming triangulation. Then 
	\begin{equation*}
		\mathbb{F} = \mathbb{F}[\T_{0}]:= \bigcup_{T_{0} \in \T_{0}} \mathbb{F}(T_{0})
	\end{equation*}
	is called \underline{masterforest} of $\T_{0}$.

	A subset $\mathcal{F} \subset \mathbb{F}$ is called \underline{forest}
  	\begin{enumerate}[label= \arabic*)]
    	\item $\T_{0} \subset \mathcal{F}$ 
    	\item All nodes $\mathcal{F} \setminus \T_{0}$ have a father.
    	\item All nodes have two children or no children at all.
  	\end{enumerate}
	Every finite forest $\mathbb{F}(\T_{0})$ corresponds to a triangulation of $\Omega$.\\
  	These triangulations are not necessarily conforming! 
\end{definition}
\begin{example}
	\underline{2D: easy!} \\
	Two steps of uniform refinement lead to a conforming grid
	\begin{figure}[H]
	\center
	\begin{tikzpicture}[scale=1.5]
	\def \xone{0};
	\def \yone{0};
	\def \h{3};

	% 2d triangle
	\coordinate (A) at (\xone ,\yone);
	\coordinate (B) at ($ (A) + (\h,0) $);
	\coordinate (C) at ($ (A) + (2*\h/3,2*\h/3) $);
	\coordinate (D) at ($ (A) + (3*\h/2,2*\h/3) $);
	\coordinate (Mab) at ( $0.5*(A) +0.5*(B) $);
	\coordinate (Mbc) at ( $0.5*(B) +0.5*(C) $);
	\coordinate (Mac) at ( $0.5*(A) +0.5*(C) $);
	\coordinate (Mcd) at ( $0.5*(C) +0.5*(D) $);
	\coordinate (Mbd) at ( $0.5*(B) +0.5*(D) $);


	\draw (A) -- (B) -- (D) -- (C) --cycle;
	\draw (B) -- (C);
	\draw[red!40] (Mab) -- (C);
	\draw[red!40] (Mbc) -- (D);

	\draw[green!40] (Mab) -- (Mac);
	\draw[green!40] (Mab) -- (Mbc);
	\draw[green!40] (Mbc) -- (Mbd);
	\draw[green!40] (Mbc) -- (Mcd);
	
	\end{tikzpicture}
	
	\caption{2D example}
	\label{ch_2_uniform_refinement_conforming_grid_2D}
	
\end{figure}

	\underline{3D:}
	\begin{itemize}
		\item Let $T_{1},T_{2}$ be neighboring tetrahedra with common side $S$. 
		\item Let $E_{1},E_{2}$ be the refinement edges of $T_{1}$ and $T_{2}$ respectively.
    \item If $E_{1}$ and $E_{2}$ are on $S$ but $E_{1}\neq E_{2}$ $\to$ big trouble, nonconformity, that cannot be healed: the new edges intersect each other at a ratio of $\frac{1}{3}$ to $\frac{2}{3}$!
	\end{itemize}
	\begin{figure}[H]
	\center
	\begin{tikzpicture}[scale=1.5]
	\def \xone{0};
	\def \yone{0};
	\def \h{3};

	% 3d triangle
	\coordinate (A) at (\xone ,\yone);
	\coordinate (B) at ($ (A) + (\h,0) $);
	\coordinate (C) at ($ (A) + (1.75*\h,\h/3) $);
	\coordinate (D) at ($ (A) + (\h,\h/3) $);
	\coordinate (E) at ($ (A) + (3*\h/5,2*\h/3) $);

	\draw (A) -- (B) -- (C) -- (E) -- cycle;
	\draw (B) -- (E);
	\draw[dotted] (A) -- (D) -- (C);
	\draw[dotted] (E) -- (D) -- (B);
	\draw[green!40] (B) -- ( $0.5*(D) + 0.5*(E) $ );
	\draw[red!40] (D) -- ( $0.5*(B) + 0.5*(E) $ );
	
	\node at ( $ 0.5*(A) + 0.5*(B) $ ) [above left] {$T_{1}$};
	\node at ( $ 0.5*(C) + 0.5*(B) $ ) [above left] {$T_{2}$};

	\end{tikzpicture}
	
	\caption{3D example}
	\label{ch_2_uniform_refinement_conforming_grid_3D}
	
\end{figure}

\end{example}
So from now on we assume:\\
\underline{Condition:} If the two refinement edges of two neighboring simplices are on the common face, then they must be equal. This condition is necessary!\nl
For $d=3$ the condition is sufficient. (A proven, but not trivial result.)\\
For $d>3$ it is not known if the condition is sufficient.

So we look for an alternative condition that implies that every uniform refinement is conforming.
\begin{definition}
	The reflected element to $T= \left\{ z_{0},\dots ,z_{d} \right\}_{t}$ is
	\begin{equation*}
		T_{R}= \left\{ z_{d}, z_{1},\dots ,z_{t},z_{d-1},\dots ,z_{t+1},z_{0} \right\}_{t}
	\end{equation*}
	$T \mapsto T_{R}$ is the unique permutation of the vertices that does not change the children.
\end{definition}
\begin{definition}
	Two neighboring elements $T$ and $T'$ are called \underline{reflected neighbors} if and only if the ordered vertices of $T$ or $T_{R}$ match those of $T'$ at all but one places.
\end{definition}
\underline{Condition:} (alternative)\\
A conforming triangulation $\T_{0}$ is called \underline{admissible}, if
\begin{enumerate}[label = \arabic*)]
	\item all elements have the same type,
	\item for all neighboring elements 
		\begin{equation*}
			T = \left\{ z_{0},\dots ,z_{d} \right\} \text{ and } T' = \left\{ z_{0}',\dots ,z_{d}' \right\}
		\end{equation*}
		with common face $S$ we have 
		\begin{itemize}
			\item if  $\overline{z_{0}z_{d}}\subset  S$ or $\overline{z_{0}'z_{d}'}\subset S$ then $T$ and $T'$ are reflected neighbors
			\item otherwise (i.e. if $\overline{z_{0}z_{d}}\not\subset  S$ and $\overline{z_{0}'z_{d}'}\not\subset S$) the two neighboring children of $T$ and $T'$ are reflected neighbors
		\end{itemize}
	\begin{figure}[H]
	\center
	\begin{tikzpicture}[scale=1]
	\def \xone{0};
	\def \yone{0};
	\def \h{3};

	% first example
	\coordinate (A) at (\xone ,\yone);
	\coordinate (B) at ($ (A) + (2*\h/3,\h) $);
	\coordinate (C) at ($ (A) + (0,2*\h) $);
	\coordinate (D) at ($ (A) + (-2*\h/3,\h) $);
	\coordinate (M) at ($ 0.5*(A) + 0.5*(C) $);


	\draw (A) -- (B) -- (C) -- (D) --cycle;
	\draw (A) -- (C);
	\draw[red!40] (B) -- (D);
	\draw[green!40] ( $0.5*(A) + 0.5*(B) $ ) -- ( $0.5*(C) + 0.5*(D)$ );
	\draw[green!40] ( $0.5*(C) + 0.5*(B) $ ) -- ( $0.5*(A) + 0.5*(D)$ );

	% right triangle
	\node at (A) [below right] {0};
	\node at (B) [right] {1};
	\node at (C) [above right] {2};

	% left triangle
	\node at (A) [below left] {0};
	\node at (D) [left] {1};
	\node at (C) [above left] {2};

	\node[red!60] at (M) [below left] {1};
	\node[red!60] at (M) [below right] {1};
	\node[red!60] at (M) [above left] {1};
	\node[red!60] at (M) [above right] {1};

	% right triangle
	\node[red!60] at (A) [right] {0};
	\node[red!60] at (B) [above] {2};
	\node[red!60] at (B) [below] {2};
	\node[red!60] at (C) [right] {0};

	% left triangle
	\node[red!60] at (A) [left] {0};
	\node[red!60] at (D) [above] {2};
	\node[red!60] at (D) [below] {2};
	\node[red!60] at (C) [left] {0};

	%-- second example --
	\coordinate (A) at ($ (A) + (2*\h,0) $);
	\coordinate (B) at ($ (A) + (2*\h/3,\h) $);
	\coordinate (C) at ($ (A) + (0,2*\h) $);
	\coordinate (D) at ($ (A) + (-2*\h/3,\h) $);
	\coordinate (M) at ($ 0.5*(A) + 0.5*(C) $);


	\draw (A) -- (B) -- (C) -- (D) --cycle;
	\draw (A) -- (C);
	\draw[red!40] (A) -- ( $0.5*(C) + 0.5*(D)$ );
	\draw[red!40] (C) -- ( $0.5*(A) + 0.5*(B)$ );
	\draw[green!40] ( $0.5*(A) + 0.5*(D) $ ) -- ( $0.5*(C) + 0.5*(D)$ );
	\draw[green!40] ( $0.5*(C) + 0.5*(D) $ ) -- (M);
	\draw[green!40] ( $0.5*(A) + 0.5*(B) $ ) -- (M);
	\draw[green!40] ( $0.5*(C) + 0.5*(B) $ ) -- ( $0.5*(A) + 0.5*(B)$ );

	% right triangle
	\node at (A) [below right] {0};
	\node at (B) [below right] {2};
	\node at (C) [above right] {1};

	% left triangle
	\node at (A) [below left] {1};
	\node at (D) [below left] {2};
	\node at (C) [above left] {0};

	% right triangle
	\node[red!60] at ( $0.5*(A) + 0.5*(B) $ ) [right] {1};
	\node[red!60] at (B) [above right] {0};
	\node[red!60] at (C) [right] {2};

	% middle right triangle
	\node[red!60] at ( $0.5*(A) + 0.5*(B) $ ) [below] {1};
	\node[red!60] at (A) [right] {0};
	\node[red!60] at ( $0.3*(A) + 0.7*(C) $ ) [right] {2};

	% middle left triangle
	\node[red!60] at ( $0.5*(C) + 0.5*(D) $ ) [above] {1};
	\node[red!60] at (C) [left] {0};
	\node[red!60] at ( $0.7*(A) + 0.3*(C) $ ) [left] {2};

	% left triangle
	\node[red!60] at ( $0.5*(C) + 0.5*(D) $ ) [left] {1};
	\node[red!60] at (A) [left] {2};
	\node[red!60] at (D) [above left] {0};

	\end{tikzpicture}
	
	\caption{2D example for the numbering rule}
	\label{ch_2_example_numbering_alg}
	
\end{figure}

\end{enumerate}
\begin{theorem}
	Under this condition all uniform refinements are conforming.
\end{theorem}
\begin{proof}[Small part of proof]
	First part:
	\begin{lemma}
		Let $T = \left\{ z_{0},\dots ,z_{d} \right\}_{0}$ be a simplex. All uniform refinements of $T$ are conforming.
	\end{lemma}
  \begin{proof}[Proof sketch]
    Consider the $g$-th uniform refinement (with $g < d$).
		\begin{enumerate}
      \item One has to show (but we do not show it here): every descendent $T'$ of $T$ with $\gen(T')=g$ has the structure
				\begin{equation*}
					T' = \left\{ z_{k_{0}},y_{g},y_{g-1},\dots ,y_{1},\underbrace{z_{k_{1}},\dots, z_{k_{d-g}} }_{\text{from original simplex}} \right\}_{g}
				\end{equation*}
				\begin{itemize}
					\item $z_{k_{0}}, \dots ,z_{k_{d-g}}$ are vertices of $T$. The indices $k_{0},\dots k_{d-g}$ are consecutive (ascending or descending)
					\item $y_{1},\dots ,y_{g}$ are vertices introduced as edge midpoints
				\end{itemize}
			\item Define a non-standard distance on the vertices of $T$:
				\begin{equation*}
					d(z_{i},z_{j}) = |i-j|
				\end{equation*}
				Distances to a new vertex $y_{k}$ are defined as $0$.
			\item The \glqq longest\grqq\ edge in $T'$ is $\overline{z_{k_{0}}z_{k_{d-g}}}$ because the indices are consecutive.
				Furthermore $\overline{z_{k_{0}}z_{k_{d-g}}}$ is the refinement edge!
		\end{enumerate}
    $\implies$ the refinement edge of any $T'$ is always the \glqq longest\grqq{} edge!
		\begin{itemize}
			\item Let $U,V$ be descendents of $T$ with $\gen(U)=\gen(V)< d$ with common facet 
				\begin{equation*}
					H = \left\{ h_{1},\dots ,h_{d} \right\}	
				\end{equation*}
				 \begin{figure}[H]
	\center
	\begin{tikzpicture}[scale=1.5]
	\def \xone{0};
	\def \yone{0};
	\def \h{3};

	% 3d triangle
	\coordinate (A) at (\xone ,\yone);
	\coordinate (B) at ($ (A) + (\h,0) $);
	\coordinate (C) at ($ (A) + (1.75*\h,\h/3) $);
	\coordinate (D) at ($ (A) + (\h,\h/3) $);
	\coordinate (E) at ($ (A) + (3*\h/5,2*\h/3) $);

	\filldraw[green!40] (B) -- (D) -- (E) -- cycle;

	\draw (A) -- (B) -- (C) -- (E) -- cycle;
	\draw (B) -- (E);
	\draw[dotted] (A) -- (D) -- (C);
	\draw[dotted] (E) -- (D) -- (B);
	
	\draw[green] (B) -- (D) -- (E) -- cycle;
	\end{tikzpicture}
	
	\caption{common facet $H$}
	\label{ch_2_common_facet}
	
\end{figure}

			\item Both $U$ and $V$ get bisected along their longest edge.
				\begin{enumerate}[label = Case \arabic*]
					\item both longest edges are not in $H$ $\to$ no problem.
					\item one longest edge is in $H$, the other is not:
						This cannot happen, because the longest edges are unique and have the same length $|d- \gen(U)|=|d-\gen(V)|$.
					\item both edges are in $H$ $\implies$ they must be the same. \qedhere
				\end{enumerate}
		\end{itemize} 
  \end{proof}
  The second part of considering a forest is omitted.
\end{proof}

\section{Refinement-Algorithm}
\underline{Goal:} Algorithm that construct a (small) conforming refinement of a given triangulation $\T$ in which all elements of a subset $\mathcal{M} \subset \T$ are refined.
\begin{enumerate}[label = \alph*)]
	\item Iterative Refinement
    \begin{algorithmic}[0]
      \Function{REFINE\_MARKED}{$\mathcal{T}$, $M$}
        \For{all $T \in M$}
        \State $\{T_0, T_1\} = \text{BISECT} (T)$
          \State $\mathcal{T} := (\mathcal{T} \setminus \{T\}) \cup \{T_0, T_1\}$
        \EndFor
        \State \Return $T$
      \EndFunction
    \end{algorithmic}
    This (usually) leads to a nonconforming grid.

		Remedy: Refine elements with hanging nodes.
    \begin{algorithmic}[0]
      \Function{REFINE}{$\mathcal{T}$, $M$}
        \While{$M \neq \emptyset$}
          \State $\mathcal{T} := \text{REFINE\_MARKED}(\mathcal{T}, M)$
          \State $M := \{T \in \mathcal{T} \ |\ T \text{ has a hanging node}\}$
        \EndWhile
        \State \Return $\mathcal{T}$
      \EndFunction
    \end{algorithmic}
    This seemingly simple algorithm is supposedly difficult to implement.
		%TODO chenag comments to Pseudocodes
		\begin{lemma}
			The algorithm terminates.
    \end{lemma}
    \begin{proof}\
			\begin{itemize}
				\item Let $\T_{*}$ be the triangulation after the first call to REFINE\_MARKED.
				\item Let $g$ be the highest generation of an element in $\T_{*}$. 
				\item The uniform refinement $\T_{g}$ is conforming.
				\item $\T_{*}\leq \T_{g}$ (i.\,e.\ $\T_{g}$ is a refinement of $T_{*}$).
        \item REFINE bisects elements with hanging nodes $\implies \T \leq \T_{g}$ for all $\T$ produced by the \glqq while\grqq{} loop. (Somehow here the condition for admissible grids comes in.)
				\item Either the loop terminates or $\T$ is bigger than its predecessor.
				\item $T_{g}$ is a upper bound $\to$ the algorithm terminates.
			\end{itemize}
		\end{proof}
		\begin{lemma}
			The grid produced by REFINE
			is the \underline{smallest} conforming refinement of $\T_{*}$.
		\end{lemma}
		\subsection{Complexity of bisection refinement}
		Consider the sequence of conforming triangulations 
		\begin{equation*}
			\T_{0} \leq \T_{1} \leq \dots \leq \T_{n} \leq \dots 
		\end{equation*}
		produced by the following algorithm.
    \begin{algorithmic}
      \For{$k \geq 0$}
    \State select a subset $\mathcal{M}_k \subseteq \mathcal{T}_k$
        \State $\mathcal{T}_{k + 1} := \text{REFINE}(\mathcal{T}_k, M_k)$
      \EndFor
    \end{algorithmic}
		Hope: There is a constant $\Lambda = \Lambda(\T_{0})$ such that 
		\begin{equation*}
      \#\T_{k+1} - \#\T_{k} \leq \Lambda \# \mathcal{M}_{k} \quad \text{for all } k \geq 0
		\end{equation*}
		\underline{This does not hold!}\\
    \begin{example}[Counterexample] \label{ex:counterexampleRefinementBoundary}
      The refinement edge is always the edge on the boundary $\implies$ grid is admissible.
      \begin{figure}[H]
	\center
	\begin{tikzpicture}[scale=1.5]
	\def \xone{0};
	\def \yone{0};
	\def \h{3};

	% first example
	\coordinate (A) at (\xone ,\yone);
	\coordinate (B) at ($ (A) + (\h,0) $);
	\coordinate (C) at ($ (A) + (\h,\h) $);
	\coordinate (D) at ($ (A) + (0,\h) $);
	\coordinate (M) at ($ 0.5*(A) + 0.5*(C) $);


	\draw (A) -- (B) -- (C) -- (D) --cycle;
	\draw (A) -- (C);
	\draw (B) -- (D);

	\node at (A) [below left] {$0$};
	\end{tikzpicture}
	
	\caption{$\T_{0}$}
	\label{ch_2_counterexample_0}
	
\end{figure}

      \begin{itemize}
        \item Pick an even number $K \in \N$ 
        \item Set $\mathcal{M}_{k} := \left\{ T \in \T_{k}: 0 \in T \right\} $ for all $k=0,1,\dots ,K-1$ 
          \begin{figure}[H]
	\center
	\begin{tikzpicture}[scale=1.5]
	\def \xone{0};
	\def \yone{0};
	\def \h{3};

	% \T_{1} and \T_{2} 
	\coordinate (A) at (\xone ,\yone);
	\coordinate (B) at ($ (A) + (\h,0) $);
	\coordinate (C) at ($ (A) + (\h,\h) $);
	\coordinate (D) at ($ (A) + (0,\h) $);
	\coordinate (M) at ($ 0.5*(A) + 0.5*(C) $);


	\draw (A) -- (B) -- (C) -- (D) --cycle;
	\draw (A) -- (C);
	\draw (B) -- (D);
	\draw[green] ( $0.5*(A) + 0.5*(B)$ ) -- (M);
	\draw[green] ( $0.5*(A) + 0.5*(D)$ ) -- (M);
	\draw[red] ( $0.5*(A) + 0.5*(D)$ ) -- ( $0.5*(A) + 0.5*(B)$ );

	\node[green] at ( $0.5*(A) + 0.5*(B) + (0,-0.1*\h)$ ) [below left] {$\T_{1}$};
	\node[red] at ( $0.5*(A) + 0.5*(B) + (0,-0.1*\h)$ ) [below right] {$\T_{2}$};

	% Arrow
	\draw[->] ( $0.5*(B) + 0.5*(C) + (0.1*\h,0)$ ) -- ( $0.5*(B) + 0.5*(C) + (0.4*\h,0)$ );

	% \T_{3} and \T_{4} 
	\coordinate (A) at ( $(A) + (1.5*\h,0)$ );
	\coordinate (B) at ($ (A) + (\h,0) $);
	\coordinate (C) at ($ (A) + (\h,\h) $);
	\coordinate (D) at ($ (A) + (0,\h) $);
	\coordinate (M) at ($ 0.5*(A) + 0.5*(C) $);


	\draw (A) -- (B) -- (C) -- (D) --cycle;
	\draw (A) -- (C);
	\draw (B) -- (D);
	\draw ( $0.5*(A) + 0.5*(B)$ ) -- (M);
	\draw ( $0.5*(A) + 0.5*(D)$ ) -- (M);
	\draw ( $0.5*(A) + 0.5*(D)$ ) -- ( $0.5*(A) + 0.5*(B)$ );

	\draw[green] ( $0.75*(A) + 0.25*(D)$ ) -- ( $0.75*(A) + 0.25*(C)$ );
	\draw[green] ( $0.75*(A) + 0.25*(B)$ ) -- ( $0.75*(A) + 0.25*(C)$ );
	\draw[red] ( $0.75*(A) + 0.25*(B)$ ) -- ( $0.75*(A) + 0.25*(D)$ );

	\node[green] at ( $0.75*(A) + 0.25*(B) + (0,-0.1*\h)$ ) [below left] {$\T_{3}$};
	\node[red] at ( $0.75*(A) + 0.25*(B) + (0,-0.1*\h)$ ) [below right] {$\T_{4}$};


	\end{tikzpicture}
	
	\caption{$\T_{1},\dots ,\T_{4}$}
	\label{ch_2_counterexample_1-4}
	
\end{figure}

        \item Set $\mathcal{M}_{K} := \left\{ T \in \T_{K}: \text{gen}(T)=K \text{ and } 0 \not\in T \right\}$ 
          \begin{figure}[H]
	\center
	\begin{tikzpicture}[scale=1.5]
	\def \xone{0};
	\def \yone{0};
	\def \h{3};

	\coordinate (A) at (\xone ,\yone);
	%  M_{K}
	\coordinate (A) at ( $(A) + (1.5*\h,0)$ );
	\coordinate (B) at ($ (A) + (\h,0) $);
	\coordinate (C) at ($ (A) + (\h,\h) $);
	\coordinate (D) at ($ (A) + (0,\h) $);
	\coordinate (M) at ($ 0.5*(A) + 0.5*(C) $);


	\filldraw[red!40] ( $0.75*(A) + 0.25*(B)$ ) -- ( $0.75*(A) + 0.25*(D)$ ) -- ( $0.75*(A) + 0.25*(C)$ ) -- cycle;

	\draw (A) -- (B) -- (C) -- (D) --cycle;
	\draw (A) -- (C);
	\draw (B) -- (D);
	\draw ( $0.5*(A) + 0.5*(B)$ ) -- (M);
	\draw ( $0.5*(A) + 0.5*(D)$ ) -- (M);
	\draw ( $0.5*(A) + 0.5*(D)$ ) -- ( $0.5*(A) + 0.5*(B)$ );

	\draw ( $0.75*(A) + 0.25*(D)$ ) -- ( $0.75*(A) + 0.25*(C)$ );
	\draw ( $0.75*(A) + 0.25*(B)$ ) -- ( $0.75*(A) + 0.25*(C)$ );
	\draw ( $0.75*(A) + 0.25*(B)$ ) -- ( $0.75*(A) + 0.25*(D)$ );

	\end{tikzpicture}
	
	\caption{marked elements for refinement $\mathcal{M}_{K}$}
	\label{ch_2_counterexample_M_K}
	
\end{figure}

      \end{itemize}
      For $k < K$ only the marked elements are refined $\implies$ $\#\T_{k+1} - \#\T_{k} = \mathcal{M}_{k} = 2$	
      \begin{figure}[H]
	\center
	\begin{tikzpicture}[scale=1.25]
	\def \xone{0};
	\def \yone{0};
	\def \h{3};

	% first refinement for M_K 
	\coordinate (A) at (\xone ,\yone);
	\coordinate (B) at ($ (A) + (\h,0) $);
	\coordinate (C) at ($ (A) + (\h,\h) $);
	\coordinate (D) at ($ (A) + (0,\h) $);
	\coordinate (M) at ($ 0.5*(A) + 0.5*(C) $);


	\draw (A) -- (B) -- (C) -- (D) --cycle;
	\draw (A) -- (C);
	\draw (B) -- (D);

	\draw ( $0.5*(A) + 0.5*(B)$ ) -- (M);
	\draw ( $0.5*(A) + 0.5*(D)$ ) -- (M);
	\draw ( $0.5*(A) + 0.5*(D)$ ) -- ( $0.5*(A) + 0.5*(B)$ );

	\draw ( $0.75*(A) + 0.25*(D)$ ) -- ( $0.75*(A) + 0.25*(C)$ );
	\draw ( $0.75*(A) + 0.25*(B)$ ) -- ( $0.75*(A) + 0.25*(C)$ );
	\draw ( $0.75*(A) + 0.25*(B)$ ) -- ( $0.75*(A) + 0.25*(D)$ );

	\draw[green] ( $0.875*(A) + 0.125*(C)$ ) -- ( $0.875*(A) + 0.125*(D) + 0.875*(A) + 0.125*(C)$ );
	\draw[green] ( $0.875*(A) + 0.125*(C)$ ) -- ( $0.875*(A) + 0.125*(B) + 0.875*(A) + 0.125*(C)$ );

	%lower red triangle
	\draw[red] ( $0.625*(A) + 0.375*(B) + 0.875*(A) + 0.125*(D)$ ) -- ( $0.75*(A) + 0.25*(B) + 0.875*(A) + 0.125*(D)$ );
	\draw[red] ( $0.625*(A) + 0.375*(B) + 0.875*(A) + 0.125*(D)$ ) -- ( $0.625*(A) + 0.375*(B)$ );
	\draw[red] ( $0.625*(A) + 0.375*(B) + 0.875*(A) + 0.125*(D)$ ) -- ( $0.75*(A) + 0.25*(B)$ );

	%upper red triangle
	\draw[red] ( $0.625*(A) + 0.375*(D) + 0.875*(A) + 0.125*(B)$ ) -- ( $0.75*(A) + 0.25*(D) + 0.875*(A) + 0.125*(B)$ );
	\draw[red] ( $0.625*(A) + 0.375*(D) + 0.875*(A) + 0.125*(B)$ ) -- ( $0.625*(A) + 0.375*(D)$ );
	\draw[red] ( $0.625*(A) + 0.375*(D) + 0.875*(A) + 0.125*(B)$ ) -- ( $0.75*(A) + 0.25*(D)$ );

	\node[green] at ( $0.75*(A) + 0.25*(B) + (0,-0.1*\h)$ ) [below left] {$+2$};
	\node[red] at ( $0.75*(A) + 0.25*(B) + (0,-0.1*\h)$ ) [below right] {$+4$};

	% Somehow this arrow format did not work properly and i decided to just make two files to avoid trouble
	
	\end{tikzpicture}
	
	\caption{first two refinements }
	\label{ch_2_counterexample_k=K_1}
	
\end{figure}

      \begin{figure}[H]
	\center
	\begin{tikzpicture}[scale=1.25]
	\def \xone{0};
	\def \yone{0};
	\def \h{3};

	% first refinement for M_K 
	\coordinate (A) at (\xone ,\yone);
	\coordinate (B) at ($ (A) + (\h,0) $);
	\coordinate (C) at ($ (A) + (\h,\h) $);
	\coordinate (D) at ($ (A) + (0,\h) $);
	\coordinate (M) at ($ 0.5*(A) + 0.5*(C) $);


	\draw (A) -- (B) -- (C) -- (D) --cycle;
	\draw (A) -- (C);
	\draw (B) -- (D);

	\draw ( $0.5*(A) + 0.5*(B)$ ) -- (M);
	\draw ( $0.5*(A) + 0.5*(D)$ ) -- (M);
	\draw ( $0.5*(A) + 0.5*(D)$ ) -- ( $0.5*(A) + 0.5*(B)$ );

	\draw ( $0.75*(A) + 0.25*(D)$ ) -- ( $0.75*(A) + 0.25*(C)$ );
	\draw ( $0.75*(A) + 0.25*(B)$ ) -- ( $0.75*(A) + 0.25*(C)$ );
	\draw ( $0.75*(A) + 0.25*(B)$ ) -- ( $0.75*(A) + 0.25*(D)$ );

	\draw ( $0.875*(A) + 0.125*(C)$ ) -- ( $0.875*(A) + 0.125*(D) + 0.875*(A) + 0.125*(C)$ );
	\draw ( $0.875*(A) + 0.125*(C)$ ) -- ( $0.875*(A) + 0.125*(B) + 0.875*(A) + 0.125*(C)$ );

	% previously lower red triangle
	\draw ( $0.625*(A) + 0.375*(B) + 0.875*(A) + 0.125*(D)$ ) -- ( $0.75*(A) + 0.25*(B) + 0.875*(A) + 0.125*(D)$ );
	\draw ( $0.625*(A) + 0.375*(B) + 0.875*(A) + 0.125*(D)$ ) -- ( $0.625*(A) + 0.375*(B)$ );
	\draw ( $0.625*(A) + 0.375*(B) + 0.875*(A) + 0.125*(D)$ ) -- ( $0.75*(A) + 0.25*(B)$ );

	% previously upper red triangle
	\draw ( $0.625*(A) + 0.375*(D) + 0.875*(A) + 0.125*(B)$ ) -- ( $0.75*(A) + 0.25*(D) + 0.875*(A) + 0.125*(B)$ );
	\draw ( $0.625*(A) + 0.375*(D) + 0.875*(A) + 0.125*(B)$ ) -- ( $0.625*(A) + 0.375*(D)$ );
	\draw ( $0.625*(A) + 0.375*(D) + 0.875*(A) + 0.125*(B)$ ) -- ( $0.75*(A) + 0.25*(D)$ );


	\draw[green] ( $0.625*(A) + 0.375*(D) + 0.875*(A) + 0.125*(B)$ ) -- ( $0.5*(A) + 0.5*(D) + 0.75*(A) + 0.25*(B)$ );
	\draw[green] ( $0.75*(A) + 0.25*(C)$ ) -- ( $0.5*(A) + 0.5*(D) + 0.75*(A) + 0.25*(B)$ );

	\draw[green] ( $0.625*(A) + 0.375*(B) + 0.875*(A) + 0.125*(D)$ ) -- ( $0.5*(A) + 0.5*(B) + 0.75*(A) + 0.25*(D)$ );
	\draw[green] ( $0.75*(A) + 0.25*(C)$ ) -- ( $0.5*(A) + 0.5*(B) + 0.75*(A) + 0.25*(D)$ );
	
	\node[green] at ( $0.75*(A) + 0.25*(B) + (0,-0.1*\h)$ ) [below right] {$+4$};

	% Somehow this arrow format did not work properly and i decided to just make two files to avoid trouble
	
	\end{tikzpicture}
	
	\caption{next refinement iteration }
	\label{ch_2_counterexample_k=K_2}
	
\end{figure}

      For $k=K$ $\implies$ $\#\T_{k+1} - \#\T_{k} = 4K +2$\\
      But $K$ can be any large even number $\implies$ Number $4K +2$ cannot be bounded by $\#\mathcal{M}=2$.
    \end{example} % end example CounterexampleRefinementBoundary:
		You can bound the average of new elements created by refinements.
    In our example this is:
		\begin{align*}
			\# \T_{K+1} - \# \T_{0} &= \sum\limits_{k=0}^{K} \# \T_{k+1} - \# \T_{k}\\
									&= 4K + 2 + 2K\\
									&\leq 3( 2K +2 )\\
									&= 3 \sum\limits_{k=0}^{K} \#\mathcal{M}_{K} 
		\end{align*}
		\begin{theorem}
			Let $\T_{0}$ be a conforming, admissible triangulation. Then there is a constant $\Lambda = \Lambda(\T_{0})$ such that 
			\begin{equation*}
				\# \T_{K} - \# \T_{0} \leq \Lambda \sum\limits_{k=0}^{K} \#\mathcal{M}_{k}
			\end{equation*}
		\end{theorem}
    \begin{proof}[Idea of the proof]
			Let $\mathcal{M} = \bigcup_{k=0} \mathcal{M}_{k}$ be the set of all elements marked by constructing the sequence
			 \begin{equation*}
				\T_{0} \leq \T_{1} \leq \T_{2}\leq \dots .
			\end{equation*}
			Define a cost function
			\begin{equation*}
				\lambda: \T \times \mathcal{M} \mapsto \R
			\end{equation*}
			such that
			\begin{enumerate}[label = \alph*)]
				\item 
					\begin{equation*}
						\sum\limits_{T \in \T\setminus \T_{0}}^{} \lambda(T,T_{*}) \leq 1 \quad \forall \, T_{*} \subset \mathcal{M}
					\end{equation*}
				\item 
					\begin{equation*}
						\sum\limits_{T_{*}\in \mathcal{M}}^{} \lambda(T,T_{*}) \geq C_{2} \quad \forall \, T \in \T \setminus \T_{0}
					\end{equation*}
			\end{enumerate}
      Finding such a cost function is highly non-trivial. (Done in Chapter 4.5.) % TODO: good reference
      But as soon as we have it:
			\begin{align*}
				C_{2} \left( \# \T_{K} - \# \T_{0} \right) &= C_{2}\# (\T\setminus \T_{0}) \\
														   &= \sum\limits_{T \in \T \setminus \T_{0} }^{} \sum\limits_{T_{*} \in \mathcal{M}}^{} \lambda(T,T_{*})\\
														   &= \sum\limits_{T_{*} \in \mathcal{M}}^{} \sum\limits_{T \in \T \setminus \T_{0} }^{} \lambda(T,T_{*})\\
														   &\overset{(a)}{\leq} C_{1} \# \mathcal{M}
			\end{align*}
		\end{proof}
\end{enumerate}

