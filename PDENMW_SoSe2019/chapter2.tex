\chapter{Refinements of triangulations/conformity}

\begin{definition}
	Let $\T_{0}$ be a conforming triangulation. Then 
	\begin{equation*}
		\mathbb{F} = \mathbb{F}[\T_{0}]:= \bigcup_{T_{0} \in \T_{0}} \mathbb{F}(T_{0})
	\end{equation*}
	is called \underline{masterforest} of $\T_{0}$.\nl
	A subset $\mathcal{F} \subset \mathbb{F}$ is called \underline{forest}
  	\begin{enumerate}[label= \arabic*)]
    	\item $\T_{0} \subset \mathcal{F}$ 
    	\item All nodes $\mathcal{F} \backslash \T_{0}$ have a father
    	\item All nodes have two children or no children at all
  	\end{enumerate}
	Every finite forest $\mathbb{F}(\T_{0})$ correspondes to a triangulation of $\Omega$.\\
  	These triangulations are not necessarily conforming! 
\end{definition}
\begin{example}
	\underline{2D: easy!} 
	two steps of uniform refinement lead to a conforming grid
	%TODO add pictures
	\underline{3D:}\\
	\begin{itemize}
		\item Let $T_{1},T_{2}$ be neighboring tetrahedra with common side $S$ 
		%TODO add picture
		\item Let $E_{1},E_{2}$ be the refinement edges of $T_{1}$ and $T_{2}$ respectively
		\item If $E_{1}$ and $E_{2}$ are on $S$ but $E_{1}\neq E_{2}$ $\to$ big trouble, nonconformity, that cannot be healed
	\end{itemize}
\end{example}
From now on:\\
\underline{Condition:} If the two refinement edges of two neighboring simplices are on the common face, then they must be equal. This condition is necessary!\nl
For $d=3$ the condition is sufficient.\\
For $d>3$ not known\\
Alternative condition( implying that every uniform refinement is conforming)
\begin{definition}
	The reflected element to $T= \left\{ z_{0},\dots ,z_{d} \right\}_{t}$
	\begin{equation*}
		T_{R}= \left\{ z_{d}, z_{1},\dots ,z_{t},z_{d-1},\dots ,z_{t+1},z_{0} \right\}_{t}
	\end{equation*}
	$T \mapsto T_{R}$ is the unique permutation of the vertices that does not change the children.
\end{definition}
\begin{definition}
	Two neighboring elements $T$ and $T'$ are called \underline{reflected neighbors} iff the ordered vertices of $T$ or $T_{R}$ match those of $T'$ at all but one places.
\end{definition}
\underline{Condition:}(alternative)\\
A conforming triangulation $\T_{0}$ is called \underline{admissible}, if
\begin{enumerate}[label = \arabic*)]
	\item all elements have the same type
	\item for all neighboring elements 
		\begin{equation*}
			T = \left\{ z_{0},\dots ,z_{d} \right\} \text{ and } T' = \left\{ z_{0}',\dots ,z_{d}' \right\}
		\end{equation*}
		with common face $S$ we have 
		\begin{itemize}
			\item if  $\overline{z_{0}z_{d}}\subset  S$ or $\overline{z_{0}'z_{d}'}\subset S$ then $T$ and $T'$ are reflected neighbors
			\item otherwise (i.e. if $\overline{z_{0}z_{d}}\not\subset  S$ and $\overline{z_{0}'z_{d}'}\not\subset S$) the two neighboring children of $T$ and $T'$ are reflected neighbors
		\end{itemize}
		%TODO add pictures
\end{enumerate}
\begin{theorem}
	Under this condition all uniform refinements are conforming.
\end{theorem}
\begin{proof}
	(small part)\\
	First part:
	\begin{lemma}
		Let $T = \left\{ z_{0},\dots ,z_{d} \right\}_{0}$ a simplex. All uniform refinements of $T$ are conforming.
	\end{lemma}
		sketch of the proof:\\
		consider the $g$-th uniform refinement $g < d$ 
		\begin{enumerate}
			\item One has to show: every descendent $T'$ of $T$ with gen$(T')=g$ has the structure
				\begin{equation*}
					T' = \left\{ z_{k_{0}},y_{g},y_{g-1},\dots ,y_{1},\underbrace{z_{k_{1}},\dots, z_{k_{d-g}} }_{\text{from original simplex}} \right\}_{g}
				\end{equation*}
				\begin{itemize}
					\item $z_{k_{0}}, \dots ,z_{k_{d-g}}$ are vertices of $T$. The indices $k_{0},\dots k_{d-g}$ are consecutive (ascending or descending)
					\item $y_{1},\dots ,y_{g}$ are vertices introduced as edge midpoints
				\end{itemize}
			\item Define non-standard distance on the vertices of $T$:
				\begin{equation*}
					d(z_{i},z_{j}) = |i-j|
				\end{equation*}
				Distances to a new vertex $y_{k}$ are defined as $0$.
			\item The \glqq longest\grqq\ edge in $T'$ is $\overline{z_{k_{0}}z_{k_{d-g}}}$ because the indices are consecutive.
				Furthermore $\overline{z_{k_{0}}z_{k_{d-g}}}$ is the refinement edge!
		\end{enumerate}
		$\implies$ the refinement edge of any $T'$ is always the \glqq longest \grqq\ edge!
		\begin{itemize}
			\item Let $U,V$ be descendents of $T$ with gen$(U)=$gen$(V)< d$ with common facet 
				\begin{equation*}
					H = \left\{ h_{1},\dots ,h_{d} \right\}	
				\end{equation*}
				%TODO add picture
			\item Both $U$ and $V$ get bisected along their longest edge
				\begin{enumerate}[label = Case \arabic*]
					\item both longest edges are not in $H$ $\to$ no problem
					\item one longest edge is in $H$, the other is not\\
						this cannot happen, because the longest edges are unique and have the same lenght $|d- \text{gen}(U)|=|d-\text{gen}(V)|$
					\item both edges are in $H$ $\implies$ they must be the same
				\end{enumerate}
		\end{itemize}
\end{proof}

\section{Refinement-Algorithm}
\underline{Goal:} Algorithm that construct a (small) conforming refinement of a given triangulation $\T$ in which all elements of a subset $\mathcal{M} \subset \T$ is refined.
\begin{enumerate}[label = \alph*)]
	\item Iterative Refinement
		%TODO insert pseudocode here
		This may lead to a nonconforming grids\\
		Remedy: Refine elements with hanging nodes
		%TODO insert pseudocode here
		%TODO chenag comments to Pseudocodes
		\begin{lemma}
			The algorithm terminates.
			\begin{itemize}
				\item Let $\T_{*}$ be the triangulation after the first call to %REFINE_MARKED
				\item Let $g$ be the highest generation of an element in $\T_{*}$ 
				\item The uniform refinement $\T_{g}$ is conforming
				\item $\T_{*}\leq \T_{g}$ (i.e. $\T_{g}$ is a refinement of $T_{*}$)
				\item %REFINE
					bisects elements with hanging nodes $\implies \T \leq \T_{g}$ for all $\T$ produced by the \glqq while \grqq\ loop
				\item Either the loop terminates or $\T$ is bigger than its predecessor
				\item $T_{g}$ is a upper bound $\to$ the algorithm terminates
			\end{itemize}
		\end{lemma}
		\begin{lemma}
			The grid produced by %REFINE
			is the \underline{smallest} conforming refinement of $\T_{*}$
		\end{lemma}
		%TODO add pictures
		\subsection{Complexity of bisection refinement}
		Consider the sequence of conforming triangulations 
		\begin{equation*}
			\T_{0} \leq \T_{1} \leq \dots \leq \T_{n} \leq \dots 
		\end{equation*}
		produced by the following algorithm.
		%PSEUDOCODE
		%for k \geq 0 do 
		%	select a subset \mathcal{M}_{k} \subset \T_{k}
		%	\T_{k+1} := REFINE( \T_{k},\mathcal{M}_{k})
		%end
		Hope: There is a constant $\Lambda = \Lambda(\T_{0})$ such that 
		\begin{equation*}
			\#\T_{k+1} - \#\T_{k} \leq \Lambda \# \mathcal{M}_{k}
		\end{equation*}
		\underline{Doesn't hold!}\\
		Counterexample:\\
		The refinement edge is always the edge on the boundary $\implies$ grid is admissible
		\begin{itemize}
			\item Pick an even number $K \in \N$ 
			\item Set $\mathcal{M}_{k} := \left\{ T \in \T_{k}: 0 \in T \right\} $ for all $k=0,1,\dots ,K-1$ 
				%TODO add pictures
			\item Set $\mathcal{M}_{K} := \left\{ T \in \T_{K}: \text{gen}(T)=K \text{ and } 0 \not\in T \right\}$ 
				%TODO add pictures
		\end{itemize}
		For $k < K$ only the marked elements are refined $\implies$ $\#\T_{k+1} - \#\T_{k} = \mathcal{M}_{k} = 2$	
		%TODO add pictures
		For $k=K$ $\implies$ $\#\T_{k+1} - \#\T_{k} = 4K +2$\\
		But $K$ can be any large even number $\implies$ Number $4K +2$ cannot be bounded by $\#\mathcal{M}=2$.\nl
		You can bound the average of new elements created by refinements.
		\begin{align*}
			\# \T_{K+1} - \# \T_{0} &= \sum\limits_{k=0}^{K} \# \T_{k+1} - \# \T_{k}\\
									&= 4K + 2 + 2K\\
									&\leq 3( 2K +2 )\\
									&= 3 \sum\limits_{k=0}^{K} \#\mathcal{M}_{K} 
		\end{align*}
		\begin{theorem}
			Let $\T_{0}$ be a conforming triangulation, admissible. Then there is a constant $\Lambda = \Lambda(\T_{0})$ such that 
			\begin{equation*}
				\# \T_{K} - \# \T_{0} \leq \Lambda \sum\limits_{k=0}^{K} \#\mathcal{M}_{k}
			\end{equation*}
		\end{theorem}
		\begin{proof}
			(Idea of the proof)\\
			Let $\mathcal{M} = \bigcup_{k=0} \mathcal{M}_{k}$ be the set of all elements marked with constructing the sequence
			 \begin{equation*}
				\T_{0} \leq \T_{1} \leq \T_{2}\leq \dots .
			\end{equation*}
			Costfunction
			\begin{equation*}
				\lambda: \T \times \mathcal{M} \mapsto \R
			\end{equation*}
			such that
			\begin{enumerate}[label = \alph*)]
				\item 
					\begin{equation*}
						\sum\limits_{T \in \T\backslash \T_{0}}^{} \lambda(T,T_{*}) \leq 1 \quad \forall T_{*} \subset \mathcal{M}
					\end{equation*}
				\item 
					\begin{equation*}
						\sum\limits_{T_{*}\in \mathcal{M}}^{} \lambda(T,T_{*}) \geq C_{2} \quad \forall T \in \T \backslash \T_{0}
					\end{equation*}
			\end{enumerate}
			with that:
			\begin{align*}
				C_{2} \left( \# \T_{K} - \# \T_{0} \right) &= C_{2}\# (\T\backslash \T_{0}) \\
														   &= \sum\limits_{T \in \T \backslash \T_{0} }^{} \sum\limits_{T_{*} \in \mathcal{M}}^{} \lambda(T,T_{*})\\
														   &\overset{(a)}{\leq} C_{1} \# \mathcal{M}
			\end{align*}
		\end{proof}
			
		

\end{enumerate}

