% !TEX root = PDENMW.tex
% This work is licensed under the Creative Commons
% Attribution-NonCommercial-ShareAlike 4.0 International License. To view a copy
% of this license, visit http://creativecommons.org/licenses/by-nc-sa/4.0/ or
% send a letter to Creative Commons, PO Box 1866, Mountain View, CA 94042, USA.

\chapter{Geometry of the bisection method}
In the previous example the function $u$ was known. Thus the grid can be constructed specifically for $u$. For finite element problems the function $u $ is not known! There you use an initial grid and modify it repeatedly.\\
possible modifications:
\begin{itemize}
	\item moving vertices (r-refinement)
	\item split elements (h-refinement)
\end{itemize} 
There are various ideas to split elements, but most of theory exists for bisection refinement.
\section{Bisection refinement}
\begin{figure}[h!]
	\center
	\begin{tikzpicture}[scale=1]
	\def \xone{0};
	\def \yone{0};
	\def \h{3};
	
	% first triangle
	\coordinate (A) at (\xone,\yone);
	\coordinate (B) at ($ (A) + (\h,0) $);
	\coordinate (C) at ($ (A) + (\h/3,2*\h/3) $);
	\coordinate (M) at ($ (A) + (\h/2,0) $);

		
	\draw (A) -- (B) -- (C) --cycle;
	\filldraw (A) circle (1pt)
			  (B) circle (1pt)
			  (C) circle (1pt)
			  (M) circle (1pt);
	
	\draw[-to] (\h,\h/3) -- ++(\h/2,0);
	
	
	% second triangle
	\coordinate (A1) at (\xone +1.7*\h,\yone);
	\coordinate (B1) at ($ (A1) + (\h,0) $);
	\coordinate (C1) at ($ (A1) + (\h/3,2*\h/3) $);
	\coordinate (M1) at ($ (A1) + (\h/2,0) $);


	\draw (A1) -- (B1) -- (C1) --cycle;
	\draw (M1) -- (C1);
	\filldraw (A1) circle (1pt)
			  (B1) circle (1pt)
			  (C1) circle (1pt)
			  (M1) circle (1pt);
	 
	\end{tikzpicture}
	
	\caption{bisection refinement}\label{tikz/chapter_mesh/bisection_refine}
	\label{ch_m_bisection_refine}
	
\end{figure}

There are two challenges:
\begin{itemize}
	\item grid conformity
	\begin{figure}[h!]
	\center
	\begin{tikzpicture}[scale=1]
	\def \xone{0};
	\def \yone{0};
	\def \h{3};
	
	% first rectangle
	\coordinate (A) at (\xone,\yone);
	\coordinate (B) at ($ (A) + (\h,0) $);
	\coordinate (C) at ($ (A) + (\h,0.7*\h) $);
	\coordinate (D) at ($ (A) + (0,0.7*\h) $);
	\coordinate (M) at ($ (A) + (\h/2,0.35*\h) $);


	\fill[red!20] (A) --(B) --(D) -- cycle;		
	\draw (A) -- (B) -- (C) --(D) --cycle;
	\draw (B) --(D);
	\filldraw (A) circle (1pt)
			  (B) circle (1pt)
			  (C) circle (1pt)
			  (D) circle (1pt)
			  (M) circle (1pt);
	
	\draw[-to] (11/10*\h,\h/3) -- ++(\h/2,0);
	
	
	% second rectangle
	\coordinate (A1) at (\xone + 2*\h,\yone);
	\coordinate (B1) at ($ (A1) + (\h,0) $);
	\coordinate (C1) at ($ (A1) + (\h,0.7*\h) $);
	\coordinate (D1) at ($ (A1) + (0,0.7*\h) $);
	\coordinate (M1) at ($ (A1) + (\h/2,0.35*\h) $);
	
	
	\fill[green!20] (A1) --(B1) --(D1) -- cycle;	
	\fill[red!20] (C1) --(B1) --(D1) -- cycle;	
	\draw (A1) -- (B1) -- (C1) --(D1) --cycle;
	\draw (B1) --(D1);
	\draw (A1) --(M1);
	\filldraw (A1) circle (1pt)
			  (B1) circle (1pt)
			  (C1) circle (1pt)
			  (D1) circle (1pt)
			  (M1) circle (1pt);
	 
	\end{tikzpicture}
	
	\caption{hanging nodes}\label{tikz/chapter_mesh/hanging_nodes}
	\label{ch_m_hanging_nodes}
	
\end{figure}
	\item elements must remain \glqq well shaped\grqq 
	\begin{figure}[h!]
	\center
	\begin{tikzpicture}[scale=1]
	\def \xone{0};
	\def \yone{0};
	\def \h{3};
	\def \r{2*\h/9 +\h*0.025}; % nicht schön, aber funktioniert
	
	% first triangle
	\coordinate (A) at (\xone,\yone);
	\coordinate (B) at ($ (A) + (\h,0) $);
	\coordinate (C) at ($ (A) + (\h/2,2*\h/3) $);
	\coordinate (M) at ($  (A) + (\h/2,\r)  $);
		
	\draw (A) -- (B) -- (C) --cycle;
	\draw[red] (A) -- (B);
	\draw[blue] (A) ++(\h/2,0) -- (M);
	\draw (M) circle (\r);
	\filldraw (A) circle (0.8pt)
			  (B) circle (0.8pt)
			  (C) circle (0.8pt)
			  (M) circle (0.8pt);
			  
	
	\def \ra{\h/13}; % nicht schön, aber funktioniert
	% second triangle
	\coordinate (D) at ($ (A) +(2*\h,0) $);
	\coordinate (E) at ($ (D) + (2*\h,0) $);
	\coordinate (F) at ($ (D) + (1*\h,\h/6) $);
	\coordinate (M2) at ($  (D) + (\h,\ra)  $);
	
	\draw (D) -- (E) -- (F) --cycle;
	\draw (M2) circle (\ra);
	\filldraw (D) circle (0.8pt)
			  (E) circle (0.8pt)
			  (F) circle (0.8pt)
			  (M2) circle (0.8pt);
	

	\node[below left] at (\h/3,0) {$h_T$};
	\node[below right] at (\h/2,\r/2) {$\rho_T$};
%	\fill[black,font=\footnotesize] (3.2,0)  node[below] {$h$}
%									(2.7,0.5) node[left] {$h$};
	\end{tikzpicture}
	
	\caption{different shapes}\label{tikz/chapter_mesh/shape_regularity}
	\label{ch_m_shape_regularity}
	
\end{figure}
\end{itemize} 
In particular: the constant $c$ in 
\begin{equation*}
	\|u - u_{N}\|_{H^{1}} \leq c\ h\ |u|_{H^{2}}
\end{equation*} 
depends on 
\begin{equation*}
	\delta = \max \delta_{T} 
\end{equation*} 
with
\begin{equation*}
	\delta_{T}= \frac{\text{radius of circum-circle of }T}{\text{radius of incircle of }T}.
\end{equation*} 
$\delta_{T}$ is called \glqq shape coefficient\grqq\ or \glqq aspect ratio\grqq.\nl
Aim: refinement strategies, that keep $\delta_{T}$ bounded.

\section{Partition of a single simplex}
\begin{definition}
	We describe a simplex $T$ as a ordered set of its vertices and a type $t\in \N$.
	\begin{equation*}
		T = \left\{z_{0}, \dots , z_{d}\right\}_{t}, \qquad t\in \left\{ 0,\dots ,d-1 \right\}
	\end{equation*}
	The refinement rule is solely based on this information.
	\begin{enumerate}[label = \arabic*)]
		\item The refinement edge is $\overline{z_{0}z_{d}}$.
		\item The new midpoint $\bar{z}= \frac{\overline{z_{0}z_{d}}}{2}$ becomes a new index.
		\item connect $\bar{z}$ to all vertices of $T$ except $z_{0},z_{d}$ \\
			$\implies$ two new simplices $T_{1},T_{2}$ with common face 
			 \begin{equation*}
				S = \left\{\overline{z},z_{1},\dots ,z_{d-1} \right\}
			\end{equation*}
		\item vertex ordering and type for the children $T_{1},T_{2}$:
			\begin{align*}
				T_{1}&= \left\{ z_{0},\overline{z},\underbrace{z_{1},\dots ,z_{t}}_{\rightarrow},\underbrace{z_{t+1},\dots ,z_{d-1}}_{{\rightarrow}} \right\}_{(t+1)\mod d}\\
				T_{2}&= \left\{ z_{d},\overline{z},\underbrace{z_{1},\dots ,z_{t}}_{\rightarrow},\underbrace{z_{d-1},\dots ,z_{t+1}}_{{\leftarrow}} \right\}_{(t+1)\mod d}\\
			\end{align*}
		$\rightarrow, \leftarrow$ means increasing or decreasing order.	\\
		$\underbrace{\left\{z_{d},\dots ,z_{0}\right\} }_{\rightarrow} =\emptyset$		
	\end{enumerate}
\end{definition}
\begin{example}\
	\begin{itemize}
		\item $d=2$:
			\begin{align*}
				T &= \left\{ z_{0},z_{1},z_{2} \right\}_{0} & \overline{z}&= \frac{1}{2}\left(z_{0} + z_{2} \right)\\
				T_{1} &= \left\{ z_{0},\overline{z},z_{1} \right\}_{1}  & T_{2} &= \left\{ z_{0},\overline{z},z_{1} \right\}_{1}
			\end{align*}
			independent of $t$
			\begin{figure}[H]
	\center
	\begin{tikzpicture}[scale=1]
	\def \xone{0};
	\def \yone{0};
	\def \h{3};

	% 2d triangle
	\coordinate (A) at (\xone ,\yone);
	\coordinate (B) at ($ (A) + (\h,0) $);
	\coordinate (C) at ($ (A) + (\h/3,2*\h/3) $);
	\coordinate (M) at ($ (A) + (\h/2,0) $);


	\filldraw[red!30] (A) -- (M) -- (C) --cycle;
	\filldraw[blue!30] (B) -- (M) -- (C) --cycle;
	\draw (A) -- (B) -- (C) --cycle;
	\filldraw (M) circle (1pt);
	
	\node at (A) [below] {2};
	\node at (B) [below] {0};
	\node at (C) [above] {1};

	\node[red!40] at (A) [left] {0};
	\node at (B) [right] {0};
	\node[red!30] at (C) [above left] {2};
	\node[blue!30] at (C) [above right] {2};
	\node[red!30] at (M) [below left] {1};
	\node[blue!30] at (M) [below right] {1};
	\end{tikzpicture}
	
	\caption{2D example for the numbering rule}
	\label{ch_1_example_numbering_alg}
	
\end{figure}

		\item $d= 3$: \quad $T = \left\{ z_{0},z_{1},z_{2},z_{3} \right\}_{t}$
			\begin{align*}
				t&=0: & T_{1}&= \left\{ z_{0},\overline{z}, z_{1},z_{2} \right\}_{1} & T_{2}&= \left\{ z_{3},\overline{z}, z_{2},z_{1} \right\}_{1}\\
				t&=1: & T_{1}&= \left\{ z_{0},\overline{z}, z_{1},z_{2} \right\}_{2} & T_{2}&= \left\{ z_{3},\overline{z}, z_{1},z_{2} \right\}_{2}\\
				t&=2: & T_{1}&= \left\{ z_{0},\overline{z}, z_{1},z_{2} \right\}_{0} & T_{2}&= \left\{ z_{3},\overline{z}, z_{1},z_{2} \right\}_{0}
			\end{align*}
	\end{itemize}
\end{example}
If you renumber the vertices of $T$ the shape of $T$ does not get modified, but the shapes of the children get modified. You can get up to $\frac{1}{2}\left(d+1 \right)!$ different shapes for the children.\\
Repeated refinement leads to an infinite binary tree $\mathbb{F}(T_{0})$.
\begin{definition}
	The \textit{generation} $g(T)$ of $T \in \mathbb{F}(T_{0})$ is its number of ancestors.
\end{definition}
\underline{Question:} Is the set of shape coefficients bounded?\nl
Let us first look at a particularly regular case:
\begin{definition}\
	\begin{enumerate}[label = \Roman*)]
		\item The Kuhn simplex has the corners
			\begin{align*}
				k_{0}&=0 \in \R^d\\
				k_{1}&=e_{1}, k_{2}=e_{2}\\
				k_{i}&= \sum_{j=1}^{i} e_{j}
			\end{align*}
			%TODO insert missing picture
		\item Let $\pi$ be a permutation of $\left\{1,\dots ,d \right\}$. The Kuhn-simplex $T_{\pi}$ has vertices 
			\begin{align*}
				k_{0}&=0 \\
				k_{i}&= \sum_{j=1}^{d} e_{\pi(j)}
			\end{align*}		
	\end{enumerate}
\end{definition}
\begin{theorem}\
	\begin{itemize}
		\item The $2^g$ descendants in the $g$-th generation of a Kuhn-simplex are all pairwise congruent.
		\item There are only $d$ different simplex shapes in the entire infinite tree $\mathbb{F}(T_{\pi})$ 
		\item All descendants of generation $d$ are similar to $T_{\pi}$ with scaling $\frac{1}{2}$ 
			%TODO insert picture
	\end{itemize}
\end{theorem}
\begin{proof_}
	(proof sketch)\\
	We show: all descendants in the $g$-th generation of a Kuhn-simplex of type $0$ ($0 \leq g < d$) are congruent to 
	\begin{equation*}
		T_{g}= \left\{0, \frac{k_{d-g+1}}{2},\dots ,\frac{k_{d}}{2},k_{1},\dots ,k_{d-g} \right\}_{g}
	\end{equation*}
	\underline{Induction over $g$}:
	\begin{align*}
		g=0&: & T_{0}= \left\{0,k_{1},\dots ,k_{d} \right\}_{0} \text{is the Kuhn simplex}\\
		g\mapsto g+1&: &\text{bisect } T_{g}\quad ( \text{bisection edge } \overline{0k_{d-g}})\\
		 && C_{0}= \left\{ 0, \frac{k_{d-g}}{2},\frac{k_{d-g+1}}{2},\dots , \frac{k_{d}}{2},k_{1},\dots ,k_{d-g-1} \right\}_{g+1}= T_{g+1}\\
		 && C_{1}= \left\{ k_{d-g}, \frac{k_{d-g}}{2},\frac{k_{d-g+1}}{2},\dots , \frac{k_{d}}{2},k_{d-g-1},\dots ,k_{1} \right\}_{g+1}= T_{g+1}
	\end{align*}
	Construct a series of reflection that maps $C_{1}$ onto $T_{g}$ :
	\begin{enumerate}
		\item Reflect at the plane perpendicular to $k_{d-g}$ in the point $\frac{k_{d-g}}{2}$ 
			\begin{itemize}
				\item $k_{d-g} \mapsto 0$ 
				\item $k_{l} \mapsto k_{d-g}-k_{l}$ ($l < d-g$)
				\item $k_{l} \mapsto k_{l}$ ($l \geq d-g$)
			\end{itemize}
		\item Swap $e_{1}\leftrightarrow e_{d-g-1}$, $e_{2} \leftrightarrow_{d-g-2},\dots $
			\begin{itemize}
				\item $0 \mapsto 0$ 
				\item $k_{d-g}-k_{l} \mapsto k_{d-g-l}$ $(l< d-g)$ 
				\item $\frac{k_{l}}{2}\mapsto \frac{k_{l}}{2}$ $(l \geq d-g)$
			\end{itemize}
	\end{enumerate}
	Result:
	\begin{equation*}
		C'_{1}= \left\{ 0, \frac{k_{d-g}}{2},\dots ,\frac{k_{d}}{2},k_{1},\dots ,k_{d-g-1} \right\}_{g+1} = T_{g+1}
	\end{equation*}
	$C_{0}$ and $C_{1}$ are congruent.
	\begin{equation*}
		T_{d}= \left\{ 0, \frac{k_{1}}{2},\dots ,\frac{k_{d}}{2} \right\}_{0} \quad \implies \text{ scaled Khun simplex}
	\end{equation*}
\end{proof_}
\begin{theorem}
	Let $T$ be an arbitrary $d$-simplex. Then there is a construct $C(T_{0})$ such that
	\begin{equation*}
		\delta_{T} \leq C(T_{0})
	\end{equation*}
	for all $T \in \mathbb{F}(T_{0})$.
\end{theorem}
\begin{proof_}\
	\begin{itemize}
		\item Let $T_{0}$ be of type 0.
		\item Let $\hat{T}_{0}$ be the Kuhn simplex of type 0.
		\item There is an affine map $F$ with $F(\hat{T}_{0})=T_{0}$ \\
			$\implies$ For each $T \in \mathbb{F}(T_{0})$ there is a unique $\hat{T} \in \mathbb{F}(\hat{T}_{0})$ such that
			\begin{equation*}
				T = F(\hat{T})
			\end{equation*}
			% insert missing pictures here
	\end{itemize}
	This does not mean that the recurrent refinement will produce no more than $d$ similarity classes, but the total number remains finite.
	 \begin{equation*}
		 \implies \delta_{T} \text{ remains bounded}
	\end{equation*}
	If $T_{0}$ has type $>0$ : Refine until reaching descendants of type $0$.
\end{proof_}



