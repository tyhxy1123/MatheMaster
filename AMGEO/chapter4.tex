\chapter{Theorems of Lie and Cartan}
\section{Jordan-Chevallery decomposition}

\section{Cartan's critertion}
\begin{remark}
    If $[L,L]$ is nilpotent $\implies$ $L$ is solvable.\\
    \underline{Engel's theorem:} $\ad_{[L,L]}x$ nilpotent $\implies$ $[L,L]$ nilpotent.
\end{remark}
\begin{lemma}
    If
    $A \subseteq B \subseteq \gl(V), \dim V < \infty$ and
    $$ M = \{x \in \gl(V) \mid [x,y] \in A \forall y \in B\}.$$
    Then $$\tr (xy) = 0 \forall y \in M \implies x \text{ nilpotent.}$$
\end{lemma}
\begin{proof}
    Let $x=n+s$ be the Jordan decomposition of $x$. Let $v_1, \ldots, v_m$ be an Eigen basis for $s$, $sv_i = a_i v_i$ for $a_i \in F$.
    In this basis $s= \diag(a_1,\ldots a_m)$. Recall we assume $\Q \subseteq F$.
    Let $E := \Span_\Q\{a_1, \ldots, a_m\} \subseteq F$
    We show that $E^{\ast} = \Hom_\Q(E,\Q) = 0$. Then 
    \begin{align*}
        E = 0 \implies \forall i,a_i=0 & \implies s =0\\
        & \implies x \text{ nilpotent}
    \end{align*}
    So let $f: E \to \Q$ be any linear functional. Define $y := \diag(f(a_1), \ldots,f(a_m)) \subseteq \gl(V).$
    Tricky part: Show $y \in M$. For this, let $r \in F[X]$ be a polynomial with $r(0)=0$ and $r(a_i - a_j) = f(a_i) - f(a_j)$
    Note: Possible is $a_i -a_j = a_m - a_n$, but $f$ linear implies that $f(a_i - a_j)=f(a_i)-f(a_j)$ and
    $a_i - a_j =$ implies $f(a_i - a_j) = 0$.
    Recall:  $\ad s$ is diagonalizable with Eigenvalues $a_i - a_j$ with Eigenbasis given by $e_{ij}$.
    Hence working in this basis of $\gl(V)$ (constructed using the basis $v_1, \ldots, v_m$ of $V$) we see 
    \begin{align*}
        r( \ad s) = \ad y
    \end{align*}
    (which is diagonal with entries $f(a_i) - f(a_j)$).
    Consider
    \begin{align*}
        A \subseteq B \text{ and }M = \{x \in \gl(V) \mid [x,y] \in A \forall y \in B\} \implies\\
        \ad x (A) \subseteq B.
    \end{align*}
    Hence $x \in M \implies y \in M$ as $\ad y = r(\ad x)$.
    So 
    \begin{align*}
        \tr(xy) = 0 & = \sum\limits_{i=1}^m a_i f(a_i) \in E
    \end{align*}
    Appplying $f \in E^{\ast}$ yields $\sum\limits_{i=1}^m f(a_i)^2 = 0$, whence
    $$ f(a_1) = \ldots f(a_m) = 0 \implies f= 0.$$
\end{proof}

\begin{lemma}
    \begin{align*}\tag{$\ast$}\label{eq:ast}
        \tr([x,y]z) = \tr(x[y,z]) \forall x,y,z \in \gl(V)   
    \end{align*}
\end{lemma}
\begin{proof}
    $$\tr([x,y]z) = \tr(x[y,z]) \iff \tr(xyz -yxz) = \tr (xyz -xzy)$$
\end{proof}

\begin{theorem}[Cartan's critertion]
    If $$\tr (xy) = 0 \forall x \in [L,L], y \in L \subseteq \gl(V), \dim V < \infty$$
    then $L$ is solvable.
\end{theorem}
\begin{proof}
    Recall: It suffices to show that all $\ad_{[L,L]} x$ are nilpotent.
    Hence it also suffices to show that $x \in [L,L]$ are nilpotent (see 3.2 in Humphreys).
    Take $A = [L,L] \subseteq B = L$ and apply the Lemma.
    Consider
    \begin{align*}
        M = \{x \in \gl(V) \mid [x,y] \in [L,L] \text{ for all } y \in L\}.
    \end{align*}
    Obviously, $L \subseteq M$. To apply the Lemma we need to show that $ \tr (xy) = 0 \forall y \in M$ for given $x \in [L,L]$.
    But \labelcref{eq:ast} says that for any $c \in M, a,b \in L$ (so $[a,b] \in [L,L]$) 
    $$ \tr([a,b]c) = \tr(a[b,c]) = 0,$$
    by assumption of the theorem.
    So the Lemma shows $x$ is nilpotent.
\end{proof}

\begin{definition}
    The \textbf{Killing form} of $L$ is the symmetric bilinear form
    \begin{align*}
        \kappa: L \tensor_F L \to , \kappa(x,y) := \tr(\ad x \compose \ad y).
    \end{align*}
\end{definition}

\begin{corollary}
    If 
    \begin{align*}
        \kappa(x,y) = 0 \forall x \in [L,L], y \in L,
    \end{align*}
    then $L$ is solvable.
\end{corollary}
\begin{proof}
    \begin{align*}
        \ad: L \to \gl(V), x \mapsto \ad x = [x, \cdot]
    \end{align*}
    is a Lie algebra homomorphism.
    \begin{align*}
        \ker \ad = Z(L)
    \end{align*}
    is a solvable ideal in $L$, so if $\im \ad \subseteq \gl(V)$ is solvable then $L$ is solvable.
    In other words:
    \begin{align*}
        0 \to Z(L) \to L \to \sfrac{L}{Z(L)} \cong \im \ad = : K \subseteq \gl(V) \to 0
    \end{align*}
    Cartan's critertion: if $\tr (xy) = 0 \forall x \in [K,K], y \in K$, then $K$ is solvable and
    for $x = \ad a, y = \ad b$
    \begin{align*}
        \kappa(a,b) = \tr (xy)
    \end{align*}
\end{proof}

\section{The Killing form}
We just saw: If $[L,L]=L^{(1)} \subseteq D_\kappa=\{x \in L \mid \kappa(x, \cdot)=0\}$
then $L$ is solvable.

\begin{lemma}
    $I \ideal L$ then for $x,y \in I$
    $$ \kappa_L(x,y) = \kappa_I(x,y).$$
\end{lemma}
\begin{proof}
    Extend a basis of $I$ to a basis of $L$. Then $\ad (x)$ has for $x \in I$ the matrix form
    $$ \begin{pmatrix}
        A & B\\
        0 & 0
    \end{pmatrix},$$
    where $A$ is the matrix of $\ad_I x: I \to I$ in the basis chosen and $B \neq 0$.
    Argument extends to $\ad x \compose \ad y$.
\end{proof}

\begin{example}
    \begin{align*}
        x = \begin{pmatrix}
            0 & 1\\
            0 & 0
        \end{pmatrix},
        y = \begin{pmatrix}
            0 & 0\\
            1 & 0
        \end{pmatrix},
        h = \begin{pmatrix}
            1 & 0\\
            0 & -1
        \end{pmatrix}\\
        \implies \kappa \text{ has matrix } \begin{pmatrix}
            0 & 0 & 4\\
            0 & 8 & 0\\
            4 & 0 & 0
        \end{pmatrix}
    \end{align*}
    calculate $\ad x, \ad y, \ad h$.
\end{example}

\begin{definition}
    Let $D_\kappa = \{a \in L \mid \kappa(a, \cdot) = 0\}$ be the radical of $\kappa$. We call
    $\kappa$ \textbf{nondegenerate}, if and only if $ D_\kappa=0$.    
\end{definition}

% \begin{example}
%     $V = \R^n$, $G = \SO(n)$, $\langle gx, gy \rangle = \langle x,y \rangle$
%     $\langle \cdot, \cdot \rangle : V \tensor V \to \R$
%     \begin{align*}
%         \Lie(G) = \so(n) = \{X \in \M_n(\R) \mid X^T = -X\} \\
%         \langle Xv, w \rangle &= (Xv)^Tw=v^TX^Tw\\
%         & v^T(-X)w = - \langle v, Xw \rangle\\
%         \langle Xv, w \rangle + \langle v, Xw \rangle = 0\\
%         \kappa([x,a],b) + \kappa(a, [x,b])\\
%         = \tr(\ad([x,a]) \compose \ad b) + \tr(\ad a \compose ) 
%     \end{align*}
% \end{example}

\begin{corollary}
    $D_\kappa \ideal L$
\end{corollary}
\begin{proof}
    $x \in D_\kappa$, that is $\kappa(x, \cdot) = 0$ and $y \in L$ , s.th.
    $$ \kappa([y,x],z)= - \kappa(x, [y,z]) = 0$$. So $[y,x] \in D_\kappa$.
\end{proof}

\begin{example}
    $L = \sl(2, F)$ basis $x,h,y$
    $\kappa$ is nondegenerate (see example above)
\end{example}

\begin{remark}
    $L$ semisimple $\iff$ $\Rad(L) = 0$, or 
    $\stackrel{\ast}{\iff}$ $L$ does not contain an nonzero abelian ideal
\end{remark}
\begin{proof}
    \underline{$\Leftarrow$:} of $\ast$ is clear: an abelian ideal is a solvable ideal
    \underline{$\implies$:} of $\ast$: $K \ideal L$, then $K^{(n)} \ideal L$. Appplying this with $K = \Rad(L)$ with the last
    $n$ for which $K^{(n)} \neq 0$ yields $\Rad(L) \neq 0 \implies$ there is a nonzero abelian ideal $0 = K^{(n+1)}$
\end{proof}

\begin{theorem}
    $L$ semisimple $\iff$ $\kappa$ is nondegenerate
\end{theorem}
\begin{proof}
    If $\Rad(L) = 0$ and $D_\kappa$ is the radical ok $\kappa$, then we can apply Cartan's critertion to show
    $\ad:L D_\kappa$ is solvable
    \begin{align*}
        \tr(\ad x \compose \ad y) = 0 \forall x \in D_\kappa, y \in L
    \end{align*}
    In particular for $y \in [D_\kappa, D_\kappa]$.
    Hence $D_\kappa$ is solvable, because with
    \begin{align*}
        \ad_L : L \to \gl(L) \text{ and}\\
        \ad_L |_{D_\kappa}: D_\kappa \to \gl(L)
    \end{align*}
    $\ker \ad_{L}|_{D_\kappa}$ is abelian, whence solvable.
    So $D_\kappa \subseteq \Rad(L) = 0$, so $\kappa$ is nondegenerate.
\end{proof}

Conversely, assume $D_\kappa = 0$. We show that every abelian ideal $I \ideal L$ is contained in $D_\kappa$.
If $y \in I, x \in L$ and $z \in L$ then
\begin{align*}
    (\ad x \compose \ad y)(z) = [x,[y,z]] \geq t\\
    (\ad x \compose \ad y)^2(z) = [x,[y,t]]
\end{align*}
So if $I \ideal L$ is an abelian ideal, then 
\begin{align*}
    (\ad x \compose \ad y)^2 (z) = 0 \forall y \in i, x,z \in L.
\end{align*}
So $\ad x \compose \ad y$ is nilpotent, so its trace
\begin{align*}
    \kappa(x,y) = \tr(\ad x \compose \ad y) = 0
\end{align*}
So $I \ideal D_\kappa = 0$. So $L$ is semisimple by $\ast$.

\begin{theorem}
    $L$ ist semisimple ,if and only if
    $$ L = L_1 \oplus \ldots \oplus L_n $$
    for simple ideals $L_i \ideal L$. Moreover, 
    $$ \kappa(x,y) = 0$$ for $x \in L_i, y \in L_j, i \neq j$ and
    $$ \kappa(x,y) = \kappa_{L_i}(x,y)$$ for $x,y \in L_i$.
\end{theorem}
\begin{proof}
    Let $I \ideal L$ and $I^{\bot} := \{x \in L \mid \kappa(x,y) = 0 \forall y \in I\}$. 
    Then $I^\bot \ideal L$ by the $\ad$-invariance of $\kappa$
    $$ \kappa(\ad(x)a, b) = - \kappa(a, \ad(x)b)$$
    So if $y \in I^\bot$ and $z \in L, x \in I$.
    \begin{align*}
        \kappa(x, [z,y]) &= \kappa(x \ad (z) y)\\
        &= - \kappa(\ad(z)x,y)\\
        &= - \kappa([z,x],y) = 0
    \end{align*}
    so $[z,y] \in I^\bot$.

    Cartan's critertion, applied to $I$ shows that $I \cap I^\bot$ is solvable.
    This is a solvable ideal, hence $0$ as $L$ is simple.
    Since $\dim I + \dim I^\bot = \dim L$, we get
    $$ L = I \oplus I^\bot$$
\end{proof}
