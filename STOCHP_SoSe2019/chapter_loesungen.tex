% This work is licensed under the Creative Commons
% Attribution-NonCommercial-ShareAlike 4.0 International License. To view a copy
% of this license, visit http://creativecommons.org/licenses/by-nc-sa/4.0/ or
% send a letter to Creative Commons, PO Box 1866, Mountain View, CA 94042, USA.

\chapter{Lösungen der Übungsaufgaben}

\section{Lösung von 
	\texorpdfstring{\hyperref[aufg:1]{Aufgabe 1}}{}
}\label{loes:1}

%TODO

\section{Lösung von 
	\texorpdfstring{\hyperref[aufg:2]{Aufgabe 2}}{}
}\label{loes:2}

%TODO

\section{Lösung von 
	\texorpdfstring{\hyperref[aufg:3]{Aufgabe 3}}{}
}\label{loes:3}

Wir zeigen, dass $\A_0$ ein $\sigma$-Algebra ist.\\
$\Omega\in\A_0$, weil $\Omega\subseteq\A$ und somit $\Omega\cup\emptyset\setminus\emptyset\in\A_0$.\nl
Sei $A_0\in\A_0$. Wir zeigen $A_0^C\in\A_0$.
Da $A_0\in\A_0$ existieren $A\in\A$ und $\P$-Nullmengen so, dass $A_0=(A\cup N)\setminus M\in\A_0$.
\begin{align*}
	A_0^C
	&=\Omega\setminus\big((A\cup N)\setminus M\big)\\
	&=\Omega\setminus\big((A\cup N)\cap M^C\big)\\
	&=\Omega\cap\big((A\cup N)\cap M^C\big)\\
	&=(A\cup N)\cap M^C\\
	\overset{\text{DM}}&=
	(A\cup N)^C\cup M\\
	&=(A^C\cap N^C)\cup M\\
	&=(A^C\cup M)\cap (N^C\cup M)\\
	&=(A^C\cup M)\setminus(N^C\cup M)^C\\
	&=(A^C\cup M)\setminus(N\cap M^C)\\
	&=(\tilde{A}\cap\tilde{N})\setminus\tilde{M}\qquad\mit\tilde{A}:=A^C\in\A,~\tilde{N}:=M,~\tilde{M}:=N\cap M^C~\P\text{-Nullmengen}\\
	&\implies A_0^C\in\A_0
\end{align*}
Sei $(A_{0,n})_{n\in\N}\subseteq\A_0$. %TODO


Wir zeigen nun die Wohldefiniertheit von $\P_0$:
%TODO

\section{Lösung von 
	\texorpdfstring{\hyperref[aufg:4]{Aufgabe 4}}{}
}\label{loes:4}

\section{Lösung von 
	\texorpdfstring{\hyperref[aufg:5]{Aufgabe 5}}{}
}\label{loes:5}

\section{Lösung von 
	\texorpdfstring{\hyperref[aufg:6]{Aufgabe 6}}{}
}\label{loes:6}

\section{Lösung von 
	\texorpdfstring{\hyperref[aufg:7]{Aufgabe 7}}{}
}\label{loes:7}

\section{Lösung von 
	\texorpdfstring{\hyperref[aufg:8]{Aufgabe 8}}{}
}\label{loes:8}

\section{Lösung von 
	\texorpdfstring{\hyperref[aufg:9]{Aufgabe 9}}{}
}\label{loes:9}

\betone{Zeige \ref{item:aufg9(1)}:}\\
Um Stationärität (im weiteren Sinne) zu prüfen, prüfen wir zuerst, dass $\E[Z(t)]$ nicht von $t$ abhängt:
\begin{align}\label{eq:loes9}
	\E\big[Z(t)\big]
	\overset{\Def}&{=}
	\E\left[\sum\limits_{j=1}^n\exp\big(\ii\mal\scaProd{a_j}{t}\big)\mal X_j\right]
	\overset{\Lin}{=}
	\sum\limits_{j=1}^n\exp\big(\ii\mal\scaProd{a_j}{t}\big)\mal\underbrace{\E[X_j]}_{
		\overset{\Vor}{=}0
	}
	=0=:M\quad\forall t\in\R^d
\end{align}

Nun müssen wir noch prüfen, dass die Korrelationsfunktion genau von $t_1-t_2$ ($t_1,t_2\in T$) abhängt:
\begin{align*}
	\E\big[Z(t_1)\mal\overline{Z(t_2)}\big]
	\overset{\Def}&{=}
	\E\left[\klammern{\sum\limits_{j=1}^n\Big(\exp\big(\ii\mal\scaProd{a_j}{t}\big)\mal X_j}\mal\overline{\klammern{\sum\limits_{k=1}^n\exp\big(\ii\mal\scaProd{a_k}{t_2}\big)\mal X_k}}\right]\\
	&=
	\E\left[\klammern{\sum\limits_{j=1}^n\Big(\exp\big(\ii\mal\scaProd{a_j}{t}\big)\mal X_j}\mal\klammern{\sum\limits_{k=1}^n\exp\big(-\ii\mal\scaProd{a_k}{t_2}\big)\mal \overline{X_k}}\right]\\
	&=\E\left[\sum\limits_{j=1}^n\sum\limits_{k=1}^n\exp\Big(\ii\mal\big(\scaProd{a_j}{t_1}-\scaProd{a_k}{t_2}\big)\Big)\mal X_j\mal \overline{X_k}\right]\\
	\overset{\Lin}&{=}
	\sum\limits_{j=1}^n\sum\limits_{k=1}^n\exp\Big(\ii\mal\big(\scaProd{a_j}{t_1}-\scaProd{a_k}{t_2}\big)\Big)\mal\underbrace{\E\big[ X_j\mal \overline{X_k}\big]}_{=0,~\falls j\neq k}\\
	&=\sum\limits_{j=1}^n\exp\big(\ii\mal\scaProd{a_j}{t_1-t_2}\big)\mal\E\big[X_j\mal \overline{X_j}\big]\\
\end{align*}
Somit haben wir die Korrelationsfunktion direkt berechnet:
\begin{align*}
	C\big(t_1,t_2\big)=C\big(t_1-t_2\big)
	&=\sum\limits_{j=1}^n\exp\big(\ii\mal\scaProd{a_j}{t_1-t_2}\big)\mal\E\big[X_j\mal \overline{X_j}\big]
\end{align*}
Nun berechnen wir noch die Kovarianzfunktion:
\begin{align*}
	\sigma\big(t_1,t_2\big)
	\overset{\Def}&{=}
	\E\Big[\big(Z(t_1)-\underbrace{M(t_1)}_{
		\overset{\eqref{eq:loes9}}{=}0
	}\big)\mal\overline{\big(Z(t_2)-\underbrace{M(t_2)}_{
		\overset{\eqref{eq:loes9}}{=}0
	}}\big)\Big]
	=C\big(t_1,t_2\big)
	=C\big(t_1-t_2\big)
\end{align*}

\betone{Zeige \ref{item:aufg9(2)}:}\\
Wir prüfen wieder, wann $\E[Z(t)]$ nicht von $t$ abhängt:
\begin{align*}
	\E\big[Z(t)\big]
	\overset{\Def}&{=}
	\E\big[g(t)\mal X\big]
	\overset{\Lin}{=}
	g(t)\mal\underbrace{\E[X]}_{
		\overset{\Vor}{<}\infty,\text{ konstant}
	}
\end{align*}
Also ist $\E[Z(t)]$ genau dann unabhängig von $t$, wenn $g\colon\R^d\to\C$ eine konstante Funktion ist.\\
In diesem Fall ist auch die zweite Eigenschaft von Stationärität (im weiteren Sinne) erfüllt:
\begin{align*}
	\E\Big[Z(t_1)\mal\overline{T(t_2)}\Big]
	\overset{\Def}&{=}
	\E\Big[g(t_1)\mal X\mal\overline{g(t_2)}\mal \overline{X}\Big]
	\overset{g\equiv\text{konst}}{=}
	\abs{g}^2\mal\E\big[X\mal\overline{X}\big]
\end{align*}
Die Korrelationsfunktion ist also auch konstant.

\section{Lösung von 
	\texorpdfstring{\hyperref[aufg:10]{Aufgabe 10}}{}
}\label{loes:10}

%TODO bisher nur rüberkopiert

\begin{align*}
\E[X(t)]
=\E[A\cdot\cos(\theta\cdot t)+B\cdot\sin(\theta\cdot t)]
=\E[A]\cdot\cos(\theta\cdot t)+\E[B]\cdot\sin(\theta\cdot t)
=0+0=0
\end{align*}

\begin{align*}
C(h) &=
\E[X(t+h)\cdot X(t)] \\
&=\E\left[(A\cos(\theta t)+B\sin(\theta t))\cdot(A\cos(\theta t+\theta h)+B\sin(\theta t+\theta h))\right]\\
&=\underbrace{\E[A^2]}_{=1}\cos^2(\theta t)\cos(\theta h)
-\underbrace{\E[A^2]}_{=1}\cos(\theta t)\sin(\theta t)\sin(\theta h)+\underbrace{\E[A\cdot B]}_{=0}\cdot...\\
&~~~+\underbrace{\E[B^2]}_{=1}\sin^2(\theta t)\cos(\theta h)+\underbrace{\E[B^2]}_{=1}\sin(\theta t)\cos(\theta t)\sin(\theta h)\\
&=\cos^2(\theta t)\cos(\theta h)+\sin^2(\theta t)\cos(\theta h)+0\\
&=\cos(\theta h)\cdot 1
\end{align*}

Zur (b):

Nein, ist nicht stationär im engeren Sinne. Gegenbeispiel:

\section{Lösung von 
	\texorpdfstring{\hyperref[aufg:11]{Aufgabe 11}}{}
}\label{loes:11}


\section{Lösung von 
	\texorpdfstring{\hyperref[aufg:12]{Aufgabe 12}}{}
}\label{loes:12}

\section{Lösung von 
	\texorpdfstring{\hyperref[aufg:13]{Aufgabe 13}}{}
}\label{loes:13}

\betone{Zeige \ref{item:aufg13_1}:}

\betone{Zeige \ref{item:aufg13_2}:}

\betone{Zeige \ref{item:aufg13_3}:}

\betone{Zeige \ref{item:aufg13_4}:}

\betone{Zeige \ref{item:aufg13_5}:}
Folgt direkt aus \ref{item:aufg13_3} und \ref{item:aufg13_4}.

\section{Lösung von 
	\texorpdfstring{\hyperref[aufg:14]{Aufgabe 14}}{}
}\label{loes:14}

\section{Lösung von 
	\texorpdfstring{\hyperref[aufg:15]{Aufgabe 15}}{}
}\label{loes:15}
