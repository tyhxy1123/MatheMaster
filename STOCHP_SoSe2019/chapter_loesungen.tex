% This work is licensed under the Creative Commons
% Attribution-NonCommercial-ShareAlike 4.0 International License. To view a copy
% of this license, visit http://creativecommons.org/licenses/by-nc-sa/4.0/ or
% send a letter to Creative Commons, PO Box 1866, Mountain View, CA 94042, USA.

\chapter{Lösungen der Übungsaufgaben}

\section{Lösung von 
	\texorpdfstring{\hyperref[aufg:1]{Aufgabe 1}}{}
}\label{loes:1}

%TODO

\section{Lösung von 
	\texorpdfstring{\hyperref[aufg:2]{Aufgabe 2}}{}
}\label{loes:2}

%TODO

\section{Lösung von 
	\texorpdfstring{\hyperref[aufg:3]{Aufgabe 3}}{}
}\label{loes:3}

\betone{Zeige \ref{item:aufg3_1}., also, dass $A_0$ eine $\sigma$-Algebra ist:}\\
$\Omega\in\A_0$, weil $\Omega\in\A$ und somit $\Omega\cup\emptyset\setminus\emptyset\in\A_0$.\nl
Sei $A_0\in\A_0$. Wir zeigen $A_0^C\in\A_0$.
Da $A_0\in\A_0$ existieren $A\in\A$ und $\P$-Nullmengen so, dass $A_0=(A\cup N)\setminus M\in\A_0$.
\begin{align*}
	A_0^C
	&=\Omega\setminus\big((A\cup N)\setminus M\big)\\
	&=\Omega\setminus\big((A\cup N)\cap M^C\big)\\
	&=\Omega\cap\big((A\cup N)\cap M^C\big)\\
	&=(A\cup N)\cap M^C\\
	\overset{\text{DM}}&=
	(A\cup N)^C\cup M\\
	&=(A^C\cap N^C)\cup M\\
	&=(A^C\cup M)\cap (N^C\cup M)\\
	&=(A^C\cup M)\setminus(N^C\cup M)^C\\
	&=(A^C\cup M)\setminus(N\cap M^C)\\
	&=(\tilde{A}\cap\tilde{N})\setminus\tilde{M}\qquad\mit\tilde{A}:=A^C\in\A,~\tilde{N}:=M,~\tilde{M}:=N\cap M^C~\P\text{-Nullmengen}\\
	&\implies A_0^C\in\A_0
\end{align*}
Zur Vereinigungsstabilität:\\
Sei $(A_{0,n})_{n\in\N}\subseteq\A_0$. 
O.B.d.A. ist 
\begin{align*}
	(A_{0,n})_{n\in\N}=\Big((A_1\cup N_1)\setminus M_1, A_2\cup N_1)\setminus M_1,\ldots\Big)
\end{align*}
Dann gilt
\begin{align*}
	\bigcup\limits_{i\in\N} A_{0,i}
	&=\bigcup\limits_{i\in\N}\big(A_i\cup N_i\big)\setminus M_i\\
	&=\klammern{\bigcup\limits_{i\in\N}\big(A_i\cup N_i\big)}\setminus\klammern{\bigcup\limits_{i\in\N} M_i}\\
	&=\klammern[\bigg]{\underbrace{\bigcup\limits_{i\in\N} A_i}_{
		 \in\A
	}\cup\underbrace{\bigcup\limits_{i\in\N} N_i}_{
		\cup\text{ v. Nullmengen ist Nullmenge}
	}}\setminus\klammern{\bigcup\limits_{i\in\N} M_i}\\
\end{align*}

\betone{Zeige \ref{item:aufg3_3}., also, dass $P_0$ ein Wahrscheinlichkeitsmaß ist:}
\begin{align*}
	\P_0(\Omega)=\P(\Omega)=1
\end{align*}
und 
\begin{align*}
	\P_0\big(A_0^C\big)
	=\P\big(A^C\big)
	=1-\P(A)=1-\P_0(A_0)
	\qquad\mit A_0=(A\cup N)\setminus M,A\in\A
\end{align*}
und zuletzt noch
\begin{align*}
	\P_0\klammern{\bigcup\limits_{i=1}^\infty A_{0,i}}
	=\P\klammern{\bigcup\limits_{i=1}^\infty A_i}
	=\sum\limits_{i=1}^\infty\P(A_i)
	=\sum\limits_{i=1}^\infty\P_0(A_{0,i})
\end{align*}

\betone{Zeige \ref{item:aufg3_2}., also, dass $P_0$ wohldefiniert ist:}\\
Sei $A_{0,1}=(A_1\cup N_1)\setminus M_1$ und $A_{0,2}=(A_2\cup N_2)\setminus M_2$.
Sei $A_{0,1}=A_{0,2}$.
Dann gilt:
\begin{align*}
	\P_0(A_{0,1})
	&=\P_0\Big(\big(A_1\cup N_1\big)\Big)-\P_0(M_1)\\
	&=\P\big(A_1\cup N_1\big)-\P(M_1)\\
	&=\P(A_1)+\underbrace{\P(N_1)}_{=0}-\underbrace{\P(A_1\cap N_1}_{=0}-\underbrace{\P(M-1)}_{=0}=\P(A_1)
\end{align*}
Da $A_{0,1}=A_{0,2}$ folgt $\P(A_1)=\P(A_2)$ und $\P_0(A_{0,2})=\P(A_2)$.
%TODO dies ist noch nicht ganz korrekt.

\section{Lösung von 
	\texorpdfstring{\hyperref[aufg:4]{Aufgabe 4}}{}
}\label{loes:4}

\section{Lösung von 
	\texorpdfstring{\hyperref[aufg:5]{Aufgabe 5}}{}
}\label{loes:5}

\section{Lösung von 
	\texorpdfstring{\hyperref[aufg:6]{Aufgabe 6}}{}
}\label{loes:6}

\section{Lösung von 
	\texorpdfstring{\hyperref[aufg:7]{Aufgabe 7}}{}
}\label{loes:7}

\section{Lösung von 
	\texorpdfstring{\hyperref[aufg:8]{Aufgabe 8}}{}
}\label{loes:8}

\section{Lösung von 
	\texorpdfstring{\hyperref[aufg:9]{Aufgabe 9}}{}
}\label{loes:9}

\betone{Zeige \ref{item:aufg9(1)}:}\\
Um Stationärität (im weiteren Sinne) zu prüfen, prüfen wir zuerst, dass $\E[Z(t)]$ nicht von $t$ abhängt:
\begin{align}\label{eq:loes9}
	\E\big[Z(t)\big]
	\overset{\Def}&{=}
	\E\left[\sum\limits_{j=1}^n\exp\big(\ii\mal\scaProd{a_j}{t}\big)\mal X_j\right]
	\overset{\Lin}{=}
	\sum\limits_{j=1}^n\exp\big(\ii\mal\scaProd{a_j}{t}\big)\mal\underbrace{\E[X_j]}_{
		\overset{\Vor}{=}0
	}
	=0=:M\quad\forall t\in\R^d
\end{align}

Nun müssen wir noch prüfen, dass die Korrelationsfunktion genau von $t_1-t_2$ ($t_1,t_2\in T$) abhängt:
\begin{align*}
	\E\big[Z(t_1)\mal\overline{Z(t_2)}\big]
	\overset{\Def}&{=}
	\E\left[\klammern{\sum\limits_{j=1}^n\Big(\exp\big(\ii\mal\scaProd{a_j}{t}\big)\mal X_j}\mal\overline{\klammern{\sum\limits_{k=1}^n\exp\big(\ii\mal\scaProd{a_k}{t_2}\big)\mal X_k}}\right]\\
	&=
	\E\left[\klammern{\sum\limits_{j=1}^n\Big(\exp\big(\ii\mal\scaProd{a_j}{t}\big)\mal X_j}\mal\klammern{\sum\limits_{k=1}^n\exp\big(-\ii\mal\scaProd{a_k}{t_2}\big)\mal \overline{X_k}}\right]\\
	&=\E\left[\sum\limits_{j=1}^n\sum\limits_{k=1}^n\exp\Big(\ii\mal\big(\scaProd{a_j}{t_1}-\scaProd{a_k}{t_2}\big)\Big)\mal X_j\mal \overline{X_k}\right]\\
	\overset{\Lin}&{=}
	\sum\limits_{j=1}^n\sum\limits_{k=1}^n\exp\Big(\ii\mal\big(\scaProd{a_j}{t_1}-\scaProd{a_k}{t_2}\big)\Big)\mal\underbrace{\E\big[ X_j\mal \overline{X_k}\big]}_{=0,~\falls j\neq k}\\
	&=\sum\limits_{j=1}^n\exp\big(\ii\mal\scaProd{a_j}{t_1-t_2}\big)\mal\E\big[X_j\mal \overline{X_j}\big]\\
\end{align*}
Somit haben wir die Korrelationsfunktion direkt berechnet:
\begin{align*}
	C\big(t_1,t_2\big)=C\big(t_1-t_2\big)
	&=\sum\limits_{j=1}^n\exp\big(\ii\mal\scaProd{a_j}{t_1-t_2}\big)\mal\E\big[X_j\mal \overline{X_j}\big]
\end{align*}
Nun berechnen wir noch die Kovarianzfunktion:
\begin{align*}
	\sigma\big(t_1,t_2\big)
	\overset{\Def}&{=}
	\E\Big[\big(Z(t_1)-\underbrace{M(t_1)}_{
		\overset{\eqref{eq:loes9}}{=}0
	}\big)\mal\overline{\big(Z(t_2)-\underbrace{M(t_2)}_{
		\overset{\eqref{eq:loes9}}{=}0
	}}\big)\Big]
	=C\big(t_1,t_2\big)
	=C\big(t_1-t_2\big)
\end{align*}

\betone{Zeige \ref{item:aufg9(2)}:}\\
\betone{Fall 1: $\E[X]\neq0$:}\\
Wir prüfen wieder, wann $\E[Z(t)]$ nicht von $t$ abhängt:
\begin{align*}
	\E\big[Z(t)\big]
	\overset{\Def}&{=}
	\E\big[g(t)\mal X\big]
	\overset{\Lin}{=}
	g(t)\mal\underbrace{\E[X]}_{
		\overset{\Vor}{<}\infty,\text{ konstant},\neq0
	}
\end{align*}
Also ist $\E[Z(t)]$ genau dann unabhängig von $t$, wenn $g\colon\R^d\to\C$ eine konstante Funktion ist.\\
In diesem Fall ist auch die zweite Eigenschaft von Stationärität (im weiteren Sinne) erfüllt:
\begin{align*}
	\E\Big[Z(t_1)\mal\overline{T(t_2)}\Big]
	\overset{\Def}&{=}
	\E\Big[g(t_1)\mal X\mal\overline{g(t_2)}\mal \overline{X}\Big]
	\overset{g\equiv\text{konst}}{=}
	\abs{g}^2\mal\E\big[X\mal\overline{X}\big]
\end{align*}
Die Korrelationsfunktion ist also auch konstant.\nl
\betone{Fall 2: $\E[X]=0$}\\
Die erste Eigenschaft der Stationarität ist trivialerweise erfüllt:
\begin{align*}
	\E\big[Z(t)\big]
	\overset{\Def}&{=}
	\E\big[g(t)\mal X\big]
	\overset{\Lin}{=}
	g(t)\mal\underbrace{\E[X]}_{
		\overset{\Vor}{=}0
	}
	=0\qquad\forall t\in\R=:T
\end{align*}

Die zweite Eigenschaft ist schon schwieriger:
\begin{align*}
	\E\Big[Z(t_1)\mal\overline{T(t_2)}\Big]
	\overset{\Def}&{=}
	\E\Big[g(t_1)\mal X\mal\overline{g(t_2)}\mal \overline{X}\Big]
	\overset{\Lin}{=}
	g(t_1)\mal\overline{g(t_2)}\mal\E\big[|X^2|\big]
\end{align*}
Somit ist $Z$ genau dann stationär (im weiteren Sinne), wenn 
\begin{align*}
	g(t_1)\mal\overline{g(t_2)}=C(t_1-t_2)
\end{align*}
für eine Funktion $C\colon\R\to\C$.
Dies führt auf die \undefine{Cauchy'sche Funktionalgleichung.}

\section{Lösung von 
	\texorpdfstring{\hyperref[aufg:10]{Aufgabe 10}}{}
}\label{loes:10}

\betone{Zeige \ref{item:Aufg:10(1)}:}\\
Um Stationärität (im weiteren Sinne) zu prüfen, prüfen wir zuerst, dass $\E[X(t)]$ nicht von $t$ abhängt:

\begin{align*}
	\E[X(t)]
	\overset{\Def}&{=}
	\E\big[A\mal\cos(\alpha\mal t)+B\mal\sin(\alpha\mal t)\big]
	\overset{\Lin}{=}
	\underbrace{\E[A]}_{
		\overset{\Vor}{=}0
	}\mal\cos(\alpha\mal t)+\underbrace{\E[B]}_{
		\overset{\Vor}{=}0
	}\mal\sin(\alpha\mal t)
	%=0+0
	=0
\end{align*}

Nun müssen wir noch prüfen, dass die Korrelationsfunktion genau von $t_1-t_2$ ($t_1,t_2\in T$) abhängt:
Seien also $t_1,t_2\in T=\R$ beliebig. 

\begin{align*}
	&\E\big[X(t_1)\mal X(t_2)\big]\\
	\overset{\Def}&{=}
	\E\Big[\big(A\mal\cos(\alpha\mal t)+B\mal\sin(\alpha\mal t)\big)\mal
	\big(A\mal\cos(\alpha\mal t)+B\mal\sin(\alpha\mal t)\big)\Big]\\
	&=\E\Big[\underbrace{A^2}_{
		\overset{\Vor}{=}1
	}\mal\cos(\alpha\mal t_1)\mal\cos(\alpha\mal t_2)+\underbrace{A\mal B}_{
		\overset{\Vor}{=}0
	}\mal\ldots+\underbrace{B\mal A}_{
		\overset{\Vor}{=}0
	}\mal\ldots+\underbrace{B^2}_{
		\overset{\Vor}{=}1
	}\mal\sin(\alpha\mal t_1)\mal\sin(\alpha\mal t_2)\Big]\\
	&=\cos(\alpha\mal t_1)\mal\cos(\alpha\mal t_2)+\sin(\alpha\mal t_1)\mal\sin(t\mal t_2)\\
	\overset{\eqref{eq:Aufg10Additionstheorem}}&{=}
	\cos(\alpha\mal t_1-\alpha\mal t_2)\\
	&=\cos\big(\alpha\mal(t_1-t_2)\big)
\end{align*}

Hierbei wird das Additionstheorem
\begin{align}\label{eq:Aufg10Additionstheorem}
	\cos(x-y)=\cos(x)\mal\cos(y)+\sin(x)\mal\sin(y)\qquad\forall x,y\in\R
\end{align}
verwendet.
Damit ist $X$ stationär (im weiteren Sinne).\nl
\betone{Zeige \ref{item:Aufg:10(2)}:}\\
Nein, $X$ ist nicht stationär im engeren Sinne. 
Zumindest nicht für alle $\alpha\in\R$.
Für $\alpha=0$ ist 
\begin{align*}
	X(t)=A\mal\cos(0\mal t)+B\mal\sin(0\mal t)=A\qquad\forall t\in T
\end{align*}
natürlich stationär im engeren Sinne, da $X$ \betone{nicht} von $t$ abhängt.
%Sei nun also o.B.d.A. $\alpha\neq0$.
Wir zeigen, dass $X$ für $\alpha\neq0$ \betone{nicht} stationär ist:
Setze $t_1:=0$ und $t_2:=\frac{\pi}{2\mal\alpha}$.
Dann besitzen die Zufallsgrößen (= einelementige Zufallsvektoren)
\begin{align*}
	X(t_1)=A\qquad\und\qquad X(t_2)=B
\end{align*}
\betone{nicht} notwendigerweise dieselbe Verteilung, denn es sind keine Voraussetzungen an die Verteilung der Zufallsgrößen $A$ und $B$ gestellt.
So ist z.B. $A\sim\Nor(0,1)$ und $B\sim\Exp(0,1)$ möglich.

\section{Lösung von 
	\texorpdfstring{\hyperref[aufg:11]{Aufgabe 11}}{}
}\label{loes:11}

Berechnen der Kovarianzfunktion:
\begin{align*}
	M(t)
	\overset{\Def}{=}	
	\E\big[X_t\big]
	\overset{\Def}{=}
	\int\limits_\Omega X_t(\omega)\ds\P(\omega)
	\overset{\Def}{=}
	\int\limits_\Omega t\mal\omega\ds\P(\omega)
	=t\mal\int\limits_0^1\omega \ds\lambda(\omega)
	=t\mal\left[\frac{\omega^2}{2}\right]_{\omega=0}^1
	=\frac{t}{2}
\end{align*}

Damit können wir jetzt die Kovarianzfunktion ausrechnen:

\begin{align*}
	\sigma(t_1,t_2)
	\overset{\Def}&{=}
	\E\Big[\big(X_{t_1}-M(t_1)\big)\mal\overline{\big(X_{t_2}-M(t_2)\big)}\Big]\\
	&=\int\limits_\Omega\klammern{t_1\mal\omega-\frac{t_1}{2}}\mal\klammern{t_2\mal\omega-\frac{t_2}{2}}\ds\P(\omega)\\
	&=\int\limits_0^1 t_1\mal\klammern{\omega-\frac{1}{2}}\mal t_2\mal\klammern{\omega-\frac{1}{2}}\ds\lambda(\omega)\\
	&=t_1\mal t_2\mal \int\limits_0^1 \klammern{\omega-\frac{1}{2}}^2\ds \lambda\\
	\overset{\text{Subst}}&=
	t_1\mal t_2\mal\int\limits_{-\frac{1}{2}}^{\frac{1}{2}} \omega^2\ds\lambda(\omega)\\
	&=t_1\mal t_2\mal\left[\frac{\omega^3}{3}\right]_{\omega=-\frac{1}{2}}^{\frac{1}{2}}\\
	&=t_1\mal t_2\mal \klammern{\frac{1}{8\mal 3}--\frac{1}{8\mal 3}}\\
	&=\frac{t_1\mal t_2}{12}
\end{align*}

Nun bestimmen wir noch die endlich-dimensionale Verteilung des Prozesses $X$:\\
Seien $n\in\N$ und $t_1,\ldots,t_n\in T$ beliebig.
Dann ist die endlich dimensionale (Rand)-verteilung
\begin{align*}
	\mu_{t_1,\ldots,t_n}(B_1\times\ldots\times B_n)
	=\lambda\klammern[\Big]{\set[\big]{\omega\in[0,1]:t_1\mal\omega,\ldots,t_n\mal\omega\in B}}
	\qquad\forall B\in\B([0,1])
\end{align*}

\section{Lösung von 
	\texorpdfstring{\hyperref[aufg:12]{Aufgabe 12}}{}
}\label{loes:12}

\section{Lösung von 
	\texorpdfstring{\hyperref[aufg:13]{Aufgabe 13}}{}
}\label{loes:13}

\betone{Zeige \ref{item:aufg13_1}:}

\betone{Zeige \ref{item:aufg13_2}:}

\betone{Zeige \ref{item:aufg13_3}:}

\betone{Zeige \ref{item:aufg13_4}:}

\betone{Zeige \ref{item:aufg13_5}:}
Folgt direkt aus \ref{item:aufg13_3} und \ref{item:aufg13_4}.

\section{Lösung von 
	\texorpdfstring{\hyperref[aufg:14]{Aufgabe 14}}{}
}\label{loes:14}

\section{Lösung von 
	\texorpdfstring{\hyperref[aufg:15]{Aufgabe 15}}{}
}\label{loes:15}
