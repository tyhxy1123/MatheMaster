% This work is licensed under the Creative Commons
% Attribution-NonCommercial-ShareAlike 4.0 International License. To view a copy
% of this license, visit http://creativecommons.org/licenses/by-nc-sa/4.0/ or
% send a letter to Creative Commons, PO Box 1866, Mountain View, CA 94042, USA.

\chapter{Allgemeine Eigenschaften}
\section{Separabilität und Messbarkeit}

Die Definition eines Prozesses $X$ \ref{def1.3.1} verlangt Messbarkeit von
\begin{align*}
	\omega\mapsto X(t,\omega)
\end{align*}
für jedes $t\in T$, es wird jedoch keine Bedingung an die Abbildung
\begin{align*}
	t\mapsto X(t,\omega)
\end{align*}
für festes $\omega\in\Omega$ gestellt.\nl
Ist zum Beispiel $T:=(a,b)\subseteq\R$ ein Intervall, also \betone{nicht abzählbar}, so hat man gelegentlich auch mit überabzählbar vielen Ereignissen zu tun, was zu Schwierigkeiten führen kann.
Beispiele:
\begin{enumerate}
	\item Die Menge
	\begin{align*}
		\set{\omega\in\Omega:X_t(\omega)\geq0~\forall t\in T}
		=\bigcap\limits_{t\in T}\set{\omega\in\Omega:X_t(\omega)\geq0}
	\end{align*}
	ist im Allgemeinen \betone{kein} Ereignis
	(alle $\omega\in\Omega$ mit nichtnegativer Trajektorie).
	\item $Y=\sup\limits_{t\in T}X_t$ braucht keine Zufallsvariable zu sein, da die Mengen
	\begin{align*}
		[Y\leq y]:=\set{\omega\in\Omega:Y(\omega)\leq y}=\bigcap\limits_{t\in T}\set{\omega\in\Omega:X_t(\omega)\leq y}
	\end{align*}
	\betone{nicht} immer Ereignisse sind.
	\item Man muss auch vorsichtig sein mit Ausdrücken wie
	\begin{align*}
		\lim\limits_{s\to t} X_s(\omega)
		\qquad\oder\qquad
		\int\limits_c^d X_t(\omega)\ds t,
	\end{align*}		
	die zwar für jedes $\omega\in\Omega$ existieren können, aber möglicherweise \betone{keine Zufallsgrößen} sind.
\end{enumerate}

Diese Schwierigkeiten motivierten die Einführung der Begriffe \undefine{Separabilität} und \undefine{Messbarkeit}.

\begin{definition}\label{def2.1.1}
	Sei $X$ ein $S$-wertiger Prozess auf $T$, wobei $S$ und $T$ topologische Räume sind.\\
	$X$ heißt \define{separabel} $\defiff$ eine abzählbare Teilmenge $T_0\subseteq T$ und ein $\P$-Nullereignis $A\in\A$ (d.h. $\P(A)=0$) mit der folgenden Eigenschaft existieren:\index{separabler Prozess}\\
	Für eine beliebige abgeschossene Menge $F\subseteq S$ und für eine beliebige offene Menge $O\subseteq T$ unterscheiden sich die Mengen
	\begin{align*}
		\set[\big]{\omega\in\Omega:X_t\in F~\forall t\in O}
	\end{align*}
	und 
	\begin{align*}
		\set[\big]{\omega\in\Omega:X_t(\omega)\in F~\forall t\in O\cap T_0}\in\A
	\end{align*}
	nur in einer Teilmenge von $A$ (d.h. ihre symmetrische Differenz ist eine Teilmenge von $A$).
	$T_0$ heißt eine \define{separierende Menge / Separabilitätsmenge} für $X$.
	\index{separierende Menge}\index{Separabilitätsmenge|see{separierende Menge}}
\end{definition}

\begin{bemerkungnr}\label{bemerkung2.1.2}
	Ein kleines Problem bleibt noch:
	Teilmengen von Nullereignissen sind im Allgemeinen keine Ereignisse.
	Dieses Problem lässt sich beheben, indem man den Grundraum vervollständigt (Bemerkung \ref{bemerkung1.3.2}\ref{item:bemerkung1.3.2(7)}).
\end{bemerkungnr}

\begin{aufgabenr}[2.1.3, \texorpdfstring{\hyperref[loes:13]{Lösung siehe Anhang}}{Lösung siehe Anhang}]\label{aufg:13}\enter
	Der Grundraum $(\Omega,\A,\P)$ sei vollständig, $(X_t)_{t\in T}$ reellwertiger und separabler Prozess, $T\subseteq\R$ ein Intervall und $O\subseteq\R$ eine offene Menge.
	Zeigen Sie, dass die folgenden Ausdrücke Zufallsgrößen sind $(s,t\in T)$:
	\begin{enumerate}
		\item $\begin{aligned}
			\sup\limits_{t\in T\cap O} X_t(\omega)
		\end{aligned}$ \label{item:aufg13_1}
		\item $\begin{aligned}
			\inf\limits_{t\in T\cap O} X_t(\omega)
		\end{aligned}$ \label{item:aufg13_2}
		\item $\begin{aligned}
			\limsup\limits_{s\to t} X_s(\omega)
		\end{aligned}$ \label{item:aufg13_3}
		\item $\begin{aligned}
			\liminf\limits_{s\to t} X_s(\omega)
		\end{aligned}$ \label{item:aufg13_4}
		\item $\begin{aligned}
			\lim\limits_{s\to t} X_s(\omega)
		\end{aligned}$ (falls existent) \label{item:aufg13_5}
	\end{enumerate}
\end{aufgabenr}

\begin{aufgabenr}[2.1.4, \texorpdfstring{\hyperref[loes:14]{Lösung siehe Anhang}}{Lösung siehe Anhang}]\label{aufg:14}\enter
	Besitzt $X$ stetige Pfade und ist $T$ separabel (d.h. $T$ enthält abzählbare dichte Teilmenge), so ist auch $X$ separabel.
	In diesem Fall ist eine beliebige abzählbare dichte Teilmenge $T_0\subseteq T$ eine Separabilitätsmenge für $X$.
\end{aufgabenr}


\begin{aufgabenr}[2.1.5, \texorpdfstring{\hyperref[loes:15]{Lösung siehe Anhang}}{Lösung siehe Anhang}]\label{aufg:15}\enter
	Wir betrachten den Wahrscheinlichkeitsraum
	\begin{align*}
		(\Omega,\A,\P)=\Big([0,1],\B\big([0,1]\big),\lambda|_{[0,1]}\Big)
	\end{align*}
	und den definierten Prozess $X$ durch
	\begin{align*}
		X_t(\omega):=\delta_t(\omega):=\left\lbrace\begin{array}{cl}
			1, &\falls t=\omega\\
			0, &\sonst
		\end{array}\right.
		\qquad\forall t,\omega\in T=\Omega=[0,1]
	\end{align*}
	Dann ist $X$ \betone{nicht} separabel.\\
	Der Prozess $Y$ mit
	\begin{align*}
		Y_t(\omega):=0\qquad\forall t,\omega\in T=\Omega=[0,1]
	\end{align*}
	hingegen ist separabel und es gilt
	\begin{align*}
		\P\big[X_t=Y_t\big]=1\qquad\forall t\in T=[0,1].
	\end{align*}
\end{aufgabenr}

\setcounter{satz}{5}

\begin{definition}\label{def2.1.6}
	Wir nennen zwei $S$-wertige Prozesse $X$ und $Y$ auf $T$ (auf denselben Wahrscheinlichkeitsraum) \define{Modifikationen voneinander}
	\index{Modifikationen}
	\begin{align*}
		\defiff \P\big[X_t=Y_t\big]=1\qquad\forall t\in T
	\end{align*}
\end{definition}

\begin{bemerkung} %No number
	Falls $X$ und $Y$ Modifikationen voneinander sind, gilt
	\begin{align*}
		\Big(X_{t_1},\ldots,X_{t_n}\Big)=\Big(Y_{t_1},\ldots,Y_{t_n}\Big)\text{f.s.} &&\forall t_1,\ldots,t_n\in T\\
	\end{align*}
	und damit besitzen $X$ und $Y$ dieselben endlich-dimensionalen Verteilungen.
\end{bemerkung}

\begin{satz}\label{satz2.1.7}
	Seien $S$ und $T$ metrische Räume, wobei $T$ separabel und $S$ kompakt ist.
	Für jeden $S$-wertigen Prozess $X$ auf $T$ existiert ein \betone{separabler} $S$-wertiger Prozess $Y$ auf $T$ so, dass 
	\begin{align*}
		\P\big[X_t=Y_t\big]=1\qquad\forall t\in T
	\end{align*}
	Also besitzt $X_t$ dann eine separable Modifikation.
\end{satz}

\begin{proof}
	Ziemlich aufwendig, daher ohne Beweis.\\
	Später betrachten wir oft Prozesse mit stetigen Pfaden, bei denen die Separabilität gegeben ist.
\end{proof}

\begin{definition}\label{def2.1.8}
	Sei $X$ ein $S$-wertiger Prozess auf $T$, wobei $(S,d)$ ein metrischer Raum und $T$ ein topologischer Raum sind.
	$X$ heißt \define{stetig in Wahrscheinlichkeit / stochastisch stetig} in $t_0\in T$
	\index{stetig in Wahrscheinlichkeit} %TODO
	\begin{align*}
		\defiff\forall\varepsilon>0:\lim\limits_{t\to t_0}
	\end{align*}
\end{definition}
