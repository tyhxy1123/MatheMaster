\chapter{Basic iterative methods}%
\label{cha:Basic iterative methods}

We want to solve the linear system
\[
	Au = f \qquad (Ax = h)
.\] 
\begin{itemize}
	\item traditional directl methodes (Gauss elemination) are insufficient:
		\begin{itemize}
			\item $\O(n^{4})$ steps
			\item $\O(n^{3})$ storage units
		\end{itemize}

	\item Iterative methods try to improve an implicit guess until some tolerance criterion is reached
		\begin{itemize}
			\item $\O(n^{2})$ storage units
		\end{itemize}
\end{itemize}

\section{Relaxation methods}%
\label{sec:Relaxation methods}

Basic relaxation methods are based on a coordinate-wise relaxation/reduction of an initial residuum
\[
\gamma = b- Ax
.\] 
In each relaxation step there methodes try to annihilate a component of the residual.

The general idea is to transform the linear system into a fixed point problem
\[
	x = x + P^{-1}(b-Ax)
,\] 
where $P$ is called a proconditioner (matrix) and should be easy to invert in the sense the $Pe=r$ can be solved easily and $P$ should be an approximation of matrix $A$.

Basic idea to solve a fixed-point problem is a fixed-point iteration
\[
	x^{k+1} = x^{k} + P^{-1}(b-Ax^{k})
.\] 
Rewriting this to make it a bit prettier results in
\[
	x^{k+1} = Gx^{k} + g =:T(x^{k})
.\] 
where $G = I-P^{-1}A$ is the iteration matrix and $g = P^{-1}b$.

\subsection{Some examples of relaxation schemes}
\label{sec:Some examples of relaxation schemes}

\begin{enumerate}[label=\Alph{enumi})]
	\item Jacobi method 

		Let $P_{J} = \diag(A)$, then obviously
		\[
			x^{k+1} = x^{k} + \diag(A)^{-1}(b-Ax^{k})
		.\] 
		written in terms of the components
		\[
			x^{k}= (\xi _{i}^{k})_{i}, \quad b=(\beta_{i})_{i}
		\] 
		gives
		\[
			a_{ii}\xi _{i}^{(k+1)} = -\sum_{\substack{j=1 \\ j \neq i}}^{n}{a_{ij}\xi _{j}^{(k)}+ \beta_{i}}
		.\] 
		Let $A=D-L-U$ be a splitting of the matrix $A$ into diagonal $D$, lower triangular part $L$ and upper triangular part $U$
		\[
		A = \begin{pmatrix}
			\diagdown & & -U \\
			  &D&    \\
			-L & & \diagdown
		\end{pmatrix}
		.\] 
		Then $P_{J} = D$
		\[
			Dx^{k+1} = (L+U)x^{k} + b
		.\] 
		For the 5-point stencil $\laplace_{h}^{(5)}$ we have 
		\[
			D = \diag(\frac{4}{h^{2}}) \qquad \text{ and } \qquad L+U = \begin{pmatrix}
				0 & 1 &  & 1 & & \\
				1 & \ddots & \ddots & 0 & \ddots &  \\
				  & \ddots & \ddots & \ddots & & 1 \\
				1 & 0 & \ddots & \ddots & \ddots & \\
				  &\ddots & & \ddots & \ddots & 1 \\
				  && 1 && 1 & 0
			\end{pmatrix}
		.\] 
			% TODO : Die Matrix nochmal in hübsch machen
		So written directly in terms of the grid vector $u_{i,j}$ we get the Jacobi scheme for inner nodes
		\[
			u_{i,j}^{k+1} = \frac{1}{4}(u_{i,j-1}^{k}+ u_{i-1,j}^{k} + u_{i,j+1}^{k} + u_{i+1,j}^{k} + h^{2}f_{i,j})
		.\] 
		% TODO : Bilder?

		\underline{Note}: Instead of $P=D$ one could implement a \underline{damped Jacobi scheme} with
		\[
		P_{wJ}=w\cdot D \qquad \text{ with } 0 < w \leq 1
		.\] 
\end{enumerate}

