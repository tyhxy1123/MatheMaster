\subsection{Triangulation}

\Large{ $B_h$ oder $B_b$ ???}

\begin{definition_}
	Let $\Omega \subset \R^d$ ($d \geq 1$) bounded domain. A partition $T_h$ of $\Omega$ with $\tau \in T_h$ is a \textit{triangulation} if
	\begin{enumerate}[label=\alph*)]
		\item $\forall \tau \in T_h$, $\tau$ is closed, $\tau^0 \neq \emptyset$, $\tau$ is conected and $\partial \tau \in C^{0,1}$
		
		\item $\bigcup \limits_{\tau \in T_h} = \overline{\Omega}$
		
		\item $\forall \tau_1,\tau_2 \in T_h:$ $\tau^0_1\cap \tau^0_2 = \emptyset$
	\end{enumerate}

	\begin{align*}
		h_{\tau} &= \diam (\tau)\\
		h &= \\underset{\tau \in T_h}{\max} h_{\tau}\\
		\rho_{\tau} &= \underset{\tau \in T_h}{\sup} \{2r: B_r \subset \tau, d\text{-ball of radius  }r \} \\
		\rho &= \\underset{\tau \in T_h}{\max} \rho_{\tau}\\
	\end{align*}
\end{definition_}
%%TODO add picture

We further require that all faces of $\tau_i \in T_h$ are faces of $\tau_i \in T_h$ or part of $\partial \Omega$.
%%TODO put picture from before here

Hanging nodes cannot guarantee continuity of functions.

$T_h$ triangulation of $\Omega$ with Lagrange elements $(\tau,P_\tau,\Sigma_\tau)$ with $\tau \in T_h$\\
$ N_h$ is set of all vertices of $T_h$. For $b\in \N_0 $ $T_h(b) $ is the set of all finite elements, for which $b$ is a vertex.

\begin{definition_}
	(Finite elements space)\\
	\begin{equation*}
		X_h = \{ v = (v_\tau)_{\tau \in T_h} \in \prod \limits_{\tau \in T_h} P_\tau :  \forall b  \in\N_h : \forall \tau_1,\tau_2 \in T_h(b): B_{b,\tau_1}(v_{\tau_1}) = B_{b,\tau_2}(v_{\tau_2})  \}
	\end{equation*}
\end{definition_}
For Lagrange elements:
\begin{equation*}
	v_{\tau_1}(b) = B_{b,\tau_1}(v_{\tau_1}) = B_{b,\tau_2}(v_{\tau_2}) = v_{\tau_2}(b)
\end{equation*}

Funnctions on $X_h$ are uniquely determined by $\{v(b): b \in N_h \}$ we thus also have 
\begin{equation*}
	X_h = \{ v: \overline{\Omega} \to \R,\ \forall \tau \in T_h, v|_\tau \in P_\tau \text{ and } \forall b \in N_h: \forall \tau_1,\tau_2 \in T_h(b) ,\ B_{b,\tau_1}(v|_{\tau_1}) = B_{b,\tau_2}(v|_{\tau_2})   \}.
\end{equation*}

\begin{example}
	linear lagrange elements $a_i$ vertex of $\tau \in T_h$,
	\begin{align*}
		B_{i,\tau}(p)  &= p(a_i) \quad \forall p \in P_\tau \\
		X_h = \{ v:\overline{\Omega} \to \R : \forall \tau \in T_h, v|_{\tau} \in P_1(\tau) ,\ v \text{ continuous in } \overline{\Omega} \}
	\end{align*}
\end{example}

$X_h$ is a finite dimensional space, has a basis
\begin{equation*}
	\Sigma_h = \{ B_{h,j} : j= 1,\dots,N  \} \quad \text{ set of DoFs} 
\end{equation*}

$\varphi_i \in X_h\ i=1,\dots,N$ with $B_{h,j}(\varphi_i) = \delta_{ij} \quad i,j=1,\dots,N$ (global basis funktions),$\varphi_i \in C^0(\overline{\Omega})$, $\forall \tau \in T_h: \varphi_i|_\tau \in P_i(\tau)$. $\varphi_i(b_j) = \delta_{ij}$ for $i,j = 1,\dots,N$ with $b_j \in N_h$ vertices of all $\tau \in T_h$.
