% This work is licensed under the Creative Commons
% Attribution-NonCommercial-ShareAlike 4.0 International License. To view a copy
% of this license, visit http://creativecommons.org/licenses/by-nc-sa/4.0/ or
% send a letter to Creative Commons, PO Box 1866, Mountain View, CA 94042, USA.

\DeclareMathOperator{\Sub}{Sub}
\newcommand{\X}{\mathfrak{X}}         % Familie
\newcommand{\Cat}{\mathscr C }        % Kategorie
\newcommand{\CatA}{\mathscr A}        % Kategorie A
\newcommand{\CatB}{\mathscr B }       % Kategorie B
\newcommand{\op}{\text{op}} 		  % dual category
\newcommand{\cat}{Cat}				  % Category property?
\newcommand{\dual}{\partial}		  % make elements dual in a category
\newcommand{\aussage}{\mathfrak{A}}   % fraktur Aussage
\newcommand{\V}{\mathscr V}			  % Kategorie V für vector räume
\newcommand{\astar}[1]{#1^{\ast}}	  % gesternte Sachen!
\newcommand{\aastar}[1]{#1^{\ast\ast}}% doppelt gesternte Sachen!


\DeclareMathOperator{\Ob}{Ob}         % Objekte
\DeclareMathOperator{\Mor}{Mor}       % Morphismen
\DeclareMathOperator{\cod}{cod}       % Codomain
\DeclareMathOperator{\codom}{codom}
% Cats
\DeclareMathOperator{\Group}{\mathbf{Group}}  % Kategorie der Gruppen
\DeclareMathOperator{\Ring}{\mathbf{Ring}}	  % Kategorie der Ringe
\DeclareMathOperator{\Top}{\mathbf{Top}}	  % Kategorie der topol. Räume  
\DeclareMathOperator{\Topp}{\Top_{\ast}}	  % Kategorie der Punkt fixierten topol. Räume  
\DeclareMathOperator{\RelSet}{\mathbf{RelSet}} 
\DeclareMathOperator{\Mat}{\mathbf Mat}   % Kategorie der Matrizen
\DeclareMathOperator{\Ab}{\mathbf{Ab}}	  % Kategorie der abelschen Gruppen
\DeclareMathOperator{\Set}{\mathbf{Set}}  % Kategorie der Mengen
\DeclareMathOperator{\Monoid}{Monoid}     % Kategorie der Monoide
\DeclareMathOperator{\HFunc}{H}			  % Hom-Functor
\DeclareMathOperator{\El}{El}			  % elements of set
%\DeclareMathOperator{\inverse}[1]{#1^{-1}} 	  % inverse of something
\newcommand{\inv}[1]{#1^{-1}} 	  % inverse of something
\undef{\id}
\DeclareMathOperator{\id}{Id}			 % Identity

% better emptyset symbol
\let\oldemptyset\emptyset
\let\emptyset\varnothing

% left and right ``cosets''
\newcommand*{\lnkset}[2]{#1 \slash #2}
\newcommand*{\rnkset}[2]{#1 \backslash #2}

% tikzcd stuff
\newcommand{\nattrans}[2]{\arrow[#1, #2]} % natural transformation arrow
%\newcommand{\nattrans}[3]{\arrow[#1, #2 , "#3"]} % natural transformation arrow
\newcommand{\arr}[2]{\arrow[#1, "#2"}
%\DeclareMathOperator{\arrtext}[4]{\begin{tikzcd}[cramped, sep=small] % Not working :(
%		#1 \arrow[#2, #3] & #4
%\end{tikzcd}}