% !TEX root = KAT.tex
% This work is licensed under the Creative Commons
% Attribution-NonCommercial-ShareAlike 4.0 International License. To view a copy
% of this license, visit http://creativecommons.org/licenses/by-nc-sa/4.0/ or
% send a letter to Creative Commons, PO Box 1866, Mountain View, CA 94042, USA.

\chapter{Kategorien: Definition und Beispiele}
\section{Zur Motivation}
Abstraktion in der Mathematik
\clearpage
\setlength{\voffset}{+45mm}
\pagestyle{empty} % disable pagenumbering
\begin{landscape}
\begin{tabular}{c||c|c|c|c}
	\makecell{konkrete\\ Objekte}
	&\makecell{Alle Teilmengen\\einer Menge $X$\\($\Potenzmenge{X}$ Potenzmenge)}
	&\makecell{Alle Unterobjekte\\eines math. Objektes ($\Sub X$)}
	&\makecell{Alle Automorphismen (Symmetrien)\\eines math. "Objektes" $X$\\($\Aut X$)}
	&\makecell{Alle Homomorphismen\\ zwischen math. Obj.\\ einer Familie $\X$}\\
	\hline\hline
	\makecell{konkrete\\ math. Struktur}
	&\makecell{Rechnen mit Teilmengen\\$\cap,\cup,\overline{\cdot},\emptyset,X$}
	&\makecell{Teilmengenbeziehung\\ "$\subseteq$"}
	&\makecell{Komposition von\\Permutationen\\
	\begin{tikzcd}[ampersand replacement=\&]
		X \arrow{r}{f} \arrow[bend right]{rr}{fg} \& X \arrow{r}{g} \& X
  	\end{tikzcd}
	}
	&\makecell{Komposition von\\ Homomorphismen\\
	\begin{tikzcd}[ampersand replacement=\&]      
		A \arrow{r}{f} \arrow{d}{fg} \& B \arrow{d}{gh} \arrow{ld}{g} \\
		C \arrow{r}{h}                  \& D                         
	\end{tikzcd}
	}\\
	\hline
	\makecell{Eigenschaften}
	&\makecell{Axiome für\\Mengenoperationen\\(z.B. $\cap$ ass., kommu.}
	&\makecell{reflexiv,\\transitiv,\\ antisymmetrisch}
	&\makecell{\ul{Ass.:} $(fg)h=f(gh)$\\\ul{Inv.:} $f^{-1}f=\id_X=ff^{-1}$\\\ul{Neu.:} $f\mal\id_X=f=\id_X f$}
	&\makecell{\ul{Ass.:} $(fg)h=f(gh)$\\\ul{Neu.:} $f\mal\id_B=\id_A\mal f=f$}\\
	\hline
	\makecell{abstrakte algebr.\\Strukturen}
	&\makecell{Boolesche Algebra\\$\langle B,\wedge,\vee,\overline{\cdot},0,1\rangle$}
	&\makecell{geordnete Menge\\$\langle P,\leq\rangle$}
	&\makecell{Gruppe\\$\langle G,\cdot,^{-1},e\rangle$}
	&\makecell{Kategorie\\(siehe \ref{section:1.2})}\\
	\hline
	\makecell{Rechtfertigung\\(Korrektheit,\\Soundness)}
	&\makecell{Jede Potenzmengenalgebra\\ist Boolsche Algebra.}
	&\makecell{$\langle\Sub X,\subseteq\rangle$ ist\\geordnete Menge}
	&\makecell{$\Aut(X)$ ist Gruppe}
	&\makecell{siehe \ref{def:set}}\\
	\hline
	\makecell{Darstellungssatz\\Vollständigkeit\\Completeness}
	&\makecell{Jede Boolsche\\ Algebra ist Unteralgebra\\einer Potenzmengenalgebra\\$\langle\Potenzmenge{X},\cap,\cup,\ldots\rangle$}
	&\makecell{ja}
	&\makecell{Satz von Cayley}
	&\makecell{siehe 3.6}\\ %TODO \ref{section:3.6}
\end{tabular}\\

Ergänzungen zu obiger Tabelle:
\begin{itemize}
	\item Satz von Cayley: Jede Gruppe ist isomorph zu Permutationsgruppe
\end{itemize}
\end{landscape}
\pagestyle{headings} %enable pagenumbering
\clearpage
\setlength{\voffset}{0mm}

\section*{Definition Kategorie}\label{section:1.2}
\setcounter{definition}{1}
\begin{definition} %TODO set counter definition +1?
	Eine \define{Kategorie} $\Cat$ ist gegeben durch:\index{Kategorie}
	\begin{itemize}
		\item eine Klasse (vgl. \ref{section:0.3}
		) $\Ob(\Cat)$\\
		Die Elemente von $\Ob(\Cat)$ heißen \define{Objekte}.\index{Objekte}
		\item Für jedes Paar $(A,B)$ von Objekten $A,B\in\Ob(\C)$ gibt es eine Menge(!) $\Cat(A,B)$\\
		Die Elemente von $\Cat(A,B)$ heißen \define{Morphismen / Pfeile.} \index{Morphismus}\index{Pfeil|see{Morphismus}}
		Diese Mengen sind paarweise disjunkt, d.h.
		\begin{align*}
			\Cat(A,B)\cap\Cat(C,D)\neq\emptyset
			\implies A=C\AND B=D
		\end{align*}
	\end{itemize}
\end{definition}

\begin{notation}
	Schreibe $A\overset{a}{\longrightarrow}B$ oder $a\colon A\to B$ für $a\in\Cat(A,B)$.
\end{notation}

\begin{lem}
	Für jedes Tripel $(A,B,C)$ von Objekten $A,B,C\in\Ob(\Cat)$ gibt es eine Abbildung, 
	\begin{align*}\index{Komposition}
		\mu_{A,B,C}^\Cat\colon\Cat(A,B)\times\Cat(B,C)\to\Cat(A,C)
	\end{align*}
	die sogenannte \define{Komposition} der Morphismen $a\in\Cat(A,B)$, $b\in\Cat(B,C)$ und wird mit $a;b$ oder $ab$ bezeichnet.
	Andere Bezeichnungen: $b\circ a$ oder $ba$.
	$$
	\begin{tikzcd}
                                   & B \arrow[rd, "b"] &   \\
A \arrow[ru, "a"] \arrow[rr, "ab"] &                   & C
	\end{tikzcd}
	$$
\end{lem}

\begin{lem}
	Für jedes Objekt $A\in\Ob(\Cat)$ gibt es einen Morphismus $1_A\in\Cat(A,A)$, genannt \define{identischer Morphismus}, falls die folgenden Axiome erfüllt sind:
	\begin{enumerate}\index{identischer Morphismus}
		\item[(ASS)] Die Morphismenkomposition ist assoziativ, d.h.
		\begin{align*}
			\forall a\in\Cat(A,B),\forall b\in\Cat(B,C),\forall c\in\Cat(C,D):(ab)c=a(bc)
		\end{align*}		 
		\item[(NEU)] Die identischen Morphismen sind neutrale Elemente bzgl. Komposition, d.h. 
		\begin{align*}
			\forall a\in\Cat(A,B):1_A a=a=a 1_B
		\end{align*}
	\end{enumerate}
	
		Schreibweise:
	$$ A\overset{1_A}{\longrightarrow}A\qquad
		 \begin{tikzcd}[cells={nodes={}}]
        \arrow[loop left, distance=3em, start anchor={[yshift=-1ex]west}, end anchor={[yshift=1ex]west}]{}{1_A} \arrow{r} \bullet 
        & \bullet%\arrow[loop right, distance=3em, start anchor={[yshift=1ex]east}, end anchor={[yshift=-1ex]east}]{}{\mathrm{id}_{B\times C}} 
    \end{tikzcd}\qquad
     \begin{tikzcd}[cells={nodes={}}]
        %\arrow[loop left, distance=3em, start anchor={[yshift=-1ex]west}, end anchor={[yshift=1ex]west}]{}{1_A}  
        \arrow{r}\bullet 
        & \bullet\arrow[loop right, distance=3em, start anchor={[yshift=1ex]east}, end anchor={[yshift=-1ex]east}]{}{1_B} 
    \end{tikzcd}	
	$$
\end{lem}

\begin{defi}
	Eine Kategorie $\Cat$ heißt \define{kleine Kategorie}\index{kleine Kategorie}\index{große Kategorie}
	\begin{align*}
		:\iff Ob(\Cat)\und\Mor(\Cat):=\bigcup\limits_{A,B\in\Ob(\Cat)}\Cat(A,B)
		\text{ Mengen sind}
	\end{align*}
	Ansonsten heißt $\Cat$ \define{große Kategorie}.
	Weitere Bezeichnungen:
	$$a\in\Cat(A,B)\qquad\text{d.h.}\qquad A\overset{a}{\longrightarrow}B$$
	
	\begin{itemize}
		\item $\dom(a):=A$ \define{Quelle} oder \define{Domain} von $a$
		\index{Quelle}\index{Domain}
		\item $\cod(a):=B$ \define{Ziel} oder \define{Codomain} von $a$
	\end{itemize}
\end{defi}
\section*{Beipiele}
\begin{beispiel}[Die Kategorie \texorpdfstring{\ul{Set}}{Set}]%TODO itemize
	\label{def:set}
	Objekte: $\Ob(\Set)$ alle Mengen\\
	Morphismen: $\Set(A,B):=$ Menge aller Abbildungen $f\colon A\to B$\\
	Komposition (Hintereinanderausführung):
	$$
		\mu_{A,B,C}^{\Set}(f,g):=f;g\\
		\begin{tikzcd}[ampersand replacement=\&]
			A \arrow{r}{f} \arrow[bend right]{rr}{fg} \& B \arrow{r}{g} \& C
	  	\end{tikzcd}
	$$
	identischer Morphismus: ist identische Abbildung
	\begin{align*}
		1_A:=\id_A\colon A\to A,\qquad x\mapsto x
	\end{align*}
	(Ass), (Neu) sind erfüllt.
\end{beispiel}


\begin{beispiel}[Die Kategorie \texorpdfstring{\ul{Group}}{Group}]
	\label{def:group}
	$\Ob(\Group):=$ alle Gruppen $\langle G,\cdot,^{-1},e\rangle$\\
	$\Group(A,B):=$ Menge aller Homomorphismen der Gruppe $A$ in die Gruppe $B$
	\begin{align*}
		\mu_{A,B,C}^{\Group}(f,g):=f;g\qquad 1_A:=\id_A
	\end{align*}
	(Ass), (Neu) erfüllt.\nl
	Analog für andere algebraische Strukturen (Morphismen sind Homomorphismen)
	\begin{itemize}
		\item \ul{Ab}: Kategorie der abelschen Gruppen
		\item \ul{Ring}: Kategorie der Ringe
		\item \ul{Ring$^1$}: Kategorie der Ringe mit Einselement
		\item \ul{Monoid}: Kategorie der Monoide
		\item \ul{Alg$(\Omega)$}: Kategorie der Algebren der Signatur $\Omega$
	\end{itemize}
\end{beispiel}

\begin{beispiel}[\ul{Top}]\label{beisp1.5Top}\enter
	$\Ob(\Top):=$ topologische Räume\\
	$\Top(X,Y):=$ Menge der stetigen Abbildungen von $X$ in $X$
	\begin{align*}
		\mu_{X,Y,Z}^{\Top}(f,g):=f;g\qquad
		1_X:=\id_X
	\end{align*}
	Wieder sind die Bedingungen (Ass),(Neu) erfüllt.
\end{beispiel}


\begin{beispiel}[$\RelSet$]\label{beisp1.6RelSet}\enter
	$\Ob(\RelSet):=\Ob(\Set)$ alle Mengen\\
	$\RelSet(A,B):=\set{\rho\mid\rho\subseteq A\times B}$ alle binären Relationen zwischen $A$ und $B$\\
	Genauer: $\RelSet(A,B):=\set{A}\times\Potenzmenge{A\times B}\times\set{B}$\\
	Komposition: für $\rho\subseteq A\times B$, $\sigma\subseteq B\times C$

	$$
	\begin{tikzcd}[ampersand replacement=\&]
		A \arrow{r}{\rho} \& B \arrow{r}{\sigma} \& C
  	\end{tikzcd}
	$$
	\begin{align*}
		\mu_{A,B,C}^{\RelSet}(\rho,\sigma):=\rho;\sigma
		:=\set{(a,c)\in A\times C\mid\exists b\in B:(a,b)\in\rho,(b,c)\in\sigma}
	\end{align*}
	heißt \define{Relationsprodukt}.
	\begin{align*}
		1_A:=\Delta_A:=\set{(a,a)\mid a\in A}
	\end{align*}
	(Ass),(Neu) erfüllt
\end{beispiel}

\begin{beispiel}[Die Kategorie \ul{Mat}$_R$]\label{beisp1.7Mat}\enter
	Sei $R$ ein Ring (z.B. reelle Zahlen)\\
	$\Ob(\Mat_R):=\N_+$\\
	Morphismen:
	$\Mat_R(m,n):=$ Menge aller $(m\times n)$-Matrizen über $R$
	$$
	\begin{tikzcd}[ampersand replacement=\&]
		m \arrow{r}{A} \& n \arrow{r}{B} \& p
  	\end{tikzcd}
	$$
	\begin{align*}
		\mu_{m,n,p}(A,B):=A\mal B\qquad 1_n:=E_n\text{ Einheitsmatrix}
	\end{align*}
\end{beispiel}

\begin{beispiel}[Geordnete Mengen als Kategorie]\label{beispiel1.8}
	$\ul{P}=\langle P,\leq\rangle$ sei geordnete Menge ($\leq$ reflexiv, antisymmetrisch, transitiv)\\
	$\ul{P}_{\text{cat}}$ ("Graph von $\leq$ als Kategorie")\\
	Objekte: $\Ob(\ul{P}_{\text{cat}}):=P$\\
	Morphismen:
	\begin{tikzcd}[ampersand replacement=\&]
		A \arrow{r}{f} \& B
  	\end{tikzcd}
  	\begin{align*}
  		\ul{P}_{\text{cat}}(a,b):=\left\lbrace\begin{array}{cl}
  			\set{(a,b)},&\falls a\leq b\\
  			\emptyset, &\sonst
  		\end{array}\right.
  	\end{align*}
  	Komposition:
  	\begin{align*}
  		\mu_{a,b,c}\big((a,b),(b,c)\big):=(a,c)
  	\end{align*}
  	Dies existiert, weil $\leq$ transitiv ist.
  	\begin{align*}
  		1_a:=(a,a)\text{ (existiert, weil $\leq$ reflexiv)}
  	\end{align*}
  	Auch hier sind (Ass),(Neu) erfüllt.
\end{beispiel}

\begin{bsp}
	$\langle P,\leq\rangle$
	Zu dem Hassediagramm
	$$
	\begin{tikzcd}
                                          & d \arrow[rd, no head] &   \\
b \arrow[ru, no head] \arrow[rd, no head] &                       & c \\
                                          & a \arrow[ru, no head] &  
\end{tikzcd}
	$$
	gehört 
	$$
\begin{tikzcd} %TODO format code xD
	& \arrow[out=90,in=50,loop, distance=3em, "1_b"] b  \arrow[rd, "{(b,d)}"] &   \\
	a \arrow[out=200,in=160, distance=3em, "1_a"] \arrow[rr, "{(a,d)}"] \arrow[ru, "{(a,b)}"] \arrow[rd, "{(a,c)}"] &                         & d \arrow[loop right, distance=3em, start anchor={[yshift=1ex]east}, end anchor={[yshift=-1ex]east}]{}{1_d}\\
	& c \arrow[out=300,in=260,loop, distance=3em, "1_c"] \arrow[ru, "{(c,d)}"] & 
\end{tikzcd}
	$$
\end{bsp}

\begin{definition}[1.9 Diagrammschemata und Diagramme]\enter
	Ein \define{Diagrammschema} ist ein gerichteter Graph $\Sigma=(V,E,\alpha,\beta)$ wobei 
	\index{Diagrammschema}
	\begin{itemize}
		\item $V$ die Menge der Punkte,
		\item $E$ Menge der Pfeile
		\item $\alpha:E\to V,~p\mapsto\alpha(p)$ \define{Anfangspunkt} von Pfeil $p$
		\item $\beta:E\to V,~p\mapsto\beta(p)$ \define{Endpunkt} von Pfeil $p$
	\end{itemize}
	$$
	\begin{tikzcd}
\alpha(p) \arrow[r, "p", bend left] & \beta(p)
\end{tikzcd}
$$
\end{definition}\enter
Ist $\Cat$ eine kleine Kategorie, so lässt sich $\Cat$ auch als Diagrammschema auffassen:
\begin{align*}
	\Sigma_\Cat:=\big(\Ob(\Cat),\Mor(\Cat),\alpha,\beta\big)\\
		\begin{tikzcd}[ampersand replacement=\&]
A \arrow[r, "f", bend left] \& B
\end{tikzcd}\qquad\begin{matrix}
	\alpha(f)& :=A~(=\dom(f))\\
	\beta(f)&:=B~(=\codom(f))
\end{matrix}
\end{align*}

\begin{bsp} %TODO should be a bullet with an identity arrow.
	$$\begin{tikzcd} 
	\arrow[loop left, distance=3em, start anchor={[yshift=-1ex]west}, end anchor={[yshift=1ex]west}]{}{1_A} \bullet 
	\end{tikzcd},
	\begin{tikzcd}
	\bullet \arrow[out=100, in=0, loop, distance=3em, "b_1"] \arrow[out=300, in=260, loop, distance=3em, "b_2"]
	\end{tikzcd},
	\begin{tikzcd}
	\bullet \arrow[r] & \bullet \arrow[r, bend left] & \bullet \arrow[l] \arrow[l, bend left]
	\end{tikzcd}$$
\end{bsp}

Umgekehrt: Zu jedem Graph $\Sigma$ gibt es eine "kleinste" Kategorie (freie Kategorie von $\Sigma$ erzeugt, die $V$ als Objekte hat (evtl. kommen noch Pfeile dazu)
\begin{definition}[Diagrammschema und Diagramme]\enter
	Sei $\Cat$ beliebige Kategorie und $\Sigma=(V,E,\alpha,\beta)$ \define{Diagrammschema}.
	Ein \define{Diagramm} (in $\Cat$) vom Typ $\Sigma$ ist "verträgliche Belegung von $V$ und $E$ mit Elemente aus $\Cat$", genauer:
	Diagramm ist Abbildungspaar
	\begin{align*}
	\left.\begin{matrix}
		\varphi_0&\colon V\to\Ob(\Cat)\\
		\varphi_1&\colon E\to\Mor(\Cat)
	\end{matrix}\right\rbrace~\varphi\colon\Sigma\to\varphi
	\end{align*}
	mit der Verträglichkeitsbedingung, d.h.
	\begin{align*}
		\begin{tikzcd}[ampersand replacement=\&]
v_1 \arrow[r, "p", bend left] \& v_2
\end{tikzcd}\implies
	\begin{tikzcd}[ampersand replacement=\&]
\varphi_0(v_1) \arrow[r, "\varphi_1(p)", bend left] \& \varphi_0(v_2)
\end{tikzcd}
	\end{align*}
	(Pfeil in $\Sigma\leadsto$ Morphismus in $\Cat$)
	Formal:
	\begin{align*}
		\forall p\in E:\varphi_1(p)\in\Cat\Big(\varphi_0\big(\alpha(p)\big),\varphi_0\big(\beta(p)\big)\Big)
\end{align*}		
	
	Ein Paar $(w_1,w_2)$ von gerichteten Wegen im Diagrammschema $\Sigma=(V,E,\alpha,\beta)$ heißt \define{Kommutativitätsbedingung}, falls $w_1$ und $w_2$ gleichen Anfangs- und gleichem Endpunkt haben und nur diese beide Punkte gemeinsam haben.\index{Kommutativitätsbedingung}\nl
	%TODO add picture
	Ein Diagramm $\varphi:\Sigma\to\Cat$ \define{erfüllt} Kommutativitätsbedingung $(w_1,w_2)$
	\begin{align*}
		:\iff\varphi_1(p_1)\mal\varphi_1(p_2)\mal\ldots\mal\varphi_1(p_n)
		=\varphi_1(P_1')\mal\varphi_1(p_2')\mal\ldots\mal\varphi_1(p_m')
	\end{align*}
	Ein Diagramm $\varphi\colon\Sigma\to\Cat$ (meist ohne Schlingen und ohne Zyklen) heißt \define{kommutativ}, wenn es jede (!) Kommutativitätsbedingung des Diagrammschemas $\Sigma$ erfüllt.
\end{definition}
\begin{bsp}
	\begin{align*}
		\begin{tikzcd}[ampersand replacement=\&]
\bullet \arrow[r, "f"] \arrow[d, "f'"'] \& \bullet \arrow[d, "g"] \\
\bullet \arrow[r, "g'"']                \& \bullet               
\end{tikzcd}
\text{ kommutativ}\iff fg=f'g'\\
	\begin{tikzcd}[ampersand replacement=\&]
\bullet \arrow[r, "g"] \arrow[rd, "e"] \& \bullet \arrow[d, "h"] \\
\bullet \arrow[r, "d"'] \arrow[u, "f"] \& \bullet               
\end{tikzcd}
\text{ kommutativ}\iff \begin{matrix}
	fe=fgh\\
	gh=e\\
	fe=d\\
	fgh=d
\end{matrix}
	\end{align*}
	Schreibweise:$
	\begin{tikzcd} %TODO add loop in the center! align hier?
		\bullet \arrow[dd] \arrow[rr] &         & \bullet \arrow[dd] \\
		& \text{Loop} &                    \\
		\bullet \arrow[rr]            &         & \bullet           
	\end{tikzcd},
	\begin{tikzcd}
		\bullet \arrow[dd] \arrow[rr] &         & \bullet \arrow[dd] \\
		& = &                    \\
		\bullet \arrow[rr]            &         & \bullet           
	\end{tikzcd},
	\begin{tikzcd}
		\bullet \arrow[dd] \arrow[rr] &         & \bullet \arrow[dd] \\
		& \text{!} &                    \\
		\bullet \arrow[rr]            &         & \bullet           
	\end{tikzcd}$
\end{bsp}
