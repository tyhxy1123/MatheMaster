\chapter{Spezielle Morphismen und Objekte in Kategorien}
\begin{bemerkung}
	Es werden bekannte Eigenschaften von bekannten mathematischen Strukturen in ``Kategoriensprache'' formuliert und sind damit auf alle Kategorieren anwendbar, (auch wo vorher nicht bekannt).
\end{bemerkung}
Sei $\Cat$ Kategorie.
\begin{definition}[Isomorphismus]
	Sei $f \in \Cat(A,B)$ $(f: \begin{tikzcd}[cramped, sep=small]
	A \arrow[r] & B
	\end{tikzcd})$ heißt \define{Isomorphismus} (\ul{iso}), wenn ein $f' \in \Cat(B,A)$ existiert, sodass
	\[
		ff' = 1_A \und f' f = 1_B.
	\]
\end{definition}
\begin{bemerkung}
	$f'$ ist dadurch eindeutig bestimmt und wird mit $\inv{f'}$ bezeichnet. ($\nearrow$ SeSt)
\end{bemerkung}
\begin{definition}[Monos und Epis]\
	\begin{enumerate}[label=]
		\item Ein $f: \begin{tikzcd}[cramped, sep=small]
		A \arrow[r] & B
		\end{tikzcd}$ heißt \define{Monomorphismus} (kurz mono), wenn gilt:\\
		$\forall X \in \Ob(\Cat) \quad \forall g,g' \in \Cat(X,A)$: $gf = g'f \implies g = g'$
		\[
		\begin{tikzcd}
		X \arrow[r, bend left, "g"] \arrow[r, bend right, "{g'}"'] & A \arrow[r, "f"] & B
		\end{tikzcd}
		\]
		\item dualer Begriff:\\
		Ein $f: \begin{tikzcd}[cramped, sep=small]
		A \arrow[r] & B
		\end{tikzcd}$ heißt \define{Epimorphismus} (epi) $\forall Y \in \Ob(\Cat)\quad \forall g,g' \in \Cat(B,Y)$ gilt
		\[
		fg = fg' \implies g=g'.
		\]
	\end{enumerate}
\end{definition}
\begin{bemerkung}
	In konkreten Theorien gilt:
	\begin{align*}
		\text{injektiv} &\implies \text{ mono}\\
		\text{surjektiv } &\implies \text{ epi}.
	\end{align*}
	($\nearrow$ Übung, Umkehrung gilt nicht immer!)
\end{bemerkung}