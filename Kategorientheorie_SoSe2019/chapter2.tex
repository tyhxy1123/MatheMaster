% !TEX root = KAT.tex
% This work is licensed under the Creative Commons
% Attribution-NonCommercial-ShareAlike 4.0 International License. To view a copy
% of this license, visit http://creativecommons.org/licenses/by-nc-sa/4.0/ or
% send a letter to Creative Commons, PO Box 1866, Mountain View, CA 94042, USA.

\chapter{Kostruktionen mit Kategorien}

\begin{definition}[duale Kategorie]\enter
	Sei $\Cat$ beliebige Kategorie.
	Die \define{duale Kategorie} $\Cat^{\op}$ entsteht aus $\Cat$ durch "Umdrehen der Pfeile".
	\begin{itemize}
		\item \ul{Objekte:}
			\begin{align*}
				\Ob(\Cat^{\op})&:=\Ob(\Cat)\qquad\begin{matrix}
						A\in\Ob(\Cat)\\
						A^\partial\in\Ob(\Cat^\op)
						\end{matrix}
			\end{align*}
		\item \ul{Morphismen:}
			\begin{align*}
				\Cat^\op(A,B)&:=\Cat(A,B)\qquad
						A^\partial\overset{f^\partial}{\longrightarrow}B^\partial\text{ für }B\overset{f}{\longrightarrow}A
			\end{align*}
		\item \ul{Komposition:}
			\begin{align*}
				\mu_{A,B,C}^{\Cat^\op}(f,g)&:=\mu_{CBA}^\Cat(g,f)\qquad\\
				1_{A^\partial}^{\Cat^\op}&:=1_A^\Cat\\
				\mu_{A,B,C}^{\Cat^{\op}}(f,g) &:= \mu_{C,A,A}(g,f)\\
			\end{align*}
		\begin{align*}
				\begin{tikzcd}[ampersand replacement=\&]
					A \& B \arrow[l, "f"'] \& C \arrow[l, "g"'] \arrow[l] \arrow[l] \arrow[ll, "gf"', bend right]\\
					A^{\partial} \& B^{\partial} \arrow[l, "f^{\partial}"'] \& C^{\partial} \arrow[l, "g^{\partial}"'] \arrow[ll, "fg", bend left]
				\end{tikzcd}
			\end{align*}
		\item \ul{Identität:}
			\begin{align*}
				1_{A^{\op}}^{\op} = 1_A
			\end{align*}
	\end{itemize}
\end{definition}

\begin{beispiel}
	\begin{align*}
	\Cat: 
	\begin{tikzcd}[ampersand replacement=\&]
	\bullet \arrow[out=200,in=160, distance=3em] \arrow[rr, bend left] \arrow[rdd] \&         \& \bullet \arrow[loop right, distance=3em, start anchor={[yshift=1ex]east}, end anchor={[yshift=-1ex]east}]{}{}\arrow[ll, bend left] \arrow[ldd] \\
	                                          \&         \&                                           \\
	                                          \& \bullet \arrow[out=300,in=260,loop, distance=3em] \&                                          
	\end{tikzcd}
	\Cat^{\op}:
	\begin{tikzcd}[ampersand replacement=\&] %TODO fix the orientation of the one identity arrow!
	\bullet \arrow[out=160,in=200, distance=3em] \arrow[rr, bend right] \&                                 \& \bullet \arrow[loop left, distance=3em, start anchor={[yshift=1ex]east}, end anchor={[yshift=-1ex]east}]{}{} \arrow[ll, bend right] \\
	                               \&                                 \&                                \\
	                               \& \bullet \arrow[out=260,in=300,loop, distance=3em] \arrow[luu] \arrow[ruu] \&                               
	\end{tikzcd}
	\end{align*}
\end{beispiel}
Ist $\aussage$ eine Aussage über Objekte und Morphismen in einer Kategorie $\Cat$, so erhält man die \betone{duale Aussage} $\aussage^{\op}$, wenn man Quelle und Ziel der Morphismen vertauscht und für jede in $\aussage$ vorkommende Morphismenkomposition $fg$ die Komposition $gf$ schreibt (d.h. man formuliert $\aussage$ in der dualen Kategorie $\Cat^{\op}$ und ersetzt diese Aussage als Aussage in $\Cat$.)\\
z.B.:
\begin{align*}
	\aussage:
	\begin{tikzcd}[ampersand replacement=\&]
		A \arrow[r, "f"] \& B
	\end{tikzcd}
	\aussage^{\op}:
	\begin{tikzcd}[ampersand replacement=\&]
		 A^{\partial} \arrow[r, "f^{\partial}"] \& B^{\partial}
	\end{tikzcd}
	\aussage^{\op}:
	\begin{tikzcd}[ampersand replacement=\&]
		A \& B \arrow[l, "f"']
	\end{tikzcd}
\end{align*}

\begin{folg}\enter
	Zu jedem Begriff gibt es einen dualen Begriff und das nennt man \betone{Dualitätsprinzip:}
	\begin{align*}
		\Cat \Vdash \aussage \Longrightarrow \Cat^{\op} \Vdash \aussage^{\op}\\
		\intertext{Ausserdem gilt:}
		(\aussage^{\op})^{\op} = \aussage \und (\Cat^{op})^{\op} = \Cat
	\end{align*}
	Insbesondere folgt daraus:\\
	Gilt eine Aussage in allen Kategorien (bzw. folgt aus gewissen gegebenen Aussagen) gilt auch die duale Aussage in allen Kategorien (bzw folgt aus den dualen gegebenen Aussagen)
\end{folg}

\begin{bemerkung}
	Die Kategorienaxiome sind selbstdual!
\end{bemerkung}

\begin{definition}[Produktkategorie]\enter
	Seien $\CatA, \CatB$ Kategorien, dann definiere die Produktkategorie durch $\CatA \times \CatB$.
\end{definition}
Nun müssen die Kategorieaxiome überprüft werden.
	\begin{itemize}
		\item \ul{Objekte:}
		\begin{align*}
			\Ob(\CatA \times \CatB) &:= \Ob(\CatA) \times \Ob(\CatB)\\
			&= \set{(A,B) \mid A \in \Ob(\CatA), B \in \Ob(\CatB)}
		\end{align*}
		\item \ul{Morphismen:}
		\begin{align*}
			&(\CatA \times \CatB)(A_1 , B_1(A_2 , B_2)\\
			&:= \CatA(A_1 , A_2) \times \CatB(B_1 , B_2)\\
			&= \set{(f,g) \mid 
			\begin{tikzcd}[ampersand replacement=\&]
			A_1 \arrow[r, "f"] \& A_2
			\end{tikzcd},
			\begin{tikzcd}[ampersand replacement=\&]
			B_1 \arrow[r, "g"] \& B_2
			\end{tikzcd}
			}
		\end{align*}
		\item \ul{Identität:} 
		\begin{align*}
			1_{(A,B)} := (1_A , 1_B)
		\end{align*}
		\item \ul{Morphismenkomposition:} erfolgt komponentenweise,
			\begin{align*}
				(f_1 ,g_1)(f_2 , g_2) := (f_1 f_2 , g_1 g_2)
			\end{align*}
	\end{itemize}
\begin{definition}[Unterkategorie]\enter
	Eine Kategorie $\CatB$ heißt Unterkategorie von $\CatA$, wenn für alle $A,B \in \Ob(\CatB) \und f \in \CatB(A,B), g\in \CatB(B,C)$ gilt:
	\begin{enumerate}
	\item $\Ob(\CatB) \subseteq \Ob(\CatA)$ \label{def:Unterkat1} \tag{Objektuntermenge}
	\item $\CatB(A,B) \subseteq \CatA(A,B)$ \label{def:Unterkat2} \tag{Morphismusuntermenge}
	\item $\mu_{A,B,C}^{\CatB}(f,g) = \mu_{A,B,C}^{\CatA}(f,g)$ \label{def:Unterkat3} \tag{Komposition}
	\item $1_A^{\CatB} = 1_A^{\CatB}$ \label{def:Unterkat4} \tag{Identität}
	\end{enumerate}
\end{definition}
\begin{bemerkung}
	\ref{def:Unterkat4} folgt nicht aus \ref{def:Unterkat1} und \ref{def:Unterkat3}!
\end{bemerkung}
\begin{notation}
	$\CatB \leqq \CatA$
\end{notation}
\begin{bsp}
	$\Ab \leqq \Group$ (vgl. \ref{def:group}) ist eine volle Unterkategorie.
\end{bsp}
$\CatB$ heißt \define{volle Unterkategorie} von $\CatA$, falls gilt \\
2' Es gilt $\CatB(A,B) = \CatA(A,B) \quad \forall A,B \in \Ob(\CatB)$.
\begin{definition}[Summe von Kategorien]\enter
	Seien $\CatA, \CatB, \Ob(\CatA) \cap \Ob(\CatB) = \emptyset$. Die \define{Summe von Kategorien} $\CatA + \CatB$ ist gegeben durch:
	\begin{itemize}
		\item \ul{Objekte:} $\Ob(\CatA + \CatB) = \Ob(\CatA) \bigcupdot \Ob(\CatB)$
		\item \ul{Morphismen:} $(\CatA + \CatB)(A,B) := 
		\begin{cases}
			\CatA(A,B) \quad &\text{ falls } A,B \in \Ob(\CatA)\\
			\CatB(A,B) \quad &\text{ falls } A,B \in \Ob(\CatB)\\
			\emptyset \quad  &\sonst 
		\end{cases}$ 
		\item \ul{Komposition:} $f,g \in \Mor(\CatA)$
		\begin{align*}
			\mu_{A,B,C}^{\CatA + \CatB} (f,g) = 
			\begin{cases}
				\mu_{A,B,C}^{\CatA} \quad &\text{ falls } A,B,C \in \Ob{\CatA}\\
				\mu_{A,B,C}^{\CatB} \quad &\text{ falls } A,B,C \in \Ob{\CatA}\\
				\emptyset \quad & \sonst
			\end{cases}
		\end{align*}
		\item \ul{Identität:}
		\begin{align*}
		1_{A,}^{\CatA + \CatB} = 
		\begin{cases}
		1_{A}^{\CatA} \quad &\text{ falls } A \in \Ob{\CatA}\\
		1_{B}^{\CatB} \quad &\text{ falls } B \in \Ob{\CatB}
		\end{cases}
		\end{align*}
	\end{itemize}
\end{definition}
\begin{definition}[Quotienten- oder Faktorkategorie]\enter
	Sei $\Cat$ Kategorie. Für jedes Paar von Objekten $(A,B)$ sei Äquivalenzrelation
	\begin{align*}
		\rho_{A,B} \subseteq \Cat(A,B) \times \Cat(A,B)
	\end{align*}
	auf der Menge der Morphismen $\Cat(A,B)$ gegeben. Es gelte \define{Kompatibilitätsbedingung}:
	\begin{align*}
		(f,f') \in \rho_{A,B} \und (g,g') \in \rho_{B,C}\\
		\implies (fg,f' g') \in \rho_{A,C},
	\end{align*}
	dann ist die \define{Faktorkategorie} definiert als
	\begin{align*}
		\lnkset{\Cat}{\rho} \quad (\Cat\text{ nach } \rho)\\
		(\rho:= (\rho_{A,B})_{A,B \in \Ob(\Cat)})
	\end{align*}
	\begin{itemize}
		\item \ul{Objekte:} $\Ob(\lnkset{\Cat}{\rho}) := \Ob(\Cat)$\\
		$\klammern{\lnkset{\Cat}{\rho}}(A,B) := \klammern{\lnkset{\Cat(A,B)}{\rho_{A,B}}}\quad$ Äquivalenzrelation
		\item \ul{Komposition:} $[f]_{\rho_{A,B}} := [fg]_{\rho_{A,C}}$ (Wohldefiniert?)
		\item \ul{Identität:} $1_{A}^{\lnkset{\Cat}{\rho}} = [1_A^{\Cat}]_{\rho_{A,A}}$
	\end{itemize}
\end{definition}
\begin{definition}[Komma-Kategorie]\enter
	Sei $\Cat, K \in \Ob(\Cat)$ Objekt fixiert. Die \define{Komma-Kategorie} oder auch ``Objekt unter $K$'' (notiert mit $(K \downarrow \Cat)$) ist gegeben durch 
	\begin{itemize}
		\item \ul{Objekte:}
		\begin{align*}
			(K \downarrow \Cat)&:= \bigcup_{A \in \Ob(\Cat)}\Cat(K,A)\\
			&= \set{(f,A)\mid f: K \to A \in \Cat}\\
			\begin{tikzcd}
				K \arrow[d, "f"'] \\ A 
			\end{tikzcd} 
			\text{ ``Pfeile von $K$ in $\Cat$ hinein''.}
		\end{align*}
		(Die Bezeichnung $(f,A)$ heißt auch \betone{$K$-wertiger Punkt} von $A$.)
		\item \ul{Morphismen:}
		\begin{align*}
			(K \downarrow \Cat)((f,A),(g,B)) := \set{K \in \Cat(A,B) \mit fK = g}
		\end{align*}
		\item \ul{Komposition:}
		\begin{align*}
			\mu_{(f,A),(g,B),(h,C)}(k,k'):= Kh' \in \Cat = \mu_{A,B,C}^{\Cat}(k,k')\\
			\begin{tikzcd}[ampersand replacement=\&]
				\& K \arrow[dl, "f"'] \arrow[d, "g"] \arrow[dr, "h"] \\
				A \arrow[r, "k"'] \& B \arrow[r, "k'"'] \& C
			\end{tikzcd}
		\end{align*}
		\item \ul{Identität:}
		\begin{align*}
			1_{(f,A)}^{(K\downarrow \Cat)} := 1_{A}^{\Cat} \qquad
			\begin{tikzcd}[ampersand replacement=\&]
				K \arrow[d, "f"'] \arrow[dr, "f"] \\
				A \arrow[r, "1_A"'] \& A
			\end{tikzcd}
		\end{align*}
	\end{itemize}
	Die Kategorie Axiome sind erfüllt!
\end{definition}
\begin{bsp}
	Sei K: 1-elementige Menge $\set{\ast}$ in $\Set$
	\begin{align*}
		(\set{\ast} \downarrow \Set) := \text{ Kategorie der Mengen mit ``ausgezeichneten Punkt''}\\
		(f: \set{\ast} \to A \text{ durch Konstante } f(\ast) \in A \text{ gegeben})
	\end{align*}
\end{bsp}
\begin{folg}
	Der duale Begriff von $(K \downarrow \Cat)$, der Komma-Kategorie ist $(\Cat \downarrow K)$ (Objekte über $K$): 
	\begin{align*}
		\Cat \downarrow K := (K^{\dual} \downarrow \Cat^{\dual}
	\end{align*}
	\begin{itemize}
		\item \ul{Objekte:}
		\begin{align*}
			\begin{tikzcd}
				A \arrow[d, "f"'] \\ K,
			\end{tikzcd} 
		\end{align*}
		wird auch mit $(A,f)$ ``$= (f^{\dual}, A^{\dual})$'' bezeichnet und wird $A$-wertige Punkte von $K$ genannt. Das heißt
		\begin{align*}
			\Ob(\Cat \downarrow K) &= \bigcup_{A \in \Ob(\Cat)}(A,K) =: \El(K)\\
			&=: \text{ \betone{Klasse} der verallgemeinerten \betone{Elemente} von } K 
		\end{align*}
		\item \ul{Morphismen:} $K \in (\Cat \downarrow K)((B,g),(A,f)$ (d.h. $K \in \Cat(B,A)$)
		\begin{align*}
			\begin{tikzcd}[ampersand replacement=\&]
				A \arrow[r, "k"] \& B\\
				K \arrow[u, "f"'] \arrow[ur, "g"'] 
			\end{tikzcd} \qquad \mit gK = f
		\end{align*}
		\item \ul{Identität:}
		\begin{align*}
			\begin{tikzcd}[ampersand replacement=\&]
				A \arrow[d, "f"'] \arrow[r, "1_A"'] \& A \arrow[dl, "f"] \\
				K 
			\end{tikzcd}
			\qquad \mit 1_{A,f} = 1_A
		\end{align*}
	\end{itemize}
\end{folg}
\begin{bemerkung}[zum Begriff verallgemeinerter Elemente]\enter
	\begin{enumerate}[label=] %TODO Write better german here!
		\item Frage:  Was ist in $\Set$ eine $A$-wertige ($=$ 1-wertige) Punkte von Menge $K$, wenn $A = \set{1}$ ist?
		\item Antwort: $\Cat(A,K), f: \set{1} \to A=$ Elemente von $K$ und $\Cat(A,K) = $ Menge $K$ %TODO vverfiy this! not sure about it.
		\item Sei nun $A = \set{1,\dots, n}$ $n$-wertige Punkte von $K$, was an an $n$-Tupel erinnert, da
		\begin{align*}
			\Cat(A,K) \cong K^n \mit f: \set{1, \dots, n} \to K
		\end{align*}
		gilt.
	\end{enumerate}
\end{bemerkung}