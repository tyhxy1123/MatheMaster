% !TEX root = KAT.tex
% This work is licensed under the Creative Commons
% Attribution-NonCommercial-ShareAlike 4.0 International License. To view a copy
% of this license, visit http://creativecommons.org/licenses/by-nc-sa/4.0/ or
% send a letter to Creative Commons, PO Box 1866, Mountain View, CA 94042, USA.

\chapter{Vorbemerkungen}
\section{Was ist Kategorientheorie?}
Was ist Kategorientheorie?\nl
Historisch:
1945 Eilenberg / McLane: Erstmaliges Auftauchen des Begriffs
$$\begin{tikzcd}
                                   & Y \arrow[rd, "g"] &   \\
X \arrow[ru, "f"] \arrow[rr, "fg"] &                   & Z
			\end{tikzcd}
$$

\section{Mathematische Vorüberlegungen}
Kenntnis der Mengensprache
\begin{align*}
	a\in B,\qquad f\colon A\to B\\
	\cap,\cup,A\setminus B\\
	A\times B=\set[\big]{(a,b)\mid a\in A,b\in B}
\end{align*}
Eine (binäre) Relation ist $\rho\subseteq A\times B$.
Man sollte außerdem kennen:
\begin{itemize}
	\item Äquivalenzrelation (reflexiv, transitiv, symmetrisch)
	\item Ordnungsrelation %TODO
	\item einige algebraische Strukturen
\end{itemize}

\section{Probleme mit der Mengenlehre}
Naiver (unreflektierter) Umgang mit Mengen führt zu Widersprüchen (Antinomie Paradoxon), z.B.
\begin{align*}
	M:=\set[\big]{X\mid X\text{ Menge und }X\not\in X}
\end{align*}
ist die Menge aller Mengen, die sich selbst nicht als Element enthalten.\\
Gilt $M\in M$?
Nein, denn für $X=M$ folgte $M\not\in M$.\\
Gilt $M\not\in M$? Nein, da dann $M$ ein solches $X\in M$ wäre, d.h. $M\in M$.
Widerspruch.\\
Ein möglicher Ausweg ist: alles heißt \define{Klasse}, aber Unterteilung in
\begin{itemize}
	\item \define{Mengen} ("kleine Mengen"): diese dürfen Elemente von anderen Klassen sein
	\item \define{(echte) Klasse} ("große Klasse"): dürfen nicht Elemente einer Klasse sein
\end{itemize}

\begin{definition} %nonumber
	$X$ heißt \define{Menge}\index{Menge}\index{Klasse}
	\begin{align*}
		:\iff\exists\text{ Klasse }Y:X\in Y
	\end{align*}
\end{definition}

Dies wurde axiomatisiert in der Zermealo-Fraenkel-Mengenlehre.\nl
Anderer Ausweg: \undefine{Universa}
\begin{definition}
	Ein \define{Universum} ist eine Menge $U$ mit folgenden Eigenschaften:\index{Universum}
	\begin{enumerate}[label=(\roman*)]
		\item $\begin{aligned}
			X\in Y\in U\implies X\in U
		\end{aligned}$
		\item $\begin{aligned}
			X,Y\in U\implies\set{X,Y}\in U
		\end{aligned}$
		\item $\begin{aligned}
			X,Y\in U\implies X\times Y\in U
		\end{aligned}$
		\item $\begin{aligned}
			X\in U\implies\Potenzmenge{X}\in U
		\end{aligned}$
		\item $\begin{aligned}
			X\in U\implies \bigcup X\in U\qquad\mit \bigcup X:=\bigcup\limits_{A\in X}A
		\end{aligned}$
		\item $\begin{aligned}
			\omega\in U
		\end{aligned}$ mit $\omega:=\N_0$
		\item $\begin{aligned}
			X\in U,~f\colon X\to U\text{ Funktion}\implies\rg(f):=\set[\big]{f(x)\mid x\in X}\in U
		\end{aligned}$
	\end{enumerate}
\end{definition}

\begin{axiom}
	Jede Menge ist in einem Universum enthalten (als Element).
\end{axiom}

\begin{definition}
	\begin{align*}\index{kleine Menge}\index{große Menge}
		A\text{ ist }\define{kleine Menge} &:\iff A\in U\\
		A\text{ ist }\define{große Menge} &:\iff A\not\in U\\
	\end{align*}
\end{definition}

Für diese Vorlesung: Wir stellen uns auf den naiven Standpunkt
"es wird schon gutgehen".


