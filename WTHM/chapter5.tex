% This work is licensed under the Creative Commons
% Attribution-NonCommercial-ShareAlike 4.0 International License. To view a copy
% of this license, visit http://creativecommons.org/licenses/by-nc-sa/4.0/ or
% send a letter to Creative Commons, PO Box 1866, Mountain View, CA 94042, USA.

\chapter{Martingalungleichungen und Rückwärtsmartingale} %5
\section{Martingalungleichungen} %5.1
\begin{itemize}
\item Doobs Maximalungleichung
\item Doobs $L_p$-Ungleichung
\item Azumas Ungleichung (Beispiel für \textit{Konzentrationsungleichung})
\end{itemize}

\setcounter{section}{5} %fix numbering
\begin{theorem}[Doobs Maximalungleichung]\label{theorem5.1DoobMaximalungleichung}\enter
Sei $(M_n)_{n\in\N}$ (Sub-)Martingal. Dann gilt:
\begin{align*}
\P\left(\max\limits_{j\leq n} M_j\geq r\right)\leq\frac{1}{r}\cdot\E\left[M_n^+\right]\qquad\forall n\in\N,\forall r\geq0
\end{align*}
Hierbei hängt die linke Seite vom gesamten Pfad ab, aber die rechte Seite hängt nur vom Endwert ab.
\end{theorem}
\begin{bemerkung}\
\begin{enumerate}[label=(\alph*)]
\item $(M_n)_{n\in\N}$ Martingal $\implies\big(|M_n|\big)_{n\in\N}$ Submartingal
\begin{align*}
\implies\P\left(\max\limits_{j\leq n} |M_j|\geq r\right)\leq\frac{\E\big[|M_n|\big]}{r}
\end{align*}
\item Es gilt die Zwischenabschätzung
\begin{align*}
\P\left(\max\limits_{j\leq n} M_j\geq r\right)
\leq
\frac{1}{r}\cdot\E\left[M_n\cdot\indi_{\left\lbrace\max\limits_{j\leq n} M_j\geq r\right\rbrace}\right]
\leq
\frac{1}{r}\cdot\E\left[M_n^+\right]
\end{align*}
\end{enumerate}
\end{bemerkung}

\begin{proof}
$(M_n)_{n\in\N}$ sei Submartingal. Für $k\leq n,A\in\F_k$ gilt wegen der Submartingaleigenschaft:
\begin{align}
M_k&\leq\E[M_n~|~\F_k]\nonumber\\
\implies\indi_A\cdot M_k &\leq\E\big[\indi_A\cdot M_n~\big|~\F_k\big]\nonumber\\
\implies\E[\indi_A\cdot M_k]&\leq\E[\indi_A\cdot M_n]\label{eqProof5.1Stern}\tag{$\ast$}
\end{align}
Definiere die Stoppzeit
\begin{align*}
\tau:=\min\big\lbrace n\in\N_0:M_n\geq r\big\rbrace.
\end{align*}
Es gilt
\begin{align*}
\tau\leq n&\implies \max\limits_{j\leq n} M_j\geq r\\
\text{ und } \max\limits_{j\leq n} M_j\geq r&\implies\tau\leq n
\end{align*}
d.h.
\begin{align}\label{eqProof5.1SternStern}\tag{$\ast\ast$}
\lbrace\tau\leq n\rbrace=\left\lbrace\max\limits_{j\leq n} M_j\geq r\right\rbrace
\end{align}
Somit
\begin{align*}
r\cdot\indi_{\lbrace\tau\leq n\rbrace}
&\leq M_\tau\cdot\indi_{\lbrace\tau\leq n\rbrace}
=\sum\limits_{k=0}^n M_k\cdot\indi_{\lbrace\tau=k\rbrace}
\end{align*}
Bilde Erwartungswerte:
\begin{align*}
r\cdot\P[\tau\leq n]
&\leq\sum\limits_{k=0}^n\E\Big[M_k\cdot\indi_{\underbrace{\lbrace\tau=k\rbrace}_{\in\F_k}}\Big]\\
&\stackrel{\eqref{eqProof5.1Stern}}{\leq}
\sum\limits_{k=0}^n\E\Big[M_n\cdot\indi_{\lbrace\tau=k\rbrace}\Big]\\
&=\E\Big[M_n\cdot\indi_{\lbrace\tau\leq n\rbrace}\Big]\\
\stackrel{\eqref{eqProof5.1SternStern}}{\implies}
r\cdot\P\left(\max\limits_{j\leq n} M_j\geq r\right)
&\leq\E\Big[M_n\cdot\indi_{\left\lbrace\max\limits_{j\leq n} M_j\geq r\right\rbrace}\Big]
\leq\E\big[M_n^+\big]
\end{align*}
\end{proof}

\begin{theorem}[$L_p$-Ungleichung]\label{theorem5.2LpUngleichung}\enter
Sei $(M_n)_{n\in\N}$ Martingal (oder positives Submartingal) und $p>1$. Dann gilt:
\begin{align*}
\big\Vert M_n^\ast\big\Vert\leq\frac{p}{p-1}\cdot\big\Vert M_n\big\Vert_p\qquad\forall n\in\N_0\qquad\mit M_n^\ast:=\max\limits_{j\leq n}|M_j|
\end{align*}
\end{theorem}
