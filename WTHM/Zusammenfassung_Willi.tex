\documentclass[12pt]{scrartcl}
\usepackage{palatino,setspace,fancyhdr}
\usepackage[left=10mm,right=10mm,top=25mm,bottom=25mm]{geometry}
\onehalfspacing
\pagestyle{fancy}
\chead{Zusammenfassung WTHM}
\lfoot{Version: \today}
\rfoot{Seite \thepage}
\lhead{}
\rhead{Willi Sontopski}

% This work is licensed under the Creative Commons
% Attribution-NonCommercial-ShareAlike 4.0 International License. To view a copy
% of this license, visit http://creativecommons.org/licenses/by-nc-sa/4.0/ or
% send a letter to Creative Commons, PO Box 1866, Mountain View, CA 94042, USA.

% PACKAGES
\usepackage[english, ngerman]{babel}	% Paket für Sprachselektion, in diesem Fall für deutsches Datum etc
\usepackage[utf8]{inputenc}	% Paket für Umlaute; verwende utf8 Kodierung in TexWorks 
\usepackage[T1]{fontenc} % ö,ü,ä werden richtig kodiert
\usepackage{amsmath} % wichtig für align-Umgebung
\usepackage{amssymb} % wichtig für \mathbb{} usw.
\usepackage{amsthm} % damit kann man eigene Theorem-Umgebungen definieren, proof-Umgebungen, etc.
\usepackage{mathrsfs} % für \mathscr
\usepackage[backref]{hyperref} % Inhaltsverzeichnis und \ref-Befehle werden in der PDF-klickbar
\usepackage{graphicx}
\usepackage{grffile}
\usepackage{setspace} % wichtig für Lesbarkeit. Schöne Zeilenabstände
\usepackage{enumitem} % für custom Liste mit default Buchstaben
\usepackage{ulem} % für bessere Unterstreichung
\usepackage{contour} % für bessere Unterstreichung
\usepackage{epigraph} % für das coole Zitat
\usepackage{float}            % figure-Umgebungen besser positionieren
\usepackage{xfrac}
\usepackage{xr} % man Referenzen aus anderen teX-files importieren und darauf verlinken
\usepackage{bbm} %sorgt für Symbol für Indikatorfunktion
\usepackage{color} % bringt Farbe ins Spiel
\usepackage{pdflscape} % damit kann man einzelne Seiten ins Querformat drehen
\usepackage{aligned-overset} % besseres Einrücken, siehe: https://tex.stackexchange.com/questions/257529/overset-and-align-environment-how-to-get-correct-alignment
\usepackage{pgfplots}
	\pgfplotsset{compat=newest}

\usepackage[
    type={CC},
    modifier={by-nc-sa},
    version={4.0},
]{doclicense} % für CC Lizenz-Vermerk

\usepackage{tikz}
	\usepackage{tikz-qtree}
	\usetikzlibrary{arrows}
	\usetikzlibrary{automata}
  \usetikzlibrary{babel}
	\usetikzlibrary{calc}
	\usetikzlibrary{cd}
	\usetikzlibrary{fit}
	\usetikzlibrary{matrix}
	\usetikzlibrary{positioning}
	\usetikzlibrary{shapes.geometric}
	
\usepackage{csquotes}
	\MakeOuterQuote{"}

\usepackage{xargs} % for multiple optional args in newcommand
\usepackage{lmodern} % provides a bigger set of font sizes
\usepackage{anyfontsize} % supports fallback scaling for non-existing font size
\usepackage{scrhack} % provides a hack for deprecated float environments used by some libs

% Ich habe gelesen, dass man folgendes Package zuletzt einbinden soll:
\usepackage[english, ngerman, capitalise]{cleveref} % bessere Verweise

% THEOREM-ENVIRONMENTS

\newtheoremstyle{mystyle}
  {20pt}   % ABOVESPACE \topsep is default, 20pt looks nice
  {20pt}   % BELOWSPACE \topsep is default, 20pt looks nice
  {\normalfont} % BODYFONT
  {0pt}       % INDENT (empty value is the same as 0pt)
  {\bfseries} % HEADFONT
  {}          % HEADPUNCT (if needed)
  {5pt plus 1pt minus 1pt} % HEADSPACE
	{}          % CUSTOM-HEAD-SPEC
\theoremstyle{mystyle}

% Definitionen der Satz, Lemma... - Umgebungen. Der Zähler von "satz" ist dem "section"-Zähler untergeordnet, alle weiteren Umgebungen bedienen sich des satz-Zählers.
\newtheorem{satz}{Satz}[section]
\newtheorem{lemma}[satz]{Lemma}
\newtheorem{korollar}[satz]{Korollar}
\newtheorem{proposition}[satz]{Proposition}
\newtheorem{beispiel}[satz]{Beispiel}
\newtheorem{definition}[satz]{Definition}
<<<<<<< HEAD
\newtheorem{bemerkungnr}[satz]{Bemerkung}
=======
\newtheorem{theorem}[satz]{Theorem}
>>>>>>> 31c1b8aef8833910045abfa6196d99bf06dfe6b5

% Bemerkungen, Erinnerungen und Notationshinweise werden ohne Numerierungen dargestellt.
\newtheorem*{bemerkung}{Bemerkung.}
\newtheorem*{erinnerung}{Erinnerung.}
\newtheorem*{notation}{Notation.}
\newtheorem*{beisp}{Beispiel.} %Beispiel ohne Nummerierung
<<<<<<< HEAD

=======
\newtheorem*{defi}{Definition.} %Definition ohne Nummerierung
>>>>>>> 31c1b8aef8833910045abfa6196d99bf06dfe6b5

% SHORTCUTS
\newcommand{\R}{\mathbb{R}}				 % reelle Zahlen
\newcommand{\Rn}{\R^n}						 % der R^n
\newcommand{\N}{\mathbb{N}}				 % natürliche Zahlen
\newcommand{\Z}{\mathbb{Z}}				 % ganze Zahlen
\newcommand{\C}{\mathbb{C}}			   % komplexe Zahlen
\renewcommand{\mit}{\text{ mit }}   % mit
\newcommand{\falls}{\text{falls }} % falls
\renewcommand{\d}{\text{ d}}        % Differential d

% ETWAS SPEZIELLERE ZEICHEN
% disjunkte Vereinigung
\newcommand{\bigcupdot}{
	\mathop{\vphantom{\bigcup}\mathpalette\setbigcupdot\cdot}\displaylimits
}
\newcommand{\setbigcupdot}[2]{\ooalign{\hfil$#1\bigcup$\hfil\cr\hfil$#2$\hfil\cr\cr}}
% großes Kreuz
\newcommand*{\bigtimes}{\mathop{\raisebox{-.5ex}{\hbox{\huge{$\times$}}}}} 

% WHITESPACE COMMANDS
% nicht restriktiver newline command
\newcommand{\enter}{$ $\newline} 
% praktischer Tabulator
\newcommand\tab[1][1cm]{\hspace*{#1}}

% TEXT ÜBER ZEICHEN
\newcommand{\stackeq}[1]{\stackrel{#1}{=}} 

% UNDERLINE
% besseres underline 
\renewcommand{\ULdepth}{1pt}
\contourlength{0.5pt}
\newcommand{\ul}[1]{
	\uline{\phantom{#1}}\llap{\contour{white}{#1}}
}


% This work is licensed under the Creative Commons
% Attribution-NonCommercial-ShareAlike 4.0 International License. To view a copy
% of this license, visit http://creativecommons.org/licenses/by-nc-sa/4.0/ or
% send a letter to Creative Commons, PO Box 1866, Mountain View, CA 94042, USA.

% hier noch ein paar Commands die nur ich nutze, weil ich sie mir im Laufe der Jahre angewöhnt habe und sie mir jetzt nicht abgewöhnen will:

\newcommand{\gdw}{\Leftrightarrow}             % genau dann, wenn
\DeclareMathOperator{\id}{id}                  % identische Abbildung
\DeclareMathOperator{\dom}{dom}                % Domain, Definitionsbereich
\DeclareMathOperator{\rg}{rg}                  % Rang, Bild einer Funktion, Rang einer Matrix
\newcommand{\K}{\mathbb{K}}                    % Körper
\newcommand{\limn}{\lim\limits_{n\to\infty}}   % genau dann, wenn
\newcommand{\sonst}{\text{sonst }}             % sonst
\renewcommand{\O}{\mathcal{O}}                 % Landau-O
\DeclareMathOperator{\Div}{div}                % Divergenz
\renewcommand{\div}{\Div}                      % Divergenz
\DeclareMathOperator{\spann}{span}             % Span
\DeclareMathOperator{\diam}{diam}       	   % Diameter, Durchmesser einer Menge
\DeclareMathOperator{\inner}{int}              % Das innere einer Menge




% Commands für WTHM
\newcommand{\F}{\mathcal{F}}				% Standard-Unter-Sigma-Algebra
\newcommand{\unab}{\rotatebox{90}{$\vDash$}} %Unabhänigkeit






\begin{document}
	\section{Bedingter Erwartungswert}
	\begin{itemize}
		\item Bedingter Erwartungswert als $L_2$-Projektion auf abgeschlossenen Unterraum $L_2(\Omega,\F,\P)\ni\E[X|\F]$, $\gdw\langle X-\E[X|F]\rangle=0~\forall F\in\F$; hat eindeutige stetige Fortsetzung auf $L_1$
		\item $X\in\L_2(\F)\implies\E[X|\F]=X$
		\item $\E[a\cdot X+b\cdot Y|\F]=a\cdot\E[X|\F]+b\cdot\E[Y|\F]\qquad\forall a,b\in\R$ (Linearität)
		\item $\big\langle\E[X|\F,Y\rangle=\langle X,\E[Y|\F]\rangle=\langle\E[X|\F],\E[Y|\F]\rangle$ (Symmetrie, gilt nur in $L_2$)
		\item $\E\big[\E[X|\F]\big|\mathcal{H}\big]=\E[X|\mathcal{H}]$ für $\mathcal{H}\subseteq\F\subseteq\A$ (Turmregel)
		\item $\E[Z\cdot X|\F]=Z\cdot\E[X|\F]$ für alle $Z$ beschränkt und messbar (also z.B. $Z\in L_2$) (Pull-Out-Property)
		\item $X\leq Y\implies \E[X|F]\leq\E[Y|\F]$ (Monotonie)
		\item $\big|\E[X|\F]\big|\leq\E\big[|X|\big|\F\big]$ (Dreiecksungleichung)
		\item $Y=\E[X|\F]$ f.s. $\Longleftrightarrow\forall F\in\F:\E[X\cdot\indi_F]=\E[Y\cdot\indi_F]$
		\item $\E[X]=\E\big[X\big|\lbrace\emptyset,\Omega\rbrace\big]$ und außerdem $\E\big[\E[X|\F]\big]=\E[X]~\forall\F\subseteq\A$ (totaler Erwartungswert)
		\item $B_b(\R):=\lbrace f:\R\to\R\mid f$ beschränkt und Borel-messbar$\rbrace$
		\item ZG $X$ heißt \define{unabhängig von} $\F\subseteq\A$, i.Z. $X\unab\F:\gdw\E\big[f(X)\cdot\indi_A\big]=\E\big[f(X)\big]\cdot\P(A)~\forall A\in\F,\forall f\in B_b(\R)$
		\item ZG $X$ heißt \define{unabhängig von ZG} $Y$, i.Z. $X\unab Y:\gdw X\unab\sigma(Y)\gdw\E\big[f(X)\cdot g(Y)\big]=\E\big[f(X)\big]\cdot\E\big[g(Y)\big]~\forall f,g\in B_b(\R)$
		\item $X\unab\F\implies\E[X|\F]=\E[X]$; der andere Extremfall war $X$ $\F$-messbar $\implies\E[X|\F]=X$.
		\item $\E[X|Y_1,\ldots,Y_n]:=\E\big[X\big|\sigma(Y_1,\ldots,Y_n)\big]$ "Beste Vorhersage von $X$ gegeben $Y$."
	\end{itemize}
	
	\section{Martingale}
	\begin{itemize}
		\item ist "faires Spiel" zwischen zwei Personen, bei dem keine Strategie einen systematischen Vorteil bringt
		\item Eine Familie $(\F_t)_{t\in I}$ von $\sigma$-Algebren $\F_t\subseteq\A$ heißt \textbf{Filtration} 
		$:\gdw\Big( s\leq t\implies\F_s\subseteq\F_t\Big)$
		Eine Filtration ist also eine aufsteigenden Folge von $\sigma$-Algebren.
		Interpretation: $\F_t$ beschreibt die zum Zeitpunkt $t$ verfügbare Information. Und der Informationsfluss wächst mit der Zeit an.
		Außerdem setzen wir $\F_\infty:=\bigcup\limits_{t\in I}\F_t$.
		\item Ein \textbf{stochastischer Prozess (SP)} ist eine Familie $(X_t)_{t\in I}$ von Zufallsvariablen.\\
		Ein stochastischer Prozess $(X_t)_{t\in I}$ heißt \textbf{adaptiert} an die Filtration $(\F_t)_{t\in I}$
		$:\gdw\forall t\in I: X_t$ ist messbar bzgl. $\F_t$
		\item Eine Folge $(e_n)_{n\in\N}$ heißt \textbf{vorhersehbar} bzgl. einer Filtration $(F_t)_{n\in\N}$
		$:\gdw\forall n\in\N:e_{n+1}$ ist messbar bzgl. $\F_n$
		Vorhersehbarkeit ist stärker als Adaptiertheit.
		\item Sei $X=(X_t)_{t\in I}$ SP und $(\F_t)_{t\in I}$ eine Filtration.
		$X$ heißt \textbf{Martingal} $:\gdw$
		\begin{enumerate}[label=(\alph*)]
			\item $(X_t)_{t\in I}$ ist adaptiert an $(\F_t)_{t\in I}$
			\item $\begin{aligned}
				X_t\in L_1(\Omega,\A,\P) \qquad \forall t\in I\qquad(\text{ d. h. } \E\big[|X_t|\big]<\infty)
			\end{aligned}$
			\item $\begin{aligned}
				\E[X_t~|~\F_s]=X_s\qquad\forall s,t\in I\mit s\leq t
			\end{aligned}$
		\end{enumerate}
		\item \textbf{Submartingal}: statt (c) gilt $\E[X_t~|~\F_s]\geq X_s$ (Der zukünftige Wert wird unterschätzt)
		\item \textbf{Supermartingal}: statt (c) gilt $\E[X_t~|~\F_s]\leq X_s$ (Der zukünftige Wert wird überschätzt)
		\item \textit{"Das Leben ist wie ein Supermartingal - die Erwartung fällt mit der Zeit."}
	\end{itemize}
	
	
	\section{Stoppzeiten und Stoppen von stochastischen Prozessen}
	
	
	\section{Martingalkonvergenz und gleichgradige Integrierbarkeit (ggi)}
	
	
	\section{Martingalungleichungen und Rückwärtsmartingale}
	
	
	\section{Inversion der Fouriertransformation und Eindeutigkeitssatz}
	
	
	\section{Der Stetigkeitssatz von Lévy}
	
	
	\section{Zentrale Grenzwertsätze (ZGS)}
	
	
	\section{Brownsche Bewegung}
	

	
	
\end{document}