% This work is licensed under the Creative Commons
% Attribution-NonCommercial-ShareAlike 4.0 International License. To view a copy
% of this license, visit http://creativecommons.org/licenses/by-nc-sa/4.0/ or
% send a letter to Creative Commons, PO Box 1866, Mountain View, CA 94042, USA.

\documentclass[12pt,a4paper]{article} 

% This work is licensed under the Creative Commons
% Attribution-NonCommercial-ShareAlike 4.0 International License. To view a copy
% of this license, visit http://creativecommons.org/licenses/by-nc-sa/4.0/ or
% send a letter to Creative Commons, PO Box 1866, Mountain View, CA 94042, USA.

% PACKAGES
\usepackage[english, ngerman]{babel}	% Paket für Sprachselektion, in diesem Fall für deutsches Datum etc
\usepackage[utf8]{inputenc}	% Paket für Umlaute; verwende utf8 Kodierung in TexWorks 
\usepackage[T1]{fontenc} % ö,ü,ä werden richtig kodiert
\usepackage{amsmath} % wichtig für align-Umgebung
\usepackage{amssymb} % wichtig für \mathbb{} usw.
\usepackage{amsthm} % damit kann man eigene Theorem-Umgebungen definieren, proof-Umgebungen, etc.
\usepackage{mathrsfs} % für \mathscr
\usepackage[backref]{hyperref} % Inhaltsverzeichnis und \ref-Befehle werden in der PDF-klickbar
\usepackage{graphicx}
\usepackage{grffile}
\usepackage{setspace} % wichtig für Lesbarkeit. Schöne Zeilenabstände
\usepackage{enumitem} % für custom Liste mit default Buchstaben
\usepackage{ulem} % für bessere Unterstreichung
\usepackage{contour} % für bessere Unterstreichung
\usepackage{epigraph} % für das coole Zitat
\usepackage{float}            % figure-Umgebungen besser positionieren
\usepackage{xfrac}
\usepackage{xr} % man Referenzen aus anderen teX-files importieren und darauf verlinken
\usepackage{bbm} %sorgt für Symbol für Indikatorfunktion
\usepackage{color} % bringt Farbe ins Spiel
\usepackage{pdflscape} % damit kann man einzelne Seiten ins Querformat drehen
\usepackage{aligned-overset} % besseres Einrücken, siehe: https://tex.stackexchange.com/questions/257529/overset-and-align-environment-how-to-get-correct-alignment
\usepackage{pgfplots}
	\pgfplotsset{compat=newest}

\usepackage[
    type={CC},
    modifier={by-nc-sa},
    version={4.0},
]{doclicense} % für CC Lizenz-Vermerk

\usepackage{tikz}
	\usepackage{tikz-qtree}
	\usetikzlibrary{arrows}
	\usetikzlibrary{automata}
  \usetikzlibrary{babel}
	\usetikzlibrary{calc}
	\usetikzlibrary{cd}
	\usetikzlibrary{fit}
	\usetikzlibrary{matrix}
	\usetikzlibrary{positioning}
	\usetikzlibrary{shapes.geometric}
	
\usepackage{csquotes}
	\MakeOuterQuote{"}

\usepackage{xargs} % for multiple optional args in newcommand
\usepackage{lmodern} % provides a bigger set of font sizes
\usepackage{anyfontsize} % supports fallback scaling for non-existing font size
\usepackage{scrhack} % provides a hack for deprecated float environments used by some libs

% Ich habe gelesen, dass man folgendes Package zuletzt einbinden soll:
\usepackage[english, ngerman, capitalise]{cleveref} % bessere Verweise

% THEOREM-ENVIRONMENTS

\newtheoremstyle{mystyle}
  {20pt}   % ABOVESPACE \topsep is default, 20pt looks nice
  {20pt}   % BELOWSPACE \topsep is default, 20pt looks nice
  {\normalfont} % BODYFONT
  {0pt}       % INDENT (empty value is the same as 0pt)
  {\bfseries} % HEADFONT
  {}          % HEADPUNCT (if needed)
  {5pt plus 1pt minus 1pt} % HEADSPACE
	{}          % CUSTOM-HEAD-SPEC
\theoremstyle{mystyle}

% Definitionen der Satz, Lemma... - Umgebungen. Der Zähler von "satz" ist dem "section"-Zähler untergeordnet, alle weiteren Umgebungen bedienen sich des satz-Zählers.
\newtheorem{satz}{Satz}[section]
\newtheorem{lemma}[satz]{Lemma}
\newtheorem{korollar}[satz]{Korollar}
\newtheorem{proposition}[satz]{Proposition}
\newtheorem{beispiel}[satz]{Beispiel}
\newtheorem{definition}[satz]{Definition}
<<<<<<< HEAD
\newtheorem{bemerkungnr}[satz]{Bemerkung}
=======
\newtheorem{theorem}[satz]{Theorem}
>>>>>>> 31c1b8aef8833910045abfa6196d99bf06dfe6b5

% Bemerkungen, Erinnerungen und Notationshinweise werden ohne Numerierungen dargestellt.
\newtheorem*{bemerkung}{Bemerkung.}
\newtheorem*{erinnerung}{Erinnerung.}
\newtheorem*{notation}{Notation.}
\newtheorem*{beisp}{Beispiel.} %Beispiel ohne Nummerierung
<<<<<<< HEAD

=======
\newtheorem*{defi}{Definition.} %Definition ohne Nummerierung
>>>>>>> 31c1b8aef8833910045abfa6196d99bf06dfe6b5

% SHORTCUTS
\newcommand{\R}{\mathbb{R}}				 % reelle Zahlen
\newcommand{\Rn}{\R^n}						 % der R^n
\newcommand{\N}{\mathbb{N}}				 % natürliche Zahlen
\newcommand{\Z}{\mathbb{Z}}				 % ganze Zahlen
\newcommand{\C}{\mathbb{C}}			   % komplexe Zahlen
\renewcommand{\mit}{\text{ mit }}   % mit
\newcommand{\falls}{\text{falls }} % falls
\renewcommand{\d}{\text{ d}}        % Differential d

% ETWAS SPEZIELLERE ZEICHEN
% disjunkte Vereinigung
\newcommand{\bigcupdot}{
	\mathop{\vphantom{\bigcup}\mathpalette\setbigcupdot\cdot}\displaylimits
}
\newcommand{\setbigcupdot}[2]{\ooalign{\hfil$#1\bigcup$\hfil\cr\hfil$#2$\hfil\cr\cr}}
% großes Kreuz
\newcommand*{\bigtimes}{\mathop{\raisebox{-.5ex}{\hbox{\huge{$\times$}}}}} 

% WHITESPACE COMMANDS
% nicht restriktiver newline command
\newcommand{\enter}{$ $\newline} 
% praktischer Tabulator
\newcommand\tab[1][1cm]{\hspace*{#1}}

% TEXT ÜBER ZEICHEN
\newcommand{\stackeq}[1]{\stackrel{#1}{=}} 

% UNDERLINE
% besseres underline 
\renewcommand{\ULdepth}{1pt}
\contourlength{0.5pt}
\newcommand{\ul}[1]{
	\uline{\phantom{#1}}\llap{\contour{white}{#1}}
}


% This work is licensed under the Creative Commons
% Attribution-NonCommercial-ShareAlike 4.0 International License. To view a copy
% of this license, visit http://creativecommons.org/licenses/by-nc-sa/4.0/ or
% send a letter to Creative Commons, PO Box 1866, Mountain View, CA 94042, USA.

% hier noch ein paar Commands die nur ich nutze, weil ich sie mir im Laufe der Jahre angewöhnt habe und sie mir jetzt nicht abgewöhnen will:

\newcommand{\gdw}{\Leftrightarrow}             % genau dann, wenn
\DeclareMathOperator{\id}{id}                  % identische Abbildung
\DeclareMathOperator{\dom}{dom}                % Domain, Definitionsbereich
\DeclareMathOperator{\rg}{rg}                  % Rang, Bild einer Funktion, Rang einer Matrix
\newcommand{\K}{\mathbb{K}}                    % Körper
\newcommand{\limn}{\lim\limits_{n\to\infty}}   % genau dann, wenn
\newcommand{\sonst}{\text{sonst }}             % sonst
\renewcommand{\O}{\mathcal{O}}                 % Landau-O
\DeclareMathOperator{\Div}{div}                % Divergenz
\renewcommand{\div}{\Div}                      % Divergenz
\DeclareMathOperator{\spann}{span}             % Span
\DeclareMathOperator{\diam}{diam}       	   % Diameter, Durchmesser einer Menge
\DeclareMathOperator{\inner}{int}              % Das innere einer Menge




% This work is licensed under the Creative Commons
% Attribution-NonCommercial-ShareAlike 4.0 International License. To view a copy
% of this license, visit http://creativecommons.org/licenses/by-nc-sa/4.0/ or
% send a letter to Creative Commons, PO Box 1866, Mountain View, CA 94042, USA.

% Commands für Stochastik / Statistik
\newcommand{\A}{\mathcal{A}}
\renewcommand{\P}{\mathbb{P}}
\newcommand{\E}{\mathbb{E}}
\newcommand{\B}{\mathcal{B}} %Borel-Sigma-Algebra
\newcommand{\Var}{\mathbb{V}\text{ar}}
\newcommand{\indi}{\mathbbm{1}} % Indikatorfunktion
\usepackage{bbm} %sorgt für Symbol für Indikatorfunktion

% Commands für WTHM
\newcommand{\F}{\mathcal{F}}				% Standard-Unter-Sigma-Algebra
\newcommand{\unab}{\rotatebox{90}{$\vDash$}} %Unabhänigkeit






\author{Willi Sontopski}

\parindent0cm %Ist wichtig, um führende Leerzeichen zu entfernen

\usepackage{scrpage2}
\pagestyle{scrheadings}
\clearscrheadfoot

\ihead{Willi Sontopski \& Robert Walter}
\chead{}
\ohead{WTHM WiSe 18 19}
\ifoot{Aufgabenblatt 5}
\cfoot{Version: \today}
\ofoot{Seite \pagemark}

\begin{document}
\section*{Aufgabe 1}
Seien $W,X,Y,Z$ unabhängig und standardnormalverteilt. Dann gilt:
\begin{enumerate}[label=\alph*)]
	\item Die charakteristische Funktion von $W\cdot X$ ist 
	\begin{align*}
		\Phi_{W\cdot X}(u)=\frac{1}{\sqrt{1+u^2}}
	\end{align*}
	\item $\det\begin{aligned}
		\begin{pmatrix}
			W & X\\
			Y & Z
		\end{pmatrix}
	\end{aligned}$ ist laplaceverteilt.
\end{enumerate}

\begin{proof}[Beweis von Jonas Kirchner an der Tafel.]
	\underline{Zeige a):}	
		\begin{align*}
		\Phi_{W\cdot X}(u)
		\overset{\text{Def}}&=
		\E\Big[\exp\big(i\cdot X\cdot W\cdot u\big)\Big]\\
		\overset{\text{unab}}&=
		\E\Big[\E\Big[\exp\big(iuWX\big)~\Big|~X\Big]\\	
		\overset{W\sim\Nor(0,1)}\implies
		\Phi_{W\cdot X}(u)
		&=\E\left[\exp\left(-\frac{u^2\cdot X^2}{2}\right)\right]\\
		&=\frac{1}{\sqrt{2\cdot\pi}}\cdot\int\limits_\R\exp\left(-\frac{u^2 y^2}{2}\right)\exp\left(-\frac{y}{2}\right)\d y\\
		&=\frac{1}{\sqrt{2\cdot\pi}}\cdot\int\limits_\R\exp\left(-\frac{y^2}{2}\cdot\left(u^2+1\right)\right)\d y
	\end{align*}
	Also ist
	\begin{align*}
		p(x)=\sqrt{\frac{u^2+1}{2\cdot\pi}}\exp\left(-\frac{y^2}{2}\cdot\left(u^2+1\right)\right)
	\end{align*}
	die Dichte von $\Nor\left(0,\frac{1}{u^2+1}\right)$. Also folgt
	\begin{align*}
		1=\int\limits_\R\sqrt{\frac{u^2+1}{2\cdot\pi}}\cdot\exp\left(-\frac{y^2}{2}\cdot\left(u^2+1\right)\right)\d y\\
		\Longleftrightarrow\int\limits_\R\exp\left(-\frac{y^2}{2}\cdot\left(u^2+1\right)\right)\d y=\sqrt{\frac{2\cdot\pi}{u^2+1}}\\
		\implies\Phi_{X\cdot W}(u)=\frac{1}{\sqrt{2\cdot\pi}}\cdot\sqrt{\frac{2\cdot\pi}{u^2+1}}=\frac{1}{\sqrt{u^2+1}}
	\end{align*}
	
	\underline{Zeige b):} Analog zu Robert
\end{proof}

\begin{proof}
	%\textbf{VERSION WILLI}
	%\begin{align*}
		%\Phi_{W\cdot X}(u)
		%\overset{\text{Def}}&=
		%\int\limits_\Omega\exp\big(i\cdot X\cdot W\cdot u\big)~\mu(\d u)\\
		%\overset{\text{Trafo}}&=
		%\int\limits_\R\exp\big(i\cdot t\cdot u)\cdot f_{X\cdot W}(t)\d t
		%\qquad\forall u\in\R
	%\end{align*}
	
	%Hierbei ist $f_{W\cdot X}$ die Dichte von $X\cdot W$. Da $W,X\sim\Nor(0,1)$ unabhängig sind,  gilt:
	%\begin{align*}
		%f_{X\cdot W}(t)
		%\overset{\text{Stoch}}&=
		%\int\limits_\R\frac{1}{|x|}\cdot f_X(x)\cdot f_W\left(\frac{t}{x}\right)\d x\\
		%\overset{\sim\Nor(0,1)}&=
		%\int\limits_\R\frac{1}{|x|}\cdot\frac{1}{\sqrt{2\cdot\pi}}\cdot\exp\left(-\frac{1}{2}\cdot x^2\right)\cdot\frac{1}{\sqrt{2\cdot\pi}}\cdot\exp\left(-\frac{1}{2}\cdot\frac{t^2}{x^2}\right)\d x\\
		%&=\frac{1}{2\cdot\pi}\cdot\int\limits_\R\frac{1}{|x|}\cdot\exp\left(-\frac{1}{2}\cdot\left(x^2+\frac{t^2}{x^2}\right)\right)\d x\\
		%&=...
	%\end{align*}
	
	%Eingesetzt erhalten wir also:
	%\begin{align*}
		%\Phi_{W\cdot X}(u)
		%&=\int\limits_\R\exp\big(i\cdot t\cdot u)\cdot f_{X\cdot W}(t)\d t\\
		%&=...
	%\end{align*}

	%\textbf{VERSION ROBERT}
	Zunächst verwenden wir den Polarisations-Trick:
	\begin{align*}
		Z &= WX= \frac{1}{4}\cdot\left((W+X)^2 - (W-X)^2\right)
	\end{align*}
	Mit dieser Darstellung lassen sich bessere Aussagen über die Verteilung von $Z$ treffen:
	\begin{align*}
		Z
		&=\frac{1}{4}\cdot\left((\underbrace{W+X}_{\sim \Nor(0, 2)})^2 - (\underbrace{W-X}_{\sim \Nor(0, 2)})^2\right)\\ 
		&\sim \frac{2}{4}\left(\underbrace{(\underbrace{F}_{\sim \Nor(0, 1)})^2}_{\sim \chi^2_1}-\underbrace{(\underbrace{G}_{\sim \Nor(0, 1)})^2}_{\sim \chi^2_1}\right) \\ 
		&\sim \frac{1}{2}(\underbrace{M}_{\sim \chi^2_1} -\underbrace{N}_{\sim \chi^2_1})
	\end{align*}
	Damit folgt:
	\begin{align*}
		\text{Cov}((W+X),(W-X)) &= \E((W+X)(W-X)) - \underbrace{\E(W+X)}_{=0}\cdot \underbrace{\E(W-X)}_{=0} \\
										&= \E(W^2-X^2) \\
										&= \E(W^2) -\E(X^2) \\
										&= \Var(W) - \Var(X) \\
										&= 0
	\end{align*}
	Das heißt, dass die Zufallsvariablen $(W+X)$ und $(W-X)$ unabhängig sind und somit auch $M$ und $N$(, da $(\cdot)^2$ borel-messbar ist).
	Nun können wir die charakteristische Funktion berechnen:
	\begin{align*}
		\E\left[e^{i\xi\cdot Z}\right]
		&=\E\left[e^{i\xi\cdot \frac{1}{2}\left(M-N\right)}\right] \\
		\implies \varphi_Z(2\xi)
		&=\E\left[e^{i\xi\cdot\left(M-N\right)}\right] \\
		\overset{\text{unab.}}&{=}\E\left[e^{i\xi M}\right]\cdot\overline{\E\big[e^{i\xi M}\big]}\\
		\overset{\text{char. F. } \chi^2_1}&{=}\frac{1}{\sqrt{1-2i\xi}}\frac{1}{\sqrt{1+2i\xi}} \\
																		&=\frac{1}{\sqrt{1+4\xi^2}}\\
		\implies \varphi_Z(\xi) &= \frac{1}{\sqrt{1+4(\frac{1}{2}\xi)^2}}\\
														&= \frac{1}{\sqrt{1+\xi^2}}
	\end{align*}

	
	\underline{Zeige b):}

	\begin{align*}
		P:=
		\left|\begin{array}{cc}
			W & X \\
			Y & Z
		\end{array}\right| &= WZ-YX
	\end{align*}
	Damit ist leicht auszurechnen:
	\begin{align*}
	 	\E\left[e^{i\xi(WZ-YX)}\right]
		\overset{\text{unab., da $x\cdot y$ mb.}}&{=}\E\left[e^{i\xi(WZ)}\cdot e^{i\xi(-YX)}\right]\\
	 	\overset{\eqref{eqAufgabe1}}&{=}
	 	\E\left[e^{i\xi(WZ)}\right]\E\left[e^{i\xi(-YX)}\right]\\
	 	&=\E\left[e^{i\xi(WZ)}\right]\overline{\E\big[e^{i\xi(YX)}\big]}\\
		&=\frac{1}{\sqrt{1+u^2}}\frac{1}{\sqrt{1+u^2}}\\
		&=\frac{1}{1+u^2}
	\end{align*}
	Dies ist die charakteristische Funktion einer Laplace-verteilten Zufallsvariable.
	Nach dem Eindeutigkeitssatz muss also $P$ Laplace-verteilt sein.\nl
	Erinnerung:
	\begin{align}\label{eqAufgabe1}
		\E\big[f(W\cdot Z)\cdot g(Y\cdot X)\big]=\E\big[f(W\cdot Z)\big]\cdot\E\big[g(Y\cdot X)\big]\qquad\forall f,g\in\B_b(\R)
	\end{align}
\end{proof}

\section*{Aufgabe 2}
Sei $\big(\phi_n(u)\big)_{n\in\N_0}$ eine Folge von CF und $(a_n)_{n\in\N_0}\subseteq\R_{\geq0}$ mit $\sum\limits_n a_n=1$.
Dann gilt:
\begin{enumerate}[label=\alph*)]
	\item $\begin{aligned}
		\psi(u):=\sum\limits_{n=0}^\infty a_n\cdot\phi_n(u)
	\end{aligned}$ ist wieder eine CF.
	\item Für festes $\phi(u)$ ist 
	\begin{align*}
		\chi(u):=\sum\limits_{n=0}^\infty a_n\cdot\big(\phi(u)\big)^n
	\end{align*}
	wieder eine CF.
	\item Für alle $\lambda>0$ und alle $p\in(0,1)$ sind
	\begin{align*}
		\exp\Big(\lambda\cdot\big(\phi(u)-1)\big)\Big),\qquad
		\frac{1-p}{1-p\cdot\phi(u)}
	\end{align*}
	ebenfalls CF. Zu welchen Verteilungen gehören diese für 
	\begin{align*}
		\phi(u)=\exp(i\cdot u)?
	\end{align*}
\end{enumerate} 

\begin{proof}
	\underline{Zeige a):}\\
	Verwende
	\textbf{Satz von Bochner.}\\
		Eine stetige Funktion $f:\R^d\to\C$ mit $f(0)=1$ ist CF einer ZG g.d.w. $f$ positiv definit.\nl
	Eine Funktion $f:\R^d\to\C$ heißt \textbf{positiv definit}
	\begin{align*}
		:\Longleftrightarrow
		\forall n\in\N,\forall\lambda_1,\ldots,\lambda_n\in\C,\forall x_1,\ldots,x_n\in\R^d:
		\sum\limits_{k,l=1}^n\lambda_k\cdot\overline{\lambda_l}\cdot f\big(x_k-x_l\big)\geq0
	\end{align*}
	
	Die Reihe
	\begin{align*}
		\sum\limits_{n=0}^\infty a_n\cdot\phi_n(u)
	\end{align*}
	ist absolut und gleichmäßig konvergent, denn
	\begin{align*}
		\left|\sum\limits_{n=0}^N a_n\cdot\phi_n(u)\right|
		&\leq\sum\limits_{n=0}^\infty a_n\underbrace{\big|\phi_n(u)\big|}_{\leq1}
		\leq\sum\limits_{n=0}^\infty a_n=1
	\end{align*}
	Somit ist
	\begin{align*}
		\psi(u):=\sum\limits_{n=0}^\infty a_n\cdot\phi_n(u)
	\end{align*}
	insbesondere stetig.
	\begin{align*}
		\psi(0)
		&=\sum\limits_{n=0}^\infty a_n\cdot\underbrace{\phi_n(0)}_{=1}
		=\sum\limits_{n=0}^\infty a_n\overset{\text{Vor}}{=}1
	\end{align*}
	Sei $N\in\N,\lambda_1,\ldots,\lambda_N\in\C,u_N\in\R^d$.
	Dann gilt:
	\begin{align*}
		\sum\limits_{k,l=1}^N\lambda_k\cdot\overline{\lambda_l}\cdot\psi\big(u_k-u_l\big)
		&=\sum\limits_{k,l=1}^N\lambda_k\cdot\overline{\lambda_l}\cdot\sum\limits_{n=0}^\infty a_n\cdot\phi_n\big(u_k-u_l\big)\\
		&=\sum\limits_{n=0}^\infty a_n\cdot\underbrace{\left(\sum\limits_{k,l=1}^N\lambda_k\cdot\overline{\lambda_l}\cdot\phi_n\big(u_k-u_l\big)\right)}_{\overset{\text{pos definit}}{\geq}0, \text{ nach S. v. Bochner}}\\
		&\geq0
	\end{align*}		
	Somit ist $\psi$ positiv definit und aus dem Satz von Bochner folgt, das $\psi$ eine charakteristische Funktion ist.\nl
	\underline{Zeige a) (Alternative vom Prof.)}\\
	Sei $(X_n)_{n\in\N}$ Folge von unabhängigen Zufallsvariablen mit
	\begin{align*}
		\phi_n(u)=\E\big[\exp(i\cdot u\cdot X_n)\big].
	\end{align*}
	Sei $N$ eine Zufallsvariable (unabhängig von $(X_n)_{n\in\N}$) mit $\P(N=k)=a_k~\forall k\in\N_0$.
	Setze $Y(\omega):=X_{N(\omega)}(\omega)~\forall\omega\in\Omega$.
	Berechne $\Phi_Y(u)$:
	\begin{align*}
		\Phi_Y(u)
		&=\E\Big[\exp\big(i\cdot u\cdot X_N\big)\Big]_{X_N}\\
		&=\E\Big[\underbrace{\E\Big[\exp(i\cdot u\cdot X_N)~\big|~N\Big]_X}_{=\phi_{n=N}(u)}\Big]_N\\
		&=\E\Big[\phi_{n=N}(u)\Big]_N\\
		&=\sum\limits_{k=0}^\infty\phi_k(u)\cdot a_k
	\end{align*}

	Hier bezeichnet $\E[~\cdot~]_X$ den Erwartungswert bezüglich der ZV $X$.
	
	\underline{Zeige b):}\\
	Sei $F$ die zugehörige Verteilung zu $\phi$.
	Setze
	\begin{align*}
		\phi_n(u):=\big(\phi(u)\big)^n.
	\end{align*}
	Dies ist die charakteristische Funktion von $X_1+\ldots+X_n$ mit $X_1,\ldots,X_n$ unabhängig mit $X_k\sim F$.
	Somit folgt die Behauptung aus a).\nl
	\underline{Zeige c):}\\
	Wähle
	\begin{align*}
		a_n:=\exp(-\lambda)\cdot\frac{\lambda^n}{n!}\geq0.
	\end{align*}
	Dann gilt:
	\begin{align*}
		\sum\limits_{n=0}^\infty a_n
		&=\exp(-\lambda)\cdot\underbrace{\sum\limits_{n=0}^\infty\frac{\lambda^n}{n!}}_{=\exp(\lambda)}=1\\
		\sum\limits_{n=0}^\infty a_n\cdot\big(\phi(u)\big)^n
		&=\exp(-\lambda)\cdot\underbrace{\sum\limits_{n=0}^\infty\frac{\lambda^n}{n!}\cdot\phi_n(u)}_{=\exp\big(\lambda\cdot\phi(u)\big)}
		=\exp\Big(\lambda\cdot\big(\phi(u)-1\big)\Big)
	\end{align*}
	Sei $a_n(1-p)\cdot p^n\geq0$ und 
	\begin{align*}
		\sum\limits_{n=0}^\infty a_n&=(1-p)\cdot\sum\limits_{n=0}^\infty p^n=1\\
		\implies
		\sum\limits_{n=0}^\infty a_n\cdot\phi_n(u)
		&=(1-p)\cdot\sum\limits_{n=0}^\infty\underbrace{p^n\cdot\phi^n(u)}_{=\big(p\cdot\phi(u)\big)^n}
		=(1-p)\cdot\frac{1}{1-p\cdot\phi(u)}
	\end{align*}		
	Speziell für $\varphi(u)=\exp(i\cdot u)$ folgt:
	\begin{align*}
		&\exp\Big(\lambda\cdot\big(\exp(i\cdot u)-1\big)\Big) &\rightsquigarrow\text{Poissonverteilung mit Par.}\lambda>0\\
		&\frac{1-p}{1-p\cdot\exp(i\cdot u)}	&\rightsquigarrow\text{geometrische Verteilung mit Par. }p
	\end{align*}
\end{proof}

\section*{Aufgabe 3}
Aus dem Stetigkeitssatz von Lévy folgt:
\begin{enumerate}[label=\alph*)]
	\item Sei $X_n$ binomialverteilt mit Parametern $n$ und $p_n=\frac{c}{n}$.
	Dann konvergiert $X_n$ für $n\to\infty$ gegen eine poissonverteilte Zufallsvariable $X$ mit Parameter $c$.
	\item Sei $X_n$ gammaverteilt mit Parametern $n$ und $\lambda>0$.
	Dann konvergiert $Y_n:=\frac{\lambda\cdot X_n-n}{\sqrt{n}}$ für $n\to\infty$ gegen eine standardnormalverteilt Zufallsvariable $Y$.
\end{enumerate}

\begin{proof}
	\underline{Zeige a):} Setze $p_n:=\frac{c}{n}$
	\begin{align*}
		&\P_n\overset{\w}{\longrightarrow} \P
		\Longleftrightarrow X_n\overset{\d}{\longrightarrow} X\\
		\P_n(X_n=k)&=\begin{pmatrix}
			n\\
			k
		\end{pmatrix}\cdot p_n^k\cdot q_n^{n-k}\\
		&=\frac{n!}{k!\cdot(n-k)!}\cdot\left(\frac{c}{n}\right)^k\cdot\left(1-\frac{c}{n}\right)^{n-k}\\
		&=\frac{n\cdot(n-1)\cdot\ldots\cdot(n-k+1)\cdot(n-k)!}{(n-k)!}\cdot\frac{c^k}{n^k}\cdot\frac{\left(1-\frac{c}{n}\right)^n}{\left(1-\frac{c}{n}\right)^k}\\
		&=\frac{c^k}{k!}\cdot\frac{(n-1)\cdot\ldots\cdot(n-k+1)}{n^{k-1}}\cdot\frac{1}{\left(-\frac{c}{n}\right)^k}\cdot\left(1-\frac{c}{n}\right)^n\\
		&=\frac{c^k}{k!}\cdot\frac{\left(1-\frac{1}{n}\right)\cdot\left(1-\frac{2}{n}\right)\cdot\ldots\cdot\left(1-\frac{k-1}{n}\right)}{\left(1-\frac{c}{n}\right)^k}\cdot\left(1-\frac{c}{n}\right)^n
	\end{align*}
	Grenzwertbildung:
	\begin{align*}
		\limn\P_n(X_n=k)
		=&\limn\frac{c^k}{k!}\cdot\frac{\left(1-\frac{1}{n}\right)\cdot\ldots\cdot\left(1-\frac{k-1}{n}\right)}{\left(1-\frac{c}{n}\right)^k}\cdot\left(1-\frac{c}{n}\right)^n\\
		=&\frac{c^k}{k!}\cdot\exp(-c) \text{ ,da} \\
		 &\limn\left(1-\frac{c}{n}\right)^n=\exp(-c) \text{ und}\\
		&\limn\frac{\left(1-\frac{1}{n}\right)\cdot\ldots\cdot\left(1-\frac{k-1}{n}\right)}{\left(1-\frac{c}{n}\right)}=1
	\end{align*}
	Also haben wir
	\begin{align*}
		\limn \P_n(X_n =k) = \frac{c^k}{k!}\cdot\exp(-c) = \P(Y=k)
	\end{align*}
	,wobei $Y\sim Poi(c)$.
	
	\underline{Zeige a), Alternative von Pascal}\\
	Sei $X_n\sim\Bin\left(n,p_n\right)$ mit $p_n := \frac{c}{n}$.
	Dann:
	\begin{align*}
		\Phi_{X_n}(u)
		&=\left(\frac{c}{n}\cdot\exp(i\cdot u)+1-\frac{c}{n}\right)^n\\
		\limn\left(\left(\frac{c}{n}\cdot\exp(i\cdot u)+1-\frac{c}{n}\right)^n\right)
		&=\limn\left(\left(\frac{c\cdot\big(\exp(i\cdot u)-1\big)}{n}+1\right)^n\right)\\
		&=\exp\Big(c\cdot\big(\exp(i\cdot u)-1\big)\Big)\\
		&=\Phi_Y(u)\mit Y\sim\Poi(c)
	\end{align*}
	Stetigkeit folgt sofort.\nl
	\underline{Zeige b):}
	\begin{align*}
		\Phi_{X_n}(u)
		&=\left(\frac{\lambda}{\lambda-u}\right)^n\\
		\Phi_{Y_n}(u)
		&=\exp\left(-i\cdot u\cdot n\cdot\frac{1}{\sqrt{n}}\right)\cdot\Phi_{X_n}\left(\frac{\lambda\cdot u}{\sqrt{n}}\right)\\
		&=\exp\big(-i\cdot u\cdot\sqrt{n})\cdot\left(\frac{\lambda}{\lambda-i\cdot\lambda\cdot\frac{u}{\sqrt{n}}}\right)^n\\
		&=\exp\big(-i\cdot u\cdot\sqrt{n}\big)\cdot\frac{1}{1-i\cdot u\cdot\frac{1}{\sqrt{n}}}\\
		&=\exp\big(-i\cdot u\cdot\sqrt{n}\big)\cdot\left(1-i\cdot\frac{u}{\sqrt{n}}\right)^{-n}\\
		&=:\big(\varphi(v_n)\big)^{-n}
	\end{align*}
	wobei $\varphi(v)=\exp(v)\cdot(1-v),v_n=\frac{i\cdot u}{\sqrt{n}}$ ist.
	Entwicklung der Exponentialfunktion:
	\begin{align*}
		\varphi(v)
		&=\Big(1+v+\frac{1}{2}\cdot v^2+\O(v^3)\Big)\cdot(1-v)
		=1-\frac{1}{2}\cdot v^2+\O(v^3)\\
		\varphi(v_n)
		&=1-\frac{1}{2}\cdot\left(\frac{i\cdot u}{\sqrt{n}}\right)^2+\O\left(\left(\frac{i\cdot u}{\sqrt{n}}\right)^3\right)\\
		&=1+\frac{1}{2}\cdot\frac{u^2}{n}+\O\left(\frac{1}{n^{\frac{3}{2}}}\right)\\
		\big(\varphi(v_n)\big)^{-n}
		&=\frac{1}{\left(1+\frac{\frac{1}{2}\cdot u^2+\O\left(\frac{1}{\sqrt{n}}\right)}{n}\right)^n}
		\overset{\text{VL}}{\longrightarrow}
		\frac{1}{\exp\left(\frac{u^2}{2}\right)}=\exp\left(-\frac{u^2}{2}\right)=\Phi_Y(u)
	\end{align*}
	wobei $Y\sim\Nor(0,1)$.
	
	
\end{proof}

\section*{Aufgabe 4}
Sei $X$ eine reellwertige Zufallsvariable mit Vertielung $F$ und seien $X_1,X_2,\ldots$ i.i.d. Kopien von $X$.
Die Verteilung $F$ heißt \textbf{stabil}
\begin{align}\label{eqStabil}\tag{Stab}
	:\Longleftrightarrow
	\exists(a_n)_{n\in\N},(b_n)_{n\in\N}\subseteq\R_{\geq0}:\forall\in\N:
	X_1+\ldots+X_n\overset{\d}{=}a_n\cdot X+b_n
\end{align}
Die Verteilung $F$ heißt \textbf{symmetrisch stabil}
\begin{align*}
	:\Longleftrightarrow
	\forall n\in\N:b_n=0
\end{align*}

\begin{enumerate}[label=\alph*)]
	\item Die Normalverteilung ist stabil mit $a_n=\sqrt{n}$ und $b_n=\mu(n-\sqrt{n})$.
	Die Cauchyverteilung ist stabil mit $a_n=n$ und $b_n=0$.
	\item Sei $X$ symmetrisch stabil.
	Zeige:
	\begin{align*}
		\big(\phi_X(u)\big)^n&=\phi_X(a_n\cdot u) & \forall n\in\N:
		a_{m\cdot n}&=a_n\cdot a_m &\forall m,n\in\N
	\end{align*}		
	%Folgender Aufgabenteil wurde in der VL ergänzt.
	Sei $b_n=0$. Dann ist \eqref{eqStabil} äquivalent zu
	\begin{align}\label{eq4Stern}\tag{$\ast$}
		\big(\Phi_X(u)\big)^n=\Phi_X(u\cdot a_n)\qquad\forall n\in\N
	\end{align}
	\item Für alle $\alpha\in(0,2]$ und alle $c>0$ existieren Zufallsvariablen $X_{\alpha,c}$ mit CF 
	\begin{align*}
		\phi(u)=\exp\left(-c\cdot|u|^\alpha\right).
	\end{align*}
	Zeige, dass $X_{\alpha,c}$ stabil ist und bestimmt $a_n$.
	Welche Parameter entsprechen der Normal- und der Cauchyverteilung?
	\item Zeige, dass
	\begin{align*}
		\phi(u)=\exp\left(-c\cdot|u|^\alpha\right).
	\end{align*}
	für $\alpha>2$ \ul{keine} CF ist.
\end{enumerate}

\begin{lösung}
	\underline{Zu a):} Vorüberlegung:
	\begin{align*}
		\Phi_{X_1+\ldots+X_n}(u)
		&=\big(\Phi(u)\big)^n\\
		\Phi_{a_n\cdot X+b_n}(u)
		&=\exp\big(i\cdot b_n)\cdot u\big)\cdot\Phi(a_n\cdot u\big)
	\end{align*}
	Beide sind stabil, wenn sie gleich sind.
	Seien $X_k\sim\Nor(\mu,\sigma^2)$, $X_1+\ldots+X_n\sim\Nor(n\cdot\mu,n\cdot\sigma^2)$.
	\begin{align*}
		X_1+\ldots+X_n
		\overset{\d}&{=}
		\underbrace{\sqrt{n}}_{=a_n}\cdot X+\underbrace{(n-\sqrt{n})\cdot\mu }_{=b_n}
	\end{align*}
	
	Für die Cauchyverteilung: $X_k\sim\Cauchy(\gamma,\beta)$ d.h. mit Dichte
	\begin{align*}
		f(x)=\frac{1}{\pi}\cdot\frac{\gamma}{\gamma^2+(x-\beta)^2}.
	\end{align*}
	Also gilt:
	\begin{align*}
		\Phi_X(s)&=\exp\big(i\cdot\beta\cdot s-\gamma\cdot|s|\big)\\
		\Phi_X^n(s)&=\exp\big(i\cdot n\cdot\beta\cdot s-\gamma_n\cdot|s|\big)&(1)\\
		\Phi_{a_n\cdot X+b_n}(s)
		&=\exp(i\cdot b_n\cdot s)\cdot\exp\big(i\cdot\beta\cdot a_n\cdot s-\gamma\cdot|a_n\cdot s|\big)\\
		&=\exp\Big(i\cdot s\cdot\big(b_n+\beta\cdot a_n\big)-\gamma\cdot a_n\cdot|s|\Big)&(2)\\
		(1)\overset{!}&{=}(2)\\
		&\left\lbrace\begin{array}{cl}
			n\cdot\beta&= b_n+\beta\cdot a_n\\
			\gamma_n&=\gamma\cdot a_n
		\end{array}\right.\implies a_n=n~\wedge~b_n=0
	\end{align*}		
	Also ist $\Cauchy(\gamma,\beta)$ symmetrisch stabil.\nl
	\underline{Zu b):}\\
	$X$ symmetrisch stabil impliziert wegen a)
	\begin{align*}
		\Phi^n(u)&=\Phi(a_n\cdot u)\\
		\Phi(a_n\cdot a_m\cdot u)&=\Big(\Phi(a_m\cdot u)\Big)^n
		=\big(\Phi(u)\big)^{n\cdot m}=\Phi(a_{m\cdot n}\cdot u)
	\end{align*}
	
	\underline{Zu b) aus der VL:} Es gilt:
	\begin{align*}
		\big(\Phi_X(u)\big)^{m\cdot n}
		&=\Big(\big(\Phi_X(u)\big)^m\Big)^n\\
		\overset{\eqref{eq4Stern}}&=
		\big(\Phi_X(a_m\cdot u)\big)^n\\
		\overset{\eqref{eq4Stern}}&=
		\Phi_X(a_m\cdot a_n\cdot u)\\
		\Phi_X(u)^{m\cdot n}
		\overset{\eqref{eq4Stern}}&=
		\Phi_X(a_{m\cdot n}\cdot u)\\
		\implies a_m\cdot a_n\cdot X\overset{\d}=a_{m\cdot n}\cdot X\\
		\implies F\left(\frac{x}{a_m\cdot a_n}\right)
		&=F\left(\frac{x}{a_{m\cdot n}}\right)\qquad\forall x\in\R\\
		\implies
		a_m\cdot a_n&=a_{m\cdot n}\qquad\forall m,n\in\N
	\end{align*}
	Welche Folgen erfüllen $a_n\cdot a_m=a_{m\cdot n}$?
	Zum Beispiel $a_n=n^{\frac{1}{\alpha}}$ für $\alpha\in(0,\infty)$ erfüllt dies.
	
	\underline{Zu c):}
	Die symmetrische $\alpha$-stabile Verteilung ist definiert durch 
	\begin{align*}
		\Phi_{c,\alpha}(u):=\exp\big(-c\cdot|u|\big)^\alpha\qquad\forall c>0,\forall\alpha\in(0,2]
	\end{align*}
	Es gilt:
	\begin{align*}
		\big(\Phi_{c,\alpha}(u)\big)^n
		&=\exp\left(-c\cdot|u|^\alpha\right)
		=\exp\left(-c\cdot\left|n^{\frac{1}{\alpha}}\cdot u\right|^\alpha\right)
		=\Phi_{c,\alpha}\left(n^{\frac{1}{\alpha}}\cdot u\right)
	\end{align*}
	Also erfüllt $\big(\Phi_{c,\alpha}(u)\big)^n$ \eqref{eq4Stern} mit $a_n=n^{\frac{1}{\alpha}}$.\\
	Spezialfälle: 
	\begin{itemize}
		\item $\alpha=2\longrightarrow$ Normalverteilung
		\item $\alpha=1\longrightarrow$ Cauchyverteilung
	\end{itemize}
	
	\underline{Zu d):}\\
	Angenommen $\Phi(u)$ ist die CF einer Zufallsvariable $X$.
	%Wir berechnen nun $\Var[X]$:
	\begin{align*}
		\Phi'(u)
		&=-c\cdot|u|^{\alpha-1}\cdot\alpha\cdot\exp\left(-c\cdot|u|^\alpha\right)
		\implies \Phi'(u)=0\\
		\Phi''(u)
		&=c^2\cdot|u|^{2\cdot(\alpha-1)}\cdot\alpha^2\cdot\exp\left(-c\cdot|u|^\alpha\right)-c\cdot\alpha\cdot(\alpha-1)\cdot|u|^{\alpha-2}\cdot\exp\left(-c\cdot|u|^\alpha\right)\\
		&\implies\Phi''(0)=0
	\end{align*}
	Der Erwartungswert $\E\big[X^2\big]$ existiert, $\E[X]=0$, $\E[X^2]$ und somit $\Var(X)=0$.
	Also ist $X=0$ fast sicher. Somit
	\begin{align*}
		\Phi_X(u)=\E\big[\exp(i\cdot u\cdot 0)\big]=1
	\end{align*}
\end{lösung}
\end{document}
