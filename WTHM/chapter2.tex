% This work is licensed under the Creative Commons
% Attribution-NonCommercial-ShareAlike 4.0 International License. To view a copy
% of this license, visit http://creativecommons.org/licenses/by-nc-sa/4.0/ or
% send a letter to Creative Commons, PO Box 1866, Mountain View, CA 94042, USA.

\chapter{Martingale} %2
\section{Definition und Eigenschaften} %2.1
\begin{beisp}
Betrachte ein Glücksspiel zwischen zwei Personen.
Die Folge $(\xi_n)_{n\in\N}$ i.i.d. und in $L_1$ beschreibe den Gewinn pro Zug $\left(\begin{cases}
	positiv & $\to$ Gewinn für Person A \\
	negativ & $\to$ Gewinn für Person B
\end{cases}\right)$. Sei
\begin{align*}
S_n:=\sum\limits_{i=1}^n\xi_i
\end{align*}
der Gesamtgewinn nach dem $n$-ten Zug und 
\begin{align*}
\F_n:=\sigma\big(\xi_1,\ldots,\xi_n\big)
\end{align*}
die Information aus dem Spiel bis zum $n$-ten Zug. Somit ist er bedingte Erwartungswert
\begin{align*}
\E\big[S_{n+k}~|~\F_n\big]
\end{align*}
die Vorhersage für Gewinn nach $(n+k)$-ten Zug basierend auf den ersten $n$ Zügen.\\
Berechnung:
\begin{align*}
\E\big[S_{n+k}~\big|~\F_n\big] 
&=\E\Big[S_n+\xi_{n+1}+\ldots+\xi_{n+k}~\Big|~\F_n\Big]\\
&=\E[S_n~|~\F_n]+\sum\limits_{i=1}^k\E\big[\underbrace{\xi_{n+i}~\big|~\F_n}_{\unab}\big]\\
&=S_n+\sum\limits_{i=1}^k\E\big[\xi_{n+i}\big]\\
&\stackrel{\text{iid.}}{=}S_n+k\cdot\E[\xi_1]\\
&=\left\lbrace\begin{array}{cll}
> S_n, & \falls \E[\xi_1]>0 & \text{vorteilhaftes Spiel für A}\\
= S_n, & \falls \E[\xi_1]=0 & \text{faires Spiel}\\
< S_n, & \falls \E[\xi_1]<0 & \text{nachteiliges Spiel für A}
\end{array}\right.
\end{align*}
\underline{Variante:} Variabler Einsatz erlaubt.
Der Einsatz im $(n+1)-k$-ten Zug wird durch 
\begin{align*}
e_{n+1}:=:e_{n+1}\big(\xi_1,\ldots,\xi_n\big)
\end{align*}
definiert und hängt von $\F_n$ ab. Man spricht hier von einer vorhersehbaren Zufallsvariable.
Der neue Gesamtgewinn unter diesem Einsatz beträgt
\begin{align*}
\tilde{S}_n:=\sum\limits_{i=1}^n e_i\cdot\xi_i.
\end{align*}
Achtung! 
\begin{align*}
\tilde{S}_n-\tilde{S}_{n-1}=e_n\cdot\xi_n
\end{align*}
ist \underline{nicht mehr iid}.\\

Nun ergibt sich durch eine beliebige Einsatzstrategie mit Informationen aus $\F_n$ die Vorhersage der nächsten Runde des Spiels:
\begin{align*}
\E[\tilde{S}_{n+1}~|~\F_n] 
&=\E\big[\tilde{S}_n+ e_{n+1}\cdot\xi_{n+1}~\big|~\F_n\big]\\
&=\tilde{S}_n+\E\big[\underbrace{e_{n+1}}_{\in\F_n}\cdot\xi_{n+1}~\big|~\F_n\big]\\
&=\tilde{S}_n+e_{n+1}\cdot\E\big[\underbrace{\xi_{n+1}~\big|~\F_n}_{\unab}\big]\\
&=\tilde{S}_n+e_{n+1}\cdot\E[\xi_1]
\end{align*}
Nehmen wir ein faires Spiel an, also $\E[\xi_1]=0$, dann erhalten wir:
\begin{align*}
\E\big[\tilde{S}_{n+1}~\big|~\F_n\big]=\tilde{S}_n
\end{align*}
Durch Iteration dieser Gleichung folgt für $k>0$:
\begin{align*}
\E\big[\tilde{S}_{n+k}~\big|~\F_n\big]
&\stackeq{\eqref{Turmregel}}
\E\Big[\underbrace{\E\big[\tilde{S}_{n+k}~\big|~\F_{n+k-1}\big]}_{=\tilde{S}_{n+k-1}}~\Big|~\F_n\Big]\\
&=\E\big[\tilde{S}_{n+1}~\big|~\F_n\big]\\
&=\tilde{S}_n
\end{align*}
Also lässt sich bei einem fairen Spiel auch durch variablen Einsatz nicht in ein vorteilhaftes Spiel verwandeln. Dies ist ein Prototyp für das Martingal.
\end{beisp}

\begin{defi}\
\begin{enumerate}[label=(\alph*)]
\item Eine Indexmenge $I$ heißt \textbf{aufsteigend geordnet} $:\gdw$ eine Ordnungsrelation ``$\leq$'' existiert mit $ \forall s,t\in I:\exists u\in I$
\begin{align*}
s\leq u\wedge t\leq u
\end{align*}
In den allermeisten Fällen haben wir eine Totalordnung, z. B. $(\N,\leq)$ oder $(\R_{\geq0},\leq)$. $I$ wird oft als Zeitachse interpretiert.
\item Eine Familie $(\F_t)_{t\in I}$ von $\sigma$-Algebren $\F_t\subseteq\A$ heißt \textbf{Filtration} genau dann, wenn gilt
\begin{align*}
s\leq t\implies\F_s\subseteq\F_t
\end{align*}
Eine Filtration ist also eine aufsteigenden Folge von $\sigma$-Algebren.\\
Interpretation: $\F_t$ beschreibt die zum Zeitpunkt $t$ verfügbare Information. Und der Informationsfluss wächst mit der Zeit an.\\
Außerdem setzen wir
\begin{align*}
\F_\infty:=\bigcup\limits_{t\in I}\F_t.
\end{align*}
\item Ein \textbf{stochastischer Prozess} ist eine Familie $(X_t)_{t\in I}$ von Zufallsvariablen.\\
Ein stochastischer Prozess $(X_t)_{t\in I}$ heißt \textbf{adaptiert} an die Filtration $(\F_t)_{t\in I}$ genau dann, wenn für alle $t\in I$ gilt:
\begin{align*}
X_t\text{ ist messbar bzgl. }\F_t
\end{align*}
\item Eine Folge $(e_n)_{n\in\N}$ heißt \textbf{vorhersehbar} bzgl. einer Filtration $(F_t)_{n\in\N}$ genau dann, wenn für alle $n\in \N$ gilt:
\begin{align*}
e_{n+1}\text{ ist messbar bzgl. }\F_n
\end{align*}
Vorhersehbarkeit ist stärker als Adaptiertheit.
\end{enumerate}
\end{defi}

\begin{defi}[Martingal]\enter
Sei $X=(X_t)_{t\in I}$ ein stochastischer Prozess und $(\F_t)_{t\in I}$ eine Filtration. $X$ heißt \textbf{Martingal} $:\gdw$ folgende drei Eigenschaften gelten:
\begin{enumerate}[label=(\alph*)]
\item $(X_t)_{t\in I}$ ist adaptiert an $(\F_t)_{t\in I}$
\item $\begin{aligned}
		X_t\in L_1(\Omega,\A,\P) \qquad \forall t\in I
\end{aligned}$
\item Für alle $s,t\in I$ mit $s\leq t$ gilt
	\[ \E[X_t~|~\F_s]=X_s \]
\end{enumerate}
Gilt statt (c) die Eigenschaft
\begin{align}\label{defSubmartingal}\tag{c'}
\E[X_t~|~\F_s]\geq X_s
\end{align}
so heißt $X$ \textbf{Submartingal}. (Der zukünftige Wert wird unterschätzt) \enter Gilt statt (c) die Eigenschaft
\begin{align}\label{defSupermartingal}\tag{c''}
\E[X_t~|~\F_s]\leq X_s
\end{align}
so heißt $X$ \textbf{Supermartingal}. (Der zukünftige Wert wird überschätzt)\\
\end{defi}

\begin{bemerkung}
Beim Submartingal steigt der Erwartungswert $\E[X_t]$ mit der Zeit und beim Supermartingal fällt der Erwartungswert mit der Zeit.\\
Begründung für Submartingal:
\begin{align*}
\E[X_t]\stackeq{\eqref{Turmregel}}
\E\big[\E[X_t~|~\F_s]~\big|~]\stackrel{\text{Mono}}{\geq}\E[X_s]\qquad s\in I
\end{align*}
``\textit{Das Leben ist ein Supermartingal - die Erwartung fällt mit der Zeit.}''
\end{bemerkung}
