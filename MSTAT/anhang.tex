\setcounter{chapter}{0}
\renewcommand{\thechapter}{\Alph{chapter}}
\chapter{Anhang}
\setcounter{equation}{1}
\section{Grundlagen, die man kennen sollen}
TODO

\begin{satz}[Maßkorrespondenzsatz]
Sei $X$ eine Zufallsvariable. 
Dann bestimmt das Maß eindeutig die Verteilung $F$ von $X$ und die Verteilung bestimmt eindeutig das Maß.
\end{satz}

\begin{notation}
$F(\d x)$ bedeutet man integriert bzgl. Maß $Q$, was durch $F$ eindeutig festgelegt ist.
\end{notation}

\begin{satz}[Transformationsformel]

\end{satz}

\begin{satz}[Lebesgue / dominierte Konvergenz]

\end{satz}

\begin{satz}[Monotone Konvergenz]

\end{satz}

\begin{satz}[Gliwenko-Cantelli]

\end{satz}

