% This work is licensed under the Creative Commons
% Attribution-NonCommercial-ShareAlike 4.0 International License. To view a copy
% of this license, visit http://creativecommons.org/licenses/by-nc-sa/4.0/ or
% send a letter to Creative Commons, PO Box 1866, Mountain View, CA 94042, USA.

\chapter{Die Finite-Elemente-Methode}
\setcounter{section}{-1}
\section{Einleitung}
In diesem Dokument werden folgende Notationen verwendet:
\begin{itemize}
\item $u_x:=\frac{\partial u(x)}{\partial x}$ für die Ableitung einer diffbaren Funktion $u, x\mapsto u(x)$
\item $\frac{\partial u}{\partial n}:=\nabla u\cdot n$ wobei $n$ die Normale an $u$ ist
\end{itemize}

\subsection{Durchbiegung einer Membran}
\begin{itemize}
\item Gegeben ist eine Membran als Graph der Funktion $u:\Omega\subseteq\R^2\to\R$.
\item Aus der Physik ist bekannt, dass die Deformationsarbeit proportional zur Flächenänderung ist. Die Flächenänderung ist
\begin{align*}
\frac{1}{2}\cdot\int\limits_\Omega\left(u^2_x+u_y^2\right)\d x\d y
\end{align*}
\item Die Energie des Systems ist 
\begin{align*}
\frac{1}{2}\cdot\int\limits_\Omega\left(u^2_x+u_y^2\right)\d x\d y-\int\limits_\Omega f\cdot u\d x\d y
\end{align*}
wobei $f$ eine von außen einwirkende Kraft ist
\item Es wirkt das \textbf{physikalische Minimierungsprinzip}, d.h. das System strebt stets einen Zustand minimaler Gesamtenergie an. Gesucht ist also eine Funktion $u$ derart, dass 
\begin{align*}
	E(u)&\leq E(v)&\forall v\in\tilde{V}\\
	\gdw ~ E(u)&\leq E(u+t\cdot v)\forall t\in\R,~&\forall v\in V
\end{align*}
Dabei ist $V$ ein Funktionenraum, dessen Funktionen auf dem Rand verschwinden.
\item Setze $\varphi(v,t):=E(u+t\cdot v)$, wobei $v$ als Parameter und $t$ als Variable aufgefasst wird. Somit lautet die notwendige Bedingung an das Energieminimum
\begin{align*}
\frac{\d\varphi}{\d t}(v,0)=0
\end{align*}
\end{itemize}
Nachrechnen:
\begin{align*}
\frac{\d\varphi}{\d t}(v,t)
&=\frac{\d}{\d t}E(u+t\cdot v)\\
&=\frac{\d}{\d t}\int\limits_\Omega\left(\frac{1}{2}\cdot\left((u+t\cdot v)_x^2+(u+t\cdot v)_y^2\right)-f(u+t\cdot v)\right)\d x\d y\\
&=\int\limits_\Omega\left((u+t\cdot v)_x\cdot v_x+(u+t\cdot v)_y\cdot v_y-f\cdot v\right)\d x\d y
\end{align*}
Setze nun $t:=0$. Dann folgt aus der notwendigen Bedingung
\begin{align*}
0=\int\limits_\Omega\left(u_x\cdot v_x+u_y\cdot v_y-f\cdot v\right)\d x\d y
\end{align*}
Es entsteht die Variationsaufgabe: Finde $u(t)$ so, dass 
\begin{align*}
\int\limits_\Omega u_x\cdot v_x+u_y\cdot v_y\d x\d y=\int\limits_\Omega f\cdot v\d x\d y~~~\forall v\in V.
\end{align*}

Durch partielle Integration (siehe Anhang für nähere Erklärung) erhält man aus dem linken Integral
\begin{align*}
\int\limits_\Omega u_x\cdot v_x+u_y\cdot v_y\d x\d y
=-\int\limits_\Omega\left(u_{xx}+u_{yy}\right)\cdot v
+\int\limits_{\partial\Omega}\underbrace{u_x\cdot v\cdot n_x+u_y\cdot v\cdot n_y}_{=(\nabla u\cdot n)\cdot v=0\text{, da $v=0$ auf }\partial\Omega}\d\gamma
\end{align*}
Somit folgt:
\begin{align*}
-\int\limits_\Omega\big(\underbrace{u_{xx}+u_{yy}}_{=\laplace u}\big)\cdot v\d x\d y=\int\limits_\Omega f\cdot v\d x\d y~\forall v\in V\\
\Longrightarrow-\laplace u\equiv f\text{ auf }\Omega\Longrightarrow\textbf{``Poisson-Gleichung''}
\end{align*}

\section{Sobolev-Räume}
Bezeichnungen für dieses Kapitel:

\begin{itemize}
\item $d\geq1$ sei die Raumdimension
\item $\Omega\subseteq\R^d$ sei offen und beschränkt
\item $p\in[1,\infty)$ reelle Zahl
\item $q\in(1,\infty]\mit\frac{1}{p}+\frac{1}{q}=1$ \textbf{konjungierter / dualer Exponent}
\item $\alpha=(\alpha_1,\ldots,\alpha_d)\in\N_0^d$ Multiindex mit
\begin{align*}
|\alpha|:=\alpha_1+\ldots+\alpha_d\\
D^\alpha\varphi:=\frac{\partial^{|\alpha|}\varphi}{\partial x_1^{\alpha_1}\hdots\partial x_d^{\alpha_d}}
\end{align*}
\item $L^p(\Omega):=\left\lbrace f:\Omega\to\R:f\text{ messbar und }\int\limits_\Omega |f(x)|^p\d\mathcal{L}(x)<\infty\right\rbrace$ Lebesgue-Räume
\end{itemize}

\textbf{Bemerkungen.}
\begin{enumerate}
\item Da $\Omega$ beschränkt ist, gilt $L^p(\Omega)\subseteq L^1(\Omega)$ und die kanonische Injektion ist stetig.
\item Es gilt die Gauß-Formel:
\begin{align}\label{GaussFormel}
\int\limits_\Omega\varphi\cdot D^\alpha\psi\d x=(-1)^{|\alpha|}\cdot\int\limits_\Omega\psi\cdot D^\alpha\varphi\d x~~~\forall \varphi,\psi\in C_0^\infty(\Omega)
\end{align}
\end{enumerate}

\begin{definition}[schwache Ableitung]
Seien $\varphi,\psi\in L^1(\Omega)$ und sei $\alpha\in\N_0^\alpha$ ein Multiindex. Dann heißt $\psi$ die \textbf{$\alpha$-te schwache Ableitung} von $\varphi:\gdw$
\begin{align*}
\forall v\in C_0^\infty(\Omega):\int\limits_\Omega\varphi\cdot D^\alpha v\d x=(-1)^{|\alpha|}\cdot\int\limits_\Omega\psi\cdot v\d x
\end{align*}
Kurzschreibweise: $\psi=D^\alpha\varphi$
\end{definition}

\textbf{Bemerkungen.}
\begin{enumerate}
\item Die $\alpha$-te schwache Ableitung ist eindeutig bestimmt im Sinne des $L^1$ (also bis auf Lebesgue-Nullmengen).
\item Ist $\varphi\in C^{|\alpha|}(\Omega)$, dann existiert die schwache $\alpha$-te 	Ableitung, die mit der klassischen Ableitung übereinstimmt.
\end{enumerate}

\begin{beisp}
$d=1$, $\Omega=(-1,1)$, $\varphi(x):=|x|$\\
Behauptung: $\varphi'(x)=\left\lbrace\begin{array}{cl}
-1, & \falls -1<x<0\\
1, & \falls 0\leq x<1
\end{array}\right.$\\
Die schwache Ableitung existiert also und der Wert an der Stelle 0 ist nicht relevant.
\begin{proof}
	Sei $v \in C_0^\infty(\Omega)$ beliebig. Dann
	\begin{align*}
		\int_{-1}^1 \varphi v' \d x &= \int_{-1}^0 \varphi v' \d x + \int_{0}^1 \varphi v' \d x \\
															&\stackeq{\text{part}} ~~~ \big[\varphi v\big]_{-1}^0 - \int_{-1}^0 (-1) v \d x + \big[\varphi v\big]_{0}^1 - \int_{0}^1 (1) v \d x \\
															&\stackeq{v = 0 \text{ auf Rand}} ~~~ - \int_{-1}^0 (-1) v \d x - \int_{0}^1 (1) v \d x \\
															&\stackeq{\text{Def}} - \int_{-1}^1 \varphi'(x)v\d x
	\end{align*}
\end{proof}
\end{beisp}

\begin{definition}[Sobolev-Räume]
Für $k\in\N_0$, $p\in[1,\infty)$ definieren wir den \textbf{Sobolev-Raum} wie folgt;
\begin{align}
W^{k,p}(\Omega):=\Big\lbrace\varphi\in L^p(\Omega):D^\alpha\varphi\text{ (schwache Ableitung) existiert und erfüllt }D^\alpha\varphi\in L^p(\Omega)~\forall|\alpha|\leq k\Big\rbrace
\end{align}
Als Norm vereinbaren wir
\begin{align*}
\Vert\varphi\Vert_{k,p,\Omega}:=\left(\sum\limits_{|\alpha|\leq k}\left\Vert D^\alpha\varphi\right\Vert^p_{L^p}\right)^{\frac{1}{p}}
=\left(\sum\limits_{|\alpha|\leq k}\int\limits_\Omega\left| D^\alpha\varphi(x)\right|^p\d x\right)^{\frac{1}{p}}
\end{align*}
Durch
\begin{align*}
|\varphi|_{k,p,\Omega}:=\left(\sum\limits_{|\alpha|= k}\left\Vert D^\alpha\varphi\right\Vert^p_{L^p}\right)^{\frac{1}{p}}
\end{align*}
wird eine Halbnorm definiert.\\
Für $p=2$ schreiben wir $H^k(\Omega):=W^{k,2}(\Omega)$.
\end{definition}

\begin{satz}[Eigenschaften der Sobolev-Räume]
Es gilt:
\begin{enumerate}
\item $\left(W^{k,p}(\Omega),\Vert\cdot\Vert_{k,p,\Omega}\right)$ ist ein Banachraum.
\item $C^\infty(\overline{\Omega})$ liegt dicht in $W^{k,p}(\Omega)$.
\item $H^k(\Omega)$ ist ein Hilbertraum mit dem Skalarprodukt
\begin{align*}
\langle \varphi,\psi\rangle_k:=\sum\limits_{|\alpha|\leq k}\int\limits_\Omega D^\alpha\varphi\cdot D^\alpha\psi\d x.
\end{align*}
\end{enumerate}
\end{satz}

\begin{satz}[Glätte von stückweise glatten Funktionen]\enter
Seien $\Omega_1,\Omega_2\subseteq\Omega$ zwei nichtleere, offene, beschränkte und disjunkte Teilmengen mit stückweise glattem Rand. Gelte
$\overline{\Omega}=\overline{\Omega_1}\cup\overline{\Omega_2}$.
Weiterhin sei $\varphi\in L^p(\Omega)$ so, dass 
\begin{align*}
\exists k\geq 1:\forall i\in\lbrace1,2\rbrace:\varphi|_{\Omega_i}\in C^k(\overline{\Omega_i}).
\end{align*}
Dann gilt:
\begin{align*}
\varphi\in W^{k,p}(\Omega)\Longleftrightarrow\varphi\in C^{k-1}(\Omega)
\end{align*}
\end{satz}
\begin{proof}
Siehe Übung.
\end{proof}

\begin{definition}
Die Vervollständigung des $C_0^\infty(\Omega)$ bzgl. der Norm $\Vert\cdot\Vert_{k,p,\Omega}$ wird mit $W_0^{k,p}(\Omega)$ bezeichnet. Außerdem setze $H_0^k(\Omega):=W_0^{k,2}(\Omega)$.
\end{definition}

\begin{definition}[Lipschitz-Rand]\enter
$\Omega$ hat einen \textbf{Lipschitz-Rand} $:\gdw\exists N\in\N,\exists U_1,\ldots,U_N\subseteq\R^d$ offen, sodass
\begin{enumerate}
\item $
\begin{aligned}
\partial\Omega\subseteq\bigcup\limits^N_{i=1} U_i
\end{aligned}$
\item $
\begin{aligned}
\forall i\in\lbrace1,\ldots,N\rbrace:\partial\Omega\cap U_i
\end{aligned}$
 ist darstellbar als Graph einer Lipschitz-stetigen Funktion
\end{enumerate}
Das Gebiet $\Omega$ wird dann \textbf{Lipschitz-Gebiet} genannt.
\end{definition}

\begin{bemerkung}
Für Lipschitz-Gebiete existiert fast überall auf $\partial\Omega$ der äußere Normalenvektor.
\end{bemerkung}

\begin{satz}[Spursatz]\enter
Seien $\Omega$ eine Lipschitz-Gebiet, $k\in\N$, $l\in\lbrace 0,\ldots,k-1\rbrace$.\\
Dann gibt es eine stetige lineare Abbildung
\begin{align*}
\gamma_l:W^{k,p}(\Omega)\rightarrow L^p(\partial\Omega)
\end{align*}
mit der Eigenschaft
\begin{align*}
\gamma_l(\varphi)=\frac{\partial^l}{\partial u^l}\varphi|_{\partial\Omega}\qquad\forall\varphi\in C^k(\overline{\Omega}).
\end{align*}
\end{satz}

\begin{bemerkung}\
\begin{itemize}
\item $\gamma_0$ bildet die Funktion $\varphi$, die auf $\Omega$ lebt, auf ihre Randdaten ab.
\item  $\gamma_1$ gibt die Normalenableitung zurück.
\item Dass Bild von $W^{k,p}(\Omega)$ unter $\gamma_l$ ist ein abgeschlossener Unterraum von $L^p(\partial\Omega)$, genauer:
\begin{align*}
\gamma_l\left(W^{k,p}(\Omega)\right)=W^{k-l-\frac{1}{p},p}(\partial\Omega)
\end{align*}
\end{itemize}
\end{bemerkung}

\begin{satz}
Sei $\Omega$ ein Lipschitz-Gebiet. Dann gilt:
\begin{align*}
W_0^{k,p}(\Omega)=\left\lbrace\varphi\in W^{k,p}(\Omega):
\forall l\in\lbrace0,\ldots,k-1\rbrace:\gamma_l(\varphi)=0\right\rbrace
\end{align*}
\end{satz}

\begin{definition}
Seien $(X,\Vert\cdot\Vert_X)$ und $(Y,\Vert\cdot\Vert_Y)$ zwei normierte Vektorräume.
\begin{enumerate}
\item Eine lineare Abbildung $A:X\to Y$ heißt \textbf{kompakt} $:\gdw$ das Bild der abgeschlossenen Einheitskugel in $X$ im Raum $Y$ relativkompakt ist.
\item $X$ heißt \textbf{stetig eingebettet} in $Y$, in Zeichen $X\hookrightarrow Y:\gdw X\subseteq Y$ und die kanonische Injektion $I:X\to Y,~\varphi\mapsto\varphi$ stetig ist.
\item $X$ heißt \textbf{kompakt eingebettet} in $Y$, in Zeichen $X\stackrel{C}{\hookrightarrow} Y:\gdw X\subseteq Y$ und die kanonische Injektion $I:X\to Y,~\varphi\mapsto\varphi$ kompakt ist.
\end{enumerate}
\end{definition}

\begin{bemerkung}\
\begin{enumerate}
\item $X\hookrightarrow Y\Longrightarrow\exists c>0:\forall\varphi\in X:\Vert\varphi\Vert_Y\leq c\cdot\Vert\varphi\Vert_X$
\item $X\stackrel{C}{\hookrightarrow} Y\Longrightarrow X\hookrightarrow Y$
\item Ist $\stackrel{C}{\hookrightarrow} Y$ und $(\varphi_n)_{n\in\N}\subseteq X$ eine beschränkte Folge in $X$, so besitzt $(\varphi_n)_{n\in\N}\subseteq Y$ eine konvergente Teilfolge.
\end{enumerate}
\end{bemerkung}

\begin{satz}[Einbettung]\
\begin{enumerate}
\item Sei $p<d$. Dann gilt:
\begin{align*}
W^{k,p}(\Omega)\hookrightarrow W^{k-1,p}(\Omega)\qquad\forall q\in\left[1,\frac{p\cdot d}{d-p}\right]\\
W^{k,p}(\Omega)\stackrel{C}{\hookrightarrow} W^{k-1,p}(\Omega)\qquad\forall q\in\left[1,\frac{p\cdot d}{d-p}\right)
\end{align*}
\item Sei $p=d$. Dann gilt:
\begin{align*}
W^{k,p}(\Omega)\hookrightarrow W^{k-1,p}(\Omega)\qquad\forall q\in\left[1,\infty\right)
\end{align*}
\item Sei $k>\frac{d}{p}$. Dann gilt:
\begin{align*}
W^{k,p}(\Omega)\stackrel{C}{\hookrightarrow} C^l(\overline{\Omega})\qquad\forall 0\leq l\leq k-\frac{d}{p}
\end{align*}
\end{enumerate}
\end{satz}

\begin{bemerkung}\
\begin{itemize}
\item Es gilt stets
\begin{align*}
W^{k,p}(\Omega)\stackrel{C}{\hookrightarrow} W^{k-1,p}(\Omega)
\end{align*}
\item Falls $p=2$, $d\in\lbrace2,3\rbrace$: 
\begin{align*}
	H^2(\Omega)\hookrightarrow C(\overline{\Omega})
\end{align*}
Für $H^1(\Omega)$-Funktionen sind Punkte weiter nicht sinnvoll.
\item Falls $p=2,~d=2$: 
\begin{align*}
H^1(\Omega\stackrel{C}{\hookrightarrow} L^q(\Omega)\qquad\forall q\in[1,\infty)
\end{align*}
\item Falls $p=2,~d=3$: 
\begin{align*}
&H^1(\Omega)\stackrel{C}{\hookrightarrow} L^q(\Omega)\qquad\forall q\in[1,6)\\
&H^1(\Omega)\stackrel{C}{\hookrightarrow} L^6(\Omega)
\end{align*}
\end{itemize}
\end{bemerkung}

\begin{beisp}
	Für $d=2$ gibt es $H^1(\Omega)$-Funktionen, die nicht stetig sind:
	\begin{itemize}
		\item Sei $\Omega=$ Einheitskreis
		\item $u(x,y):=\ln\left(\ln\left(\frac{4}{\sqrt{x^2+y^2}}\right)\right)$
	\end{itemize}
	$u$ hat einen Pol im Ursprung, d.h. $u\not\in C(\overline{\Omega})$, aber:\enter
	\ul{Behauptung:} 
	\[\|u\|^2_{1,2,\Omega}\leq 8\pi + \frac{2\pi}{\ln(4)}< \infty\]
	\[\implies u \in H^1(\Omega)\]
	\begin{proof}
	\[\|u\|^2_{1,2,\Omega} = \underbrace{\int_\Omega |u|^2 \d x \d y}_{:=\text{ I}} +
	\underbrace{\int_\Omega |u_x|^2 + |u_y|^2 \d x \d y}_{:=\text{ II}}\]
	Es gilt für alle $z\geq 4$:
	\[\ln(\ln(z))\leq \sqrt{z}\]
	Mit der Koflächenformel erhalten wir:
	\begin{align*}
		\text{I} &= \int_\Omega |u|^2 \d x \d y \\
			&\leq \int_\Omega \left(\frac{4}{\sqrt{x^2 + y^2}}\right)^2 \d x \d y \\
			&\stackeq{\text{Koflä.}} \int_0^{2\pi}\int_0^1\left(\frac{4}{r}\right)^2 r \d r \d \varphi \\
			&= 8 \pi
	\end{align*}
	Als nächstes betrachten wir II. Dafür schauen wir uns zunächst die partiellen Ableitungen von $u$ an:
	\begin{align*}
		u_x &= \frac{1}{\ln\left(\frac{4}{r}\right)}\frac{r}{4}\left(-\frac{4}{r^2}\right)\frac{x}{r} = - \frac{x}{r^2\ln\left(\frac{4}{r}\right)} \\
		u_y &= - \frac{y}{r^2\ln\left(\frac{4}{r}\right)} \\
		\implies u_x^2 + u_y^2 &= \frac{x^2+y^2}{\left(r^2\ln\left(\frac{4}{r}\right)\right)^2} \\
		&= \frac{r^2}{\left(r^2\ln\left(\frac{4}{r}\right)\right)^2} \\
		&= \frac{1}{r^2\left(\ln\left(\frac{4}{r}\right)\right)^2} \\
	\end{align*}
	Also schlussendlich:
	\begin{align*}
		\text{II} ~~ &\stackeq{\text{vgl. I}} \int_0^{2\pi}\int_0^1\frac{r}{r^2\left(\ln\left(\frac{4}{r}\right)\right)^2} \d r \d \varphi \\
			 &= 2\pi \int_0^1 \frac{1}{r^2\left(\ln\left(\frac{4}{r}\right)\right)^2} \d r \\
			 &=\left[2\pi\left(\frac{1}{\ln\left(\frac{4}{r}\right)}\right)\right]_{r=0}^1 \\
			 &= \frac{2\pi}{\ln(4)}
	\end{align*}
\end{proof}
\end{beisp}

\newpage
\begin{proposition}\label{prop1.11}
Sei $\Omega\subseteq\R^d$ ein beschränktes Lipschitz-Gebiet und sei $F$ lineares, stetiges Funktional auf $H^1(\Omega)$ und $B$ eine beschränkte, symmetrische Bilinearform auf $H^1(\Omega)$ mit
\begin{align*}
B(u,u)\geq0\qquad\forall u\in H^1(\Omega).
\end{align*}
Sei $g$ eine Funktion mit $g(x)=1~\forall x\in\overline{\Omega}$ so, dass
\begin{align*}
|F(g)|+\sqrt{B(g,g)}>0.
\end{align*}
Dann gilt:\\
Die Normen $\Vert\cdot\Vert_{1,2,\Omega}$ und
\begin{align*}
\Vertiii{f}:=|f|_{1,2,\Omega}+|F(f)|+\sqrt{B(f,f)}
\end{align*}
sind äquivalent auf $H^1(\Omega)$, d. h. es existieren $c_1,c_2>0$ so, dass
\begin{align*}
c_1\cdot\Vert f\Vert_{1,2,\Omega}\leq\Vertiii{f}
\leq
c_2\cdot\Vert f\Vert_{1,2,\Omega}
\qquad\forall f\in H^1(\Omega)
\end{align*}
\end{proposition}
\begin{proof}\enter
	\underline{Zeige $\Vertiii{\cdot}$ ist eine Norm:}
Dies wird in der Übung gezeigt.\enter \enter
\underline{Zeige $\Vertiii{f}\leq c_2\cdot\Vert f\Vert_{1,2,\Omega}$:}
\begin{align*}
\Vertiii{f} &=
|f|_{1,2,\Omega}+|F(f)|+\sqrt{B(f,f)}\\
&\leq
\Vert f\Vert_{1,2,\Omega} +c_F\cdot\Vert f\Vert_{1,2,\Omega}+\sqrt{c_B\cdot\Vert f\Vert^2_{1,2,\Omega}}\\
&\leq
(1+c_F+\sqrt{c_B})\cdot\Vert f\Vert_{1,2,\Omega}
\end{align*}

%(iii)
\underline{Zeige $\Vert f\Vert_{0,2,\Omega}\leq c\cdot\Vertiii{f}$ indirekt:}
Angenommen es gibt eine Folge $(f_n)_{n\in\N}\subseteq H^1(\Omega)$ mit
\begin{align}\label{proof1.9}\tag{$\ast$}
\Vert f_n\Vert_{0,2,\Omega}=1\text{ und }\Vertiii{f_n}<\frac{1}{n}~\forall n\in\N
\end{align}
Dann gilt für alle $n\in\N$:
\begin{align*}
|f_n|_{1,2,\Omega}<\frac{1}{n}
\implies
\Vert f_n\Vert^2_{1,2,\Omega}\leq 2\
\end{align*}
Deshalb ist $(f_n)_{n\in\N}$ eine beschränkte Folge in $H^1(\Omega)$. Wegen $H^1(\Omega)\stackrel{C}{\hookrightarrow} L^2(\Omega)$ gibt eine Teilfolge $(f_m)_{m\in\N}$, die in $L^2(H)$ konvergiert.
\begin{align*}
&\stackrel{\eqref{proof1.9}}{\implies}
(f_m)_{m\in\N}\text{ ist eine Cauchy-Folge in }H^1(\Omega)\\
&\implies
 f_m\stackrel{m\to\infty}{\longrightarrow} f\in H^1(\Omega)\\
 &\implies
 \Vert f\Vert_{0,2,\Omega}=1\text{ und } |f|_{1,2,\Omega}=0\\
 &\implies
 f\text{ ist konstant, also } f\equiv \lambda\cdot g\mit\lambda\in\R
\end{align*}
Mit $\Vertiii{f}=0$ erhalten wir
\begin{align*}
\Vertiii{f} = 0 
&=|f|_{1,2,\Omega}+|F(f)|+\sqrt{B(f,f)}\\
&=0+|\lambda|\cdot|F(g)|+|\lambda|\cdot\sqrt{B(g,g)}\\
&=|\lambda|\cdot\underbrace{\Big(|F(g)|+\sqrt{B(g,g)}\Big)}_{>0}\\
&\implies\lambda=0\implies f\equiv 0
\end{align*}
Das ist ein Widerspruch zur Annahme $\Vert f\Vert_{0,2,\Omega}=1$. Also folgt
\begin{align*}
\Vert f\Vert_{0,2,\Omega}\leq c\cdot\Vertiii{f}\qquad\forall f\in H^1(\Omega).
\end{align*}

\underline{Zeige $\|f\|_{1,2,\Omega}\leq c_1 \Vertiii{f}$}

\begin{align*}
\Vert f\Vert_{1,2,\Omega}
&\leq
|f|_{1,2,\Omega}+\Vert f\Vert_{0,2,\Omega}\\
&\leq
\Vertiii{f}+c\cdot\Vertiii{f}\\
&=c_1\cdot\Vertiii{f}\qquad\forall f\in H^1(\Omega)
\end{align*}
\end{proof}

\begin{korollar}[Poincaré-Ungleichung]\enter %1.12
Aus Proposition \ref{prop1.11} folgt mit
\begin{align*}
B(u,v)=0\quad\text{ und }\quad F(u)=\int\limits_\Omega u(x)\d x
\qquad\forall u,v\in H^1(\Omega)
\end{align*}
die Äquivalenz der Normen
\begin{align*}
\Vert\cdot\Vert_{1,2,\Omega}\text{ und } |\cdot|_{1,2,\Omega}+|F(\cdot)|.
\end{align*}
Setze
\begin{align*}
V:=\left\lbrace\varphi\in H^1(\Omega):\int\limits_\Omega u(x)\d x=0\right\rbrace.
\end{align*}
Dann sind sogar $\Vert\cdot\Vert_{1,2,\Omega}$ und $|\cdot|_{1,2,\Omega}$ äquivalent auf $V$.
\end{korollar}
\begin{proof}
Es ist zu zeigen, dass $F$ ein lineares stetiges Funktional auf $H^1(\Omega)$  ist. Das ist einfach.
\end{proof}

\begin{proposition}[Friedrichs Ungleichung]\label{prop1.13FriedrichsUngleichung}\enter
$|\cdot|_{k,p,\Omega}$ ist eine Norm auf $W_0^{k,p}(\Omega)$, welche äquivalent zu der Norm $\Vert\cdot\Vert_{k,p,\Omega}$ ist.\\

Bemerkung: Für $k=1,~p=2$ ist das eine direkte Folgerung auf Proposition \ref{prop1.11} mit 
\begin{align*}
B(u,v)\equiv0\text{ und } F(u)=\int\limits_{\partial\Omega} u\d \gamma.
\end{align*}
\end{proposition}

\section{Abstrakte Variations-Analysis}
Die \textbf{Poissongleichung} ist
\begin{align}
	\left\{
	\begin{array}{r l l}\label{PoissonEquation}\tag{Poi}
		-\laplace u &\equiv f &\text{ auf }\Omega \\
		u &\equiv 0 &\text{ auf }\partial\Omega=:\Gamma.
	\end{array}\right.
\end{align}
Durch Multiplikation mit $v$ und Integration erhalten wir
\begin{align*}
&\stackrel{ v|_{\partial\Omega}\equiv0}{\implies}
\int\limits_\Omega f\cdot v\d x
=\int\limits_\Omega-\laplace u\cdot v\d x
=\int\limits_\Omega\nabla u\cdot\nabla v\d x-\underbrace{\int\limits_{\partial\Omega}\frac{\partial u}{\partial n}v\d x}_{=0}
\end{align*}
Definiere:
\begin{align*}
\text{Bilinearform } a(u,v)&:=\int\limits_\Omega\nabla u\cdot\nabla v\d x\\
\text{Linearform } l(v)&:=\int\limits_\Omega f\cdot v\d x
\end{align*}
$a$ und $l$ sind wohldefiniert für $f\in L^2(\Omega),~u,v\in H^1(\Omega)$. Um $v|_{\partial\Omega}=0$ und $u|_{\partial\Omega}=0$ sicherzustellen, muss $u,v\in H^1_0(\Omega)$ gelten.\\

\textbf{Variationsformulierung der Poissongleichung:}\\
Finde $u\in H_0^1(\Omega)$ so, dass
\begin{align*}
a(u,v)=l(v)\qquad\forall v\in H^1_0(\Omega).
\end{align*}
Dieses Problem gilt es zu lösen. Allgemeiner formuliert erhält man das folgende Problem:\\
Sei $V$ ein Hilbertraum. Finde $u\in V$ so, dass
\begin{align*}
a(u,v)=l(v)\qquad\forall v\in V.
\end{align*}

Eigenschaften der oben definieren Bilinearform $a$:
\begin{itemize}
\item $a$ ist stetig auf $V$, d. h.
\begin{align*}
\exists M>0:\forall u,v\in V:\big|a(u,v)\big|\leq M\cdot\Vert u\Vert_V\cdot\Vert v\Vert_V
\end{align*}
denn:
\begin{align*}
\big| a(u,v)\big|=\left|\int\limits_\Omega\nabla u\cdot\nabla v\d x\right|
\leq
|u|_{1,2,\Omega}\cdot|v|_{1,2,\Omega}
\leq
\underbrace{1}_{=:M}\cdot\Vert u\Vert_{1,2,\Omega}\cdot\Vert v\Vert_{1,2,\Omega}
\end{align*}
\item $a$ ist \textbf{koerziv}, d. h.
\begin{align*}
\exists\alpha>0:\forall v\in V:a(v,v)\geq\alpha\cdot\Vert v\Vert^2_V
\end{align*} 
denn:
\begin{align*}
a(v,v)
=\int\limits_\Omega\nabla v\cdot\nabla v\d x
=\int\limits_\Omega\sum\limits_{i=1}^d\left(\frac{\partial v}{\partial x_i}\right)^2\d x
=|v|^2_{1,2,\Omega}
\stackrel{\text{Fried}}{\geq}
\frac{1}{c_F^2}\cdot\Vert v\Vert^2_{1,2,\Omega}
\end{align*}
\end{itemize}
Eigenschaften der oben definierten Linearform $l$:
\begin{itemize}
\item $l$ ist stetig auf $V$, d. h.
\begin{align*}
\exists M^\ast>0:\forall v\in V: \big|l(v)\big|\leq M^\ast\cdot\Vert v\Vert_V
\end{align*}
denn:
\begin{align*}
\big| l(v)\big|=\left|\int\limits_\Omega f\cdot v\d x\right|
\stackrel{\text{CS}}{\leq}
\Vert f\Vert_{0,2,\Omega}\cdot\Vert v\Vert_{0,2,\Omega}
\stackrel{\text{CS}}{\leq}
\underbrace{\Vert f\Vert_{0,2,\Omega}}_{=:M^\ast}\cdot\Vert v\Vert_{1,2,\Omega}
\end{align*}
\end{itemize}

\begin{theorem}[Lax-Milgram]\label{theorem2.1LaxMilgram}\enter
Sei $V$ ein Hilbertraum, $a$ eine stetige, koerzive Bilinearform auf $V$ und $b$ eine stetige Linearform auf $V$.\\
Dann hat das Variationsproblem
\begin{align*}
\text{Bestimme $u\in V$ so, dass }a(u,v)=b(v)\qquad\forall v\in V.
\end{align*}
eine eindeutige Lösung.
\end{theorem}
\begin{proof}
\underline{Schritt 1:}\\ Transformation in einen dualen Operator über dem Dualraum $V^\ast$ von $V$. Die Abbildung $v\mapsto a(u,v)$ definiert für alle festen $u\in V$ eine stetige Linearform auf $V$. Wir bezeichnen diese mit $Au\in V^\ast$
\begin{align*}
	\langle Au,v\rangle=(Au)(v):&=a(u,v)\qquad\forall u,v\in V \\
	b\in V^\ast: \langle b,v\rangle&=b(v)
\end{align*}
Damit folgt:
\begin{align*}
\text{Variationsproblem}&\Longleftrightarrow\langle Au,v\rangle=\langle b,v\rangle\qquad\forall v\in V\\
&\Longleftrightarrow Au=b\text{ in }V^\ast
\end{align*}
\underline{Schritt 2:}\\
Die Abbildung $V\to V^\ast$ kann auch als Abbildung $A:V\to V$ aufgefasst werden. Eigenschaften von $A$:
\begin{enumerate}[label=(\alph*)]
\item $A$ ist linear auf $V$
\begin{align*}
\big\langle A(\alpha u+\beta v),w\big\rangle
&=a(\alpha u+\beta v,w)\\
&=\alpha a(u,w)+\beta a(v,w)\\
&=\alpha\langle Au,w\rangle+\beta\langle Av,w\rangle\qquad\forall u,v,w,\in V
\end{align*}
\item $A$ ist stetig auf $V$:
\begin{align*}
\Vert A v\Vert_{V}
&=\Vert A u\Vert_{V^\ast}\\
&=\sup\limits_{v\neq0}\frac{\big|\langle Au,v\rangle\big|}{\Vert v\Vert_V}\\
&=\sup\limits_{v\neq0}\frac{\big|a(u,v)\big|}{\Vert v\Vert_V}\\
&\leq
\sup\limits_{v\neq0}\frac{M\cdot\Vert u\Vert_V\cdot\Vert v\Vert_V}{\Vert v\Vert_V}\\
&=M\cdot\Vert u\Vert_V
\end{align*}
\end{enumerate}
\underline{Schritt 3:} Setze
\begin{align*}
u^{n+1}:=P(u^n):=u^n+\tau\cdot(b-A u^n)\qquad\forall\tau\in\R_{\neq0}
\end{align*}
$u^0$ ist eine beliebiger Startwert.
\begin{align*}
P:V\to V,\qquad u\mapsto u+\tau\cdot(b-Au)
\end{align*}
Jeder Fixpunkt von $P$ ist eine Lösung von $Au=b$ und umgekehrt.\\
Es bleibt also zu zeigen, das $P$ einen eindeutigen Fixpunkt hat. Dafür nutzen wir den \textit{Banach'schen Fixpunktsatz} \ref{BanachscherFixpunktsatz}.\\

Wähle also $N=V$. $P$ ist eine Kontraktion, wenn $\tau$ richtig gewählt wird.
\begin{align*}
\Vert P(u)-P)v)\Vert^2_V
&=\big\langle P(u)-P(v),P(u)-P(v)\big\rangle_V\\
&=\big\langle u-v-\tau\cdot A(u-v),u-v-\tau\cdot A(u-v)\big\rangle_V\\
&=\langle u-v,u-v\rangle_V-2\cdot\tau\cdot\underbrace{\big\langle A(u-v),u-v\big\rangle_V}_{\stackrel{\text{Def.}}{=}a(u-v,u-v)\stackrel{\text{koerz}}{\geq}\alpha\cdot\Vert u-v\Vert^2_V}+\tau^2\cdot\underbrace{\big\langle A(u-v),A(u-v)\big\rangle_V}_{=\Vert A(u-v)\Vert^2_V\leq M^2\cdot\Vert u-v\Vert^2_V}\\
&\stackrel{\tau>0}{\leq}
\Vert u-v\Vert^2_V\cdot\big(1-2\cdot\tau\cdot\alpha+\tau^2\cdot M^2\big)
\end{align*}
Wähle $\tau:=\frac{\alpha}{M^2}>0$. Dann gilt:
\begin{align*}
1-2\cdot\tau\cdot\alpha+\tau^2\cdot M^2=1-\frac{\alpha^2}{M^2}>0
\end{align*}
$P$ ist eine Kontraktion, also hat $P$ nach dem Banachschen Fixpunktsatz einen eindeutigen Fixpunkt und damit hat auch $Au=b$ eine eindeutige Lösung.
\end{proof}

Schwierigkeit:
\begin{align*}
a(u,v)=b(v)\qquad\forall v\in V
\end{align*}
ist ein unendlichdimensionales Problem.\\
\underline{Idee:} Approximation in einem endlich-dimensionalen Unterraum\\
Sei $V_h\subseteq V$ ein endlich-dimensionaler Unterraum von $V$. Betrachte das Problem
\begin{align*}
\text{Finde }u_n\in V_h \text{ so, dass }\qquad \\
a(u_n,v_h)=l(v_h)\qquad\forall v_h\in V_h
\end{align*}
Da $a$ und $l$ auf $V_h$ die Voraussetzungen des Satzes von Lax-Milgram auf $V_h$ erfüllen, gibt es ein eindeutiges $u_n\in V_h$, das das Problem löst.\\

Sei $\lbrace\varphi_i\rbrace_{i=1}^n$ eine Basis von $V_h$.
\begin{align*}
\Big(a(u_n,v_h)=l(v_h)\qquad\forall v_h\in V_h\Big)\Longleftrightarrow
\Big(a(u_n,\varphi_i)=l(\varphi_i)\qquad\forall i\in\lbrace1,\ldots,n\rbrace\Big)
\end{align*}
und 
\begin{align*}
u_n=\sum\limits_{j=1}^n u_j\cdot\varphi_j,\qquad u_1,\ldots,u_n\in\R,j\\
\sum\limits_{j=1}^n a(\varphi_j,\varphi_i)\cdot u_j=l(\varphi_i)\qquad\forall i\in\lbrace1,\ldots,n\rbrace\\
\end{align*}
Daraus ergibt sich ein lineares Gleichungssystem:
\begin{align*}
\left.\begin{array}{ll}
A:=(a_{i,j})_{i,j=1,\ldots,n} &a_{i,j}:=a\big(\varphi_j,\varphi_i\big)\\
b:=(b_i)_{i=1,\ldots,n} &b_i:=l(\varphi_i)\\
u:=(u_j)_{j=1,\ldots,n}
\end{array}\right\rbrace A\cdot u=b\qquad\text{ LGS}
\end{align*}

\begin{theorem}[Céa's Lemma]\enter
Sei $V$ ein Hilbertraum, $V_h\subseteq V$ ein abgeschlossener Unterraum, $a$ eine stetige, koerzive Bilinearform auf $V$ und $l$ eine stetige Linearform. Dann gilt:
\begin{align*}
\exists C=\frac{M}{\alpha}>0:\Vert u-u_n\Vert_V\leq C\cdot\inf\limits_{v_h\in V_h}\Vert u-v_h\Vert_V
\end{align*}
\end{theorem}
\begin{proof}
\begin{align*}
a(u,v)&=l(v) &\forall v\in V\\
a(u_n,v_h)&= l(v_h) &\forall v_h\in V_h\subseteq V
\end{align*}
Differenz: 
\begin{align*}
a(u-u_n,v_h)=\qquad v_h\in V_h
\end{align*}
Also:
\begin{align*}
\alpha\cdot\Vert u-u_n\Vert^2_V
&\leq
a(u-u_n,u-u_n)\\
&=a(u-u_n,u-v_v+v_h- u_n)\\
&=a(u-u_n,u-v_h)+\underbrace{a(u-u_n,\underbrace{v_h-u_n}_{\in V_h})}_{=0}\\
&\stackrel{\text{stetig}}{\leq}
M\cdot\Vert u-u_n\Vert_V\cdot\Vert u-v_h\Vert_V\\
\implies
\Vert u-u_n\Vert_V&\leq\frac{M}{\alpha}\cdot\Vert u-v_h\Vert_V\qquad\forall v_h\in V_h\\
\implies
\Vert u-u_n\Vert_V&\leq \frac{M}{\alpha}\cdot\inf\limits_{v_h\in V_h}\Vert u-v_h\Vert_V
\end{align*}
%ToDo: v_h ist eigentlich v_h -.- Das habe ich beim Abschreiben verkackt.
\end{proof}

\begin{theorem}[Dualitätsargument, Aubin-Nietsche]\enter
Sei $V$ ein Hilbertraum, $V_h\subseteq V$ ein abgeschlossener Unterraum, $a$ eine stetige, koerzive Bilinearform und $l$ eine stetige Linearform. Sei weiterhin $H$ ein Hilbertraum mit Skalarprodukt $\langle\cdot,\cdot\rangle_H$ und Norm $\Vert\cdot\Vert_H$ so, dass $V\hookrightarrow H$ und $V$ liegt dicht in $H$ bzgl. $\Vert\cdot\Vert_H$. Sei zusätzlich $u_\varphi\in V$ für jedes feste $\varphi\in H$ die eindeutige Lösung von
\begin{align}\label{Dtheorem2.3}\tag{D}
a(v,u_\varphi)=\langle\varphi,v\rangle_H\qquad\forall v\in V
\end{align}
Dann gilt:
\begin{align*}
\Vert u-u_h\Vert_H\leq M\cdot\Vert u-u_h\Vert_V\cdot\sup\limits_{\begin{subarray}{c}\varphi\in H\\\Vert\varphi\Vert_H=1\end{subarray}}\inf\limits_{v_h\in V}\Vert u_\varphi-v_h\Vert_V
\end{align*}
\end{theorem}

\begin{proof}
$v\mapsto\langle\varphi,v\rangle_H$ ist eine stetige Linearform auf $V$. Aus Lax-Milgram \ref{theorem2.1LaxMilgram} folgt, dass es eine eindeutige Lösung $u_\varphi\in V$ gibt.
 
\begin{align*}
\langle u-u_h,\varphi\rangle_H
&=\langle\varphi,u-u_h\rangle_H\\
&=a(u-u_h,u_\varphi)\\
&=a(u-u_h,u_\varphi-v_h)\mit v_h\in V\text{ beliebig}\\
&\leq M\cdot\Vert u-u_h\Vert_V\cdot\Vert u_\varphi-v_h\Vert_V\\
\Vert u-u_h\Vert_H
&=\sup\limits_{\begin{subarray}{c}\varphi\in H\\\Vert\varphi\Vert=1\end{subarray}}\langle u-u_h,\varphi\rangle_H\\
&\leq M\cdot\Vert u-u_h\Vert_V\cdot\sup\limits_{\begin{subarray}{c}\varphi\in H\\\Vert\varphi\Vert=1\end{subarray}}\inf\limits_{v_h\in V}\Vert u_\varphi-v_h\Vert_V
\end{align*}
\end{proof}

\begin{theorem}\label{theorem2.4} %2.4
Seien $(X,\Vert\cdot\Vert_X)$, $(Y,\Vert\cdot\Vert_Y)$ Banachräume  mit $X\stackrel{C}{\hookrightarrow} Y$ und $a_0,a_1$ stetige Bilinearformen auf $X$. Gelte weiterhin:
\begin{itemize}
\item $a_0$ ist symmetrisch und koerziv auf $X$
\item $\begin{aligned}
\exists A>0:\forall u,v\in X:\big|a_1(u,v)\big|\leq A\cdot\Vert u\Vert_X\cdot\Vert v\Vert_Y
\end{aligned}$
\end{itemize}
Erfülle
\begin{align*}
a(u,v):=a_0(u,v)+a_1(u,v)\qquad\forall u,v\in X
\end{align*}
die Ungleichung
\begin{align*}
a(v,v)>0\qquad\forall v\in X\setminus\lbrace0\rbrace.
\end{align*}
Dann haben die beiden Probleme
\begin{align*}
\text{Finde }u\in X\text{ so, dass }
a(u,v)=l(v)\qquad\forall v\in X\\
\text{Finde }\tilde{u}\in X\text{ so, dass }
a(v,\tilde{u})=l(v)\qquad\forall v\in X
\end{align*}
jeweils eindeutig lösbar für alle stetigen Linearformen $l$.
\end{theorem}

\begin{theorem} %2.5
Gleiche Voraussetzungen wie Theorem \ref{theorem2.4}.\\
Sei zusätzlich $X_h\subseteq X$ ein endlich-dimensionaler Unterraum von $X$.\\
Dann ist das Problem
\begin{align*}
\text{Finde }u_h\in X_h\text{ so, dass }
a(u_h,v_h)=l(v_h)\qquad\forall v_h\in X_h
\end{align*}
eindeutig lösbar für jede feste stetige Linearform $l$. Zusätzlich gilt
\begin{align*}
\Vert u-u_h\Vert_X\leq\frac{M}{\alpha}\cdot\inf\limits_{v_h\in X_h}\Vert u-u_h\Vert_X
\end{align*}
unter der Voraussetzung
\begin{align*}
a(v,v)&\geq\beta\cdot\Vert v\Vert^2_X &\forall v\in X\\
\big|a(u,v)\big|&\leq M\cdot\Vert u\Vert_X\cdot\Vert v\Vert_X &\forall u,v\in X
\end{align*}
\end{theorem}

\section{Schwache Lösungen} %3.
Hier: skalare, lineare, elliptische partielle Differentialgleichungen zweiter Ordnung, also
\begin{align*}
-\sum\limits_{i,j=1}^d\frac{\partial}{\partial x_i}\left(A_{i,j}(x)\cdot\frac{\partial u}{\partial x_j}\right)+\sum\limits_{i=1}^d a_i(x)\cdot\frac{\partial u}{\partial x_i}+\alpha(x)\cdot u=f(x)\text{ in }\Omega
\end{align*}
Vereinfachende Annahmen:
%RobertToDo:
\begin{align*}
f\in L^2(\Omega),\alpha \in C(\Omega), a_i\in C^1(\Omega),i\in\lbrace1,\ldots,d\rbrace\\
A_{i,j}\in C^1(\Omega), i,j\in\lbrace1,\ldots,d\rbrace, A_{i,j}(x)=A_{j,i}(x)\forall x\in\Omega, i,j\in\lbrace1,\ldots,d\rbrace\\
a=\big(a_1,\ldots a_d\big)^T, A=\big(A_{i,j}\big)^d_{i,j=1}\\
0<\lambda_0:=\inf\limits_{x\in\Omega}\inf\limits_{z\in\R^d\setminus\lbrace0\rbrace}\frac{z^T\cdot A(x)\cdot z}{z^T\cdot z}
\end{align*}

\subsection*{Einige Spezialfälle}
\begin{itemize}
\item Poisson-Gleichung: $\alpha\equiv 0,~a\equiv0,~A=I$
\item Membran-Gleichung: $\alpha\equiv0,~a\equiv 0$
\item Reaktionsdiffusionsgleichung: $a\equiv 0$
\item Konvektions-Diffusions-Gleichung: ist der allgemeine Fall
\end{itemize}

\subsection*{Randbedingungen}
\begin{itemize}
\item Die \textbf{homogene Dirichlet-Randbedingung} ist
\begin{align*}
u\equiv0\qquad\text{auf}\qquad\Gamma:=\partial\Omega
\end{align*}
\item Die \textbf{Neumann-Randbedingung} ist
\begin{align*}
n^T\cdot A\cdot\nabla u\equiv q\qquad\text{auf}\qquad\Gamma,\qquad\forall g\in L^2(\Gamma)
\end{align*}
wobei $n$ der äußere Normalenvektor ist.
\item Mischform: Dirichlet-Randbegingung auf $\Gamma_D$ und Neumann-Randbedingung auf $\Gamma_N$ für $\Gamma=\Gamma_D\stackrel{\cdot}{\cup}\Gamma_N,~\Gamma_D\cap\Gamma_N=\emptyset$
\item Die \textbf{Robin-Randbedingung} ist
\begin{align*}
\gamma\cdot u+n^T\cdot A\cdot\nabla u\equiv h\qquad\text{auf}\qquad\Gamma
\end{align*}
\end{itemize}

%Was ist das hier?
Sei $v\in C_0^\infty(\Omega)$. Dann gilt:
\begin{align*}
\int\limits_\Omega f(x)\cdot v\d x
&\stackeq{\text{Def}}
\int\limits_\Omega-\sum\limits_{i,j=1}^d\frac{\partial}{\partial x_i}\left(A_{i,j}(x)\cdot\frac{\partial u}{\partial x_j}\right)\cdot v+\sum\limits_{i=1}^d a_i(x)\cdot\frac{\partial u}{\partial x_i}\cdot v+\alpha(x)\cdot u\cdot v\d x\\ 
&\stackeq{\text{partGauß}}
\int\limits_\Omega\sum\limits_{i,j=1}^d A_{i,j}(x)\cdot\frac{\partial u}{\partial x_j}\cdot\frac{\partial v}{\partial x_i}\d x+\int\limits_\Omega\sum\limits_{i=1}^d a_i(x)\cdot\frac{\partial u}{\partial x_i}\cdot v+\alpha(x)\cdot u\cdot v\d x\\
&\qquad
-\underbrace{\int\limits_\Gamma\underbrace{\sum\limits_{i,j=1}^d n_i\cdot A_{i,j}(x)\cdot\frac{\partial u}{\partial x_j}}_{=n^T\cdot A\cdot\nabla u}\cdot v\d\gamma}_{=0}\\
&=
\int\limits_\Omega\nabla u^T\cdot A\cdot\nabla v+a\cdot\nabla u\cdot v+\alpha\cdot u\cdot v\d x
\end{align*}

\begin{definition}\ %3.1
\begin{enumerate}[label=(\roman*)]
\item Eine Funktion $u\in H_0^1(\Omega)$ heißt \textbf{schwache Lösung einer Konvektions-Diffusions-Gleichung mit homogener Dirichlet-Randbedingung}
\begin{align*}
:\Longleftrightarrow
\int\limits_\Omega\nabla u^T\cdot A\cdot\nabla v+a\cdot\nabla u\cdot v+\alpha\cdot u\cdot v\d x
=\int\limits_\Omega f(x)\cdot v\d x
\qquad\forall v\in H_0^1(\Omega)
\end{align*}
\item Eine Funktion 
\begin{align*}
u\in H_D^1(\Omega):=\big\lbrace\varphi\in H^1(\Omega):\varphi|_{\Gamma_D}\equiv 0\big\rbrace
\end{align*}
heißt \textbf{schwache Lösung einer Konvektions-Diffusions-Gleichung  mit gemischten Randbedingungen}
\begin{align*}
:\Longleftrightarrow
\int\limits_\Omega \nabla u^T\cdot A\cdot\nabla v+a\cdot\nabla u\cdot v+\alpha\cdot u\cdot v\d x
=\int\limits_\Omega f(x)\cdot v\d x+\int\limits_{\Gamma_N}g\cdot v\d\gamma
\qquad\forall v\in H^1_D(\Omega)
\end{align*}
\item Eine Funktion $u\in H^1(\Omega)$ heißt \textbf{schwache Lösung einer Konvektions-Diffusions-Gleichung mit Neumann-Randbedingung}
\begin{align*}
:\Longleftrightarrow
\int\limits_\Omega\nabla u^T\cdot A\cdot\nabla v+a\cdot\nabla u\cdot v+\alpha\cdot u\cdot v\d x
=\int\limits_\Omega f(x)\cdot v\d x+\int\limits_{\Gamma} g\cdot v\d\gamma
\qquad\forall v\in H^1(\Omega)
\end{align*}
\end{enumerate}
\end{definition}

\begin{bemerkung} %NoNumber
\begin{itemize}
\item Jede ``klassische'' Lösung ist auch eine schwache Lösung. 
\item Jede schwache Lösung, welche glatt genug ist ($C^2$) ist auch eine ``klassische'' Lösung.
\item schwächere Annahmen an die Problemdaten:
\begin{align*}
&\alpha\in L^\infty(\Omega), &\alpha\geq0\\
&a_i\in L^\infty(\Omega), &\div(a)=\sum\limits_{i=1}^d\frac{\partial a_i}{\partial x_i}\in L^\infty(\Omega)\\
&A_{i,j}\in L^\infty(\Omega)
\end{align*}
\item Unter der inhomogenen Dirichletbedingung
\begin{align*}
u|_\Gamma\equiv u_D
\end{align*}
gehört die schwache Lösung zu
\begin{align*}
u_D+H_0^1(\Omega)=\big\lbrace v\in H^1(\Omega):v|_\Gamma\equiv u_D\big\rbrace
\end{align*}
Der Testraum ist immer $H^1_0(\Omega)$.
\end{itemize}
\end{bemerkung}

\begin{theorem}[Existenz und Eindeutigkeit von schwachen Lösungen]\ 
\begin{enumerate}[label=(\roman*)]
\item Sei 
\begin{align*}
\alpha-\frac{1}{2}\cdot\div(a)\geq0.
\end{align*}
Dann hat die Konvektions-Diffusions-Gleichung mit homogenen Dirichlet-Randbedingungen eine eindeutige schwache Lösung.
\item Sei 
\begin{align*}
\alpha-\frac{1}{2}\cdot\div(a)\geq0\qquad\text{ und }\qquad a\cdot n\geq0\text{ auf }\Gamma_N.
\end{align*}
Dann hat die Konvektions-Diffusions-Gleichung mit gemischten Randbedingungen eine eindeutige Lösung.
\item Sei 
\begin{align*}
\alpha-\frac{1}{2}\cdot\div(a)\geq0,\qquad \alpha\geq\alpha_0>0\qquad\text{ und }\qquad a\cdot n\geq 0\text{ auf }\Gamma.
\end{align*}
Dann hat die Konvektions-Diffusions-Gleichung mit Neumann-Randbedingung eine eindeutige Lösung.
\item Sei
\begin{align*}
\alpha=0,\qquad-\frac{1}{2}\cdot\div(a)=0\qquad\text{und}\qquad a\cdot n\geq 0\text{ auf }\Gamma\\
\text{und gelte}\qquad\int\limits_\Omega f(x)\d x+\int\limits_\Gamma g\d\gamma=0.
\end{align*}
Dann hat die Konvektions-Diffusions-Gleichung mit Neumann-Randbedingung eine eindeutige schwache Lösung $u$, welche 
\begin{align*}
\int\limits_\Omega u(x)\d x=0
\end{align*}
erfüllt.
\end{enumerate}
\end{theorem}
