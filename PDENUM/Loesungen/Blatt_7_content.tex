% This work is licensed under the Creative Commons
% Attribution-NonCommercial-ShareAlike 4.0 International License. To view a copy
% of this license, visit http://creativecommons.org/licenses/by-nc-sa/4.0/ or
% send a letter to Creative Commons, PO Box 1866, Mountain View, CA 94042, USA.
% vim: set noexpandtab:

\section{Aufgabenblatt 7}
\subsection*{Aufgabe 6.3}
Betrachte
\begin{align*}
	-ε Δ u + b·∇u + cu = f \text{ in } u=0 \text{ auf } \Rand{Ω}
\end{align*}
mit $0 < ε \ll 1$ unter der Annahme
\begin{align*}
	c-\frac12 \div(b)\geq c_0 > 0 \text{ in } Ω
\end{align*}
Weiter untersuchen wir das Standard-Galerkin-FEM Verfahren auf einer regulären, affin äquivalenten Zerlegung von $Ω$. 
Es ist bekannt, dass die entstehende Bilinearform $a$ koerziv auf $H_0^1(Ω)$ bezüglich der Norm
\begin{align*}
	\norm{·} \coloneq \left(ε \halfnorm{·}_{1, 2, Ω}^2+ \norm{·}_{0, 2, Ω}^2\right)^{\frac12}
\end{align*}
ist.
\subsubsection*{Aufgabe 6.3 (a)}
Warum ist auf dem Standardweg (mit Hilfe des Céa-Lemmas \ref{theorem2.2CeasLemma}) in diesem Fall keine von $ε$ unabhängige Fehlerschätzung in der $\norm{·}_{ε}$-Norm möglich?

\begin{lösung}
	Da die Rechnung mit der Divergenz des Vektorfeldes $b$ zwei Mal auftaucht,
	erwähne ich sie hier.%
  %
	\begin{lemma}[Divergenz-Lemma]\enter \label{thm:aufg6.3-divergenz_lemma}
		Für $b ∈ C^1(Ω)$ und $u, v ∈ H_0^1(Ω)$ gilt
		\begin{align*} 
			∫_{Ω} b · ∇u v &= - ∫_{Ω} u b · ∇v + \div b u v \\
			\text{und } ∫_Ω \left(b·∇u\right) u &= -\frac 12 ∫_{Ω} \div b \abs u^2.
		\end{align*}
	\end{lemma}

	\begin{proof}
		Die Produktregel für $b_i u$ ($i$ durchläuft die Koordinaten in $Ω$) ergibt
		$(b_i u)_{x_i} = (b_i)_{x_i} u + b_i u_{x_i}$, also
		$(b_i u)_{x_i} - (b_i)_{x_i} u = b_i u_{x_i}$.
		Mit partieller Integration sehen wir für $u, v ∈ H_0^1(Ω)$
		\begin{align*}
			∫_{Ω} \left(b · ∇u\right) v
			= ∫_{Ω} Σ_i b_i u_{x_i} v
			&= ∫_{Ω} Σ_i {(b_i u)}_{x_i} v - {(b_i)}_{x_i} u v
			= - ∫_{Ω} Σ_i b_i u v_{x_i} + {(b_i)}_{x_i} u v \\
			&= - ∫_{Ω} u \left(b · ∇v\right) + \div(b) u v. \\
			\intertext{Für den Fall $u = v$ vereinfacht sich dies zu}
			∫_{Ω} \left(b · ∇u\right) u
			&= - ∫_{Ω} \left(b · ∇u\right) u + \div b \abs u^2 \\
			⇒ ∫_Ω \left(b·∇u\right) u &= -\frac 12 ∫_{Ω} \div b \abs u^2. \qedhere
		\end{align*}
	\end{proof}

	Die Bedingung an $c$ und $b$ garantiert, dass es eine eindeutige Lösung der
	Gleichung gibt. Céas Lemma \ref{theorem2.2CeasLemma} gibt uns eine Fehlerschranke,
	die von der Stetigkeits- und Koerzivitätsschranke der Bilinearform abhängig ist.
	Die Bilinearform $a$ ist
	\begin{align*}
		a(u, v) = ∫_{Ω} ε ∇u · ∇v + b·∇u v + cuv \quad u, v ∈ H_0^1(Ω)
	\end{align*}
	Um die Stetigkeit von $a$ nachzuweisen, rechne, wie üblich:
	\begin{align*}
		\abs {a(u, v)} & \leq \left(∫_{Ω} ε \abs{∇ u}^2 \right)^{\frac12}
		\left(∫_{Ω} ε \abs{∇ v}^2 \right)^{\frac12}
		+ \norm{b}_{∞} \left(∫_{Ω} \abs{∇u}^2\right)^{\frac12}
		\left(∫_{Ω} \abs{v}^2\right)^{\frac12} \\
		&\tab + \norm{c}_{∞} \left(∫_{Ω} \abs{u}^2\right)^{\frac12}
		\left(∫_{Ω} \abs{v}^2\right)^{\frac12} \\
		&= ε^{\frac12} \halfnorm{u}_{1, 2, Ω} ε^{\frac12} \halfnorm{v}_{1,2,Ω}
		+ \norm{b}_{∞} \halfnorm{u}_{1, 2, Ω} \norm{v}_{0, 2, Ω}
		+ \norm{c}_{∞} \norm{u}_{0, 2, Ω} \norm{v}_{0, 2, Ω} \\
		&\leq ε^{-\frac12}\tilde M(b, c) \left(ε \halfnorm{u}_{1, 2, Ω} + \norm{u}_{0, 2, Ω} \right)^{\frac12}
		\left(ε \halfnorm{v}_{1, 2, Ω} + \norm{v}_{0, 2, Ω}\right)^{\frac12} \\
		&= M(ε, b, c) \norm{u}_{ε} \norm{v}_{ε}.
	\end{align*}
	Die Stetigkeitskonstante $M$ ist also von $ε$ abhängig und steigt mit fallendem $ε$.
	Das wird auch nicht durch die Koerzivitätskonstante $α$ ausgeglichen.
	Berechne diese
	\begin{align*}
		a(u, u)
		&= ∫_{Ω} ε \abs{∇u}^2 + b · ∇u u + c u^2 \\
		\overset{\ref{thm:aufg6.3-divergenz_lemma}}&=
		ε \halfnorm u_{1, 2, Ω}^2 + ∫_{Ω} \left(c - \frac12 \div b\right) \abs u^2 \\
		\overset{\text{Vor.}}&{\geq}
		ε \halfnorm u_{1, 2, Ω}^2 + c_0 \norm{u}_{0, 2, Ω}^2
		\geq \min\{1, c_0\} \norm u_ε.
	\end{align*}
	Die Koerzivitätskonstante ist also $α = \min\{1, c_0\}$ und damit von $ε$ unabhängig.
	Damit ergibt Céas Lemma keine $ε$-unabhängige Abschätzung, denn
	es besagt
	\begin{align*}
		\norm{u - u_h}_{ε} \leq \frac M{α} \inf_{v_h ∈ V} \norm{u - v_h}_{ε}.
	\end{align*}
\end{lösung}

\subsubsection*{Aufgabe 6.3 (b)}
Beweisen Sie stattdessen Term für Term für lineare Elemente eine Fehlerabschätzung der Form
\begin{align*}
	\norm {u-u_h}_{ε} \leq C \left(ε^{\frac12}h + h + h^2 \right)\halfnorm{u}_{2, 2, Ω}.
\end{align*}

\begin{proof}
	Wir nutzen die Galerkin-Orthogonalität.
	%TODO
\end{proof}

\subsection{Aufgabe 7.1}
Sei $\Omega\subseteq\R^d$ ein beschränktes Gebiet mit Lipschitz-Rand $\partial\Omega$.
Die primale gemischte Methode für die Poisson-Gleichung
\begin{align*}
	-\Delta u=-\div(\grad((u))=f\text{ in }\Omega
\end{align*}
mit homogenen Dirichlet-Randbedingungen $u=0$ auf $\partial\Omega$ ist gegeben durch:\nl
Finde $(\sigma,u)\in L^2(\Omega)^d\times H_0^1(\Omega)$ so, dass
\begin{align*}
	\begin{array}{rll}
		(\sigma,\tau)_{0,\Omega}-(\tau,\nabla u)_{0,\Omega}&=0 &\forall\tau\in L^2(\Omega)^d\\
		-(\sigma,\nabla v)_{0,\Omega}&=-(f,v)_{0,\Omega} &\forall v\in H_0^1(\Omega)
	\end{array}
\end{align*}

\subsubsection{Aufgabe 7.1 (a)}
Leiten Sie die Formulierung im Detail her.
Schreiben Sie dazu die Poisson-Gleichung zunächst durch Einführung einer zweiten Variable $\sigma=\nabla u$ in ein System partieller Differentialgleichungen erster Ordnung um.

\begin{lösung}
	%TODO
\end{lösung}

\subsubsection{Aufgabe 7.1 (b)}
Obiges gemischtes Problem liefert ein stabiles Sattelpunktproblem.

\begin{proof}
	%TODO
\end{proof}

\subsection{Aufgabe 7.2}
Sei $\Omega\subseteq\R^d$ ein beschränktes Gebiet mit Lipschitz-Rand $\partial\Omega$.
Die duale gemischte Methode für die Poisson-Gleichung 
\begin{align*}
	-\Delta u=-\div(\grad((u))=f\text{ in }\Omega
\end{align*}
mit homogenen Dirichlet-Randbedingungen $u=0$ auf $\partial\Omega$ ist gegeben durch:\nl
Finde $(\sigma,u)\in H(\div,\Omega)\times L^2(\Omega)$: so, dass
\begin{align*}
	\begin{array}{rll}
		(\sigma,\tau)_{0,\Omega}+(\div(\tau),u)_{0,\Omega}&=0&\forall\tau\in H(\div,\Omega)\\
		(\div(\sigma),v)_{0,\Omega} &=-(f,v)_{0,\Omega} &\forall v\in L^2(\Omega)
	\end{array}
\end{align*}
Der verwendete Raum
\begin{align*}
	H(\div,\Omega):=\left\lbrace\tau\in L^2(\Omega)^d:\div(\tau)\in L^2(\Omega)\right\rbrace
\end{align*}
kann auch als Vervollständigung von $C^\infty(\Omega)^d$ bzgl. der Norm 
\begin{align*}
	\Vert v\Vert_{H(\div,\Omega)}:=\left(\Vert v\Vert_{0,\Omega}^2+\Vert\div(v)\Vert_{0,\Omega}^2\right)^{\frac{1}{2}}
\end{align*}
definiert werden.

\subsubsection{Aufgabe 7.2 (a)}
Auf den ersten Blick schein $u$ nur in $L^2(\Omega)$ zu liegen.
Zeigen Sie, dass tatsächlich aber $u\in H_0^1(\Omega)$ ist.
Was fällt bei den Randbedingungen auf?

\begin{proof}
	%TODO
\end{proof}

\subsubsection{Aufgabe 7.2 (b)}
Obige gemischte Formulierung liefert ein stabiles Sattelpunktsproblem.

\begin{proof}
	%TODO
\end{proof}
