% This work is licensed under the Creative Commons
% Attribution-NonCommercial-ShareAlike 4.0 International License. To view a copy
% of this license, visit http://creativecommons.org/licenses/by-nc-sa/4.0/ or
% send a letter to Creative Commons, PO Box 1866, Mountain View, CA 94042, USA.

\section{Charakteristische Funktionen von Zufallsvektoren}

Sind $X,Y$ reelle Zufallsvariablen über $(\Omega,\A,\P)$, so ist die Zufallsvariable $Z:=X+\ii\mal Y$ eine komplexe Zufallsvariable.

\begin{definition}\label{def4.1}
	Eine Zufallsvariable $Z=X+\ii\mal Y$ heißt \define{integrierbar}, falls $X$ und $Y$ (reell-)-integrierbar sind.
	In diesem Fall definiere das Integral über $Z$:
	\begin{align*}
		\int\limits_\Omega Z\ds\P
		:=\int\limits_\Omega X\ds\P+\ii\mal\int\limits_\Omega Y\ds\P\in\C
	\end{align*}
	Wie im Reellen:
	\begin{align*}
		\E\eckigeKlammern{Z}:=\int\limits_\Omega Z\ds\P=\E\eckigeKlammern{X}+\ii\mal\E\eckigeKlammern{Y}
	\end{align*}
\end{definition}

\begin{satz}\label{satz4.2}
	Seien $Z,\tilde{Z}$ komplexe Zufallsvariablen.
	Dann gilt:
	\begin{enumerate}[label=(\arabic*)]
		\item $Z,\tilde{Z}$ integrierbar und $a,b\in\C\implies a\mal Z+b\mal\tilde{Z}$ integrierbar und 
		\begin{align*}
			\int\limits_\Omega a\mal Z+b\mal\tilde{Z}\ds\P
			=a\mal\int\limits_\Omega Z\ds\P+b\mal\int\limits_\Omega\tilde{Z}\ds\P
		\end{align*}
		\label{item:satz4.2_1}
		\item $Z$ integrierbar $\implies \conj{Z}:=X-\ii\mal Y$ integrierbar und
		\begin{align*}
			\int\limits_\Omega\conj{Z}\ds\P=\conj{\int\limits_\Omega Z\ds\P}
		\end{align*}
		\label{item:satz4.2_2}
		\item Falls $Z$ integrierbar ist, gilt:
		\begin{align*}
			\abs{\int\limits_\Omega Z\ds\P}\leq\int\limits_\Omega\abs{Z}\ds\P
		\end{align*}
		\label{item:satz4.2_3}
		\item $Z$ integrierbar $\iff\int\limits_\Omega\abs{Z}\ds\P<\infty$
		\label{item:satz4.2_4}
	\end{enumerate}
\end{satz}

\begin{proof}
	\ref{item:satz4.2_1} und \ref{item:satz4.2_2} folgen aus der Linearität in Definition \ref{def4.1} (zur Übung).\nl
	\betone{Zeige \ref{item:satz4.2_3}:}\\
	Benutze die Polarkoordinatendarstellung
	\begin{align}\label{eq:ProofSatz4.2_i}\tag{i}
		\E\eckigeKlammern{Z}
		&=r\mal\exp(\ii\mal\theta),\qquad r:=\abs[\big]{\E\eckigeKlammern{Z}}\\
		\label{eq:ProofSatz4.2_ii}\tag{ii}
		\Re\klammern[\big]{\exp\klammern{-\ii\mal\theta}\mal Z}
		&\leq\abs[\big]{\exp\klammern{-\ii\mal\theta}}
		=\underbrace{\abs[\big]{\exp\klammern{-\ii\mal\theta}}}_{=1}\mal\abs{Z}
		=\abs{Z}
	\end{align}
	Folglich ist $\Re\klammern[\big]{\exp\klammern{-\ii\mal\theta}\mal Z}$ wegen \ref{item:satz4.2_4}.
	Also folgt
	\begin{align*}
		\abs[\big]{\E\eckigeKlammern{Z}}
		&=\exp\klammern{-\ii\mal\theta}\mal\E\eckigeKlammern{Z}\\
		\overset{\ref{item:satz4.2_1}}&{=}
		\underbrace{\E\eckigeKlammern[\big]{\exp\klammern{-\ii\mal\theta}\mal Z}}_{
			\in\R
		}\\
		&=\E\eckigeKlammern[\Big]{\underbrace{\Re\klammern[\big]{\exp\klammern{-\ii\mal\theta}\mal Z}}_{
			\overset{\ref{item:satz4.2_2}}{\leq}\abs{Z}
		}}\\
		\overset{\text{reeller $\E$ ist monoton}}&{\leq}
		\E\eckigeKlammern[\big]{\abs{Z}}
	\end{align*}
	\betone{Zeige \ref{item:satz4.2_4}:}\\
	Folgt aus
	\begin{align*}
		\Re(Z),\Im(Z)\leq\abs{Z}\leq\abs[\big]{\Re(Z)}+\abs[\big]{\Im(Z)}
	\end{align*}
\end{proof}

\begin{satz}[Komplexer Fubini]\label{satz4.3KomplexerFubini}\enter
	Seien $Z_1,\ldots,Z_n$ unabhängige und integrierbare komplexe Zufallsvariablen.\\
	Dann ist $\prod\limits_{i=1}^n Z_i$ integrierbar und 
	\begin{align}\label{eq:Satz4.3KomplexerFubini_4.1}\tag{4.1}
		\E\eckigeKlammern{\prod\limits_{i=1}^n Z_i}
		=\prod\limits_{i=1}^n\E\eckigeKlammern{Z_i}
	\end{align}
\end{satz}

\begin{proof}
	Beweis durch Induktion: Fall $n=1$ ist nichts zu zeigen.
	Zeige dies für $n=2$:
	\begin{align*}
		\abs{Z_1\mal Z_2}
		&=\abs{Z_1}\mal\abs{Z_2}\\
		\E\eckigeKlammern[\big]{\abs{Z_1}\mal\abs{Z_2}}
		\overset{\text{reeller Fubini}}&{=}
		\E\eckigeKlammern[\big]{\abs{Z_1}}\mal\E\eckigeKlammern[\big]{\abs{Z_2}}
	\end{align*}
	Somit ist $Z_1\mal Z_2$ nach Satz \ref{satz4.2}\ref{item:satz4.2_4} integrierbar.
	Es gilt:
	\begin{align*}
		Z_1\mal Z_2
		&=\klammern[\big]{\Re(Z_1)+\ii\mal\Im(Z_1)}\mal\klammern[\big]{\Re(Z_2)+\ii\mal\Im(Z_2)}\\
		&=\Re(Z_1)\mal\Re(Z_2)-\Im(Z_1)\mal\Im(Z_2)+\ii\mal\klammern[\big]{\Re(Z_1)\mal\Im(Z_2)+\Im(Z_1)\mal\Re(Z_2)}
	\end{align*}
	Bilden des Erwartungswertes auf beiden Seiten liefert:
	\begin{align*}
		\E\eckigeKlammern{Z_1\mal Z_2}
		&=\E\eckigeKlammern[\big]{\Re(Z_1)\mal\Re(Z_2)-\Im(Z_1)\mal\Im(Z_2)}
		+\ii\mal\E\eckigeKlammern[\big]{\Re(Z_1)\mal\Im(Z_2)+\Im(Z_1)\mal\Re(Z_2)}\\
		\overset{\text{reeller Fall + reeller Fubini}}&{=}
		\E\eckigeKlammern{Z_1}\mal\E\eckigeKlammern{Z_2}
	\end{align*}
	Es folgt \eqref{eq:Satz4.3KomplexerFubini_4.1} wegen Linearität \ref{satz4.2}\ref{item:satz4.2_1} und wegen reellem Fubini (und wegen des \undefine{Blockungslemmas}).
	Somit ist der Fall $n=2$ gezeigt.
	Jetzt folgt die Behauptung aus vollständiger Induktion.
\end{proof}

\begin{definition}\label{def4.4}
	Sei $X=\klammern{X_1,\ldots,X_d}'$ ein Zufallsvektor im $\R^d$.
	Dann heißt
	\begin{align*}
		\varphi_X\colon\R^d\to\C,\qquad
		\varphi_X(z):=\E\eckigeKlammern[\Big]{\exp\klammern[\big]{\ii\mal\scaProd{z}{X}}}
		\qquad\forall z\in\R^d
	\end{align*}
	\define{charakteristische Funktion von $X$}.
	\index{charakteristische Funktion}
\end{definition}

\begin{beispiel}\label{beisp4.5}
	Sei $X\sim\Nor(0,1)$.
	Dann gilt:
	\begin{align*}
		\varphi_X(z)
		&=\exp\klammern{-\frac{1}{2}\mal z^2}\qquad\forall z\in\R
	\end{align*}
	Die charakteristische Funktion $X$ stimmt also bis auf einen Vorfaktor mit der Dichtefunktion von $X$ überein.
\end{beispiel}

\begin{satz}\label{satz4.6}
	Sei $X$ ein Zufallsvektor im $\R^d$.
	Dann gilt:
	\begin{enumerate}[label=(\arabic*)]
		\item $\begin{aligned}
			\varphi_X(0)=1,\qquad\abs[\big]{\varphi_X(z)}\leq 1\qquad\forall z\in\R^d
		\end{aligned}$
		\label{item:satz4.6_1}
		\item $\begin{aligned}
			\varphi_X(-z)=\conj{\varphi_X(z)}\qquad\forall z\in\R^d
		\end{aligned}$
		\label{item:satz4.6_2}
		\item Mit $\begin{aligned}
			A\in M(k\times d)
		\end{aligned}$ und $b\in\R^k$ gilt
		\begin{align*}
			\varphi_{A\mal X+b}(z)=\exp\klammern[\big]{\ii\mal\scaProd{z}{b}}\mal\varphi_X\klammern{A'\mal z}\qquad\forall z\in\R^k
		\end{align*}
		\label{item:satz4.6_3}
	\end{enumerate}
\end{satz}

\begin{proof}
	\betone{Zeige \ref{item:satz4.6_1}:}
	\begin{align*}
		\varphi_X(0)
		&=\E\eckigeKlammern[\Big]{\underbrace{\exp\klammern[\big]{\ii\mal\underbrace{\scaProd{0}{X}}_{=0}}}_{=1}}
		=1\\
		\abs[\big]{\varphi_X(z)}
		\overset{\ref{satz4.2}\ref{item:satz4.2_3}}&{=}
		\E\eckigeKlammern[\bigg]{\underbrace{\abs[\Big]{\exp\klammern[\big]{\ii\mal\scaProd{z}{X}}}}_{=1}}=1
	\end{align*}
	\betone{Zeige \ref{item:satz4.6_2}:}
	\begin{align*}
		\varphi_X(-z)
		&=\E\eckigeKlammern[\Big]{\exp\klammern[\big]{\ii\mal\scaProd{-z}{X}}}\\
		&=\E\eckigeKlammern[\Big]{\conj{\exp\klammern[\big]{\ii\mal\scaProd{z}{X}}}}\\
		\overset{\ref{satz4.2}\ref{item:satz4.2_2}}&{=}
		\conj{\E\eckigeKlammern[\Big]{\exp\klammern[\big]{\ii\mal\scaProd{z}{X}}}}\\
		&=\conj{\varphi_X(z)}
	\end{align*}
	
	\betone{Zeige \ref{item:satz4.6_3}:}
	\begin{align*}
		\varphi_{A\mal X+b}(z)
		&=\E\eckigeKlammern[\Big]{\underbrace{\exp\klammern[\big]{\ii\mal\underbrace{\scaProd{z}{A\mal X+b}}_{
			=\scaProd{z}{A\mal X}+\scaProd{z}{b}
		}}}_{
			=\exp\klammern[\big]{\ii\mal\scaProd{z}{A\mal X}}\mal\exp\klammern[\big]{\ii\mal\scaProd{z}{b}}
		}}\\
		\overset{\ref{satz4.2}\ref{item:satz4.2_1}}&{=}
		\exp\klammern[\big]{\ii\mal\scaProd{z}{b}}\mal\E\eckigeKlammern[\Big]{\exp\klammern[\big]{\ii\mal\scaProd{z}{A\mal X}}}
	\end{align*}
	Da 
	\begin{align*}
		\scaProd{z}{A\mal X}
		&=\scaProd{A\mal X}{z}
		=\klammern{A\mal X}'\mal z
		=X'\mal\klammern{A'\mal z}
		=\scaProd{X}{A'\mal z}
		=\scaProd{A'\mal z}{X}
	\end{align*}
	folgt die Behauptung:
	\begin{align*}
		\varphi_{A\mal X+b}(z)
		&=\exp\klammern[\big]{\ii\mal\scaProd{z}{b}}\mal\E\eckigeKlammern[\Big]{\exp\klammern[\big]{\ii\mal\scaProd{A'\mal z}{X}}}\\
		&=\exp\klammern[\big]{\ii\mal\scaProd{z}{b}}\mal\varphi_X\klammern{A'\mal z}.
	\end{align*}
\end{proof}

\begin{lemma}\label{lemma4.7}
	Seien $X_1,\ldots,X_n$ iid $\sim\Nor(0,1)$ und sei $X:=\klammern{X_1,\ldots,X_n}'$.
	Dann gilt:
	\begin{align*}
		\varphi_X(z)&=\exp\klammern{-\frac{1}{2}\mal\norm{z}^2}\qquad\forall z\in\R^n
	\end{align*}
\end{lemma}

\begin{proof}
	\begin{align*}
		\varphi_X(z)
		\overset{\Def}&=
		\E\eckigeKlammern[\Big]{\exp\klammern[\big]{\ii\mal\scaProd{z}{X}}}\\
		\overset{\Def}&=
		\E\eckigeKlammern{\exp\klammern{\ii\mal\sum\limits_{j=1}^n z_j\mal X_j}}\\
		&=\E\eckigeKlammern{\exp\klammern{\sum\limits_{j=1}^n \ii\mal z_j\mal X_j}}\\
		&=\E\eckigeKlammern[\Bigg]{\prod\limits_{j=1}^n\underbrace{\exp\klammern{\ii\mal z_j\mal X_j}}_{
			\text{unab. wg. \undefine{Blockungslemma}}
		}}\\
		\overset{\ref{satz4.3KomplexerFubini}}&{=}
		\prod\limits_{j=1}^n\underbrace{\E\eckigeKlammern[\big]{\exp\klammern{\ii\mal z_j\mal X_j}}}_{
			=\varphi_{X_j}(z_j)
			\overset{\ref{beisp4.5}}{=}
			\exp\klammern{-\frac{1}{2}\mal z_j^2}
		}\\
		&=\exp\klammern{-\frac{1}{2}\mal\sum\limits_{j=1}^n z_j^2}\\
		&=\exp\klammern{-\frac{1}{2}\mal\norm{z}^2}
	\end{align*}
\end{proof}

Zum Abschluss der wichtige sogenannte \define{Eindeutigkeitssatz für charakteristische Funktionen}:

\begin{satz}[Eindeutigkeitssatz für charakteristische Funktionen]\label{satz4.8EindeutigkeitssatzCF}\enter
	Seien $X,Y$ Zufallsvektoren im $\R^d$ mit Verteilungen $\L(X)$ bzw. $\L(Y)$.
	Dann gilt:
	\begin{align*}
		X\overset{\L}{=}Y\defiff		
		\L(X)=\L(Y)\iff\varphi_X\equiv\varphi_Y\iff\forall z\in\R^d:\varphi_X(z)=\varphi_Y(z)
	\end{align*}
\end{satz}

\begin{proof}
	Siehe Wahrscheinlichkeitstheorie.
\end{proof}

\begin{bemerkung}
	$X$ und $Y$ müssen \betone{nicht} über demselben Wahrscheinlichkeitsraum definiert sein, also z.B. $X\colon(\Omega,\A,\P)\to\R^d$ und $Y\colon\klammern{\tilde{\Omega},\tilde{A},\tilde{P}}\to\R^d$.
	\begin{align*}
		X\overset{\L}{=}Y
		\iff\L(X)
		\overset{\Def}{=}
		\P\circ X^{-1}
		=\tilde{\P}\circ Y^{-1}
		\overset{\Def}{=}
		\L(Y)
	\end{align*}
\end{bemerkung}





