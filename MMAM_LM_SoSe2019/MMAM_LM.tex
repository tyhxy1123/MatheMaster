% This work is licensed under the Creative Commons
% Attribution-NonCommercial-ShareAlike 4.0 International License. To view a copy
% of this license, visit http://creativecommons.org/licenses/by-nc-sa/4.0/ or
% send a letter to Creative Commons, PO Box 1866, Mountain View, CA 94042, USA.

\documentclass[]{scrreprt}

\newcommand{\directoryPrefix}{../latex/} 
\input{\directoryPrefix packages}
\input{\directoryPrefix theoremenvironments}
\input{\directoryPrefix font}
\input{\directoryPrefix commands}
\input{\directoryPrefix commands_Willi}
% This work is licensed under the Creative Commons
% Attribution-NonCommercial-ShareAlike 4.0 International License. To view a copy
% of this license, visit http://creativecommons.org/licenses/by-nc-sa/4.0/ or
% send a letter to Creative Commons, PO Box 1866, Mountain View, CA 94042, USA.

\DeclareMathOperator{\WN}{WN}          % weißes Rauschen
\DeclareMathOperator{\IID}{IID}        % unabhängig identisch verteilt % lokale Commands und packages werden auch ausgelagert. Lokal bedeutet nur diese Vorlesung betreffend.

\makeindex % erstellt Stichwortverzeichnis
\setlength{\parindent}{0px} % verhindert das Einrücken eines Absatzes bei einer leeren Codezeile.
\renewcommand{\thesection}{\arabic{section}} % nützlich, falls es keine Chapter, sondern nur sections gibt.

\title{
	Vorlesung\\
	Lineare Modelle\\
}
%\subject{Thema}
\subtitle{Sommersemester 2019}
\author{
	Vorlesung:  Prof. Dr. Dietmar Ferger\\
	Mitschrift: Willi Sontopski
}
\date{\today}
\publishers{\url{https://github.com/LostInDarkMath/MatheMaster}}
%\dedication{Widmung}

\begin{document}
	%\begin{hack}: fix overfull hbox warning caused by \tableofcontents when exceeding 100 pages	
	\makeatletter
  	\renewcommand{\@pnumwidth}{2em}
  	%\renewcommand{\@tocrmarg}{3em} % currently unnecessary
  	\makeatother
  	%\end{hack}
  	
	\pagenumbering{gobble}	% disable pagenumbering
	\maketitle
	%\epigraph{Hier steht eine Lebensweisheit.}{\textit{Der Admin}}
	%\newpage
	\doclicenseThis
	\tableofcontents
	\pagenumbering{arabic}	% enable pagenumbering
	
	\addchap{Lineare Modelle} % sinnvoll in Kombination mit Zeile 15
	% This work is licensed under the Creative Commons
% Attribution-NonCommercial-ShareAlike 4.0 International License. To view a copy
% of this license, visit http://creativecommons.org/licenses/by-nc-sa/4.0/ or
% send a letter to Creative Commons, PO Box 1866, Mountain View, CA 94042, USA.

\section{Einführung in algebraische Modellierung}

%Vorlesung vom 02.04.2019 war scheinbar inhaltslos, deshalb lasse ich die mal

Die \define{Strukturmathematik} stellt sich die Aufgabe Strukturen zu modellieren und technisch wie sprachlich zugänglich zu machen, sogenannte \define{Theoriebildung}.
Dies ist ein evokutionärer Prozess in dem neue Begriffe entstehen und alte untergehen.

\subsection{Warum Modellierung und Formalisierung?}
Natürliche Sprache

\begin{figure}[H] % oder ht!
	\begin{center}
		\input{./tikz/natSprache}
		%\caption{Modellierung}
		%\label{Abb:natSpracheModellierung}
	\end{center}
\end{figure}

\betone{Problem:} Fehlkommunikation\\
Das fehlende Glied ist die Formale Sprache:

\begin{figure}[H] % oder ht!
	\begin{center}
		\input{./tikz/natSpracheBesser}
		%\caption{Modellierung}
		%\label{Abb:natSpracheModellierung}
	\end{center}
\end{figure}

Die Formale Sprache erlaubt Absraktion und Vergleich.
Die automatisierte Sprache ist algorithmisch reichhaltig, aber strukturell arm.
Im Sinne der mathematischen Beschreibung / Modellierung ist ein Modell-Vergleich oft nicht möglich, wenn ich kein abstraktes Modell habe!\nl
Es gibt zwei Wege, Wissen zu erlangen:
\begin{enumerate}
	\item \define{Top Down:} deduktive Methode
	\item \define{Bottom Up:} induktive Methode
\end{enumerate}

Technisches Makro sind symbolische Abkürzungen, z.B. "$\R$".
Im Gegensatz dazu gibt es auch verbale Makros, z.B. "reelle Zahlen".

\subsection{Mengenbasierte Modellierung / strukturelle Modellierung}
\begin{definition}
	Eine \define{Inzidenzstruktur} ist ein Tripel $\Inz=(P,B,I)$ wobei $P,B,I$ Mengen sind mit $I\subseteq P\times B$. 
	Interpretation:
	\index{Inzidenzstruktur}
	\begin{itemize}
		\item $P$ ist Menge von Punkten / Points
		\item $B$ ist Menge von Blöcken / Blocks
		\item $I$ Inzidenzrelation (z.B. Lines)
	\end{itemize}
	Für $(p,b)\in I$ schreiben wir auch $pIb$ (incidence) und sagen:\\
	"Der Punkt $p$ inzidiert mit dem Block $b$ in $\Inz$."\\
	Für $p\in P$ sei
	\begin{align*}
		pI:=\set{b\in B\mid pIb}
	\end{align*}
	d.h. die Menge aller mit dem Punkt $p$ inzidierenden Blöcke.\\
	Analog sei für $b\in B$ stets
	\begin{align*}
		Ib:=\set{p\in P\mid pIb}
	\end{align*}
	die Menge aller mit dem Block $b$ inzidierenden Punkte.
\end{definition}

Strukturelle Modellierung ist eine mengenbasierte Modellierung.
Dies ist ein Gegensatz z.B. zur \textbf{Prozedurale Modellierung}.

\begin{beispiel}\
	\begin{itemize}
		\item $P:=$ Eckenmenge eines Würfels
		\item $B:=$ Flächenmenge des Würfels
		\item Inzidenz: Punkt ist Eckpunkt von Würfelfläche
	\end{itemize}
\end{beispiel}

\begin{satz}[Prinzip des doppelten Abzählens]\enter
	Ist $\Inz=(P,B,I)$ endliche Inzidenzstruktur (d.h. $P$ und $B$ endlich), so gilt
	\begin{align*}
		\sum\limits_{p\in P}\# pI=\#I=\sum\limits_{b\in B}\# Ib
	\end{align*}
	Hierbei ist $\#M$ die Anzahl der Elemente von $M$.
\end{satz}

\begin{definition}
	$\Inz=(P,B,I)$ heißt \define{taktische Konfiguration}
	\index{taktische Konfiguration} 
	\begin{align*}
		:\iff\exists r_\Inz,k_\Inz\in\N:\forall p\in P,\forall b\in B:\# pI=r_\Inz\und\#Ib=k_\Inz
	\end{align*}
\end{definition}

\begin{lemma}
	Sei $y$ taktische Konfiguration von $J=(P,B,I)$. Dann gilt:
	\begin{align*}
		v_\Inz\mal r_\Inz=b_\Inz\mal k_\Inz
		\qquad\mit\qquad
		v_\Inz:=\# P\und b_\Inz:=\#B
	\end{align*}
\end{lemma}

\begin{proof}
	Doppelte Abzählung: 
	\begin{align*}
		v_\Inz\mal r_\Inz=\sum\limits_{p\in P}\underbrace{\# pI}_{=r_\Inz}=\sum\limits_{b\in B}\underbrace{\#Ib}_{=k_\Inz}=b_\Inz\mal r_\Inz
	\end{align*}
\end{proof}


\begin{definition}
	$\big(v_\Inz,r_\Inz;b_\Inz,k_\Inz)$ ist das \define{Parametertupel} von $\Inz$.
	\index{Parametertupel}\index{duales Parametertupel}\\
	Das dazu \define{duale Parametertupel} ist $\big(b_\Inz,k_\Inz;v_\Inz,r_\Inz\big)$.
\end{definition}

\begin{beispiel}
	Für unsere Würfelinzidenzstruktur ist das Parametertupel:\\
	(Anzahl der Punkte, Anzahl der Geraden pro Punkt, Anzahl der Geraden, Anzahl der Punkte pro Gerade).
	\begin{enumerate}
		\item Tetraeder (dual Tetraeder): $(4,3;4,3)$ 
		\item Hexader (dual: Oktaeder): $(8,3;6,4)$
		\item Dodekaeder (dual: Ikosaeder): $(20,3;12,5)$
		\item Dreieck: $(3,2;3,2)$ ist auch selbstdual: Dreieck $\leftrightarrow$ Dreiseit
	\end{enumerate}
\end{beispiel}

\begin{beispiel}[Veblen-Young-Configuration]\
	\begin{figure}[H] % oder ht!
		\begin{center}
			% This work is licensed under the Creative Commons
% Attribution-NonCommercial-ShareAlike 4.0 International License. To view a copy
% of this license, visit http://creativecommons.org/licenses/by-nc-sa/4.0/ or
% send a letter to Creative Commons, PO Box 1866, Mountain View, CA 94042, USA.



\tikzset{every picture/.style={line width=0.75pt}} %set default line width to 0.75pt        

\begin{tikzpicture}[x=0.75pt,y=0.75pt,yscale=-1,xscale=1]
%uncomment if require: \path (0,300); %set diagram left start at 0, and has height of 300

%Shape: Circle [id:dp8256548323361356] 
\draw   (15,144) .. controls (15,139.03) and (19.03,135) .. (24,135) .. controls (28.97,135) and (33,139.03) .. (33,144) .. controls (33,148.97) and (28.97,153) .. (24,153) .. controls (19.03,153) and (15,148.97) .. (15,144) -- cycle ;
%Shape: Circle [id:dp9999136902725693] 
\draw   (162,30) .. controls (162,25.03) and (166.03,21) .. (171,21) .. controls (175.97,21) and (180,25.03) .. (180,30) .. controls (180,34.97) and (175.97,39) .. (171,39) .. controls (166.03,39) and (162,34.97) .. (162,30) -- cycle ;
%Shape: Circle [id:dp7722874472312242] 
\draw   (86,89) .. controls (86,84.03) and (90.03,80) .. (95,80) .. controls (99.97,80) and (104,84.03) .. (104,89) .. controls (104,93.97) and (99.97,98) .. (95,98) .. controls (90.03,98) and (86,93.97) .. (86,89) -- cycle ;
%Shape: Circle [id:dp1591555646648798] 
\draw   (194,158) .. controls (194,153.03) and (198.03,149) .. (203,149) .. controls (207.97,149) and (212,153.03) .. (212,158) .. controls (212,162.97) and (207.97,167) .. (203,167) .. controls (198.03,167) and (194,162.97) .. (194,158) -- cycle ;
%Shape: Circle [id:dp04988684193722637] 
\draw   (121,112) .. controls (121,107.03) and (125.03,103) .. (130,103) .. controls (134.97,103) and (139,107.03) .. (139,112) .. controls (139,116.97) and (134.97,121) .. (130,121) .. controls (125.03,121) and (121,116.97) .. (121,112) -- cycle ;
%Shape: Circle [id:dp15540025125472967] 
\draw   (99,153) .. controls (99,148.03) and (103.03,144) .. (108,144) .. controls (112.97,144) and (117,148.03) .. (117,153) .. controls (117,157.97) and (112.97,162) .. (108,162) .. controls (103.03,162) and (99,157.97) .. (99,153) -- cycle ;
%Straight Lines [id:da9051529769910057] 
\draw    (24,144) -- (171,30) ;


%Straight Lines [id:da9722531504709364] 
\draw    (24,144) -- (203,158) ;


%Straight Lines [id:da879150422612323] 
\draw    (95,89) -- (203,158) ;


%Straight Lines [id:da809884536800273] 
\draw    (171,30) -- (108,153) ;

\end{tikzpicture}

			\caption{Veblen-Young-Konfiguration: $(6,2;4,3)$}
			%\label{Abb:natSpracheModellierung}
		\end{center}
	\end{figure}
	\begin{figure}[H] % oder ht!
		\begin{center}
			% This work is licensed under the Creative Commons
% Attribution-NonCommercial-ShareAlike 4.0 International License. To view a copy
% of this license, visit http://creativecommons.org/licenses/by-nc-sa/4.0/ or
% send a letter to Creative Commons, PO Box 1866, Mountain View, CA 94042, USA.


\tikzset{every picture/.style={line width=0.75pt}} %set default line width to 0.75pt        

\begin{tikzpicture}[x=0.75pt,y=0.75pt,yscale=-1,xscale=1]
%uncomment if require: \path (0,300); %set diagram left start at 0, and has height of 300

%Shape: Square [id:dp9706132534648327] 
\draw   (201,62) -- (379,62) -- (379,240) -- (201,240) -- cycle ;
%Straight Lines [id:da9618724140233941] 
\draw    (201,62) -- (379,240) ;


%Straight Lines [id:da3907949926394131] 
\draw    (379,62) -- (201,240) ;


%Shape: Circle [id:dp5133007782867983] 
\draw   (192,62) .. controls (192,57.03) and (196.03,53) .. (201,53) .. controls (205.97,53) and (210,57.03) .. (210,62) .. controls (210,66.97) and (205.97,71) .. (201,71) .. controls (196.03,71) and (192,66.97) .. (192,62) -- cycle ;
%Shape: Circle [id:dp3477876429055702] 
\draw   (192,240) .. controls (192,235.03) and (196.03,231) .. (201,231) .. controls (205.97,231) and (210,235.03) .. (210,240) .. controls (210,244.97) and (205.97,249) .. (201,249) .. controls (196.03,249) and (192,244.97) .. (192,240) -- cycle ;
%Shape: Circle [id:dp36747093980837164] 
\draw   (370,240) .. controls (370,235.03) and (374.03,231) .. (379,231) .. controls (383.97,231) and (388,235.03) .. (388,240) .. controls (388,244.97) and (383.97,249) .. (379,249) .. controls (374.03,249) and (370,244.97) .. (370,240) -- cycle ;
%Shape: Circle [id:dp00705377401053342] 
\draw   (370,62) .. controls (370,57.03) and (374.03,53) .. (379,53) .. controls (383.97,53) and (388,57.03) .. (388,62) .. controls (388,66.97) and (383.97,71) .. (379,71) .. controls (374.03,71) and (370,66.97) .. (370,62) -- cycle ;

\end{tikzpicture}

			\caption{duale Veblen-Young-Konfiguration: $(4,3;6,2)$}
			%\label{Abb:natSpracheModellierung}
		\end{center}
	\end{figure}	
\end{beispiel}

\begin{beispiel}
	Berühmte taktische Konfiguation ist $(10,3;10,3)$.\\
	Sei $[n]:=\set{1,\ldots,n}$ für $n\in\N$ und $\ul{n}:=\set{0,\ldots,n-1}$.
	Sei $\begin{pmatrix}
		M\\k
	\end{pmatrix}:=\set{X\subseteq M\mid\#X=k}$ für $M$ Menge.
	Dann gilt $\#\begin{pmatrix}
		M\\k
	\end{pmatrix}=\begin{pmatrix}
		\#M\\k
	\end{pmatrix}$.
	Seien $i,j,n\in\N$ mit $i\leq j\leq n$.
	Setze
	\begin{align*}
		J^n_{(i,j)}&:=\klammern{\begin{pmatrix}
			[n]\\i
		\end{pmatrix},\begin{pmatrix}
			[n]\\ j
		\end{pmatrix},I^n_{(i,j)}}\qquad\mit\\
		I^n_{(i,j)}&:=\set{(p,b)\in\begin{pmatrix}
			[n]\\ i
		\end{pmatrix}\times\begin{pmatrix}
			[n]\\j
		\end{pmatrix}\mid p\subseteq b}
	\end{align*}
	Dann ist $J^n_{(i,j)}$ taktische Konfiguration mit dem Parametertupel
	\begin{align*}
		\klammern{
		\begin{pmatrix}
			n\\i
		\end{pmatrix},
		\begin{pmatrix}
			n-i\\j-i
		\end{pmatrix},
		\begin{pmatrix}
			n\\j
		\end{pmatrix},
		\begin{pmatrix}
			j\\i
		\end{pmatrix}
		}
	\end{align*}
	Wann ist dies gleich $(10,3;10,3)$?
	Oder gleich $(4,3;6,2)$? Antwort:
	\begin{align*}
		(4,3;6,2)=\klammern{\begin{pmatrix}
			4\\1
		\end{pmatrix},\begin{pmatrix}
			4-1\\2-1
		\end{pmatrix},
		\begin{pmatrix}
			4\\2
		\end{pmatrix},
		\begin{pmatrix}
			2\\1
		\end{pmatrix}}
	\end{align*}
	dh. $(i,j,n)=(1,2,4)$.
\end{beispiel}

\begin{definition}
	Die \define{duale Inzidenzstruktur} einer Inzidenzstruktur $\Inz=(P,B,I)$ ist
	\index{duale Inzidenzstruktur}
	\begin{align*}
		\Inz^{\op}:=\big(B,P,I^\op\big)
		\qquad\mit\qquad
		I^\op:=\set{(b,p)\in B\times P\mid p I b}
	\end{align*}
\end{definition}

\begin{beispiel}
	\begin{align*}
		\Big(\Inz_{(i,j)}^n\Big)^\op=(B,P,R^\op)
		\qquad\mit\qquad
		R^\op:=\set{(b,p)\in\begin{pmatrix}
			[n]\\
			j
		\end{pmatrix}\times
		\begin{pmatrix}
			[n]\\
			i
		\end{pmatrix}:p\subseteq b}
	\end{align*}
\end{beispiel}

\begin{definition}
	Seien $\Inz=(P,B,I)$ und $\Inz'=(P',B',I')$ Inzidenzstrukturen.
	Dann heißt ein Abbildungspaar $(\varphi,\psi)$ bestehend aus Abbildungen $\varphi\colon P\to P'$ und $\psi\colon B\to B'$ ein \define{Morphismus} von $\Inz$ nach $\Inz'$
	\index{Morphismus}\index{Isomorphismus}
	\begin{align*}
		\defiff\forall p\in P,\forall b\in B:pIb\implies\varphi(p)I'\psi(b)
	\end{align*}

	Ein Abbildungspaar $(\varphi,\psi)$ bestehend aus Bijektionen $\varphi\colon P\to P'$ und $\psi\colon B\to B'$ heißt \define{Isomorphismus (Iso)} von $\Inz$ nach $\Inz'$1
	\begin{align*}
		\defiff\forall p\in P,\forall b\in B:pIb\iff\varphi(p)I'\psi(b)
\end{align*}		
	
	Wir sagen, $\Inz$ ist \define{isomorph} zu $\Inz'$, i.Z. $\Inz\simeq\Inz'$, falls es einen Isomorphismus von $\Inz$ nach $\Inz'$ gibt.
\end{definition}

\begin{lemma}\
	\begin{enumerate}
		\item Ist $(\varphi,\psi)$ ein Isomorphismus von $\Inz$ nach $\Inz'$, so ist auch $\big(\varphi^{-1},\psi^{-1}\big)$ ein Isomorphismus von $\Inz'$ nach $\Inz$.\label{item:lemma1.13_1}
		\item Ist $(\varphi,\psi)$ Morphismus von $\Inz$ nach $\Inz'$, so ist $(\psi,\varphi)$ Morphismus von $\Inz^\op$ nach $(\Inz')^\op$.\label{item:lemma1.13_2}
	\end{enumerate}
\end{lemma}

\begin{proof}
	\betone{Zeige \ref{item:lemma1.13_1}:}\\
	Seien $p'\in P'$ und $b'\in B'$ mit $p'I'b'$.
	Zu zeigen ist
	\begin{align*}
		\varphi^{-1}(b')=\psi^{-1}(b')
	\end{align*}
	Wegen $p'I'b'$ gilt für $p:=\varphi^{-1}(p')$ und $b:=\psi^{-1}(b')$ stets $\varphi(p)=p'$ und $sp(b)=b'$, also $\varphi(p)I'\psi(b)$.
	Also gilt auch $pIb$ d.h. $\varphi^{-1}(p)I\psi^{-1}(b)$.\nl
	\betone{Zeige \ref{item:lemma1.13_1}:} Total klar.
\end{proof}

\begin{satz}
	Seien $i,j,n\in\N$ mit $i\leq j\leq n$.
	Dann gilt
	\begin{align*}
		\Big(\Inz_{(i,j)}^n\Big)^\op\simeq\Inz_{(n-j,n-i)}^n
	\end{align*}
	vermöge $(\varphi,\psi)$ mit
	\begin{align*}
		\varphi\colon		
		\underbrace{
		\begin{pmatrix}
			[n]\\
			j
		\end{pmatrix}}_{
		\text{Punktmenge von }\big(\Inz_{(i,j)}^n\big)^\op
		}&\to\underbrace{\begin{pmatrix}
			[n]\\
			n-j
		\end{pmatrix}}_{
		\text{Punktmenge von }\Inz_{(n-j,n-i)}^n
		}
		&&
		b\mapsto[n]-b\\
	\psi\colon\underbrace{\begin{pmatrix}
			[n]\\
			i
		\end{pmatrix}}_{
		\text{Blockmenge von }\big(\Inz_{(i,j)}^n\big)^\op
		}&\to\underbrace{\begin{pmatrix}
			[n]\\
			n-i
		\end{pmatrix}}_{
		\text{Blockmenge von }\Inz_{(n-j,n-i)}^n
		}
		&&
		p\mapsto[n]-p
	\end{align*}
	Also sind die Parametertupel von $\Big(\Inz_{(i,j)}^n\Big)^\op$ gegeben durch
	\begin{align*}
		\klammern{
		\begin{pmatrix}
			n\\
			j
		\end{pmatrix},
		\begin{pmatrix}
			n-i\\
			j-i
		\end{pmatrix};
		\begin{pmatrix}
			n\\
			i
		\end{pmatrix},
		\begin{pmatrix}
			j\\ 
			i
		\end{pmatrix}}
	\end{align*}
	und von $\Inz_{(n-j,n-i)}^n$ gegeben durch
	\begin{align*}
		\klammern{
		\begin{pmatrix}
			n\\
			n-j
		\end{pmatrix},
		\begin{pmatrix}
			n-(n-j)\\
			(n-i)-(n-j)
		\end{pmatrix},
		\begin{pmatrix}
			n\\
			n-i
		\end{pmatrix},
		\begin{pmatrix}
			n-i\\
			n-j
		\end{pmatrix}}
	\end{align*}
	gleich.
\end{satz}

\begin{definition}
	Eine Inzidenzstruktur $\Inz$ heißt \define{selbstdual}\index{selbstdual}
	\begin{align*}
		\defiff\Inz\simeq\Inz^\op
	\end{align*}
\end{definition}

Wann ist $\Inz_{(i,j)}^n$ selbstdual?

\begin{lemma}
	Für $0<i\leq j\leq n$ gilt:
	\begin{align*}
		\Inz_{(i,j)}^n\simeq\Inz_{(i',j')}^{n'}\iff
		(i,j,n)=(i',j',n')
	\end{align*}
\end{lemma}

\begin{satz}
	\begin{align*}
		\Inz_{(i,j)}^n\text{ selbstdual}
		&\iff \Inz_{(i,j)}^n\simeq\Big(\Inz_{(i,j)}^n\Big)^\op\simeq \Inz_{(n-j,n-i)}^n\\
		&\iff i=n-j\und j=n-i\\
		&\iff i+j=n
	\end{align*}
\end{satz}

\begin{beispiel}
	Für welches Tripel $(i,j,n)$ hat die Inzidenzstruktur $\Inz_{(i,j)}^n$ das Parametertupel $(10,3;10,3)$?\\
	Antwort: für $(i,j,n)=(2,3,5)$.
	Geometrische Realisierung von $\Inz_{(2,3)}^5$:
	\begin{figure}[H] % oder ht!
		\begin{center}
			% This work is licensed under the Creative Commons
% Attribution-NonCommercial-ShareAlike 4.0 International License. To view a copy
% of this license, visit http://creativecommons.org/licenses/by-nc-sa/4.0/ or
% send a letter to Creative Commons, PO Box 1866, Mountain View, CA 94042, USA.




\tikzset{every picture/.style={line width=0.75pt}} %set default line width to 0.75pt        

\begin{tikzpicture}[x=0.75pt,y=0.75pt,yscale=-1,xscale=1]
%uncomment if require: \path (0,300); %set diagram left start at 0, and has height of 300

%Shape: Circle [id:dp08312604554148328] 
\draw  [fill={rgb, 255:red, 80; green, 227; blue, 194 }  ,fill opacity=1 ] (192,142) .. controls (192,138.13) and (195.13,135) .. (199,135) .. controls (202.87,135) and (206,138.13) .. (206,142) .. controls (206,145.87) and (202.87,149) .. (199,149) .. controls (195.13,149) and (192,145.87) .. (192,142) -- cycle ;
%Shape: Circle [id:dp5479371678347926] 
\draw  [fill={rgb, 255:red, 80; green, 227; blue, 194 }  ,fill opacity=1 ] (182,100) .. controls (182,96.13) and (185.13,93) .. (189,93) .. controls (192.87,93) and (196,96.13) .. (196,100) .. controls (196,103.87) and (192.87,107) .. (189,107) .. controls (185.13,107) and (182,103.87) .. (182,100) -- cycle ;
%Shape: Circle [id:dp9035059763139627] 
\draw  [fill={rgb, 255:red, 241; green, 11; blue, 11 }  ,fill opacity=1 ] (203,182) .. controls (203,178.13) and (206.13,175) .. (210,175) .. controls (213.87,175) and (217,178.13) .. (217,182) .. controls (217,185.87) and (213.87,189) .. (210,189) .. controls (206.13,189) and (203,185.87) .. (203,182) -- cycle ;
%Shape: Circle [id:dp13405737177609034] 
\draw  [fill={rgb, 255:red, 65; green, 117; blue, 5 }  ,fill opacity=1 ] (25,144) .. controls (25,140.13) and (28.13,137) .. (32,137) .. controls (35.87,137) and (39,140.13) .. (39,144) .. controls (39,147.87) and (35.87,151) .. (32,151) .. controls (28.13,151) and (25,147.87) .. (25,144) -- cycle ;
%Shape: Circle [id:dp820938089239013] 
\draw  [fill={rgb, 255:red, 240; green, 19; blue, 19 }  ,fill opacity=1 ] (229,11) .. controls (229,7.13) and (232.13,4) .. (236,4) .. controls (239.87,4) and (243,7.13) .. (243,11) .. controls (243,14.87) and (239.87,18) .. (236,18) .. controls (232.13,18) and (229,14.87) .. (229,11) -- cycle ;
%Shape: Circle [id:dp9814130832419268] 
\draw  [fill={rgb, 255:red, 80; green, 227; blue, 194 }  ,fill opacity=1 ] (136,187) .. controls (136,183.13) and (139.13,180) .. (143,180) .. controls (146.87,180) and (150,183.13) .. (150,187) .. controls (150,190.87) and (146.87,194) .. (143,194) .. controls (139.13,194) and (136,190.87) .. (136,187) -- cycle ;
%Shape: Circle [id:dp7483452900473003] 
\draw  [fill={rgb, 255:red, 80; green, 227; blue, 194 }  ,fill opacity=1 ] (241,85) .. controls (241,81.13) and (244.13,78) .. (248,78) .. controls (251.87,78) and (255,81.13) .. (255,85) .. controls (255,88.87) and (251.87,92) .. (248,92) .. controls (244.13,92) and (241,88.87) .. (241,85) -- cycle ;
%Shape: Circle [id:dp47277380471474084] 
\draw  [fill={rgb, 255:red, 80; green, 227; blue, 194 }  ,fill opacity=1 ] (269,238) .. controls (269,234.13) and (272.13,231) .. (276,231) .. controls (279.87,231) and (283,234.13) .. (283,238) .. controls (283,241.87) and (279.87,245) .. (276,245) .. controls (272.13,245) and (269,241.87) .. (269,238) -- cycle ;
%Shape: Circle [id:dp9584244691590221] 
\draw  [fill={rgb, 255:red, 80; green, 227; blue, 194 }  ,fill opacity=1 ] (221,140) .. controls (221,136.13) and (224.13,133) .. (228,133) .. controls (231.87,133) and (235,136.13) .. (235,140) .. controls (235,143.87) and (231.87,147) .. (228,147) .. controls (224.13,147) and (221,143.87) .. (221,140) -- cycle ;
%Straight Lines [id:da8654056524389862] 
\draw    (32,144) -- (328,257.93) ;


%Straight Lines [id:da5764695965444305] 
\draw    (32,144) -- (356,54) ;


%Straight Lines [id:da5308065029099274] 
\draw    (236,11) -- (143,187) ;


%Straight Lines [id:da046230899122107094] 
\draw    (236,11) -- (276,238) ;


%Shape: Circle [id:dp5984512451328349] 
\draw  [fill={rgb, 255:red, 252; green, 11; blue, 11 }  ,fill opacity=1 ] (213,124) .. controls (213,120.13) and (216.13,117) .. (220,117) .. controls (223.87,117) and (227,120.13) .. (227,124) .. controls (227,127.87) and (223.87,131) .. (220,131) .. controls (216.13,131) and (213,127.87) .. (213,124) -- cycle ;
%Straight Lines [id:da11514776685343242] 
\draw [color={rgb, 255:red, 248; green, 12; blue, 12 }  ,draw opacity=1 ]   (189,99) -- (210,182) ;


%Straight Lines [id:da39941550536718873] 
\draw [color={rgb, 255:red, 250; green, 7; blue, 7 }  ,draw opacity=1 ]   (220,124) -- (143,187) ;


%Straight Lines [id:da1529104012025022] 
\draw [color={rgb, 255:red, 247; green, 6; blue, 6 }  ,draw opacity=1 ]   (189,100) -- (143,187) ;


%Straight Lines [id:da4295459927621441] 
\draw [color={rgb, 255:red, 240; green, 7; blue, 7 }  ,draw opacity=1 ]   (220,124) -- (279,240) ;


%Straight Lines [id:da36586521305300124] 
\draw [color={rgb, 255:red, 248; green, 5; blue, 5 }  ,draw opacity=1 ]   (248,85) -- (210,182) ;


%Straight Lines [id:da9039017754327724] 
\draw [color={rgb, 255:red, 245; green, 166; blue, 35 }  ,draw opacity=1 ]   (236,11) -- (210,182) ;



% Text Node
\draw (9,143) node [color={rgb, 255:red, 65; green, 117; blue, 5 }  ,opacity=1 ] [align=left] {$z$};
% Text Node
\draw (220,62) node [color={rgb, 255:red, 245; green, 166; blue, 35 }  ,opacity=1 ] [align=left] {$a$};


\end{tikzpicture}

			\caption{Desargues Konfiguration}
			%\label{Abb:natSpracheModellierung}
		\end{center}
	\end{figure}
	\begin{itemize}
		\item Die beiden roten Dreiecke sind zentral perspektiv bzgl. des Zentrums $z$.
		\item Die drei roten Punkte liegen auf der Geraden $a$.
		\item Satz: Sind zwei Dreiecke sind zentral perpspektiv, so sind sie auch axial perspektiv und umgekehrt.
		\item Zeige $\Inz_{(2,3)}^5\simeq$ Desargue Konfiguration
	\end{itemize}
\end{beispiel}

% hier könnte was fehlen (15.04.19)

\begin{theorem}[Desargues Theorem]\enter
	Im \undefine{projektiven Raum} (bzw. in der desargueschen projektiven Ebene) gilt:
	\begin{enumerate}
		\item Sind zwei Dreiecke zentral perspektiv, so sind sie auch axial perspektiv.
		\item Sind zwei Dreiecke axial perspektiv, so sind sie auch zentral perspektiv.
	\end{enumerate}
\end{theorem}

\begin{beispiel}[Pappos-Konfiguration]\enter
	Die \define{Pappos-Konfiguration} ist eine $(9,3;9,3)$-Konfiguration.
\end{beispiel}

\begin{definition}
	Ein Isomorphismus einer Inzidenzstruktur $\Inz$ auf sich heißt auch \define{Automorphismus} von $\Inz$.
	Bezeichne $\Aut(\Inz)$ die Menge der Automorphismen von $\Inz$ und
	\begin{align*}
		\Aut(\Inz):=\big(\Aut(\Inz),\circ,\id_\Inz,^{-1}\big)
	\end{align*}
	Seien Morphismen
	\begin{align*}
		\Inz\overset{(\varphi,\psi)}{\longrightarrow}\Inz'\overset{(\varphi',\psi')}{\longrightarrow}\Inz''
	\end{align*}		
	gegeben. Dann ist
	\begin{align*}
		(\varphi',\psi')\circ(\varphi,\psi):=\big(\varphi'\circ\varphi,\psi'\circ\psi\big)
	\end{align*}
	ein Morphismus von $\Inz$ nach $\Inz''$, die sogenannte  \define{kontravariante Verkettung} von $(\varphi,\psi)$ mit $(\varphi',\psi')$.
	\index{kontravariante Verkettung}\nl
	Ist $\Inz=(P,B,I)$, so bezeichne $\id_\Inz:=\big(\id_P,\id_B\big)$ den \define{identischen Morphismus} von $\Inz$.
	\index{identischer Morphismus}
\end{definition}

\begin{lemma}
	ist $(\varphi,\psi)$ ein Isomorphismus, so ist auch 
	\begin{align*}
		(\varphi,\psi)^{-1}:=\klammern{\varphi^{-1},\psi^{-1}}
	\end{align*}
	ein Isomorphismus.
	Dieser heißt \define{der zu $(\varphi,\psi)$ inverse Morphismus}.
	\index{inverser Morphismus}
\end{lemma}

\subsection{Taktische Konfoguration via Unterverband eines endlichdimensionalen Vektorraumes}

Sei $V$ $n$-dimensionaler Vektorraum über  $\F_g$ ($q$ Primzahlpotenz, $n\in\N$).
Für $i,j\in\N$ mit $0\leq i\leq j\leq $n sei dann
\index{Unterraumverband}
\begin{align*}
	\Inz^n_{(i,j)}&:=\klammern{\begin{pmatrix}
		\L\\
		i
	\end{pmatrix},\begin{pmatrix}
		\L\\
		j
	\end{pmatrix},
	I^\L_{(i,j)}
	}\mit\\
	\L&:=L(V):=\big(L(V),\leq\big)\und\\
	L(V)&:=\set{U\mid U\leq V}~\text{\define{Unterraumverband} von $V$ und}\\
	\begin{pmatrix}
		\L\\
		i
	\end{pmatrix}
	&:=\set{p\in L(V)\mid\dim(p)=i}
	\und\\
	\begin{pmatrix}
		\L\\
		j
	\end{pmatrix}
	&:=\set{b\in L(V)\mid\dim(b)=j}
	\text{ sowie }\\
	I^\L_{(i,j)}&:=\set{(p,b)\in\begin{pmatrix}
		\L\\
		i
	\end{pmatrix}\times\begin{pmatrix}
		\L\\
		j
	\end{pmatrix}:p\leq b}
\end{align*}

\begin{proposition}
	Dann ist $\Inz^\L_{(i,j)}$ taktische Konfiguration mit Parametertupel
	\begin{align*}
		\klammern{
			\begin{pmatrix}
				n\\
				i
			\end{pmatrix}_p,\begin{pmatrix}
				n-i\\
				j-i
			\end{pmatrix}_p;
			\begin{pmatrix}
				n\\
				j
			\end{pmatrix}_q,\begin{pmatrix}
				j\\
				i
			\end{pmatrix}_q
		}
	\end{align*}
	wobei $\begin{pmatrix}
		n\\
		i
	\end{pmatrix}_q$ den Gaußschen Binomialkoeffizient von $n$ über $i$ bezeichnet.
\end{proposition}

\begin{notation}
	\begin{align*}
		\begin{pmatrix}
			n\\
			i
		\end{pmatrix}_q
		:=\#\begin{pmatrix}
			\L\\
			i
		\end{pmatrix}
	\end{align*}
	Für $u,w\in L(V)$ mit $u\leq w$ sei
	\begin{align*}
		\L(w):=\set{t\in L(V)\mit t\leq w}
	\end{align*}
	und
	\begin{align*}
		\L/_u:=\set{t\in L(V)\mid u\leq t}
		\qquad\und\qquad
		[u,w]_\L:=\set{t\in L(V)\mid u\leq t\leq w}
	\end{align*}
	Hierbei heißt $[u,w]_\L$ \define{Intervall in $\L=\L(V)$}.
	\index{Intervall}
	\begin{align*}
		\L(u)=\big(\L(u),\leq\big)
	\end{align*}
	ist Menge der Unterräume, die in $w$ enthalten sind bzw. der Unterraumverband von $U$.
\end{notation}

%\begin{beispiel}
	Setze $0_\L:=\set{\vec{0}}$ und $1_\L:=V$.
	Dann ist
	\begin{align*}
		L(u)=\big[0_\L,u\big]_\L
	\end{align*}
	Sei $\L|T:=\big(T,\leq\cap T\times T)$ die Einschränkung von $\L$ auf $T$ für $T\subseteq L$.
	Dann gilt:
	\begin{align*}
		\L(w)&=\L|\big[0_\L,w\big]_\L\\
		\L/u&:=\L|\big[u,1_\L\big]_\L\cong\L(V/u)
	\end{align*}
	Achtung: $L/u$ klappt nicht sauber!\\
	$\leadsto$ "abuse of notation": $L/_u:=[u,1_\L]$.\\
	Aber $\L(w)$ ist gleichzeitig der Unterraumverband von $w$ (als $\F_q$-Vektorraum).
%\end{beispiel}
Genauer:
\begin{align*}
	\big(\L(V)\big)(w)=\L(w)
\end{align*}

Probleme treten oft bei Paradigmenwechsel auf.

\begin{erinnerung}
	\begin{align*}
		V/u=w/u=\set{x+u\mid x\in V}\text{ als Vektorraum}
	\end{align*}
\end{erinnerung}

\begin{satz}
	\begin{align*}
		\L(V)/u\cong\L(U/v)
		\qquad\text{mittels}\qquad
		w\mapsto w/u\\
		\und\qquad
		\dim(V(u)=\dim(V)-\dim(U)
	\end{align*}
	Für $u\leq_\L w$ sei 
	\begin{align*}
		\Delta(u,w):=-\dim(u)+\dim(w)=\dim(w/u)
	\end{align*}
\end{satz}

\begin{align*}
	\begin{pmatrix}
		n\\
		i
	\end{pmatrix}_q\overset{\dim(V)=n}{=}
	\begin{pmatrix}
		\dim(V)\\
		i
	\end{pmatrix}:=
	\#\begin{pmatrix}
		\L(V)\\
		i
	\end{pmatrix}
\end{align*}
falls $V$ $n$-dimensionaler Vektorraum über $\F_q$.

\begin{figure}[H]
	\begin{center}
			% This work is licensed under the Creative Commons
% Attribution-NonCommercial-ShareAlike 4.0 International License. To view a copy
% of this license, visit http://creativecommons.org/licenses/by-nc-sa/4.0/ or
% send a letter to Creative Commons, PO Box 1866, Mountain View, CA 94042, USA.



\tikzset{every picture/.style={line width=0.75pt}} %set default line width to 0.75pt        

\begin{tikzpicture}[x=0.75pt,y=0.75pt,yscale=-1,xscale=1]
%uncomment if require: \path (0,300); %set diagram left start at 0, and has height of 300

%Curve Lines [id:da9559714668165431] 
\draw    (258,209.93) .. controls (225,179.93) and (200,118.93) .. (254,61.93) ;


%Curve Lines [id:da870462767582893] 
\draw    (258,209.93) .. controls (302,166.93) and (314,100.93) .. (254,61.93) ;


%Shape: Circle [id:dp6905850303089136] 
\draw  [fill={rgb, 255:red, 0; green, 0; blue, 0 }  ,fill opacity=1 ] (249,66.93) .. controls (249,64.17) and (251.24,61.93) .. (254,61.93) .. controls (256.76,61.93) and (259,64.17) .. (259,66.93) .. controls (259,69.69) and (256.76,71.93) .. (254,71.93) .. controls (251.24,71.93) and (249,69.69) .. (249,66.93) -- cycle ;
%Shape: Circle [id:dp3933822162001085] 
\draw  [fill={rgb, 255:red, 0; green, 0; blue, 0 }  ,fill opacity=1 ] (253,209.93) .. controls (253,207.17) and (255.24,204.93) .. (258,204.93) .. controls (260.76,204.93) and (263,207.17) .. (263,209.93) .. controls (263,212.69) and (260.76,214.93) .. (258,214.93) .. controls (255.24,214.93) and (253,212.69) .. (253,209.93) -- cycle ;
%Straight Lines [id:da2624167968218917] 
\draw    (230,97.93) -- (286,97.93) ;


%Straight Lines [id:da8790065247209863] 
\draw    (224,146.93) -- (296,146.93) ;


%Straight Lines [id:da09868631062113131] 
\draw    (248.04,100.64) -- (277,147.93) ;

\draw [shift={(247,98.93)}, rotate = 58.52] [fill={rgb, 255:red, 0; green, 0; blue, 0 }  ][line width=0.75]  [draw opacity=0] (8.93,-4.29) -- (0,0) -- (8.93,4.29) -- cycle    ;
%Shape: Circle [id:dp07389781898233405] 
\draw  [fill={rgb, 255:red, 0; green, 0; blue, 0 }  ,fill opacity=1 ] (244,98.93) .. controls (244,97.28) and (245.34,95.93) .. (247,95.93) .. controls (248.66,95.93) and (250,97.28) .. (250,98.93) .. controls (250,100.59) and (248.66,101.93) .. (247,101.93) .. controls (245.34,101.93) and (244,100.59) .. (244,98.93) -- cycle ;
%Shape: Circle [id:dp15699358050616063] 
\draw  [fill={rgb, 255:red, 0; green, 0; blue, 0 }  ,fill opacity=1 ] (274.03,147.49) .. controls (274.83,145.93) and (276.8,144.87) .. (278.43,145.11) .. controls (280.07,145.36) and (280.76,146.82) .. (279.97,148.38) .. controls (279.17,149.93) and (277.2,151) .. (275.57,150.75) .. controls (273.93,150.51) and (273.24,149.04) .. (274.03,147.49) -- cycle ;
%Straight Lines [id:da5019157895054271] 
\draw  [dash pattern={on 4.5pt off 4.5pt}]  (247,98.93) -- (247,141.93) ;
\draw [shift={(247,143.93)}, rotate = 270] [color={rgb, 255:red, 0; green, 0; blue, 0 }  ][line width=0.75]    (10.93,-3.29) .. controls (6.95,-1.4) and (3.31,-0.3) .. (0,0) .. controls (3.31,0.3) and (6.95,1.4) .. (10.93,3.29)   ;

%Straight Lines [id:da5990497530420301] 
\draw  [dash pattern={on 4.5pt off 4.5pt}]  (275.57,150.75) -- (275.98,100.93) ;
\draw [shift={(276,98.93)}, rotate = 450.48] [color={rgb, 255:red, 0; green, 0; blue, 0 }  ][line width=0.75]    (10.93,-3.29) .. controls (6.95,-1.4) and (3.31,-0.3) .. (0,0) .. controls (3.31,0.3) and (6.95,1.4) .. (10.93,3.29)   ;


% Text Node
\draw (201,37) node   {$\dim$};
% Text Node
\draw (258,48) node   {$1_{\mathbb{L}}$};
% Text Node
\draw (198,90) node   {$j$};
% Text Node
\draw (198,146) node   {$i$};
% Text Node
\draw (195,207) node   {$0$};
% Text Node
\draw (319,147) node   {$\begin{pmatrix}
	\L\\j
\end{pmatrix}$};
% Text Node
\draw (318,92) node   {$\begin{pmatrix}
	\L\\i
\end{pmatrix}$};
% Text Node
\draw (267,226) node   {$0_{\mathbb{L}}$};
% Text Node
\draw (249,86) node   {$b$};
% Text Node
\draw (278,157) node   {$p$};


\end{tikzpicture}

			%\caption{Desargues Konfiguration}
			%\label{Abb:natSpracheModellierung}
		\end{center}
\end{figure}

Seien $p\in\begin{pmatrix}
	\L\\
	i
\end{pmatrix}$ und $b\in\begin{pmatrix}
	\L\\
	j
\end{pmatrix}$.
Dann ist
\begin{align*}
	p I_{(i,j)}^\L&=\set{b\in'\begin{pmatrix}
		\L\\
		j
	\end{pmatrix}:p\leq_\L b'}\\
	I_{(i,j)}^\L b&=\set{p'\in\begin{pmatrix}
		\L\\
		i
	\end{pmatrix}:p'\leq_\L b}\\
	I_{(i,j)}^\L b&=\begin{pmatrix}
		\L(b)\\
		i
	\end{pmatrix}\mapsto
	\begin{pmatrix}
		\dim(b)\\
		i
	\end{pmatrix}_q=\begin{pmatrix}
		j\\
		i
	\end{pmatrix}_q
\end{align*}

\begin{satz}
	Es gilt weiter
	\begin{align*}
		pI^\L_{(i,j)}&=\set{b'\in\begin{pmatrix}
			\L\\
			j
		\end{pmatrix}:p\leq_\L b'}
		\overset{1-1}{\leftrightarrow}
		\set{b'/p\mid b'\in\begin{pmatrix}
			\L\\
			j
		\end{pmatrix}:p\leq_\L b}
		=\begin{pmatrix}
			\L(V/p)\\
			j-i
		\end{pmatrix}
		\mapsto\begin{pmatrix}
			n-i\\
			j-i
		\end{pmatrix}\\
		&\dim(b'/p)=\dim(b')-\dim(p)=j-i
	\end{align*}
	Ergebnis:
	\begin{align*}
		\Inz^\L_{(i,j)}
		=\klammern{
			\begin{pmatrix}
				\L\\
				i
			\end{pmatrix},
			\begin{pmatrix}
				\L\\
				j
			\end{pmatrix},
			I_{(i,j)}^\L
		}
	\end{align*}
	hat Parametertupel
	\begin{align*}
		\klammern{
			\begin{pmatrix}
				n\\
				i
			\end{pmatrix}_q,\begin{pmatrix}
				n-i\\
				j-i
			\end{pmatrix}_q,\begin{pmatrix}
				n\\
				j
			\end{pmatrix}_q,\begin{pmatrix}
				j\\
				i
			\end{pmatrix}_q
		}
	\end{align*}
\end{satz}

\begin{definition}
	Ist $f\colon\N\to\N_+$ eine Abbildung, so sei 
	\begin{align*}
		f!\colon\N\to\N_+,\qquad n\mapsto f(1)\mal f(2)\mal\ldots\mal f(n)
	\end{align*}
	Setze dann
	\begin{align*}
		\begin{pmatrix}
			n\\
			i
		\end{pmatrix}_f
		:=\frac{(f!)(n)}{(f!)(i)\mal(f!)(n-i)}
	\end{align*}
	Sonderfall: $(f!)(1)=f(1)$ und $(f!)(0)=1$
\end{definition}

\begin{satz}
	Dann gilt
	\begin{align*}
		\begin{pmatrix}
			n\\
			i
		\end{pmatrix}_f
		\mal\begin{pmatrix}
			n-i\\
			j-i
		\end{pmatrix}_f
		=\begin{pmatrix}
			n\\
			j
		\end{pmatrix}_f\mal\begin{pmatrix}
			j\\
			i
		\end{pmatrix}_f
	\end{align*}
	für beliebige $i,j\in\N$ mit $i\leq j\leq n$.
\end{satz}

\begin{beispiel}
	\begin{align*}
		f(n)&:=\#\set{t\in\L\big(\F^n_q\big)\mid\dim(t)1}\\
		f(1)&:=1\\
		\begin{pmatrix}
			n\\
			1
		\end{pmatrix}_f
		&=f(n)=\frac{q^n-1}{q-1}
	\end{align*}
\end{beispiel}

Wiederholung:
$V\simeq\F_q^n$, $\L=\L(V)=(L(V),\leq)$ Unterraumverband.
Hierbei ist "$\leq$" die Untervektorraum-Enthalten-Seins-Relation.
\begin{align*}
	L_i&:=\set{U\leq V\mid \dim(U)=i}
\end{align*}
Dann ist 
\begin{align*}
	\Inz_{i,j}^\L=\big(L_i,L_j,I_{i,j}\big)
	\qquad\mit\qquad
	I_{i,j}:=\set{(p,b)\in L_i\times L_j\mid p\leq b}
\end{align*}
taktische Konfiguration mit Parametertupel
\begin{align*}
	\klammern{
		\begin{pmatrix}
			n\\
			i
		\end{pmatrix}_q,\begin{pmatrix}
			n-i\\
			j-i
		\end{pmatrix}_q,\begin{pmatrix}
			n\\
			j
		\end{pmatrix}_q,\begin{pmatrix}
			j\\
			i
		\end{pmatrix}_q
	}
\end{align*}
wobei
\begin{align*}
	\begin{pmatrix}
		n\\
		1
	\end{pmatrix}_q
	=\frac{q^n-1}{q-1}
\end{align*}

\begin{bemerkungnr}
	$\begin{pmatrix}
		n\\
		i
	\end{pmatrix}_q$ heißt \define{Gauß-Koeffizient}
	\index{Gauß-Koeffizient}
	\begin{align*}
		\begin{pmatrix}
			n\\
			i
		\end{pmatrix}_q
		=\frac{
			\big(q^n-1\big)\mal\ldots\mal\big(q^{n-ii+1}-1\big)
		}{
			(q-1)\mal\ldots\mal\big(q^i-1\big)
		}
	\end{align*}
	Dies folgt aus
	\begin{align*}
		\begin{pmatrix}
			n\\
			i
		\end{pmatrix}_q\mal\begin{pmatrix}
			n-i\\
			j-i
		\end{pmatrix}_q
		=\begin{pmatrix}
			n\\
			j
		\end{pmatrix}_q\mal\begin{pmatrix}
			j\\
			i
		\end{pmatrix}_q
	\end{align*}
	und 
	\begin{align*}
		\begin{pmatrix}
			n\\
			1
		\end{pmatrix}_q=\frac{q^n-1}{q-1}
	\end{align*}
	
	Es gilt irgendwie auch:
	\begin{align*}
		\#V&=q^n\\
		\#\big(V\setminus\set{\vec{0}}\big)&=q^n-1\\
		\#\F_g v&=q\qquad\forall v\in V\setminus\set{\vec{0}}\\
		\#\big(\F_q\setminus\set{0}\big)v&=q-1\\
		\implies (q-1)\mal\begin{pmatrix}
			n\\
			1
		\end{pmatrix}_q&=q^n-1
	\end{align*}
\end{bemerkungnr}

\begin{bemerkungnr}
	Sei $f\colon\N\to\N_+$ Abbildung. 
	Setze
	\begin{align*}
		f!\colon\N\to\N_+,\qquad n\mapsto \prod\limits_{i\in[n]} f(i)=f(1)\mal f(2)\mal\ldots\mal f(n)
	\end{align*}
	Beachte $\prod\limits_{i\in\emptyset} f(i)=1$.
	Für $i,n\in\N$ sei
	\begin{align*}
		\begin{pmatrix}
			n\\
			i
		\end{pmatrix}_f
		:=\left\lbrace\begin{array}{cl}
			\frac{(f!)(n)}{(f!)(i)\mal(f!)(n-i)},&\falls i\leq n\\
			0, &\falls i>n
		\end{array}\right.
	\end{align*}
	Also ist 
	\begin{align*}
		\begin{pmatrix}
			n\\
			i
		\end{pmatrix}_f
		=\frac{
			f(n)\mal\ldots\mal f(n-i+1)
		}{
			f(1)\mal\ldots\mal f(i)
		}\qquad\forall i\leq n
	\end{align*}
	Für eine Abbildung
	\begin{align*}
		[~]\colon\N\times\N\to\N,\qquad
		(i,n)\mapsto\begin{bmatrix}
			n\\
			i
		\end{bmatrix}\\
		\mit~
		\supp([~])
		\overset{\Def}{=}
		\set{(i,n)\in\N\times\N\mid i\leq n}
		=\set{(i,n)\in\N\times\N\mid i\leq n}
	\end{align*}
	 Link: \url{https://de.wikipedia.org/wiki/Inzidenzalgebra}
	 Es gilt stets:
	 \begin{align}\label{eq:24.4Stern}\tag{$*$}
	 	\begin{bmatrix}
	 		n\\
	 		i
	 	\end{bmatrix}\mal\begin{bmatrix}
	 		n-i\\
	 		j-i
	 	\end{bmatrix}
	 	=\begin{bmatrix}
	 		n\\
	 		j
	 	\end{bmatrix}\mal\begin{bmatrix}
	 		j\\
	 		i
	 	\end{bmatrix}
	 	\qquad\forall i\leq j\leq n
	 \end{align}
	 Das heißt
	 \begin{align*}
	 	\klammern{
			\begin{bmatrix}
				n\\
				i
			\end{bmatrix},
			\begin{bmatrix}
				n-i\\
				j-i
			\end{bmatrix};\begin{pmatrix}
				n\\
				j
			\end{pmatrix},\begin{bmatrix}
				j\\
				i
			\end{bmatrix}
	 	}
	 \end{align*}
	 ist formales taktisches Konfigurationsquadrupel.
\end{bemerkungnr}

Ist umgekehrt $[~]\colon\leq_\N\to\N_+$ mit \eqref{eq:24.4Stern} gegeben, so ist $[~]=(~)_f$ für
\begin{align*}
	f\colon\N\to\N_+,\qquad n\mapsto\begin{bmatrix}
		n\\
		1
	\end{bmatrix}
\end{align*}
Folglich ist $f(1)=1$.

\begin{bemerkungnr}\
	\begin{enumerate}[label=(\arabic*)]
		\item Warnung! Gegenbeispiel:
	$[~]\colon\leq_\N\to\N_+,~(i,n)\mapsto 2$ erfüllt \eqref{eq:24.4Stern}, aber es existiert \betone{kein} $f\colon\N\to\N_+$ mit $[~]=(~)_f$.
		\item $[~]\colon\leq_\N\to\N_+,~(i,n)\mapsto1$ erfüllt $[~]=(~)_f$ für $f\colon\N\to\N_+,~ n\mapsto 1$.
		\item \eqref{eq:24.4Stern} ist ein Beispiel einer Funktionalgleichung.
	\end{enumerate}
\end{bemerkungnr}

\begin{beispiel}
	Weitere "interessanter" taktischer Konfigurationen via freier endlichen Ring Moduln.\nl
	Sei $\S=\big(S,+,\mal,0,1\big)$ ein \define{Semiring (mit Eins)}, d.h. $\S_{\add}:=\big(S,+,\mal\big)$ ist kommutatives Monoid und $\S_{\mult}:=\big(S,\mal,1\big)$ ist Monoid derart, dass das Distributivgesetz
	\index{Semiring}
	\begin{align*}
		a\mal(b+c)=(a\mal b)+(a\mal c)=:a\mal b+a\mal c\qquad\forall a,b,c\in S
	\end{align*}
	gilt.
	Sind $\S=\big(S,+,\mal,0,1\big)$ und $\S'=\big(S',+',\mal',0',1'\big)$ Semiringe, so heiße eine Abbildung $\varphi\colon S\to S'$ \define{Morphismus}, falls $\varphi$ Morphismus von $\S_{\add}$ nach $\S'_{\add}$ und $\varphi$ Morphismus von $\S_{\mult}$ nach $\S_{\mult}'$ bildet, d.h. es gilt für alle $a,b\in S$:
	\index{Morphismis}
	\begin{enumerate}
		\item $\begin{aligned}
			\varphi(a+b)&=\varphi(a)+'\varphi(b)
		\end{aligned}$
		\item $\begin{aligned}
			\varphi(a\mal b)=\varphi(a)\mal'\varphi(b)
		\end{aligned}$
		\item $\begin{aligned}
			a\mal 0=0=0\mal a
		\end{aligned}$
		\item $\begin{aligned}
			1\neq 0
		\end{aligned}$
		\item $\begin{aligned}
			\varphi(1)=1'
		\end{aligned}$
	\end{enumerate}
	
	Ist $\M=(M,+\vec{0)})$ kommutativer Monoid und $\S=(S,+,\mal,0,1)$ Semiring (SR), so heiße 
	\index{Sclaling}
	\begin{align*}
		\scal\colon S\to\End(\M)
	\end{align*}
	ein \define{Scaling}, falls
	\begin{align*}
		\End(\M):=\big(\End(M),+,\mal,0,\id_M\big)
	\end{align*}
	Endomorphismen-Semiring von $\M$.\\
	$\scal$ Morphismus von $\S$ nach $\End(\M)$.\nl
	Die Abbildung $S\times M\to M,~(s,m)\mapsto sm:=\big(\scal(s)\big)m$
	heiße das zugehörige \define{Skalarprodukt}.
	\index{Skalarprodukt}\\
	Wir nennen dann $(\M,\S,\scal)$ einen \define{Semiring-Modul} bzw. sagen $\M$ bildet (bzgl. $\scal$) einen Modul über $\S$.
	\index{Semiring-Modul}
\end{beispiel}

\begin{bemerkungnr}
	Sei $\F_q$ endlicher Körper ($q$-elementig) und sei $V$ $n$-dimensionaler Vektorraum über $\F_q$ ($n\in\N_+$).
	Für $i,j\in\N$ mit $i\leq j\leq n$ und $\L:=\L(V)$ Unterraumverband von $V$ ist dann
	\begin{align*}
		\Inz_{(i,j)}^\L&:=\big(L_i,L_j,I_{i,j}\big)\mit\\
		L_i&:=\set{p\leq V\mid\dim(p)=i}
		\und \\
		L_j&:=\set{b\leq V\mid\dim(b)=j}
		\und\\
		I_{i,j}&:=\set{(p,b)\in L_i\times L_j\mid p\leq b}
	\end{align*}
	eine taktische Konfiguration mit Parametertupel
	\begin{align*}
		\klammern{
			\begin{pmatrix}
				n\\
				i
			\end{pmatrix}_q,
			\begin{pmatrix}
				n-i\\
				j-i
			\end{pmatrix}_q;
			\begin{pmatrix}
				n\\
				j
			\end{pmatrix}_q,
			\begin{pmatrix}
				j\\
				i
			\end{pmatrix}_q
		}
	\end{align*} wobei $\begin{pmatrix}
		n\\
		i
	\end{pmatrix}_q:=\begin{pmatrix}
		n\\
		i
	\end{pmatrix}_f$ für $f\colon\N\to\N_+,~i\mapsto q^i-1$ gerader der Gaußkoeffizient zu $n$ über $i$ bzgl. $q$ ist.
	Gilt $0<i<j<n$, so ist
	\begin{align*}
		\Aut\big(\Inz_{i,j}^\L\big)\cong\PGL_n(\F_q)
	\end{align*}
	(ohne Beweis). Auf der linken Seite steht die projektive lineare Gruppe von $\F_q^n$.
	\begin{align*}
		f!\colon\N\to\N_+,\qquad n\mapsto\prod\limits_{i\in[n]} f(i)\\
		\begin{pmatrix}
			n\\
			i
		\end{pmatrix}_f:=\frac{(f!)(n)}{(f!)(i)\mal (f!)(n-i)}\qquad\forall 0\leq i\leq n
	\end{align*}
	Die Abbildung $[~]\colon\leq_\N\to\N_+$ erfüllt
	\begin{align*}
		\begin{bmatrix}
			n\\
			i
		\end{bmatrix}\mal
		\begin{bmatrix}
			n-i\\
			j-i
		\end{bmatrix}=\begin{bmatrix}
			n\\
			j
		\end{bmatrix}\mal
		\begin{bmatrix}
			j\\
			i
		\end{bmatrix}\qquad\forall i,j,n\in\mit i\leq j\leq n
	\end{align*}
	Und $\begin{bmatrix}
		1\\
		1
	\end{bmatrix}=1$ genau dann, wenn $[~]=(~)_f$ für $f\colon\N\to\N_+,~n\mapsto\begin{bmatrix}
		n\\
		1
	\end{bmatrix}$
\end{bemerkungnr}

\textbf{Frage:} Lässt sich obiges auf endliche freie Ring-Moduln ausweiten?

\subsection{Ausflug: Semiring-Moduln (Semimodule)}
Sei $\S=(S,+,\mal,0,1)$ ein Semiring ($\S_{\add}$ kommutatives Monoid und $\S_{\mult}$ Monoid)
Es gilt also:
\begin{enumerate}
	\item $a\mal(b+c)=(a\mal b)+(a\mal c)=:a\mal c+b\mal c$ Distri
	\item $(a+b)\mal c=a\mal c+b\mal c$ Distri
	\item $0\mal a=0=a\mal 0$ Nullabsorption
	\item $a\neq 0$ Reichhaltigkeit
\end{enumerate}

\begin{beispiel}
	\begin{itemize}
		\item Boolscher Semiring: $\big(\set{0,1},+,\mal,0,1\big)$ mit $1+1=1$
		\item $\ul{\N}=\big(\N,+,\mal,0,1\big)$ ist der natürliche Semiring
		\item $\F_2=\big(\set{0,1},+,\mal,0,1\big)$ mit $1+1=0$
		\item Ringe, Körper $\Z,\Q,\R,\C$
		\item \define{tropischer Semiring} $\big([0,\infty],\min,+,\infty,0\big)$
	\end{itemize}
\end{beispiel}

\begin{definition}
	Seien $\S=(S,+,\mal,0,1)$ und $\S'=(S',+',\mal',0',1')$ Semiringe.
	Eine Abbildung $\varphi\colon S\to S'$ heißt \define{Morphismus} von $\S$ nach $\S'$, falls $\varphi$ Morphismus von $\S_{\add}$ nach $\S_{\add}'$ und $\varphi$ Morphismus von $\S_{\mult}$ nach $\S_{\mult}'$.\nl
	Ein bijektiver Morphismus heißt \define{Isomorphismus}.
	Ein Morphismus in sich heißt \define{Endomorphismus}.


\begin{itemize}
	\item $\S$ \define{Ring}: $\S$ Semiring und $\S_{\add}$ Gruppe
	\item $\S$ \define{Divisionsring / Schiefkörper:} $\S$ Ring und $\S_{\mult}^\times=\big(S\setminus\set{0},\mal,1\big)$ Gruppe
	\item $\S$ \define{Körper}: $\S$ Divisionsring und $\S_{\mult}$ kommutativ
\end{itemize}

\end{definition}

Ist $\M=(M,+,\vec{0})$ kommutatives Monoid, so ist $\End(M)$ Menge der Endomorphismen von $\M$ d.h. $\varphi\colon M\to M$ Abbildung mit
\begin{align*}
	\varphi(x+y)=\varphi(x)+\varphi(y)\qquad\forall x,y\in M
\end{align*}
und $\varphi(\vec{0})=\vec{0}$.
Dann ist 
\begin{align*}
	\End(\M):=\big(\End(M),+,\circ,0,\id_M\big)
\end{align*}
wobei $\circ$ die Abbildungsverkettung ist und
	\begin{align*}
		\varphi+\psi\colon:M\to M,\qquad x\mapsto \varphi(x)+\psi(x)\qquad\forall \varphi,\psi\in\End(M)\\
		0\colon M\to M,\qquad x\mapsto\vec{0}
	\end{align*}
	ein Semiring, der sogenannte \define{Endomorphismen-Semiring zu $\M$}.
	
\begin{lemma}
	Ist $\M$ kommutative Gruppe, so ist $\End(\M)$ ein Ring.
\end{lemma}

\subsubsection{Lineare Algebra über Semiring-Moduln}
Sei $\S=(S,+,\mal,0,1)$ Semiring ("abstrakte Skalare", skalarer Bereich) und $\M=(M,+,\vec{0})$ kommutatives Monoid ("abstrakte Vektoren", vektorieller Bereich) und sei $\scal\colon\End(\M)$ Morphismus, genannt \define{Scaling}.
\index{Scaling}
Dann nennen wir $\mathcal{M}:=(\M,\S,\scal)$ auch \define{Semiring-Modul}.
\index{Semiring-Modul}
Es wird durch $\mathcal{M}$ ein \define{skalare Multiplikation}
\index{skalare Multiplikation}
\begin{align*}
	S\times M\to M,\qquad (s,m)\mapsto s\mal m:=\big(\scal(s)\big) m
\end{align*}
induziert. Dies ist die \define{kontravariante Sicht}.\index{kontravariante Sicht}\\
\define{Kovariante Sicht:}\index{kovariante Sicht}
\begin{align*}
	M\times S\to M,\qquad (m,s)\mapsto m\mal s:=m(s-\scal)
\end{align*}
wobei $s-\scal$ das zu $s\in S$ gehörige $s$-Scaling sei, d.h. $s-\scal:=\scal(s)$.\\
Für $\mathcal{M}$ schreiben wir auch $\S^\M$ oder sagen $\M$ ist Semiring über $\S$ via $\scal$.\\
(Kovariante bzw. Kontravariante Sicht ist nur eine Frage der Notation)

\begin{beispiel}[Freie Semiring-Moduln]\enter
	Sei $\S=(S,+,\mal,0,1)$ Semiring und sei $N$ eine Menge.
	Dann sei
	\begin{align*}
		\Mod(\S,N):=\klammern{\S_{\add}^{(N)},\S,\scal}
		\mit
		\S_{\add}^{(N)}:=\set{u\in S^N\mid\supp(u)\text{ endlich}}
		\mit\\
		\supp(u):=\set{i\in N\mid u(i)\neq 0}
		u+w\colon N\to S,\qquad i\mapsto u(i)+w(i)
		\vec{0}\colon N\to S,\qquad i\mapsto 0
	\end{align*}
	$\scal\colon S\to\End\klammern{\S_{\add}^{(N)}}$ via jedes $s\in S$ erfüllt $\scal(s)\colon S^{(N)}\to S^{(N)},~u\mapsto s\mal u:=(s\mal u(i))_{i\in\N}$, d.h. $s\mal u\colon N\to S,~i\mapsto s\mal u(i)$.
	ein Semiring-Modul, der sogenannte $N$-freie Semiring-Modul über $\S$.
\end{beispiel}


	% This work is licensed under the Creative Commons
% Attribution-NonCommercial-ShareAlike 4.0 International License. To view a copy
% of this license, visit http://creativecommons.org/licenses/by-nc-sa/4.0/ or
% send a letter to Creative Commons, PO Box 1866, Mountain View, CA 94042, USA.

\section{Elemente der Linearen Algebra}
Ziel: Bereitstellung einiger Hilfsmittel aus der linearen Algebra.
Literatur:
\begin{enumerate}[label=(\arabic*)]
	\item G. Fischer (1997) \emph{Lineare Algebra} \cite{fischerLinAlg}
	\item M. Koecher (1997) \emph{Lineare Algebra und analytische Geometrie} \cite{koecher2013lineare}
\end{enumerate}

Im gesamten Abschnitt ist
\begin{align*}
	\R^n:=\set{\begin{pmatrix}
		x_1\\
		\vdots\\
		x_n
	\end{pmatrix}
	:x_i\in\R\quad\forall 1\leq i\leq n}\qquad\forall n\in\N
\end{align*}
der gewöhnliche \define{euklidische Raum} versehen mit dem kanonischen Skalarprodukt
\begin{align*}
	\scaProd{x}{y}:=\sum\limits_{i=1}^n x_i\mal y_i\qquad\forall x,y\in\R^n
\end{align*}
und zugehöriger \define{euklidische Norm}
\begin{align*}
	\norm{x}:=\sqrt{\scaProd{x}{x}}
	=\sqrt{\sum\limits_{i=1}^n x_i^2}
	\qquad\forall x\in\R
\end{align*}

\setcounter{satz}{-1}
\begin{definition}\label{def2.0}\
	\begin{enumerate}[label=(\arabic*)]
		\item $x,y\in\R^n$ heißen \define{orthogonal}, in Zeichen
		\index{orthogonal}
		\begin{align*}
			x\perp y:\iff\scaProd{x}{y}=0
		\end{align*}
		\item Sei $U\subseteq\R^n$ und $x\in\R^n$. Dann setze
		\begin{align*}
			x\perp U:\iff\forall u\in U: x\perp u
		\end{align*}
		\item Das \define{orthogonale Komplement} von $U\subseteq\R^n$ ist
		\index{orthogonales Komplement}
		\begin{align*}
			U^\perp:=\set{x\in\R^n:x\perp U}
		\end{align*}
	\end{enumerate}
\end{definition}

\begin{satz}\label{satz2.1}\
	\begin{enumerate}[label=(\arabic*)]
		\item $\begin{aligned}
			x\perp y\implies \norm{x+y}^2=\norm{x}^2+\norm{y}^2
		\end{aligned}$ (Pythagoras)\label{item:satz2.1(1)}
		\index{Pythagoras}
		\item $U^\perp$ ist Untervektorraum (UVR / UR) von $\R^n$\label{item:satz2.1(2)}
		\item Falls $U\subseteq\R^n$ UVR von $\R^n$ ist, gilt:\label{item:satz2.1(3)}
		\begin{enumerate}[label=(\roman*)]
			\item $\begin{aligned}\label{item:satz2.1(3i)}
				U\cap U^\perp=\set{0}
			\end{aligned}$
			\item $\begin{aligned}
				\big(U^\perp)^\perp=U\label{item:satz2.1(3ii)}
			\end{aligned}$
		\end{enumerate}
	\end{enumerate}
\end{satz}

\begin{proof}
	\betone{Zeige \ref{item:satz2.1(1)}:}
	\begin{align*}
		\norm{x+y}^2
		&=\scaProd{x+y}{x+y}
		=\underbrace{\scaProd{x}{x}}_{=\norm{x}^2}+2\mal\underbrace{\scaProd{x}{y}}_{=0}+\underbrace{\scaProd{y}{y}}_{=\norm{y}^2}
	\end{align*}
	\betone{Zeige \ref{item:satz2.1(2)}:} Seien $x,y\in U^\perp$ und seien $\alpha,\beta\in\R$
	\begin{align*}
		\scaProd{\alpha\mal x+\beta\mal y}{u}
		\overset{\Lin}&{=}
		\alpha\mal\underbrace{\scaProd{x}{u}}_{=0}+\beta\mal\underbrace{\scaProd{y}{u}}_{=0}\qquad\forall u\in U\\
		&\implies\alpha\mal x+\beta\mal y\in U^\perp
	\end{align*}
	\betone{Zu \ref{item:satz2.1(3)} zeige \ref{item:satz2.1(3i)}:}
	Da $U$ und $U^\perp$ Vektorräume sind, gilt $0\in U,U^\perp$ und somit $U\cap U^\perp\supseteq\set{0}$.
	\begin{align*}
		x\in U\cap U^\perp\implies\scaProd{x}{x}=0
		\implies\norm{x}=0
		\implies x=0
	\end{align*}
	\betone{Zu \ref{item:satz2.1(3)} zeige \ref{item:satz2.1(3ii)}:} 
	Zeige "$\supseteq$":
	Sei $x\in U$ und $y\in U^\perp$. Zeige
	\begin{align}\label{eq:ProofSatz2.1Stern}\tag{$*$}
		\scaProd{x}{y}=0
	\end{align}
	Da $y\in U^\perp$, gilt $\scaProd{y}{u}=0$ für alle $u\in U$, also insbesondere für $u=x\in U$.
	Somit folgt \eqref{eq:ProofSatz2.1Stern}.\\
	Für Teil "$\subseteq$" siehe Koecher \cite{koecher2013lineare}, Seite 160.
\end{proof}

\begin{satz}\label{satz2.2}
	Seien $U,V\subseteq\R^n$ UVR des $\R^n$ und 
	\begin{align*}
		U+V:=\set{u+v:u\in U,v\in V}\subseteq\R^n
	\end{align*}
	die \define{Summe} von $U$ und $V$.
	Dann ist $U+V$ wieder ein UVR von $\R^n$.
	Falls $U\cap V=\set{0}$, so schreibt man
	\begin{align*}
		U\oplus V:=U+V
	\end{align*}
	für die \define{direkte Summe} von $U$ und $V$.
	\index{direkte Summe}
	Es gilt:
	\begin{align*}
		x\in U\oplus V\implies\exists! u\in U,\exists! v\in V: x=u+v
	\end{align*}
	Wichtig ist hierbei, dass $u\in U$, $v\in V$ \betone{eindeutig} bestimmt sind.
\end{satz}

\begin{proof}
	Dass $U+V$ bzw. $U\oplus V$ ein  UVR ist, ist klar.\\
	Sei $x\in U\oplus V$ und angenommen $x=u+v=\tilde{u}+\tilde{v}$ mit $u,\tilde{u}\in U$ und $v,\tilde{v}\in V$.
	Dann folgt:
	\begin{align*}
		u+v=\tilde{u}+\tilde{v}
		\implies
		\underbrace{u-\hat{u}}_{\in U}=\underbrace{\tilde{v}-v}_{\in V}\in U\cap V
		\overset{\Vor}{=}\set{0}\\
		\implies u=\tilde{u}\und\tilde{v}=v
	\end{align*}
\end{proof}

\begin{satz}\label{satz2.3}
	Für jeden UVR $U\subseteq\R^n$ des $\R^n$ gilt:
	\begin{align*}
		\R^n=U\oplus U^\perp
	\end{align*}
\end{satz}

\begin{proof}
	Siehe Fischer \cite{fischerLinAlg}, Seite 283.
\end{proof}

\begin{definition}\label{def2.4}
	Sei $U\subseteq\R^n$ UVR des $\R^n$.	
	Gemäß Satz \ref{satz2.2} und Satz \ref{satz2.3} existiert für jedes $x\in\R^n$ genau ein Paar 
	$(u,v)\in U\times U^\perp$ mit $x=u+v$.
	Die Abbildung
	\begin{align*}
		P=P_U\colon \R^n\to U\subseteq\R^n,\qquad P(x):= u
	\end{align*}
	ist wegen Satz \ref{satz2.2} und Satz \ref{satz2.3} wohldefiniert.
	Weitere Bezeichnung:
	$P=P_U$ heißt \define{orthogonale Projektion auf $U$}.
	\index{orthogonale Projektion}
	Analog: $P_{U^\perp}(x):=v$.
\end{definition}

\begin{satz}\label{satz2.5}\
	\begin{enumerate}[label=(\arabic*)]
		\item $P=P_U$ ist linear \label{item:satz2.5(1)}
		\item $\begin{aligned}
			P(u)=u\qquad\forall u\in U \label{item:satz2.5(2)}
		\end{aligned}$
	\end{enumerate}
\end{satz}

\begin{proof}
	\betone{Zeige \ref{item:satz2.5(1)}:}
	Seien $x_1,x_2\in\R^n$. Dann:
	\begin{align*}
		x_i&=u_i+v_i,\qquad\exists u_i\in U,~v_i\in U^\perp\\
		i\in\set{1,2}\implies
		x_1+x_2&=\big(u_1+v_1\big)+\big(u_2+v_2\big)
		\overset{\text{Kommu+Asso}}{=}\big(\underbrace{u_1+u_2}_{\in U}\big)+\big(\underbrace{v_1+v_2}_{\in U^\perp}\big)\\
		\implies P(x_1+x_2)&=P\klammern[\Big]{\big(\underbrace{u_1+u_2}_{\in U}\big)+\big(v_1+v_2\big)}
		\overset{\Def}{=}u_1+u_2
		\overset{\Def}{=}P(x_1)+P(x_2)\\
		&\implies P\text{ ist additiv}\\
		P(\lambda\mal x_1)
		\overset{\text{Distri}}&{=}
		\P\big(\underbrace{\lambda\mal u_1}_{\in U}+\lambda\mal v_1)
		\overset{\Def}{=}\lambda\mal u_1
		=\lambda\mal P(x_1)
	\end{align*}
	\betone{Zeige \ref{item:satz2.5(2)}:}
	\begin{align*}
		u=\underbrace{u}_{\in U}+\underbrace{0}_{\in U^\perp}
		\implies P(u)=u
	\end{align*}
\end{proof}

\begin{notation}
	Sei $M(m\times n)$ die Menge aller $m\times n$-Matrizen.
	Für $A\in M(m\times n)$ sei $A':=A^T\in M(n\times m)$ die Transponierte von $A$
	Beachte: $\scaProd{u}{v}=u'\mal v$
\end{notation}

\begin{satz}\label{satz2.6}
	Zu jeder linearen Abbildung $A\colon\R^n\to\R^m$ existiert genau ein $\ul{A}\in M(m\times n)$ mit
	\begin{align}\label{eq:satz2.6}\tag{$*$}
		A(x)=\ul{A}\mal x\qquad\forall x\in\R^n
	\end{align}
\end{satz}

\begin{proof}
	Sei $e_i\in\R^n$ der $i$-te Einheitsvektor im $\R^n$ ($i\in\set{1,\ldots,n}$)
	und sei
	\begin{align*}
		\ul{A}:=\klammern[\Big]{A(e_1),\ldots,A(e_n)},
	\end{align*}
	also $A(e_i)$ ist die $i$-te Spalte von $\ul{A}$.
	Also ist $\ul{A}\in M(m\times n)$ und $\ul{A}$ erfüllt \eqref{eq:satz2.6}, denn mit $x=\begin{pmatrix}
		x_1\\
		\vdots\\
		x_n
	\end{pmatrix}$ gilt:
	\begin{align*}
		\ul{A}\mal x
		&=\ul{A}\klammern{\sum\limits_{i=1}^n x_i\mal e_i}
		\overset{\text{Distri}}{=}
		\sum\limits_{i=1}^n x_i\mal\underbrace{\ul{A}\mal e_i}_{
			\begin{subarray}{c}
				=i\text{-te Spalte}\\
				\text{von }\ul{A}
			\end{subarray}
		}
		=\sum\limits_{i=1}^n x_i\mal A(e_i)
		\overset{\Lin}{=}
		A\klammern{\sum\limits_{i=1}^n x_i\mal e_i}
		=A(x)
	\end{align*}
	\betone{Zeige Eindeutigkeit:}
	\begin{align*}
		A(x)=\ul{A}\mal x&=\ul{B}\mal x &\forall& x\in\R^n\\
		\implies (\ul{A}-\ul{B})\mal x&=0 &\forall& x\in\R^n
	\end{align*}
	Wähle $x=e_i$. Dann folgt $\ul{A}=\ul{B}$.
\end{proof}

Wegen Satz \ref{satz2.6} identifiziere $A$ mit der zugehörigen \define{Darstellungsmatrix}
\index{Darstellungsmatrix}
$\ul{A}$, also
\begin{align*}
	A(x)\cong A\mal x
\end{align*}

Es gilt für lineare Abbildungen $A,B$: 
\begin{enumerate}[label=(\roman*)]
	\item $\begin{aligned}
		\ul{A+B}=\ul{A}+\ul{B} \label{item:nachSatz2.6(i)}
	\end{aligned}$
	\item $\begin{aligned}
		\ul{\lambda\mal A}=\lambda\mal\ul{A} \label{item:nachSatz2.6(ii)}
	\end{aligned}$
	\item $\begin{aligned}
		\ul{A\mal B}=\ul{A}\mal\ul{B} \label{item:nachSatz2.6(iii)}
	\end{aligned}$
\end{enumerate}

Wegen Satz \ref{satz2.6} und \ref{item:nachSatz2.6(i)} und \ref{item:nachSatz2.6(ii)} ist die Abbildung
\begin{align*}
	\Hom(\R^n,\R^m)\to M(m\times n),\qquad A\mapsto\ul{A}
\end{align*}
ein \define{Isomorphismus}.
Hierbei ist $\Hom(\R^n,\R^m)$ der Vektorraum aller linearen Abbildungen von $\R^n$ nach $\R^m$.

\begin{satz}\label{satz2.7}
	Sei $P_U$ die orthogonale Projektion auf UVR $U\subseteq\R$.
	Dann gilt:
	\begin{align*}
		P^2=P,\qquad \ul{P^2}=\ul{P}
	\end{align*}
\end{satz}

\begin{proof}
	Sei $x\in\R^n$. Dann gilt:
	\begin{align*}
		x=u+v\qquad\exists u\in U,v\in U^\perp
		\implies P^2(x)\overset{\Def}{=}(P\circ P)(x)
		=P\big(P(x)\big)
		=P(u)
		\overset{\ref{satz2.5}\ref{item:satz2.5(2)}}{=}
		u
	\end{align*}
	Zur zweiten Gleichung:
	\begin{align*}
		\ul{P^2}=\ul{P\circ P}\overset{\ref{item:nachSatz2.6(iii)}}{=}\ul{P}^2
	\end{align*}
\end{proof}

\begin{satz}\label{satz2.8}
	Sei $P_U$ die orthogonale Projektion auf UVR $U\subseteq\R^n$.
	Dann gilt:
	\begin{enumerate}[label=(\arabic*)]
		\item $P$ ist \define{selbstadjungiert}, d.h. \label{item:satz2.8(1)}
		\index{selbstadjungiert}
		\begin{align*}
			\scaProd{P(x)}{y}=\scaProd{x}{P(y)}\qquad\forall x,y\in\R^n
		\end{align*}
		\item $\ul{P'}=\ul{P}$, d.h. $\ul{P}$ ist \define{symmetrisch} \label{item:satz2.8(2)}
	\end{enumerate}
\end{satz}

\begin{proof}
	\betone{Zeige \ref{item:satz2.8(1)}:}\\
	Sei $x=u+v$, $y=\tilde{u}+\tilde{v}$ für $u,\tilde{u}\in U$, $v,\tilde{v}\in U^\perp$.
	Dann gilt:
	\begin{align*}
		\scaProd{P(x)}{y}
		&=\scaProd{u}{\tilde{u}+\tilde{v}}
		=\scaProd{u}{\tilde{u}}+\underbrace{\scaProd{u}{\tilde{v}}}_{\overset{\tilde{v}\in U^\perp}{=}0}
		=\scaProd{u}{\tilde{u}}
		\overset{v\in U^\perp}{=}
		\scaProd{u+v}{\tilde{u}}
		=\scaProd{x}{P(y)}
	\end{align*}
	\betone{Zeige \ref{item:satz2.8(2)}:}
	\begin{align*}
		\scaProd{P(x)}{y}
		&=P(x)'\mal y
		=(\ul{P}\mal x)'\mal y
		=x'\mal \ul{P}'\mal y
		\overset{\ref{item:satz2.8(1)}}{=}
		\scaProd{x}{P(y)}
		=x'\mal P(y)
		=x'\mal\ul{P}\mal y\\
		\implies x'\mal \ul{P}'\mal y&=x'\mal\ul{P}\mal y\qquad\forall x,y\in\R^n
	\end{align*}
	Wähle $x=e_i$, $y=e_j$ für $i,j\in\set{1,\ldots,n}$.
	Somit folgt schon $\ul{P}'=\ul{P}$
\end{proof}

\begin{erinnerung}
	Für eine Matrix $M\in M(m\times n)$ und einen (passenden) Einheitsvektor $e_i$  gilt: %Fergersche Regel
	\begin{itemize}
		\item $M\mal e_i$ ist die $i$-te Spalte von $M$
		\item $e_i\mal M$ ist die $i$-te zeile von $M$
	\end{itemize}
\end{erinnerung}

\begin{satz}\label{satz2.9}
	Sei $P_U$ die orthogonale Projektion $U$ und $Q:=P_{U^\perp}$ die orthogonale Projektion auf $U^\perp$.
	Dann gilt:
	\begin{enumerate}[label=(\arabic*)]
		\item $\begin{aligned}
			 Q=\id_{\R^n}-P
		\end{aligned}$ \label{item:satz2.9(1)}
		\item $\begin{aligned}
			 \ul{Q}=I_n-\ul{P}
		\end{aligned}$ \label{item:satz2.9(2)}
	\end{enumerate}
	Hierbei ist $I_n$ die $n$-te Einheitsmatrix.
\end{satz}

\begin{proof}
	\betone{Zeige \ref{item:satz2.9(1)}:}\\
	Sei $x=u+v$ mit $u\in U$,  $v\in U^\perp$.
	Dann gilt:
	\begin{align*}
		Q(x)
		\overset{\Def}{=}
		v=x-u=\id(x)-P(x)=(\id-P)(x)
	\end{align*}
	\betone{Zeige \ref{item:satz2.9(2)}:}
	\begin{align*}
		Q\overset{\ref{item:satz2.9(1)}}&{=}
		\id-P
		\overset{\ref{item:nachSatz2.6(i)}}{=}
		\ul{\id}-\ul{P}
		=I_n-\ul{P}
	\end{align*}
\end{proof}

\begin{satz}\label{satz2.10}
	Sei $U\subseteq\R^n$ UVR von $\R^n$ und sei $x\in\R^n$.
	$u_0\in U$ heißt \define{Bestapproximation von $x$ in $U$}
	\index{Bestapproximation}
	\begin{align*}
		:\iff\norm{x-u_0}\leq\norm{x-u}\qquad\forall u\in U
	\end{align*}
	Also:
	\begin{align*}
		u_0=\argmin\limits_{u\in U}\norm{x-u}
	\end{align*}
	Es gelten:
	\begin{enumerate}[label=(\arabic*)]
		\item $\begin{aligned}
			 P_U(x) \label{item:satz2.10(1)}
		\end{aligned}$ ist die einzige Bestapproximation von $x$ in $U$.  
		\item $\begin{aligned}
			 x-P_U(x)\perp U\Big(\iff x-P_u(x)\in U^\perp\Big) \label{item:satz2.10(2)}
		\end{aligned}$ 
		\item $\begin{aligned}
			u\in U,~x-u\perp U\implies u=P_U(x) \label{item:satz2.10(3)}
		\end{aligned}$
	\end{enumerate}
\end{satz}

\begin{figure}[H] % oder ht!
	\begin{center}
		% This work is licensed under the Creative Commons
% Attribution-NonCommercial-ShareAlike 4.0 International License. To view a copy
% of this license, visit http://creativecommons.org/licenses/by-nc-sa/4.0/ or
% send a letter to Creative Commons, PO Box 1866, Mountain View, CA 94042, USA.


\tikzset{every picture/.style={line width=0.75pt}} %set default line width to 0.75pt        

\begin{tikzpicture}[x=0.75pt,y=0.75pt,yscale=-1,xscale=1]
%uncomment if require: \path (0,300); %set diagram left start at 0, and has height of 300

%Shape: Axis 2D [id:dp7098119764904821] 
\draw  (50,256.93) -- (558,256.93)(100.8,4.93) -- (100.8,284.93) (551,251.93) -- (558,256.93) -- (551,261.93) (95.8,11.93) -- (100.8,4.93) -- (105.8,11.93)  ;
%Straight Lines [id:da9071558730776327] 
\draw [line width=3]    (61,275.93) -- (518,41.93) ;


%Straight Lines [id:da7398348018165936] 
\draw    (210,68.93) ;


%Shape: Circle [id:dp7764692603132283] 
\draw   (223,61.43) .. controls (223,58.95) and (225.01,56.93) .. (227.5,56.93) .. controls (229.99,56.93) and (232,58.95) .. (232,61.43) .. controls (232,63.92) and (229.99,65.93) .. (227.5,65.93) .. controls (225.01,65.93) and (223,63.92) .. (223,61.43) -- cycle ;
%Straight Lines [id:da721943207149605] 
\draw    (229.5,65.93) -- (286.5,160.43) ;



% Text Node
\draw (531,43.93) node  [align=left] {$U$};
% Text Node
\draw (223,44.93) node  [align=left] {$x$};
% Text Node
\draw (290,177.93) node  [align=left] {$P_U(x)$};


\end{tikzpicture}

		\caption{Skizze zur Bestapproximation}
		\label{Abb:Bestapproximation}
	\end{center}
\end{figure}

\begin{proof}
	\betone{Zeige \ref{item:satz2.10(1)}:}\\
	Sei $u\in U$ beliebig.
	Dann folgt aus der Definition der Projektion:
	\begin{align*}
		x&=P_U(x)+v\qquad\exists v\in U^\perp\\
		\implies\norm{x-u}^2&=\norm[\Big]{\big(\underbrace{P_U(x)-u}_{\in U}\big) +v}^2
		\overset{\ref{satz2.1}\ref{item:satz2.1(1)}}{=}
		\underbrace{\norm{P_U(x)-u}^2}_{\geq0}+\norm{v}^2
		\geq\norm{v}^2=\norm{x-P_U(x)}^2
	\end{align*}
	Und Gleichheit gilt genau dann, wenn $u=P_U(x)$.\nl
	\betone{Zeige \ref{item:satz2.10(2)}:}
	\begin{align*}
		x&=P_U(x)+P_{U^\perp}(x)\\
		&\implies x-P_U(x)=P_{U^\perp}(x)\\
		&\implies x-P_U(x)\in U^\perp
	\end{align*}
	\betone{Zeige \ref{item:satz2.10(3)}:}
	\begin{align*}
		x-u\in U^\perp
		\implies x=\underbrace{u}_{\in U}+\underbrace{(x-u)}_{\in U^\perp}
		\implies u=P_U(x)
	\end{align*}
\end{proof}

\begin{satz}\label{satz:2.11}
	Sei
	\begin{align*}
		A=\big(a_1,\ldots,a_n\big)=\begin{pmatrix}
			b_1'\\
			\vdots\\
			b_m'
		\end{pmatrix}\in M(m\times n)
	\end{align*}
	d.h. $a_i\in\R^m$ ist $i$-te Spalte von $A$ bzw. $b_i\in\R^n$ und $b_i'=i$-te Zeile von $A$.
	Dann gilt:
	\begin{enumerate}[label=(\alph*)]
		\item $\begin{aligned}
			 \dim\Big(\spann\big(a_1,\ldots,a_n\big)\Big)
			 =\dim\Big(\spann\big(b_1,\ldots,b_m\big)\Big) 
			 \label{item:satz2.11(a)}
		\end{aligned}$\\
		Der gemeinsame Wert $\Rg(A)$ heißt \define{Rang} von $A$.
		\index{Rang}  
		\item $\begin{aligned}
			 \Rg(A)=\Rg(A')=
			 \label{item:satz2.11(b)}
		\end{aligned}$ maximale Anzahl linear unabhängiger Zeilen / Spalten von $A$
		\item $\begin{aligned}
			\Rg(A)\leq\min\set{m,n}
			\label{item:satz2.11(c)}
		\end{aligned}$
		\item $\begin{aligned}
			B\in M(n\times p)\implies\left\lbrace
			\begin{array}{l}
				\Rg(A\mal B)\leq\min\set{\Rg(A),\Rg(B)}\\
				\Rg(A\mal B)\geq\Rg(A)+\Rg(B)-n
			\end{array}\right.
			\label{item:satz2.11(d)}
		\end{aligned}$
		\item Seien $S\in M(m\times m),~T\in M(n\times n)$ invertierbar.
		Dann gilt: \label{item:satz2.11(e)}
		\begin{align*}
			\rg(S\mal A\mal T)=\Rg(A)
		\end{align*}
	\end{enumerate}
\end{satz}

\begin{proof}
	Siehe Koecher \cite{koecher2013lineare}, Seite 57 ff.
\end{proof}

\begin{satz}[Dimensionsformel]\label{satz:2.12Dimensionsformel}\enter
	Sei $A\in M(m\times n)$. Setze
	\index{Bild einer Matrix}\index{Kern einer Matrix}\index{Dimensionsformel}
	\begin{align*}
		\Kern(A):=\set{x\in\R^n:A\mal x=0}\qquad
		\Bild(A):=\set{A\mal x:x\in\R^n}
	\end{align*}
	Dann gilt:
	\begin{align*}
		\dim\big(\Kern(A)\big)+\dim\big(\Bild(A)\big)=n
	\end{align*}
\end{satz}

\begin{proof}
	Siehe Koecher \cite{koecher2013lineare}, Seite 39 ff.
\end{proof}

\begin{satz}\label{satz:2.13}
	Sei $A\in M(m\times n)$ mit $m\geq n$.
	Dann gilt: \index{Vollrang}
	\begin{align*}
		A'\mal A\text{ invertierbar}\iff\Rg(A)=n\iff: A\text{ hat \define{Vollrang}}
	\end{align*}
\end{satz}

\begin{proof}
	\betone{Zeige "$\Longrightarrow$":}
	\begin{align*}
		n&=\Rg(I_n)
		=\Rg\klammern{A'\mal A\mal(A'\mal A)^{-1}}
		\overset{\text{\ref{satz:2.11}\ref{item:satz2.11(e)}}}{=}
		\rg(A'\mal A)
		\overset{\text{\ref{satz:2.11}\ref{item:satz2.11(d)}+\ref{satz:2.11}\ref{item:satz2.11(b)}}}{\leq}
		\Rg(A)
		\leq n
	\end{align*}
	\betone{Zeige "$\Longleftarrow$":} Es gilt:
	\begin{align*}
		\Bild(A)=\spann\big(a_1,\ldots,a_n\big)
	\end{align*}
	wobei $a_i:=A\mal e_i$ die $i$-te Spalte von $A$ ist, denn:
	\begin{align}
		A\mal x \nonumber
		&=A\klammern{\sum\limits_{i=1}^n x_i\mal e_i}
		=\sum\limits_{i=1}^n x_i\mal A\mal e_i
		=\sum\limits_{i=1}^n x_i\mal A_i
		\qquad\forall x=\big(x_1,\ldots,x_n\big)'\in\R^n\\
		\overset{\ref{satz:2.12Dimensionsformel}}&{\implies}\nonumber
		\dim\big(\Kern(A)\big)=0\\
		&\implies \Kern(A)=\set{0}\label{eq:ProofSatz2.13Stern}\tag{$*$}\\
		&\implies\Kern(A'\mal A)=\set{0}, \label{eq:ProofSatz2.13SternStern}\tag{$**$}
	\end{align}
	denn:
	\begin{align*}
		A'\mal A=0
		&\implies (x'\mal A')\mal A\mal x=(A\mal x)'\mal A\mal x=\norm{A\mal x}^2
		\implies\norm{A\mal x}=0\\
		&\implies A\mal x=0\implies x\in\Kern(A)
		\overset{\eqref{eq:ProofSatz2.13Stern}}{\implies}x=0
	\end{align*}
	Aus \eqref{eq:ProofSatz2.13SternStern} folgt, dass $A'\mal A$ injektiv ist.
	Gemäß Satz \ref{satz:2.12Dimensionsformel} ist $A'\mal A$ auch surjektiv, also insgesamt bijektiv.
	Also ist $A'\mal A$ invertierbar.
\end{proof}

\begin{satz}\label{satz2.14}
	Sei $A\in M(m\times n),~B\in M(n\times m)$.
	Dann gilt: \index{Spur einer Matrix}
	\begin{align*}
		\Spur(A\mal B)=\Spur(B\mal A)
		\qquad\mit\qquad
		\Spur(A):=\sum\limits_{i=1}^n a_{i,i}
	\end{align*}
	(Die Spur einer Matrix ist die Summe der Diagonalelemente.)
\end{satz}

\begin{proof}
	\begin{align*}
		\Spur(A\mal B)
		\overset{\Def}&{=}
		\sum\limits_{i=1}^m(A\mal B)_{i,i}\\
		&=\sum\limits_{i=1}^m\klammern{\sum\limits_{j=1}^n(A)_{i,j}\mal(B)_{j,i}}\\
		&=\sum\limits_{j=1}^n\sum\limits_{i=1}^m (B)_{j,i}\mal(A)_{i,j}\\
		&=\sum\limits_{j=1}^n(B\mal A)_{j,j}\\
		&=\Spur(B\mal A)
	\end{align*}
\end{proof}

\begin{erinnerung}\
	\begin{enumerate}
		\item $A=\big(a_1,\ldots,a_n\big)\in M(n\times n)$ heißt \define{orthonormal / orthogonal}
		\begin{align*}
			:&\iff\forall i,j\in\set{1,\ldots,n}:\scaProd{a_i}{a_j}=\delta_{i,j}\\
			&\iff A^T=A^{-1}
		\end{align*}
		\item Sei $A\in M(n\times n)$. 
		Dann heißt $\lambda\in\C$ \define{Eigenwert (EW)} von $A$ \index{Eigenwert}\index{Eigenvektor}
		\begin{align*}
			:\iff\exists x\in\R^n\setminus\set{0}:A\mal x=\lambda\mal x
		\end{align*}		 
		In diesem Fall heißt $x$ der zum Eigenwert $\lambda$ gehörende \define{Eigenvektor (EV)} von $A$.
	\end{enumerate}
\end{erinnerung}

\begin{satz}\label{satz:2.15}
	Sei $A\in M(n\times n)$ symmetrisch.
	Dann existiert eine orthogonale / orthonormale Matrix
	$B=\big(v_1,\ldots,v_n\big)$ bestehend aus Eigenvektoren (EV) von $A$ mit
	\begin{align*}
		B'\mal A\mal B=\begin{pmatrix}
			\lambda_1 & 0 & 0\\
			0 & \ddots & 0\\
			0 & 0 & \lambda_n
		\end{pmatrix}=:D=:\Diag\big(\lambda_1,\ldots,\lambda_n\big)
	\end{align*}
	Dabei ist $\lambda_i\in\R$ Eigenwert (EW) von $A$ zum Eigenwert $v_i$ ($i\in\set{1,\ldots,n}$).
\end{satz}

\begin{proof}
	\betone{Teil 1:} Siehe Koecher \cite{koecher2013lineare} Seite 194.\nl
	\betone{Teil 2:} Zunächst gilt $B'\mal B=I_n$, denn
	\begin{align*}
		\scaProd{v_i}{v_j}=v_i'\mal v_j=\delta_{i,j}=\left\lbrace\begin{array}{cl}
			0,&\falls i\neq j\\
			1,&\falls i=j
		\end{array}\right.
	\end{align*}
	Also ist $B'$ Linksinverses von $B$ und damit auch Rechtsinverses.
	\begin{align*}
		A\mal v_i
		&=B\mal D\mal\underbrace{B'\mal v_i}_{=e_i}
		=B\mal D\mal e_i
		=B\mal\lambda_i\lambda e_i
		=\lambda_i\mal B\mal e_i
		=\lambda_i\lambda\mal v_i
	\end{align*}
	Dass die Eigenwerte hierbei reellwertig sind, folgt aus der Symmetrie.
\end{proof}

\begin{satz}\label{satz:2.16}
	Sei $A$ symmetrisch und \define{idempotent}, d.h. $A^2=A$.
	Dann gilt: \index{idempotente Matrix}
	\begin{align*}
		\Rg(A)=\Spur(A)
	\end{align*}
\end{satz}

\begin{proof}
	Für $\lambda_1,\ldots,\lambda_d$ in Satz \ref{satz:2.15} gilt o.B.d.A. (sonst $v_i$'s vertauschen) $\lambda_1\geq\lambda_2\geq\ldots\geq\lambda_n$.\\
	Sei $x$ Eigenvektor von $A$ zum Eigenwert $\lambda$.
	Dann gilt:
	\begin{align*}
		\lambda\mal x=A\mal x=A(A\mal x)=A\mal\lambda\mal x=\lambda\mal A\mal x=\lambda^2\mal	x\\
		\overset{x\neq0}{\implies}
		\lambda=\lambda^2\in\R\implies \lambda\in\set{0,1}
	\end{align*}
	Gemäß Satz \ref{satz:2.15} existiert eine orthogonale Matrix $B$ derart, dass
	\begin{align}\label{eq:ProofSatz2.16Star}\tag{$*$}
		B'\mal A\mal B&=\begin{pmatrix}
			1 & 0 &\hdots & \hdots & \hdots & 0\\
			0 & \ddots & \ddots & \ddots & \ddots & \vdots\\
			\vdots & \ddots & 1 & \ddots & \ddots & \vdots\\
			\vdots & \ddots & \ddots & 0 & \ddots & \vdots\\
			\vdots & \ddots & \ddots & \ddots &  \ddots & \vdots\\
			0 & \hdots & \hdots & \hdots & \hdots & 0
		\end{pmatrix}=:D\\ \nonumber
		\implies\Rg(A)
		\overset{\text{\ref{satz:2.11}\ref{item:satz2.11(e)}}}&{=}
		\rg(B'\mal A\mal B)\\ \nonumber
		\overset{\eqref{eq:ProofSatz2.16Star}}&{=}
		\Rg(D)\\\nonumber
		\overset{\text{\ref{satz:2.11}\ref{item:satz2.11(a)}}}&{=}
		\Spur(D)\\\nonumber
		\overset{\eqref{eq:ProofSatz2.16Star}}&{=}
		\Spur\big(B'\mal(A\mal B)\big)\\\nonumber
		\overset{\ref{satz2.14}}&{=}
		\Spur\big(A\mal\underbrace{(B\mal B')}_{=I_n}\big)\\\nonumber
		&=\Spur(A)
	\end{align}
\end{proof}

\begin{satz}\label{satz2.17}
	Sei $A\in M(m\times n)$. Dann gilt:
	\begin{align*}
		\Kern(A)&=\big(\Bild(A')\big)^\perp\\
		\overset{\text{\ref{satz2.1}\ref{item:satz2.1(3)}\ref{item:satz2.1(3ii)}}}{\iff}
		\Kern(A)^\perp&=\Bild(A')
	\end{align*}
\end{satz}

\begin{proof}
	Sei $x\in\Kern(A)$.
	\begin{align*}
		&x\in\Kern(A)\\
		&\iff A\mal x=0\\
		&\iff\scaProd{b_i}{x}=0 &&\forall i\in\set{1,\ldots,n}\text{wobei $b_i':=i$-te Zeile von $A$}\\
		&\iff\scaProd{y}{x}=0 &&\forall y\in\spann\big(b_1,\ldots,b_m\big)\\
		&\iff x\perp\spann\big(b_1,\ldots,b_m\big)
		\overset{\text{vgl.\ref{satz:2.13}}}{=}		
		\Bild(A)\\
		&\iff x\in\Bild(A')^\perp
	\end{align*}
	denn:
	\begin{align*}
		\scaProd{\sum\limits_{i=1}^m \alpha_i\mal b_i}{x}=\sum\limits_{i=1}^m\underbrace{\scaProd{b_i}{x}}_{=0~\forall i}=0
	\end{align*}
\end{proof}

\begin{satz}\label{satz:2.18}
	Seien $U_1,U_2\subseteq\R^n$ Untervektorräume des $\R^n$.
	Dann gilt
	\begin{enumerate}[label=(\arabic*)]
		\item $\begin{aligned}
			\big(U_1+U_2\big)^\perp=U_1^\perp\cap U_2^\perp
			\label{item:satz2.18(1)}
		\end{aligned}$
		\item $\begin{aligned}
			\big(U_1\cap U_2\big)^\perp=U_1^\perp+ U_2^\perp
			\label{item:satz2.18(2)}
		\end{aligned}$
	\end{enumerate}
\end{satz}

\begin{proof}
	\betone{Zu \ref{item:satz2.18(1)} zeige "$\subseteq$":}
	\begin{align*}
		U_1+U_2\overset{0\in U_1\cap U_2}{\supseteq} U_i\qquad\forall i\in\set{1,2}\\
		\implies \big(U_1+U_2\big)^\perp\subseteq U_i^\perp
		\implies\big(U_1+U_2)^\perp\subseteq U_1^\perp\cap U_2^\perp
	\end{align*}
	\betone{Zu \ref{item:satz2.18(1)} zeige "$\supseteq$":}\\
	Sei $x\in U_1^\perp\cap U_2^\perp$.
	Sei $y\in U_1+U_2$.	
	Dann gilt:
	\begin{align*}
		\exists u_1\in U_1,u_2\in U_2: y=u_1+u_2\\
		\implies\scaProd{x}{y}\overset{\Lin}{=}\underbrace{\scaProd{x}{u_1}}_{
			\overset{x\in U_1^\perp}{=}0		
		}+\underbrace{\scaProd{x}{u_2}}_{
			\overset{x\in U_2^\perp}{=}0
		}
		\implies x\perp U_1+U_2
		\implies x\in\big(U_1+U_2\big)^\perp
	\end{align*}
	\betone{Zeige \ref{item:satz2.18(2)}:}\\
	Wende \ref{item:satz2.18(1)} auf $U_1^\perp$ und $U_2^\perp$ an:
		\begin{align*}
		\big(U_1\cap U_2\big)^\perp
		\overset{\ref{satz2.1}}&{=}
		\Big(\big(U_1^\perp\big)^\perp\cap\big(U_2^\perp\big)^\perp\Big)^\perp
		\overset{\ref{item:satz2.18(1)}}{=}
		\Big(\big(U_1^\perp+U_2^\perp\big)^\perp\Big)^\perp
		\overset{\ref{satz2.1}}{=}
		U_1^\perp+U_2^\perp
	\end{align*}		
\end{proof}

\begin{satz}\label{satz2.19}
	Sei $A\in M(m\times n)$, $U\subseteq\R^n$ Untervektorraum von $\R^n$.
	Dann gilt:
	\begin{align*}
		\big(\Kern(A)\cap U\big)^\perp
		=\Bild\big(P_U A'\big)
	\end{align*}
\end{satz}

\begin{proof}
	\betone{Zeige "$\subseteq$":}
	\begin{align*}
		\big(\Kern(A)\cap U\big)^\perp
		\overset{\ref{satz:2.18}\ref{item:satz2.18(2)}}&{=}
		\big(\Kern(A)^\perp+U^\perp\big)\cap U
		\overset{\ref{satz2.17}}{=}
		\big(\Bild(A')+U^\perp\big)\cap U
	\end{align*}
	Sei nun $x\in\big(\Bild(A')+U^\perp\big)\cap U$.
	Dann: $\exists y\in\R^m,\exists v\in U^\perp$ mit
	\begin{align}\label{eq:Proof2.19Stern}\tag{$*$}
		x=A'\mal y+v
		\overset{\ref{satz2.5}}{=}
		A'\mal y+P_{U^\perp} v
		\overset{\ref{satz2.9}}{=}
		A'\mal y+\big(I_n-P_U\big)\mal v\\ \nonumber
		\implies x
		\overset{\ref{satz2.5},x\in U}{=}
		P_U x
		\overset{\eqref{eq:Proof2.19Stern}}{=}
		P_U A'\mal y+\underbrace{P_U\big(I_n-P_u\big)}_{=P_U-P_U^2\overset{\ref{satz2.7}}{=}0}\mal v
		=\big(P_U A'\big)\mal y\in\Bild\big(P_U A'\big)
	\end{align}		
	
	\betone{Zeige "$\supseteq$":}\\
	Sei $x\in\Bild\big(P_U A'\big)$.
	Dann gilt
	\begin{align*}
		x=P_U A'\mal y\qquad\exists y\in\R^m
	\end{align*}
	und somit $x\in U$.
	Ferner sei $z\in\Kern(A)\cap U$. Folglich:
	\begin{align*}
		\scaProd{x}{z}=x'\mal z
		=y'\mal(A')'\mal P_U' z
		\overset{\ref{satz2.8}}{=}
		y'\mal A\mal P_U z
		\overset{z\in U,\ref{satz2.5}}{=}
		y'\mal (A\mal z)
		\overset{z\in\Kern(A)}{=}
		y'\mal 0
		=0
	\end{align*}
	Also ist $\scaProd{x}{z}=0$ für alle $z\in\Kern(A)\cap U$.
	Somit ist $x\in\big(\Kern(A)\cap U\big)^\perp$.
\end{proof}

\begin{satz}\label{satz2.20}
	Seien $U_1\subseteq U_2\subseteq\R^n$ Untervektorräume.
	Dann gilt:
	\begin{align*}
		P_{U_1}\mal P_{U_2}=P_{U_2}\mal P_{U_1}=P_{U_1}
	\end{align*}
\end{satz}

\begin{proof}
	\begin{align*}
		P_{U_2}\underbrace{\big(P_{U_1} x\big)}_{\in U_1\subseteq U_2}
		\overset{P_{U_1}x\in U_2,\ref{satz2.5}}{=}
		P_{U_1}x\qquad\forall x\in\R^n
	\end{align*}
	Ferner sei $x\in\R^n$. 
	Dann ist $x=u_1+v_1=u_2+v_2$ mit $u_i\in U_i$ und $v_i\in U_i^\perp$, $i\in\set{1,2}$.
	Sowie
	\begin{align}\label{eq:Proof2.20Stern}\tag{$*$}
		U_2^\perp\subseteq U_1^\perp
	\end{align}		
	Damit:
	\begin{align*}
		P_{U_1}P_{U_2}(x)
		&=P_{U_1}\big(P_{U_2}(x)\big)\\
		&=P_{U_1}(u_2)\\
		\overset{\ref{satz2.9}}&{=}
		\big(\id-P_{U_1^\perp}\big)(u_2)\\
		&=u_2-P_{U_1^\perp}(\underbrace{u_2}_{=x-v_2})\\
		\overset{\Lin}&{=}
		u_2-P_{U_1^\perp}(x)+\underbrace{P_{U_1^\perp}(\underbrace{v_2}_{
			\overset{\eqref{eq:Proof2.20Stern}}{=}u_1^\perp		
		})}_{=v_2}\\
		&=x-P_{U_1^\perp}(x)\\
		&=\big(\id-P_{U_1^\perp}\big)(x)\\
		\overset{\ref{satz2.9}}&{=}
		P_{U_1}(x)\\
		\implies P_{U_1}P_{U_2}=P_{U_1}
	\end{align*}
\end{proof}

\begin{satz}\label{satz2.21}
	Sei $U_1\subseteq U_2\subseteq\R^n$ Untervektorräume.
	Dann gilt:
	\begin{align*}
		P_{U_2}=P_{U_1}+P_{U_1^\perp\cap U_2}
	\end{align*}
\end{satz}

\begin{proof}
	Sei zunächst $x\in U_2$.
	Es gilt wegen Satz \ref{satz2.3}:
	\begin{align*}
		x=u+v\qquad\exists u\in U_1,v\in U_1^\perp
	\end{align*}
	Insbesondere:
	\begin{align*}
		u=P_{U_1}(x),\qquad v=P_{U_1^\perp}(x)
	\end{align*}
	Da $x\in U_2$ und $u\in U_2$ (wegen $U_1\subseteq U_2$) folgt $v=x-u\in U_2$.
	Also $v\in U_1^\perp\cap U_2$. Es folgt:
	\begin{align*}
		v\overset{\ref{satz2.5}}{=}
		P_{U_1^\perp\cap U_2}(v)
		=P_{U_1^\perp\cap U_2}\big(P_{U_1^\perp}(x)\big)
		\overset{\ref{satz2.20}}{=}
		P_{U_1^\perp\cap U_2}(x)
	\end{align*}
	Somit:
	\begin{align}\label{eq:Proof2.21Stern}\tag{$*$}
		P_{U_2}(x)
		\overset{x\in U_2,\ref{satz2.5}}{=}
		x
		=u+v
		=P_{U_1}(x)+P_{U_1^\perp\cap U_2}(x)
		\qquad\forall x\in U_2
	\end{align}
	Sei jetzt $x\in\R^n$ beliebig.
	Dann folgt:
	\begin{align*}
		P_{U_2}(x)
		\overset{\ref{satz2.7}}{=}
		P_{U_2}\big(\underbrace{P_{U_2}(x)}_{\in U_2}\big)
		\overset{\eqref{eq:Proof2.21Stern}}{=}
		P_{U_1}\big(P_{U_2}(x)\big)+P_{U_1^\perp\cap U_2}\big(P_{U_2}(x)\big)
		\overset{\ref{satz2.20}}{=}
		P_{U_1}(x)+P_{U_1^\perp\cap U_2}(x)
	\end{align*}
\end{proof}

Der folgende Satz hilft uns, die Existenz des Schätzers sicherzustellen und gibt uns eine Möglichkeit, diesen auszurechnen.

\begin{notation}
	\begin{align*}
		\argmin\limits_{y\in\R^p} f(y):=\set{z\in\R^p:f(z)\leq f(y)~\forall y\in\R^p}
	\end{align*}
\end{notation}

\begin{satz}\label{satz2.22}
	Sei $A\in M(n\times p)$ mit $n\geq p$ und $x\in\R^n$ fest.
	Dann gilt:
	\begin{enumerate}[label=(\alph*)]
		\item Die Minimalstelle \label{item:satz2.22(a)}
		\begin{align*}
			y_0\in\argmin\limits_{y\in\R^p}\norm{x-A\mal y}
			=\argmin\limits\set[\big]{\norm{x-A\mal y}:y\in\R^p}
		\end{align*}
		existiert und erfüllt die \define{Normalgleichung} \index{Normalgleichung}
		\begin{align*}
			A'\mal A\mal y_0=A'\mal x.
		\end{align*}
		Umgekehrt ist jede Lösung der Normalgleichung eine Minimalstelle
		\begin{align*}
			\argmin\limits_{y\in\R^p}\norm{y-A\mal x}.
		\end{align*}
		Ferner gilt:
		\begin{align*}
			A\mal y_0=P_{\Bild(A)}(x)
		\end{align*}
		\item Falls $\Rg(A)=p$ (also Vollrang), so gilt \label{item:satz2.22(b)}
		\begin{align*}
			y_0=\big(A'\mal A\big)^{-1}\mal A'\mal x
			\qquad\und\qquad
			P_{\Bild(A)}=A\mal\big(A'\mal A\big)^{-1}\mal A'
		\end{align*}
	\end{enumerate}
\end{satz}

\begin{proof}
	\betone{Zeige \ref{item:satz2.22(a)}:}\\
	$U:=\Bild(A)$ ist Untervektorraum des $\R^n$.
	Nach Satz \ref{satz2.10}\ref{item:satz2.10(1)} ist $u_0:=P_U(x)$ eine eindeutige Minimalstelle von
	\begin{align*}
		\set[\big]{\norm{x-u}:u\in U}=\set[\big]{\norm{x-A\mal y}:y \in\R^p}
	\end{align*}
	Da $u_0\in\Bild(A)$, ist
	\begin{align*}
		M:=A^{-1}\big(\set*{u_0}\big)=\set{y\in\R^p:A\mal y=u_0}
		\neq\emptyset
	\end{align*}
	und für alle $y_0\in M$ gilt:
	\begin{align*}
		\norm{x-A\mal y_0}=\norm{x-u_0}
		\leq\norm{x-A\mal y}\qquad\forall y\in\R^p
	\end{align*}
	d.h. $y_0$ ist Minimalstelle und wegen $y_0\in M$ gilt:
	\begin{align}\label{eq:Proof2.22Stern}\tag{$*$}
		A\mal y_0&=u_0=P_{\Bild(A)}(x)
	\end{align}
	Mit Satz \ref{satz2.10}\ref{item:satz2.10(2)} folgt:
	\begin{align*}
		&\hspace{10mm}x-A\mal y_0\perp A\mal y\qquad\forall y\in\R^p\\
		&\iff &0&=\scaProd{x-A\mal y_0}{A\mal y}\\
		&&\overset{\text{Sym}}&{=}
		\scaProd{A\mal y}{x-A\mal y_0}\\
		&&&=(A\mal y)'\mal(x-A\mal y_0)\\
		&&&=y'\mal A'\mal(x-A\mal y_0)\\
		&&&=y'\mal(A'\mal x-A'\mal A\mal y_0)\\
		&&&=\scaProd{y}{A'\mal x-A'\mal A\mal y_0}\qquad\forall y\in\R^p\\
		&\iff A'\mal x-A'\mal A\mal y_0\perp\R^d\\
		&\iff A'\mal x-A'\mal A\mal y_0=0\\
		&\iff A'\mal A\mal y_0=A'\mal x
	\end{align*}
	Hierbei geht ein:
	\begin{align*}
		z\perp\R^p\iff\scaProd{z}{y}=0~\forall y\in\R^p\\
		\overset{y=z}{\implies}\scaProd{z}{z}=0
		\implies z=0\\
		z=0\implies\scaProd{z}{y}=0~\forall y\in\R^p
	\end{align*}
	Umgekehrt ist jede Lösung $y_0$ Der Normalgleichung Minimalstellen von 
	\begin{align*}
		y\mapsto\norm{x-A\mal y},
	\end{align*}
	denn:
	\begin{align*}
		A'\mal A\mal y_0=A'\mal x
		\overset{\text{s.o.}}&{\iff}
		x-\underbrace{A\mal y_0}_{\in\Bild(A)}\perp\Bild(A)\\
		\overset{\ref{satz2.10}\ref{item:satz2.10(3)}}&{\implies}
		A\mal y_0=P_{\Bild(A)}(x)=u_0\\
		&\implies y_0\text{ ist Minimalstelle}
	\end{align*}
	
	\betone{Zeige \ref{item:satz2.22(b)}:}\\
	Gelte also $\Rg(A)=p$.
	Dann folgt aus Satz \ref{satz:2.13}, dass $A'\mal A$ invertierbar ist.
	Aus \ref{item:satz2.22(a)} folgt dann
	\begin{align*}
		y_0&=\big(A'\mal A\big)^{-1}\mal A'\mal x\\
		\overset{\eqref{eq:Proof2.22Stern}}{\implies}
		P_{\Bild(A)}(x)=A\mal(A'\mal A)^{-1}\mal A'\mal x\qquad\forall x\in\R^n
	\end{align*}
\end{proof}

\begin{definition}\label{def2.23}
	$A\in M(n\times n)$ symmerisch heißt
	\begin{enumerate}
		\item \define{positiv definit}
		\index{positiv definit}
		\begin{align*}
			\defiff\forall x\in\R^n\setminus\set{0}:x'\mal A\mal x>0
		\end{align*}
		\item \define{nichtnegativ definit}
		\index{nichtnegativ definit}
		\begin{align*}
			\defiff\forall x\in\R^n\setminus\set{0}:x'\mal A\mal x\geq0
		\end{align*}
	\end{enumerate}			
\end{definition}

\begin{satz}\label{satz2.24}
	Sei $A\in M(n\times n)$ positiv definit.
	Dann gilt:
	\begin{enumerate}[label=(\alph*)]
		\item Alle Eigenwerte von $A$ sind positiv.
		\label{item:satz2.24(a)}
		\item Es existiert $R\in M(n\times n)$ positiv definit mit $A=R^2$.
		Bezeichnung: $A^{\frac{1}{2}}:=R$
		\label{item:satz2.24(b)}
		\item $A$ ist invertierbar und $A^{-1}$ ist positiv definit.
		\label{item:satz2.24(c)}
	\end{enumerate}
\end{satz}

\begin{proof}
	\betone{Zeige \ref{item:satz2.24(a)}:}\\
	Sei $\lambda$ Eigenwert zum Eigenvektor $x\neq0$.
	Dann gilt $A\mal x=\lambda\mal x$ nach Definition des Eigenwertes.
	\begin{align*}
		0
		\overset{A\text{ p.d.}}&{<}
		x'\mal A\mal x
		=x'\mal\lambda\mal x
		=\lambda\mal\norm{x}^2
		\overset{\norm{x}>0}{\implies}
		\lambda>0
	\end{align*}
	
	\betone{Zeige \ref{item:satz2.24(b)}:}\\
	Gemäß Satz \ref{satz:2.15} existiert eine orthogonale Matrix $B\in M(n\times n)$ mit
	\begin{align*}
		A=B\mal D\mal B',\qquad D:=\diag\big(\lambda_1,\ldots,\lambda_n\big)
	\end{align*}
	wobei $\lambda_1,\ldots,\lambda_n$ Eigenwerte von $A$ sind, die wegen \ref{item:satz2.24(a)} alle positiv sind.
	Setze
	\begin{align*}
		R:=B\mal\sqrt{D}\mal B',\qquad\sqrt{D}:=\diag\klammern{\sqrt{\lambda_1},\ldots,\sqrt{\lambda_n}}\\
		\implies
		R^2=B\mal \sqrt{D}\mal\underbrace{B'\mal B}_{=I_n}\mal \sqrt{D}\mal B'
		=B\mal\sqrt{D}\mal\sqrt{D}\mal B'
		=B\mal D\mal B'
		=A
	\end{align*}
	Ferner ist $R$ symmetrisch, denn 
	\begin{align*}
		R'\overset{\Def}{=}\big(B\mal\sqrt{D}\mal B'\big)'=B''\mal\sqrt{D}'\mal B'
		\overset{D~\diag}{=}B\mal\sqrt{D}\mal B'
	\end{align*}
	Weiterhin gilt für alle $x\neq0$:
	\begin{align*}
		x'\mal R\mal x
		\overset{\Def}&{=}
		x'\mal B\mal\sqrt{D}\mal\underbrace{B'\mal x}_{=:y}
		=y'\mal\sqrt{D}\mal y=\sum\limits_{i=1}^n\underbrace{\sqrt{\lambda_i}}_{>0}\mal y_i^2
		>0
	\end{align*}
	und $y\neq 0$, da $x\neq0$ und $\Kern(B')=\set{0}$ (weil $B'$ invertierbar).\nl
	\betone{Zeige \ref{item:satz2.24(c)}:}\\
	$A=R\mal R$ und $\R$ ist invertierbar, da $B$, $\sqrt{D}$ und $B'$ invertierbar sind.
	Somit ist auch $A$ invertierbar.
	\begin{align*}
		A^{-1}=R^{-1}\mal R^{-1}=S\mal S
		\qquad\mit\qquad S:=R^{-1}\text{ invertierbar und symmetrisch}\\
		\implies x'\mal A^{-1}\mal x=x'\mal S'\mal S\mal x=\scaProd{S\mal x}{S\mal x}=\norm{S\mal x}^2>0,
	\end{align*}
	da $x\neq0$ und $\Kern(S)=\set{0}$.\\
	($S$ ist symmetrisch wegen $S'=(R^{-1})'=(R')^{-1}=R^{-1}=S$.)
\end{proof}

\begin{bemerkungnr}\label{bem2.25} %ohne Nummer
	Falls $A$ in Satz \ref{satz2.24} nichtnegativ definit ist (statt positiv definit), so gelten \ref{item:satz2.24(a)} und \ref{item:satz2.24(b)} analog mit "nichtnegativ" anstelle von "positiv".
	Dies zeigt der Beweis von \ref{item:satz2.24(a)} und \ref{item:satz2.24(b)}.
\end{bemerkungnr}

	% This work is licensed under the Creative Commons
% Attribution-NonCommercial-ShareAlike 4.0 International License. To view a copy
% of this license, visit http://creativecommons.org/licenses/by-nc-sa/4.0/ or
% send a letter to Creative Commons, PO Box 1866, Mountain View, CA 94042, USA.

\section{Parameterschätzung in linearen Modellen}

\begin{definition}\label{def3.1}\
	\begin{enumerate}[label=(\arabic*)]
		\item Sei $Z:=\big(Z_{i,j}\big)_{\begin{subarray}{c}
			1\leq i\leq m\\
			1\leq j\leq n
		\end{subarray}}\in M(m\times n)$
		mit integrierbaren Komponenten $Z_{i,j}$, welche reelle Zufallsvariablen sind.
		Dann: \label{item:def3.1(1)}
		\begin{align*}
			\E[Z]:=\Big(\E\big[Z_{i,j}\big]\Big)_{\begin{subarray}{c}
			1\leq i\leq m\\
			1\leq j\leq n
		\end{subarray}}
		\end{align*}
		Speziell für 
		\begin{align*}
			Z=\big(Z_1,\ldots,Z_m\big)'
		\end{align*}
		ist 
		\begin{align*}
			\E[Z]=\begin{pmatrix}
				\E\big[Z_1\big]\\
				\vdots\\
				\E\big[Z_n\big]
			\end{pmatrix}=\Big(\E\big[Z_1\big],\ldots,\E\big[Z_n\big]\Big)'
		\end{align*}
		\item Seien
		\begin{align*}
			Z=\big(Z_1,\ldots,Z_m\big)',\qquad
			Y=\big(Y_1,\ldots,Y_n\big)'.
		\end{align*}		 
		Dann, falls existent
		\begin{align*}
			\Cov(Y,Z)&:=\E\Big(\big(Y-\E[Y]\big)\mal\big(Z-\E[Z]\big)'\Big)\\
			\overset{\ref{item:def3.1(1)}}&{~=}
			\klammern[\bigg]{\E\Big[\big(Y_i-\E[Y_i]\big)\mal\big(Z_j-\E[Z_j]\big)\Big]}_{\begin{subarray}{c}
			1\leq i\leq m\\
			1\leq j\leq n
		\end{subarray}}\\
		&~=\Big(\Cov(Y_i,Z_j)\Big)_{\begin{subarray}{c}
			1\leq i\leq m\\
			1\leq j\leq n
		\end{subarray}}
		\end{align*}
		Speziell ist\index{Kovarianzmatrix}
		\begin{align*}
			\Var(Y):=\Cov(Y,Y)=\Big(\Cov(Y_i,Y_j)\Big)_{\begin{subarray}{c}
			1\leq i\leq m\\
			1\leq j\leq n
		\end{subarray}}\in M(n\times n)
		\end{align*}
		die \define{Kovarianzmatrix} von $Y$.
	\end{enumerate}
\end{definition}

\begin{satz}\label{satz3.2}
	Seien $Y$ und $Z$ Zufallsvektoren in $\R^n$ bzw. $\R^m$, $A\in M(m\times n)$, $b\in\R^m$ beide deterministisch.
	Dann gilt:
	\begin{enumerate}[label=(\arabic*)]
		\item $\begin{aligned}
			\E\big[A\mal Y+Z\big]=A\mal\E[Y]+\E[Z]
		\end{aligned}\qquad$ ($\E$ ist linear)
		\label{item:satz3.2(1)}
		\item $\begin{aligned}
			\Var\big(A\mal Y+b\big)=A\mal\Var(Y)\mal A'
		\end{aligned}$
		\label{item:satz3.2(2)}
	\end{enumerate}
\end{satz}

\begin{proof}
	Nachrechnen! (Zur Übung, folgt aus Matrixmultiplikation + den Eigenschaften im reellen Fall)
	%\betone{Zeige \ref{item:satz3.2(1)}:}\\
	
	%\betone{Zeige \ref{item:satz3.2(2)}:}\\
\end{proof}

Wir betrachten jetzt das lineare Modell (vergleiche \eqref{eq:1.2}) mit 
\begin{align}\label{eq:3.1}\tag{3.1}
	Y&=X\mal\beta+\varepsilon\\
	\E[\varepsilon]&=0\label{eq:3.2}\tag{3.2}\\
	\Var(\varepsilon)&=\sigma^2\mal I_n\mit\sigma^2>0\label{eq:3.3}\tag{3.3}\\
	n&\geq p\label{eq:3.4}\tag{3.4}
\end{align}
Beachte ($\varepsilon=\big(\varepsilon_1,\ldots,\varepsilon_n\big)'$):
\begin{align*}
	\eqref{eq:3.3}&\iff\Cov(\varepsilon_i,\varepsilon_j)=\sigma^2\mal\delta_{i,j}
	&&\forall i,j\\
	&\iff\Cov\big(\varepsilon_i,\varepsilon_j\big)=0
	&&\forall i\neq j\und\Var(\varepsilon_i)=\sigma^2
\end{align*}
Also gilt: $\varepsilon_1,\ldots,\varepsilon_n$ iid $\implies\eqref{eq:3.3}$\\
($p$ ist die Anzahl der Modellparameter)
\begin{align*}
	\eqref{eq:3.1}\iff Y_i=\sum\limits_{j=1}^p X_{i,j}\mal\beta_j+\varepsilon_i\quad\forall i
\end{align*}

\subsection{Minimum-Quadrat-Schätzer}

Ziel: Schätzung von $\beta$.
Idee nach Gauß und Legendre: 
\define{Methode der kleinsten Quadrate}, nämlich:
Minimierung der \define{Fehlerquadratsumme}
\index{Fehlerquadratsumme}\index{Methode der kleinsten Quadrate}
\begin{align*}
	\sum\limits_{i=1}^n\varepsilon_i^2
	\overset{\Def}&{=}
	\norm{\varepsilon}^2
	\overset{\eqref{eq:3.1}}{=}
	\norm{Y-X\mal\beta}^2,
\end{align*}
d.h. finde $\hat{\beta}$ mit
\begin{align*}
	&\hspace{11.5mm}\norm{Y-X\mal\hat{\beta}}^2\leq\norm{Y-X\mal\beta}^2&&\forall\beta\in\R^p\\
	&\iff
	\norm{Y-X\mal\hat{\beta}}\leq\norm{Y-X\mal\beta}&&\forall\beta\in\R^p\\
	&\iff\hat{\beta}(\omega)\in\argmin\limits_{\beta\in\R^p}\norm{Y(\omega)-X\mal\beta}
\end{align*}
Die Lösung ist bekannt.
Es gilt

\begin{satz}\label{satz3.3}
	Es gelten \eqref{eq:3.1} und \eqref{eq:3.4}.
	Dann gilt:
	\begin{enumerate}[label=(\arabic*)]
		\item Die Minimalstelle $\hat{\beta}$ existiert und erfüllt die \define{Normalgleichung} \label{item:satz3.3(1)}
		\index{Normalgleichung}
		\begin{align}\label{eq:satz3.3Stern}\tag{$*$}
			X'\mal X\mal\hat{\beta}=X'\mal Y
		\end{align}
		und umgekehrt ist jede Lösung von \eqref{eq:satz3.3Stern} eine Minimalstelle.\\
		Bezeichnet $L:=\Bild(X)$, so gilt:
		\begin{align*}
			P_L(Y)=X\mal\hat{\beta}
		\end{align*}
		\item Falls $\Rg(X)=p$ (also Vollrang), so gilt:\label{item:satz3.3(2)}
		\begin{align*}
			\hat{\beta}&=\big(X'\mal X)^{-1}\mal X'\mal Y
			\qquad\und\qquad
			P_L=X\mal\big(X'\mal X\big)^{-1}\mal X'
		\end{align*}
		Der Schätzer $\hat{\beta}$ heißt \define{Minimum-Quadrat-Schätzer (MQS) /\\ Kleinst-Quadrate-Schätzer (KQS) / least squares estimator (lse)}.
		\index{Minimum-Qudrat-Schätzer}
		\index{Kleinst-Quadrate-Schätzer}
		\index{least squares estimator|see{Minimum-Quadrat-Schätzer}}
	\end{enumerate}
\end{satz}

\begin{proof}
	Nutze Satz \ref{satz2.22} mit $A\leftrightarrow X$, $x\leftrightarrow Y(\omega)$ und $y\leftrightarrow\beta$.
\end{proof}

\begin{figure}[H]
	\begin{center}
		% This work is licensed under the Creative Commons
% Attribution-NonCommercial-ShareAlike 4.0 International License. To view a copy
% of this license, visit http://creativecommons.org/licenses/by-nc-sa/4.0/ or
% send a letter to Creative Commons, PO Box 1866, Mountain View, CA 94042, USA.



\tikzset{every picture/.style={line width=0.75pt}} %set default line width to 0.75pt        

\begin{tikzpicture}[x=0.75pt,y=0.75pt,yscale=-1,xscale=1]
%uncomment if require: \path (0,300); %set diagram left start at 0, and has height of 300

%Shape: Axis 2D [id:dp7098119764904821] 
\draw  (50,256.93) -- (558,256.93)(100.8,4.93) -- (100.8,284.93) (551,251.93) -- (558,256.93) -- (551,261.93) (95.8,11.93) -- (100.8,4.93) -- (105.8,11.93)  ;
%Straight Lines [id:da9071558730776327] 
\draw [line width=3]    (61,275.93) -- (518,41.93) ;


%Straight Lines [id:da7398348018165936] 
\draw    (210,68.93) ;


%Straight Lines [id:da721943207149605] 
\draw    (229.5,65.93) -- (286.5,160.43) ;


%Shape: Circle [id:dp22369371272099947] 
\draw   (277.13,159.1) .. controls (277.87,153.92) and (282.66,150.33) .. (287.84,151.07) .. controls (293.01,151.81) and (296.61,156.6) .. (295.87,161.77) .. controls (295.13,166.94) and (290.34,170.54) .. (285.16,169.8) .. controls (279.99,169.06) and (276.39,164.27) .. (277.13,159.1) -- cycle ;
%Shape: Circle [id:dp5421173957527967] 
\draw   (349.75,120.88) .. controls (350.49,115.71) and (355.28,112.12) .. (360.45,112.86) .. controls (365.63,113.59) and (369.22,118.39) .. (368.48,123.56) .. controls (367.74,128.73) and (362.95,132.33) .. (357.78,131.59) .. controls (352.6,130.85) and (349.01,126.06) .. (349.75,120.88) -- cycle ;
%Shape: Circle [id:dp28472585647194326] 
\draw   (220.13,64.6) .. controls (220.87,59.42) and (225.66,55.83) .. (230.84,56.57) .. controls (236.01,57.31) and (239.61,62.1) .. (238.87,67.27) .. controls (238.13,72.44) and (233.34,76.04) .. (228.16,75.3) .. controls (222.99,74.56) and (219.39,69.77) .. (220.13,64.6) -- cycle ;

% Text Node
\draw (531,43.93) node  [align=left] {$L$};
% Text Node
\draw (223,44.93) node  [align=left] {$Y$};
% Text Node
\draw (294,183.93) node  [align=left] {$X\mal\hat{\beta}$};
% Text Node
\draw (383,138.93) node  [align=left] {$X\mal\beta$};


\end{tikzpicture}

		\caption{Minimum-Quadrat-Schätzer}
		\label{Abb:MinQuaSchätzer}
	\end{center}
\end{figure}

\begin{satz}\label{satz3.4}
	Angenommen es gelten \eqref{eq:3.1}, \eqref{eq:3.2}, \eqref{eq:3.3} und \eqref{eq:3.4}.
	Falls $\Rg(X)=p$, so gilt:
	\begin{align}\label{eq:3.5}\tag{3.5}
		\E\big[\hat{\beta}\big]&=\beta\qquad\forall\beta\in\R^p\\
		\Var\big(\hat{\beta}\big)&=\sigma^2\mal\big(X'\mal X\big)^{-1}\label{eq:3.6}\tag{3.6}
	\end{align}
\end{satz}

\begin{proof}
	\begin{align*}
		\E\big[\hat{\beta}\big]
		\overset{\ref{satz3.3}\ref{item:satz3.3(2)}}&{=}
		\E\Big[\underbrace{\big((X'\mal X)^{-1}\mal X'\big)}_{=:A}\mal Y\Big]\\
		\overset{\ref{satz3.2}\ref{item:satz3.2(1)}}&{=}
		\big(X'\mal X\big)^{-1}\mal X'\mal\E[Y]\\
		\overset{\eqref{eq:3.1}}&{=}
		\big(X'\mal X\big)^{-1}\mal X'\mal\underbrace{\E[X\mal\beta+\varepsilon]}_{
			=\underbrace{\E[X\mal\beta]}_{
				=X\mal\beta
			}+\underbrace{\E[\varepsilon]}_{
				\overset{\eqref{eq:3.2}}{=}0		
			}=X\mal\beta
		}\\
		&=\big(X'\mal X\big)^{-1}\mal\big(X'\mal X\big)\mal\beta\\
		&=I_p\beta\\
		&=\beta
	\end{align*}
	Zur Varianz:
	\begin{align*}
		\Var\big(\hat{\beta}\big)
		\overset{\ref{satz3.3}\ref{item:satz3.3(2)}}&{=}
		\Var\Big(\big(X'\mal X\big)^{-1}\mal X'\mal Y)\\
		\overset{\ref{satz3.2}\ref{item:satz3.2(2)}}&{=}
		\big(X'\mal X\big)^{-1}\mal X'\mal\Var(Y)\mal\Big(\big(X'\mal X\big)^{-1}\mal X'\Big)'\\
		&=\big(X'\mal X\big)^{-1}\mal X'\mal\underbrace{\Var(Y)}_{
			\overset{\eqref{eq:3.1}}{=}\Var(X\mal\beta+\varepsilon)
			\overset{\ref{satz3.2}\ref{item:satz3.2(2)}}{=}
			\Var(\varepsilon)
			\overset{\eqref{eq:3.3}}{=}
			\sigma^2\mal I_n
		}\mal X\mal\big(X'\mal X\big)^{-1}\\
		&=\sigma^2\mal\big(X'\mal X\big)^{-1}\mal\underbrace{\big(X'\mal X\big)\mal\big(X'\mal X\big)^{-1}}_{=I_p}\\
		&=\sigma^2\mal\big(X'\mal X\big)^{-1}
	\end{align*}
\end{proof}

Beachte, der Designmatrix $X$ kommt durch die \define{Varianzformel} \eqref{eq:3.6} eine statistische Bedeutung zu ($\leadsto$ design of experiments).
\index{Varianzformel}

\begin{beispiel}[Einfache lineare Regression]
\label{beisp3.5einfacheLineareRegression}
	\begin{align*}
		Y_i&=a+b\mal x_i+\varepsilon_i,\qquad\forall i\in\set{1,\ldots,n}
	\end{align*}

	\begin{figure}[H]
		\begin{center}
			% This work is licensed under the Creative Commons
% Attribution-NonCommercial-ShareAlike 4.0 International License. To view a copy
% of this license, visit http://creativecommons.org/licenses/by-nc-sa/4.0/ or
% send a letter to Creative Commons, PO Box 1866, Mountain View, CA 94042, USA.





\tikzset{every picture/.style={line width=0.75pt}} %set default line width to 0.75pt        

\begin{tikzpicture}[x=0.75pt,y=0.75pt,yscale=-1,xscale=1]
%uncomment if require: \path (0,300); %set diagram left start at 0, and has height of 300

%Shape: Axis 2D [id:dp07321911068108622] 
\draw  (4,261.94) -- (518,261.94)(55.4,10) -- (55.4,289.93) (511,256.94) -- (518,261.94) -- (511,266.94) (50.4,17) -- (55.4,10) -- (60.4,17)  ;
%Straight Lines [id:da07148662198339073] 
\draw [color={rgb, 255:red, 245; green, 166; blue, 35 }  ,draw opacity=1 ][line width=3]    (2,268.93) -- (483,30.93) ;


%Shape: Circle [id:dp26668633556271415] 
\draw   (92,241) .. controls (92,237.13) and (95.13,234) .. (99,234) .. controls (102.87,234) and (106,237.13) .. (106,241) .. controls (106,244.87) and (102.87,248) .. (99,248) .. controls (95.13,248) and (92,244.87) .. (92,241) -- cycle ;
%Shape: Circle [id:dp3403607469057863] 
\draw   (98,186) .. controls (98,182.13) and (101.13,179) .. (105,179) .. controls (108.87,179) and (112,182.13) .. (112,186) .. controls (112,189.87) and (108.87,193) .. (105,193) .. controls (101.13,193) and (98,189.87) .. (98,186) -- cycle ;
%Shape: Circle [id:dp4758519789604272] 
\draw   (194,197) .. controls (194,193.13) and (197.13,190) .. (201,190) .. controls (204.87,190) and (208,193.13) .. (208,197) .. controls (208,200.87) and (204.87,204) .. (201,204) .. controls (197.13,204) and (194,200.87) .. (194,197) -- cycle ;
%Shape: Circle [id:dp35126366068887793] 
\draw   (446,38) .. controls (446,34.13) and (449.13,31) .. (453,31) .. controls (456.87,31) and (460,34.13) .. (460,38) .. controls (460,41.87) and (456.87,45) .. (453,45) .. controls (449.13,45) and (446,41.87) .. (446,38) -- cycle ;
%Shape: Circle [id:dp6450412209324151] 
\draw   (402,53) .. controls (402,49.13) and (405.13,46) .. (409,46) .. controls (412.87,46) and (416,49.13) .. (416,53) .. controls (416,56.87) and (412.87,60) .. (409,60) .. controls (405.13,60) and (402,56.87) .. (402,53) -- cycle ;
%Shape: Circle [id:dp4264033728988835] 
\draw   (388,98) .. controls (388,94.13) and (391.13,91) .. (395,91) .. controls (398.87,91) and (402,94.13) .. (402,98) .. controls (402,101.87) and (398.87,105) .. (395,105) .. controls (391.13,105) and (388,101.87) .. (388,98) -- cycle ;
%Shape: Circle [id:dp41793329654483513] 
\draw   (292,119) .. controls (292,115.13) and (295.13,112) .. (299,112) .. controls (302.87,112) and (306,115.13) .. (306,119) .. controls (306,122.87) and (302.87,126) .. (299,126) .. controls (295.13,126) and (292,122.87) .. (292,119) -- cycle ;
%Shape: Circle [id:dp4914935297369122] 
\draw   (136,190) .. controls (136,186.13) and (139.13,183) .. (143,183) .. controls (146.87,183) and (150,186.13) .. (150,190) .. controls (150,193.87) and (146.87,197) .. (143,197) .. controls (139.13,197) and (136,193.87) .. (136,190) -- cycle ;
%Shape: Circle [id:dp23764538796154067] 
\draw   (173,160) .. controls (173,156.13) and (176.13,153) .. (180,153) .. controls (183.87,153) and (187,156.13) .. (187,160) .. controls (187,163.87) and (183.87,167) .. (180,167) .. controls (176.13,167) and (173,163.87) .. (173,160) -- cycle ;
%Shape: Circle [id:dp6151497769773182] 
\draw   (249,156) .. controls (249,152.13) and (252.13,149) .. (256,149) .. controls (259.87,149) and (263,152.13) .. (263,156) .. controls (263,159.87) and (259.87,163) .. (256,163) .. controls (252.13,163) and (249,159.87) .. (249,156) -- cycle ;
%Shape: Circle [id:dp6112008925466065] 
\draw   (148,208) .. controls (148,204.13) and (151.13,201) .. (155,201) .. controls (158.87,201) and (162,204.13) .. (162,208) .. controls (162,211.87) and (158.87,215) .. (155,215) .. controls (151.13,215) and (148,211.87) .. (148,208) -- cycle ;
%Shape: Circle [id:dp7443465654150703] 
\draw   (340,111) .. controls (340,107.13) and (343.13,104) .. (347,104) .. controls (350.87,104) and (354,107.13) .. (354,111) .. controls (354,114.87) and (350.87,118) .. (347,118) .. controls (343.13,118) and (340,114.87) .. (340,111) -- cycle ;
%Straight Lines [id:da5984247302071123] 
\draw    (179,251.93) -- (179,270.93) ;


%Straight Lines [id:da6676175412533971] 
\draw    (59,158.93) -- (76,158.93) ;


%Straight Lines [id:da2593592347170056] 
\draw    (189,75.93) -- (180.24,147.02) ;
\draw [shift={(180,149)}, rotate = 277.02] [color={rgb, 255:red, 0; green, 0; blue, 0 }  ][line width=0.75]    (10.93,-3.29) .. controls (6.95,-1.4) and (3.31,-0.3) .. (0,0) .. controls (3.31,0.3) and (6.95,1.4) .. (10.93,3.29)   ;

%Shape: Circle [id:dp9148664206548816] 
\draw   (82,169) .. controls (82,165.13) and (85.13,162) .. (89,162) .. controls (92.87,162) and (96,165.13) .. (96,169) .. controls (96,172.87) and (92.87,176) .. (89,176) .. controls (85.13,176) and (82,172.87) .. (82,169) -- cycle ;

% Text Node
\draw (470,70) node   {$y( x) =a+bx$};
% Text Node
\draw (287,107) node [rotate=-333.51] [align=left] {\textcolor[rgb]{0.96,0.65,0.14}{unbekannte Regressionsgerade}};
% Text Node
\draw (512,271) node   {$x$};
% Text Node
\draw (47,11) node   {$y$};
% Text Node
\draw (178,282) node   {$x_{i}$};
% Text Node
\draw (46,156) node   {$Y_{i}$};
% Text Node
\draw (191,58) node   {$( x_{i} ,Y_{i})$};


\end{tikzpicture}


			\caption{Einfache lineare Regression}
			%\label{Abb:MinQuaSchätzer}
		\end{center}
	\end{figure}
	
	\begin{figure}[H]
		\begin{center}
			% This work is licensed under the Creative Commons
% Attribution-NonCommercial-ShareAlike 4.0 International License. To view a copy
% of this license, visit http://creativecommons.org/licenses/by-nc-sa/4.0/ or
% send a letter to Creative Commons, PO Box 1866, Mountain View, CA 94042, USA.




% Pattern Info
 
\tikzset{
pattern size/.store in=\mcSize, 
pattern size = 5pt,
pattern thickness/.store in=\mcThickness, 
pattern thickness = 0.3pt,
pattern radius/.store in=\mcRadius, 
pattern radius = 1pt}
\makeatletter
\pgfutil@ifundefined{pgf@pattern@name@_yznbr6opy}{
\pgfdeclarepatternformonly[\mcThickness,\mcSize]{_yznbr6opy}
{\pgfqpoint{0pt}{0pt}}
{\pgfpoint{\mcSize+\mcThickness}{\mcSize+\mcThickness}}
{\pgfpoint{\mcSize}{\mcSize}}
{
\pgfsetcolor{\tikz@pattern@color}
\pgfsetlinewidth{\mcThickness}
\pgfpathmoveto{\pgfqpoint{0pt}{0pt}}
\pgfpathlineto{\pgfpoint{\mcSize+\mcThickness}{\mcSize+\mcThickness}}
\pgfusepath{stroke}
}}
\makeatother
\tikzset{every picture/.style={line width=0.75pt}} %set default line width to 0.75pt        

\begin{tikzpicture}[x=0.75pt,y=0.75pt,yscale=-1,xscale=1]
%uncomment if require: \path (0,300); %set diagram left start at 0, and has height of 300

%Shape: Axis 2D [id:dp17046936193864992] 
\draw  (50,264.53) -- (573,264.53)(102.3,8.93) -- (102.3,292.93) (566,259.53) -- (573,264.53) -- (566,269.53) (97.3,15.93) -- (102.3,8.93) -- (107.3,15.93)  ;
%Straight Lines [id:da10502832048562816] 
\draw [color={rgb, 255:red, 11; green, 17; blue, 252 }  ,draw opacity=1 ][line width=3]    (40,195.93) -- (529,43.93) ;


%Shape: Circle [id:dp3936057208076633] 
\draw  [fill={rgb, 255:red, 0; green, 0; blue, 0 }  ,fill opacity=1 ] (124,160.5) .. controls (124,158.01) and (126.01,156) .. (128.5,156) .. controls (130.99,156) and (133,158.01) .. (133,160.5) .. controls (133,162.99) and (130.99,165) .. (128.5,165) .. controls (126.01,165) and (124,162.99) .. (124,160.5) -- cycle ;
%Shape: Circle [id:dp7335662721698364] 
\draw  [fill={rgb, 255:red, 0; green, 0; blue, 0 }  ,fill opacity=1 ] (166,141.5) .. controls (166,139.01) and (168.01,137) .. (170.5,137) .. controls (172.99,137) and (175,139.01) .. (175,141.5) .. controls (175,143.99) and (172.99,146) .. (170.5,146) .. controls (168.01,146) and (166,143.99) .. (166,141.5) -- cycle ;
%Shape: Circle [id:dp10562516602478167] 
\draw  [fill={rgb, 255:red, 0; green, 0; blue, 0 }  ,fill opacity=1 ] (269,91.5) .. controls (269,89.01) and (271.01,87) .. (273.5,87) .. controls (275.99,87) and (278,89.01) .. (278,91.5) .. controls (278,93.99) and (275.99,96) .. (273.5,96) .. controls (271.01,96) and (269,93.99) .. (269,91.5) -- cycle ;
%Shape: Circle [id:dp6248807101111838] 
\draw  [fill={rgb, 255:red, 0; green, 0; blue, 0 }  ,fill opacity=1 ] (323,45.5) .. controls (323,43.01) and (325.01,41) .. (327.5,41) .. controls (329.99,41) and (332,43.01) .. (332,45.5) .. controls (332,47.99) and (329.99,50) .. (327.5,50) .. controls (325.01,50) and (323,47.99) .. (323,45.5) -- cycle ;
%Shape: Circle [id:dp719119066288937] 
\draw  [fill={rgb, 255:red, 0; green, 0; blue, 0 }  ,fill opacity=1 ] (144,180.5) .. controls (144,178.01) and (146.01,176) .. (148.5,176) .. controls (150.99,176) and (153,178.01) .. (153,180.5) .. controls (153,182.99) and (150.99,185) .. (148.5,185) .. controls (146.01,185) and (144,182.99) .. (144,180.5) -- cycle ;
%Shape: Circle [id:dp7264172592917753] 
\draw  [fill={rgb, 255:red, 0; green, 0; blue, 0 }  ,fill opacity=1 ] (364,28.5) .. controls (364,26.01) and (366.01,24) .. (368.5,24) .. controls (370.99,24) and (373,26.01) .. (373,28.5) .. controls (373,30.99) and (370.99,33) .. (368.5,33) .. controls (366.01,33) and (364,30.99) .. (364,28.5) -- cycle ;
%Shape: Circle [id:dp33818115228089896] 
\draw  [fill={rgb, 255:red, 0; green, 0; blue, 0 }  ,fill opacity=1 ] (51,210.5) .. controls (51,208.01) and (53.01,206) .. (55.5,206) .. controls (57.99,206) and (60,208.01) .. (60,210.5) .. controls (60,212.99) and (57.99,215) .. (55.5,215) .. controls (53.01,215) and (51,212.99) .. (51,210.5) -- cycle ;
%Shape: Circle [id:dp34245504510141156] 
\draw  [fill={rgb, 255:red, 0; green, 0; blue, 0 }  ,fill opacity=1 ] (283,64.5) .. controls (283,62.01) and (285.01,60) .. (287.5,60) .. controls (289.99,60) and (292,62.01) .. (292,64.5) .. controls (292,66.99) and (289.99,69) .. (287.5,69) .. controls (285.01,69) and (283,66.99) .. (283,64.5) -- cycle ;
%Shape: Circle [id:dp5914292952911566] 
\draw  [fill={rgb, 255:red, 0; green, 0; blue, 0 }  ,fill opacity=1 ] (223,114.5) .. controls (223,112.01) and (225.01,110) .. (227.5,110) .. controls (229.99,110) and (232,112.01) .. (232,114.5) .. controls (232,116.99) and (229.99,119) .. (227.5,119) .. controls (225.01,119) and (223,116.99) .. (223,114.5) -- cycle ;
%Shape: Circle [id:dp40497536712327753] 
\draw  [fill={rgb, 255:red, 0; green, 0; blue, 0 }  ,fill opacity=1 ] (82,214.5) .. controls (82,212.01) and (84.01,210) .. (86.5,210) .. controls (88.99,210) and (91,212.01) .. (91,214.5) .. controls (91,216.99) and (88.99,219) .. (86.5,219) .. controls (84.01,219) and (82,216.99) .. (82,214.5) -- cycle ;
%Shape: Circle [id:dp39909888798506066] 
\draw  [fill={rgb, 255:red, 0; green, 0; blue, 0 }  ,fill opacity=1 ] (237,149.5) .. controls (237,147.01) and (239.01,145) .. (241.5,145) .. controls (243.99,145) and (246,147.01) .. (246,149.5) .. controls (246,151.99) and (243.99,154) .. (241.5,154) .. controls (239.01,154) and (237,151.99) .. (237,149.5) -- cycle ;
%Shape: Circle [id:dp2019267403650239] 
\draw  [fill={rgb, 255:red, 0; green, 0; blue, 0 }  ,fill opacity=1 ] (348,74.5) .. controls (348,72.01) and (350.01,70) .. (352.5,70) .. controls (354.99,70) and (357,72.01) .. (357,74.5) .. controls (357,76.99) and (354.99,79) .. (352.5,79) .. controls (350.01,79) and (348,76.99) .. (348,74.5) -- cycle ;
%Shape: Circle [id:dp5259745267856665] 
\draw  [fill={rgb, 255:red, 0; green, 0; blue, 0 }  ,fill opacity=1 ] (294,113.5) .. controls (294,111.01) and (296.01,109) .. (298.5,109) .. controls (300.99,109) and (303,111.01) .. (303,113.5) .. controls (303,115.99) and (300.99,118) .. (298.5,118) .. controls (296.01,118) and (294,115.99) .. (294,113.5) -- cycle ;
%Shape: Circle [id:dp17390342487641564] 
\draw  [fill={rgb, 255:red, 0; green, 0; blue, 0 }  ,fill opacity=1 ] (416,112.5) .. controls (416,110.01) and (418.01,108) .. (420.5,108) .. controls (422.99,108) and (425,110.01) .. (425,112.5) .. controls (425,114.99) and (422.99,117) .. (420.5,117) .. controls (418.01,117) and (416,114.99) .. (416,112.5) -- cycle ;
%Shape: Circle [id:dp23479676072523903] 
\draw  [fill={rgb, 255:red, 0; green, 0; blue, 0 }  ,fill opacity=1 ] (401,29.5) .. controls (401,27.01) and (403.01,25) .. (405.5,25) .. controls (407.99,25) and (410,27.01) .. (410,29.5) .. controls (410,31.99) and (407.99,34) .. (405.5,34) .. controls (403.01,34) and (401,31.99) .. (401,29.5) -- cycle ;
%Straight Lines [id:da34048831480133634] 
\draw [color={rgb, 255:red, 245; green, 166; blue, 35 }  ,draw opacity=1 ][line width=1.5]    (368.5,28.5) -- (391,86.93) ;


%Straight Lines [id:da8628989912348813] 
\draw [color={rgb, 255:red, 245; green, 166; blue, 35 }  ,draw opacity=1 ][line width=1.5]    (352.5,74.5) -- (358,92.93) ;


%Straight Lines [id:da6248165946669281] 
\draw [color={rgb, 255:red, 245; green, 166; blue, 35 }  ,draw opacity=1 ][line width=1.5]    (327.5,45.5) -- (345,97.93) ;


%Straight Lines [id:da057361574656471515] 
\draw [color={rgb, 255:red, 245; green, 166; blue, 35 }  ,draw opacity=1 ][line width=1.5]    (287.5,64.5) -- (303,113.5) ;


%Straight Lines [id:da9723378016055195] 
\draw [color={rgb, 255:red, 245; green, 166; blue, 35 }  ,draw opacity=1 ][line width=1.5]    (273.5,91.5) -- (283,119.93) ;


%Straight Lines [id:da7072892699379895] 
\draw [color={rgb, 255:red, 245; green, 166; blue, 35 }  ,draw opacity=1 ][line width=1.5]    (227.5,114.5) -- (241.5,149.5) ;


%Straight Lines [id:da5331650878440136] 
\draw [color={rgb, 255:red, 245; green, 166; blue, 35 }  ,draw opacity=1 ][line width=1.5]    (170.5,141.5) -- (176,150.93) ;


%Straight Lines [id:da2225989489764092] 
\draw [color={rgb, 255:red, 245; green, 166; blue, 35 }  ,draw opacity=1 ][line width=1.5]    (148.5,180.5) -- (144,163.93) ;


%Straight Lines [id:da30768836888532625] 
\draw [color={rgb, 255:red, 245; green, 166; blue, 35 }  ,draw opacity=1 ][line width=1.5]    (128.5,160.5) -- (128.5,165) ;


%Straight Lines [id:da9543043664995713] 
\draw [color={rgb, 255:red, 245; green, 166; blue, 35 }  ,draw opacity=1 ][line width=1.5]    (86.5,214.5) -- (76,184.93) ;


%Straight Lines [id:da3131414955517793] 
\draw [color={rgb, 255:red, 245; green, 166; blue, 35 }  ,draw opacity=1 ][line width=1.5]    (55.5,210.5) -- (48,193.93) ;


%Shape: Square [id:dp4250994342007913] 
\draw  [color={rgb, 255:red, 245; green, 166; blue, 35 }  ,draw opacity=1 ][pattern=_yznbr6opy,pattern size=6pt,pattern thickness=0.75pt,pattern radius=0pt, pattern color={rgb, 255:red, 245; green, 166; blue, 35}][line width=1.5]  (405.5,29.5) -- (452.86,13.48) -- (468.89,60.84) -- (421.52,76.86) -- cycle ;
%Straight Lines [id:da5455233967320696] 
\draw    (409,255.93) -- (409,274.93) ;


%Straight Lines [id:da8891249987228257] 
\draw [color={rgb, 255:red, 245; green, 166; blue, 35 }  ,draw opacity=1 ][line width=1.5]    (410,82.93) -- (420.5,112.5) ;


%Straight Lines [id:da2990647448398298] 
\draw    (93,34.93) -- (110,34.93) ;



% Text Node
\draw (410,281) node   {$x_{i}$};
% Text Node
\draw (515,22) node   {$Y_{i} -( \alpha +\beta \cdot x_{i})$};
% Text Node
\draw (68,23) node   {$\alpha +\beta \cdot x_{i}$};


\end{tikzpicture}

			\caption{Einfache lineare Regression - Methode der kleinsten Fehlerquadrate}
			%\label{Abb:MinQuaSchätzer}
		\end{center}
	\end{figure}

	\begin{align*}
		\underbrace{\begin{pmatrix}
			Y_1\\
			\vdots\\
			\vdots\\
			Y_n
		\end{pmatrix}}_{=Y}
		&=\underbrace{\begin{pmatrix}
			1 & x_1\\
			\vdots & x_2\\
			\vdots & \vdots\\
			1 & x_n
		\end{pmatrix}}_{=X}\mal\underbrace{\begin{pmatrix}
			a\\
			b
		\end{pmatrix}}_{=\beta}+\underbrace{\begin{pmatrix}
			\varepsilon_1\\
			\vdots\\
			\vdots\\
			\varepsilon_n
		\end{pmatrix}}_{=\varepsilon}
	\end{align*}
	Angenommen, $\Rg(X)=2$ ($\overset{!}{\iff}\exists i\neq j:x_i\neq x_j$)
	\begin{align*}
		X'\mal X
		&=\begin{pmatrix}
			1 & \hdots & 1\\
			x_1 & \hdots & x_n
		\end{pmatrix}\mal\begin{pmatrix}
			1 & x_1\\
			\vdots &  \vdots\\
			1 & x_n
		\end{pmatrix}
		=\begin{pmatrix}
			n & \sum\limits_{i=1}^n x_i\\
			\sum\limits_{i=1}^n x_i & \sum\limits_{i=1}^n x_i^2
		\end{pmatrix}
		=\begin{pmatrix}
			n & n\mal\overline{x_n}\\
			n\mal\overline{x_n} & \sum\limits_{i=1}^n x_i^2
		\end{pmatrix}\\
		X'\mal Y
		&=\begin{pmatrix}
			1 & \hdots & 1\\
			x_1 & \hdots & x_n
		\end{pmatrix}\mal\begin{pmatrix}
			Y_1\\
			\vdots\\
			Y_n
		\end{pmatrix}
		=\begin{pmatrix}
			\sum\limits_{i=1}^n Y_i\\
			\sum\limits_{i=1}^n x_i\mal Y_i
		\end{pmatrix}
		=\begin{pmatrix}
			n\mal\overline{Y_n}\\
			\sum\limits_{i=1}^n x_i\mal Y_i
		\end{pmatrix}\\
		\mit\qquad \overline{x_n}&:=\frac{1}{n}\mal\sum\limits_{i=1}^n x_i
		\qquad\und\qquad
		\overline{Y_n}:=\frac{1}{n}\mal\sum\limits_{i=1}^n Y_i
	\end{align*}
	Normalgleichung \eqref{eq:satz3.3Stern}:
	\begin{align*}
		\begin{pmatrix}
			n & n\mal\overline{x_n}\\
			n\mal\overline{x_n} & \sum\limits_{i=1}^n x_i^2
		\end{pmatrix}\mal\begin{pmatrix}
			\alpha\\
			\beta
		\end{pmatrix}=\begin{pmatrix}
			n\mal\overline{Y_n}\\
			\sum\limits_{i=1}^n x_i\mal Y_i
		\end{pmatrix}\\
		\iff
		\left\lbrace\begin{array}{l}
			(1)~~
			n\mal\alpha+n\mal\overline{x_n}\mal\beta=n\mal\overline{Y_n}
			\iff\alpha=\overline{Y_n}-\overline{x_n}\mal\beta\\
			(2)~~n\mal\overline{x_n}\mal\alpha+\sum\limits_{i=1}^n x_i^2\mal\beta=\sum\limits_{i=1}^n x_i\mal Y_i
		\end{array}\right.
	\end{align*}
	Teilen durch $n$ in (2) und einsetzen von (1) in (2) liefert:
	\begin{align*}
		\overline{x_n}\mal \overline{Y_n}\underbrace{-\overline{x_n}^2\mal\beta+\frac{1}{n}\mal\sum\limits_{i=1}^n x_i^2\mal\beta}_{
			=\beta\mal\klammern{\frac{1}{n}\mal\sum\limits_{i=1}^n \big(x_i^2-\overline{x_n}^2\big)}
		}
		&=\frac{1}{n}\mal\sum\limits_{i=1}^n x_i\mal Y_i\\
		\implies
		\beta
		&=\frac{
			\frac{1}{n}\mal\sum\limits_{i=1}^n x_i\mal Y_i-\overline{x_n}\mal\overline{Y_n}
		}{
			\frac{1}{n}\mal\sum\limits_{i=1}^n \big(x_i^2-\overline{x_n}^2\big)
		}\\
		&=
		\frac{
			\frac{1}{n}\mal\klammern{\sum\limits_{i=1}^n x_i\mal Y_i-n\mal\overline{x_n}\mal\overline{Y_n}}
		}{
			\frac{1}{n}\mal\sum\limits_{i=1}^n \big(x_i^2-\overline{x_n}^2\big)
		}\\
		\overset{\mal\frac{n}{n-1}}&{=}
		\frac{
			\frac{1}{n-1}\mal\klammern{\sum\limits_{i=1}^n x_i\mal Y_i-n\mal\overline{x_n}\mal\overline{Y_n}}		
		}{
			\frac{1}{n-1}\mal\klammern{\sum\limits_{i=1}^n x_i^2-n\mal \overline{x_n}^2}
		}\\
		\overset{!}&{=}
		\frac{
			\frac{1}{n-1}\mal\sum\limits_{i=1}^n\big(x_i-\overline{x_n}\big)\mal\big(Y_i-\overline{Y_n}\big)
		}{
			\frac{1}{n-1}\mal \sum\limits_{i=1}^n\big(x_i-\overline{x_n}\big)^2
		}\\
		&=:\frac{s_{x,Y}}{s_x^2}
	\end{align*}
	Hierbei heißen $s_{x,Y}$ \define{empirische Kovarianz} und $s_x^2$ \define{empirische Varianz}.
	\index{empirische Varianz}\index{empirische Kovarianz}
	\begin{align*}
		\hat{\beta}=\begin{pmatrix}
			\hat{a}\\
			\hat{b}
		\end{pmatrix}\qquad\mit\qquad
		\hat{a}:=\overline{Y_n}-\overline{x_n}\mal\hat{b},\qquad
		\hat{b}:=\frac{s_{x,Y}}{s_x^2}
	\end{align*}
\end{beispiel}

\subsection{Optimalität des MQS} %3.2

Betrachten im linearen Modell $Y=X\mal\beta+\varepsilon$ \define{lineare} Funktionen $\Psi$ von $\beta$, d.h.
\begin{align*}
	\Psi\colon\R^p\to\R,\qquad\Psi(\beta)=c'\mal\beta\qquad\forall \beta	\in\R^p,
\end{align*}
wobei $c\in\R^p$ fest vorgegeben (bekannt).

\begin{beisp}\
	\begin{enumerate}[label=(\arabic*)]
		\item $\begin{aligned}
			c=e_j\implies\Psi(\beta)=e_j'\mal\beta=\beta_j
		\end{aligned}$
		\item $\begin{aligned}
			c=e_i-e_j\implies\Psi(\beta)=\beta_i-\beta_j
		\end{aligned}$
	\end{enumerate}
\end{beisp}

\begin{definition}\label{def3.6}\
	\begin{enumerate}[label=(\arabic*)]
		\item $\Psi(\beta)=c'\mal\beta$ heißt \define{schätzbar} $\defiff\exists a\in\R^n$ so, dass
		\index{schätzbar}
		\begin{align*}
			\hat{\Psi}:=a'\mal Y=\sum\limits_{i=1}^n a_i\mal Y_i
		\end{align*}
		ein \define{erwartungstreuer Schätzer} für $\Psi$ ist, d.h.
		\index{erwartungstreuer Schätzer}
		\begin{align*}
			\E_\beta\big[\hat{\Psi}\big]=\Psi(\beta)\qquad\forall \beta\in\R^p
		\end{align*}
		\item Schätzer der Form $a'\mal Y$ heißen \define{lineare Schätzer}.
		\index{lineare Schätzer}
	\end{enumerate}
	Kurz: $c'\mal\beta$ ist schätzbar $\iff\exists$ linearer und erwartungstreuer Schätzer für $c'\mal\beta$.
\end{definition}

\begin{lemma}\label{lemma3.7}\
	\begin{enumerate}[label=(\arabic*)]
		\item $\begin{aligned}
			\Psi(\beta)=c'\mal\beta
		\end{aligned}$ schätzbar $\iff\exists a\in\R^n$ mit
		\label{item:lemma3.7(1)}
		\begin{align}\label{eq:3.7}\tag{3.7}
			c'=a'\mal X\qquad\Big(\iff c=X'\mal a\Big)
		\end{align}
		d.h.
		\begin{align*}
			c\in\Bild(X')=\set{X'\mal y:y\in\R^n}
		\end{align*}
		\item Falls $\Rg(X)=p$ (also Vollrang), so ist $c'\mal\beta$ für alle $c\in\R^p$ schätzbar.
		\label{item:lemma3.7(2)}
	\end{enumerate}
\end{lemma}

\begin{proof}
	\betone{Zu \ref{item:lemma3.7(1)} zeige "$\Longrightarrow$":}\\
	Es existiert $a\in\R^n$ so, dass:
	\begin{align*}
		c'\mal\beta
		&=\E[a'\mal Y]
		\overset{\Lin}{=}
		a'\mal\E[Y]
		=a'\mal X\mal\beta &&\forall\beta\in\R^p\\
		\overset{\beta:=e_j}{\iff}
		c'&=a'\mal X
	\end{align*}		
	
	\betone{Zu \ref{item:lemma3.7(1)} zeige "$\Longleftarrow$":}\\
	Es ist $\hat{\Psi}:=a'\mal Y$ ein linearer Schätzer und es gilt:
	\begin{align*}
		\E\big[\hat{\Psi}\big]
		&=a'\mal\E[Y]
		=\underbrace{a'\mal X}_{
			\overset{\eqref{eq:3.7}}{=}c'		
		}\mal\beta
		=c'\mal\beta
		=\Psi(\beta)
		\qquad\forall \beta\in\R^p
	\end{align*}
	Also ist $c'\mal\beta$ schätzbar.\nl
	\betone{Zeige \ref{item:lemma3.7(2)}:}\\
	Setze
	\begin{align*}
		a':=c'\mal\big(X'\mal X\big)^{-1}\mal X'.
	\end{align*}
	Die Inverse hier existiert wegen Satz \ref{satz:2.13}.
	Dann gilt:
	\begin{align*}
		a'\mal X
		&=c'
		\overset{\ref{item:lemma3.7(1)}}{\implies}\text{ Behauptung}
	\end{align*}
\end{proof}

\begin{lemma}\label{lemma3.8}
	Sei $\Psi(\beta)=c'\mal\beta$ schätzbar.
	Dann gilt:
	\begin{enumerate}[label=(\arabic*)]
		\item Es existiert ein $b\in\R^n$ mit
		\begin{align}\label{eq:lemma3.8Stern}\tag{$*$}
			b'\mal Y\text{ ist erwartungstreu für }\Psi
		\end{align}
		\label{item:lemma3.8(1)}
		\item Es existiert genau ein $a\in L:=\Bild(X)\subseteq\R^n$ derart, dass
		\label{item:lemma3.8(2)}
		\begin{align*}
			\hat{\Psi}:=a'\mal Y
		\end{align*}
		erwartungstreu für $\Psi$ ist, nämlich
		\begin{align*}
			a=P_L(b)
		\end{align*}
		für jedes $b$ mit \eqref{eq:lemma3.8Stern}.
	\end{enumerate}
\end{lemma}

\begin{proof}
	\betone{Zeige \ref{item:lemma3.8(1)}:}\\
	Gilt gemäß Definition \ref{def3.6}.\nl
	\betone{Zeige \ref{item:lemma3.8(2)}:}
	\begin{align*}
		b&=a+\tilde{a},\qquad a=P_L(b)\in L,~\tilde{a}\in L^\perp
	\end{align*}
	Es gilt für alle $\beta\in\R^p$:
	\begin{align*}
		\Psi(\beta)
		\overset{\ref{item:lemma3.8(1)}}&{=}
		\E[b'\mal Y]\\
		&=b'\mal X\mal\beta\\
		&=\big(a'+\tilde{a}'\big)\mal X\mal\beta\\
		&=a'\mal X\mal\beta+\underbrace{\tilde{a}'\mal X\mal\beta}_{
			=\scaProd[\big]{\tilde{a}}{\underbrace{X\mal\beta}_{\in L}}
			\overset{\tilde{a}\in L^\perp}{=}0
		}\\
		&=a'\mal\underbrace{X\mal\beta}_{
			=\E[Y]
		}\\
		\overset{\Lin}&{=}
		\E[a'\mal Y]
	\end{align*}
	Somit folgt
	\begin{align}\label{eq:ProofLemma3.8Plus}\tag{+}
		\hat{\Psi}:=a'\mal Y\text{ ist erwartungatreu für $\Psi$ mit }a=P_L(b)\in L
	\end{align}
	Bleibt Eindeutigkeit zu zeigen:
	Sei $\overline{a}\in L$ mit $\overline{a}'\mal Y$ ist erwartungstreu für $\Psi$.
	Dann folgt:
	\begin{align*}
		&a'\mal X\mal\beta
		\overset{\Lin}{=}
		\E[a'\mal Y]
		\overset{\eqref{eq:ProofLemma3.8Plus}}{=}
		c'\mal\beta
		=\E[\overline{a}'\mal Y]
		\overset{\Lin}{=}
		\overline{a}'\mal X\mal\beta
		&&\forall \beta\in\R^p\\
		&\implies
		0=\big(a'-\overline{a}'\big)\mal X\mal\beta
		=\big(a-\overline{a}\big)'\mal X\mal\beta&&\forall\beta\in\R^p\\
		&\implies a-\overline{a}\in L^\perp
	\end{align*}
	Aber $a-\overline{a}\in L$, da $L$ Untervektorraum.
	Somit ist 
	\begin{align*}
		a-\overline{a}\in L^\perp\cap L\overset{\ref{satz2.1}\ref{item:satz2.1(3)}\ref{item:satz2.1(3i)}}{=}
	\set{0}
	\implies \overline{a}=a
	\end{align*}		
\end{proof}

\begin{satz}[Gauß-Markov-Theorem]\label{satz3.9GaussMarkovTheorem}\enter
	Sei $\Psi(\beta)=c'\mal\beta$ schätzbar und $\hat{\Psi}:=c'\mal\beta$, $\hat{\beta}=$ MQS für $\beta$.
	Dann gilt:
	\begin{enumerate}[label=(\arabic*)]
		\item $\hat{\Psi}$ ist linearer und erwarungstreuer Schätzer für $\Psi$ und der einzige unter allen linearen erwarungstreuen Schätzern für $\psi$ mit \betone{minimaler Varianz}.
		\label{item:satz3.9(1)}
		\item $\hat{\Psi}=a'\mal Y$ für ein eindeutig bestimmtes $a\in L$ und 
		\label{item:satz3.9(2)}
		\begin{align*}
			\Var\big(\hat{\Psi}\big)=\sigma^2\mal\norm{a}^2
		\end{align*}
	\end{enumerate}
\end{satz}

\begin{proof}
	Gemäß Lemma \ref{lemma3.8} existiert genau ein $a\in L$ mit
	%\begin{enumerate}[label=(\roman*)]
	\begin{align}\label{eq:ProofSatz3.9(i)}\tag{i}
		a'\mal Y\text{ erwartungstreu für }\Psi
	\end{align}
		%\item $a'\mal Y$ erwartungstreu für $\Psi$
	%\end{enumerate}
	Es gilt
	\begin{align}\label{eq:ProofSatz3.9RoterStern}\tag{$*$}
		a'\mal Y
		\overset{a\in L,\ref{satz2.5}\ref{item:satz2.5(2)}}&{=}
		\big(P_L a\big)'\mal Y
		=a'\mal P_L' Y
		\overset{\ref{satz2.8}\ref{item:satz2.8(2)}}{=}
		a'\mal\underbrace{P_L Y}_{
			=X\mal\hat{\beta}
		}
		\overset{\ref{satz3.3}\ref{item:satz3.3(1)}}{=}
		a'\mal X\mal\hat{\beta}
	\end{align}
	Ferner gilt:
	\begin{align*}
		a'\mal X\mal\beta
		&=a'\mal\E[Y]
		\overset{\Lin}{=}
		\E[a'\mal Y]
		\overset{\eqref{eq:ProofSatz3.9(i)}}{=}
		c'\mal\beta
		\qquad\forall\beta\in\R^p
	\end{align*}
	Wende nun \eqref{eq:ProofSatz3.9RoterStern} mit $\beta:=\hat{\beta}$ an:
	\begin{align}\label{eq:ProofSatz3.9(ii)}\tag{ii}
		a'\mal Y=c'\mal\beta=\hat{\Psi}
	\end{align}
	Also ist $\hat{\Psi}$ linearer erwartungstreuer Schätzer für $\Psi$.
	Ferner:
	\begin{align}\label{eq:ProofSatz3.9(iii)}\tag{iii}
		\Var(\hat{\Psi})
		&=\Var(a'\mal Y)
		\overset{\ref{satz3.2}\ref{item:satz3.2(2)}}{=}
		a'\mal\underbrace{\Var(Y)}_{
			\overset{\ref{satz3.2}}{=}
			\Var(\varepsilon)
			\overset{\eqref{eq:3.2}}{=}
			\sigma^2\mal I_n
		}\mal a
		=\sigma^2\mal\norm{a}^2
	\end{align}

	Sei $\tilde{\Psi}=b'\mal Y$ ein beliebiger linearer, erwartungstreuer Schätzer für $\Psi$.
	Dann folgt (vergleiche \eqref{eq:ProofSatz3.9(iii)}):
	\begin{align}\label{eq:ProofSatz3.9(iii)Strich}\tag{iii'}
		\Var(\tilde{\Psi})=\sigma^2\mal\norm{b}^2
	\end{align}
	Es gilt:
	\begin{align}\label{eq:ProofSatz3.9SternStern}\tag{$**$}
		b&=P_L(b)+v
		\overset{\ref{lemma3.8}\ref{item:lemma3.8(2)}}{=}
		a+v\qquad v\in L^\perp
	\end{align}
	Dann folgt aus Pythagoras:
	\begin{align}\label{eq:ProofSatz3.9(iv)}\tag{iv}
		\norm{b}^2
		\overset{\text{Pythago}}&{=}
		\norm{a}^2+\norm{v}^2
		\geq\norm{a}^2
		\overset{\eqref{eq:ProofSatz3.9(iii)},\eqref{eq:ProofSatz3.9(iii)Strich}}{\implies}
		\Var(\tilde{\Psi})=\Var(\hat{\Psi})
	\end{align}
	 \betone{Eindeutigkeit:}\\
	 Angenommen $\tilde{\Psi}=b'\mal Y$ hat Minimalvarianz.
	 Dann gilt:
	 \begin{align*}
	 	\norm{a}^2=\norm{b}^2
	 	\implies\norm{v}^2=0
	 	\implies v=0
	 	\overset{\eqref{eq:ProofSatz3.9SternStern}}{\implies}
	 	b=a
	 	\overset{\eqref{eq:ProofSatz3.9(i)}}{=} %könnte auch (ii) sein
		\tilde{\Psi}=\hat{\Psi}
	 \end{align*}
	Gemeinsam mit \eqref{eq:ProofSatz3.9(i)}, \eqref{eq:ProofSatz3.9(ii)}, \eqref{eq:ProofSatz3.9(iii)} und \eqref{eq:ProofSatz3.9(iv)} folgen die Behauptungen.

	%\betone{Zeige \ref{item:satz3.9(1)}:}\\
	
	%\betone{Zeige \ref{item:satz3.9(2)}:}\\
\end{proof}

\begin{bemerkungnr}\label{bemerkung3.10}\
	\begin{enumerate}[label=(\arabic*)]
		\item In Satz \ref{satz3.9GaussMarkovTheorem} wird \betone{nicht} $\Rg(X)=p$ (also Vollrang) gefordert.
		\label{item:bem3.10_1}
		\item Sprechweise im Englischen: $c'\mal\hat{\beta}$ is \define{BLUE}
		 (:= best linear unbiased estimator)\\
		 "unbiased" wird mit "unverzerrt" übersetzt, was einfach erwartungstreu bedeutet.
		 \label{item:bem3.10_2}
		 \item Falls $\Rg(X)=p$ (also Vollrang), dann gilt:
		 \begin{align*}
		 	\Var(\hat{\Psi})=\Var\big(c'\mal\hat{\beta}\big)
		 	\overset{\ref{satz3.2}}{=}
		 	c'\mal\underbrace{\Var(\hat{\beta})}_{
		 		\overset{\ref{def3.6}}{=}\sigma^2\mal\big(X'\mal X\big)^{-1}
		 	}=\sigma^2\mal c'\mal(X'\mal X)^{-1}\mal c
		 \end{align*}
		 \label{item:bem3.10_3}
		 \item Speziell für $c=e_j$ liefert $\hat{\beta}_j=j$-te Komponente von $\hat{\beta}$ ist der BLUE für $\beta_j$
		 \label{item:bem3.10_4}
	\end{enumerate}
\end{bemerkungnr}

\subsection{Schätzung von \texorpdfstring{$\sigma^2$}{Sigma Quadrat}}
Wegen \eqref{eq:3.2} und \eqref{eq:3.3} gilt
\begin{align*}
	\sigma^2=\Var(\varepsilon_i)=\E[\varepsilon^2]\qquad\forall i\in\set{1,\ldots,n}
\end{align*}
Es folgt:
\begin{align*}
	\E\eckigeKlammern{\frac{1}{n}\mal\sum\limits_{i=1}^n\varepsilon_i^2}
	\overset{\Lin}{=}\E[\varepsilon_i^2]=\sigma^2
\end{align*}
Aber 
$
	\frac{1}{n}\mal\sum\limits_{i=1}^n\varepsilon_i^2
$
ist \betone{kein Schätzer} für $\sigma^2$, da der Fehlervektor 
$
	\varepsilon=\klammern[\big]{\varepsilon_1,\ldots,\varepsilon_n}
$\\1
\betone{nicht beobachtbar}.
\betone{Idee}:
Ersetze
$
	\varepsilon
	\overset{\eqref{eq:3.1}}{=}
	Y-X\mal\beta
$
durch \betone{bekanntes}
\begin{align*}
	\hat{\varepsilon}:=Y-X\mal\hat{\beta}=Y-\hat{Y},
\end{align*}
wobei 
$	
	\hat{Y}:=X\mal\hat{\beta}
$
das \define{den Daten angepassten (fitted)} $Y$.
$\hat{\varepsilon}$ heißt Vektor der \define{Residuen} (geschätzte Fehler).
\index{Residuenvektor}\\
Erhalte (bis auf Faktor $\frac{1}{n}$):
\begin{align*}
	\RSS:=\sum\limits_{i=1}^n\hat{\varepsilon}_i^2
	=\hat{\varepsilon}'\mal\hat{\varepsilon}
	=\norm{\hat{\varepsilon}}^2
\end{align*}
Das RSS steht für \define{Residual sums of squares} also \define{Residuenquadratsumme}.
\index{Residuenquadratsumme}

\begin{satz}\label{satz3.11}
	Sei $\Rg(X)=p<n$. Dann gilt:
	\begin{align*}
		\S^2:=\frac{1}{n-p}\mal\RSS>0
	\end{align*}
	ist \betone{erwartungstreuer} Schätzer für $\sigma^2$, d.h.
	\begin{align*}
		\E_\beta[S^2]=\sigma^2\qquad\forall \beta\in\R^p
	\end{align*}
\end{satz}

Zwei Hilfsmittel für den Beweis von Satz \ref{satz3.11}:

\begin{lemma}\label{lemma3.12}
	Sei $\Rg(X)=p$. 
	Dann gilt mit $L:=\Bild(X)$:
	\begin{align*}
		\Rg\big(I_n-P_L\big)
		=\Spur\big(I_n-P_L\big)
		=n-p
	\end{align*}
\end{lemma}

\begin{proof}
	\begin{align*}
		I_n-P_L
		\overset{\ref{satz2.9}}&{=}
		P_{L^\perp}
	\end{align*}
	Also ist $I_n-P_L$ symmetrische und idempotente Matrix gemäß Satz \ref{satz2.7} und Satz \ref{satz2.8}.
	Also ist
	\begin{align*}
		\Rg\big(I_n-P_L\big)
		\overset{\ref{satz2.1}\ref{item:satz2.1(2)}}&{=}
		\Spur\big(I_n-P_L\big)\\
		\overset{\Spur\text{ additiv}}&{=}
		\Spur(I_h)-\Spur(P_L)\\
		&=n-\Spur(P_L)\\
		\overset{\ref{satz3.3}\ref{item:satz3.3(2)}}&{=}
		n-\Spur\Big(X\mal\big(X'\mal X\big)^{-1}\mal X'\Big)\\
		\overset{\ref{satz2.14}}&{=}
		n-\Spur(I_p)\\
		&=n-p
	\end{align*}
\end{proof}

\begin{lemma}\label{lemma3.13}
	Sei $A\in M(n\times n)$ deterministisch und $Z$ ein Zufallsvektor in $\R^n$.
	Dann gilt für den Erwartungswert der quadratischen Form:
	\begin{align*}
		\E\eckigeKlammern[\big]{Z'\mal A\mal Z}
		=\Spur\klammern[\big]{A\mal\Var(Z)}+
		\E[Z]'\mal A\mal\E[Z]
	\end{align*}
\end{lemma}

\begin{proof}
	Setze $\theta:=\E[Z]$.
	Idee: Wir betrachten den zentrierten Vektor.
	\begin{align}\nonumber
		\E\eckigeKlammern[\big]{(Z-\theta)'\mal A\mal\underbrace{(Z-\theta)}_{
			\text{zentriert}
		}}
		&=\E\eckigeKlammern{Z'\mal A\mal Z-\theta'\mal A\mal Z-Z'\mal A\mal\theta+\theta'\mal A\mal\theta}\\\nonumber
		\overset{\ref{satz3.2}}&{=}
		\E\eckigeKlammern{Z'\mal A\mal Z}-\theta'\mal A\mal\underbrace{\E[Z]}_{=\theta}-\E[Z]'\mal A\mal\theta+\theta'\mal A\mal\theta\\\nonumber
		&=\E\eckigeKlammern{Z'\mal A\mal Z}-\theta'\mal A\mal\theta\\
		\implies
		\E\eckigeKlammern{Z'\mal A\mal Z}
		&=\underbrace{\E\eckigeKlammern[\big]{(Z-\theta)'\mal A\mal(Z-\theta)}}_{=:\mu}+\theta'\mal A\mal\theta
		\label{eqProofLemma3.13Stern}\tag{$*$}
	\end{align}
	Jetzt rechnen wir $\mu$ aus:
	\begin{align*}
		\mu&=\E\eckigeKlammern{\sum\limits_{1\leq i,j\leq n}\big(Z_i-\theta_i\big)\mal A_{i,j}\mal\big(Z_j-\theta_j\big)}\\
		\overset{\Lin}&{=}
		\sum\limits_{1\leq i,j\leq n}A_{i,j}\mal\E\eckigeKlammern[\Big]{\big(Z_i-\underbrace{\theta_i}_{=\E[Z_i]}\big)\mal\big(Z_j-\underbrace{\theta_j}_{=\E[Z_j]}\big)}\\
		&=\sum\limits_{1\leq i,j\leq n}A_{i,j}\mal\Cov\big(Z_i,Z_j\big)\\
		&=\sum\limits_{1\leq i,j\leq n}A_{i,j}\mal\big(\Var(Z)\big)_{i,j}\\
		\overset{\text{Symm}}&{=}
		\sum\limits_{i=1}^n\sum\limits_{j=1}^n A_{i,j}\mal\big(\Var(Z)\big)_{j,i}\\
		&=\sum\limits_{i=1}^n\klammern{\sum\limits_{j=1}^n A_{i,j}\mal\big(\Var(Z)\big)_{j,i}}\\
		&=\sum\limits_{i=1}^n\Big(A\mal \Var(Z)\Big)_{i,i}\\
		&=\Spur\big(A\mal\Var(Z)\big)
	\end{align*}
	Aus \eqref{eqProofLemma3.13Stern} folgt die Behauptung.
\end{proof}

\begin{proof}[Beweis von Satz \ref{satz3.11}]%\enter
	\begin{align*}
		\hat{\varepsilon}
		\overset{\Def}&{=}
		Y-X\mal\hat{\beta}
		\overset{\ref{satz3.3}}{=}
		Y-P_L Y
		=\big(I_n-P_L\big)Y
		\overset{\ref{satz2.9}}{=}
		P_{L^\perp}Y\\
		\implies
		(n-p)\mal S^2
		\overset{\Def}&{=}
		\RSS\\
		&=\hat{\varepsilon}'\mal\hat{\varepsilon}\\
		&=\big(P_{L^\perp} Y\big)'\mal\big(P_{L^\perp} Y\big)\\
		&=Y'\mal P_{L^\perp}'\mal P_{L^\perp}\mal Y\\
		\overset{\text{Symm}}&{=}
		Y'\mal P_{L^\perp}\mal P_{L^\perp}\mal Y\\
		\overset{\text{Idempotenz}}&{=}
		Y'\mal P_{L^\perp}\mal Y\\
		\implies
		(n-p)\mal\E\big[S^2\big]
		&=\E\eckigeKlammern{Y'\mal\big(I_n-P_L)\mal Y}\\
		\overset{\ref{lemma3.13}}&{=}
		\Spur\klammern[\Big]{\big(I_n-P_L\big)\mal\underbrace{\Var(Y)}_{
			=\sigma^2\mal I_n		
		}}+\underbrace{\E[Y]'\mal\underbrace{\big(I_h-P_L\big)}_{
			=P_{L^\perp}
		}\underbrace{\mal\E[Y]}_{
			=X\mal\beta\in L
		}}_{
			=0
		}\\
		&=\sigma^2\mal\Spur(I_n-P_L)\\
		\overset{\ref{lemma3.12}}&{=}
		\sigma^2\mal(n-p)
	\end{align*}
	Division durch $(n-p)\overset{\Vor}{>}0$ liefert die Behauptung $\E[S^2]=\sigma^2$.
\end{proof}

\begin{bemerkung}
	Es gilt
	\begin{align*}
		\E\eckigeKlammern{\frac{1}{n}\mal\RSS}
		\overset{\Lin}{=}
		\frac{1}{n}\mal\E\eckigeKlammern{\RSS}
		\overset{\Def}{=}
		\frac{1}{n}\mal\E\eckigeKlammern{\sigma^2\mal(n-p)}
		\overset{\ref{satz3.11}}{=}
		\frac{n-p}{n}\mal\sigma^2
	\end{align*}
	Der Schätzer $\frac{1}{n}\mal\RSS$ ist also \betone{nicht} erwartungstreu, da $\frac{n-p}{n}\neq 0$ für alle $p>0$.
	Für große $n$ allerdings ist $\frac{n-p}{n}\approx1$.
	Somit ist der Schätzer $\frac{1}{n}\mal\RSS$ immerhin \define{asymptotisch erwartungstreu}.
\end{bemerkung}

\begin{beispiel}[Einfache lineare Regression, vgl. Beispiel \ref{beisp3.5einfacheLineareRegression}]\label{beisp3.14einfacheLinareRegression}
	\begin{align*}
		Y_i&=a+b\mal x_i+\varepsilon_i &&\forall i\in\set{1,\ldots,n}\\
		\hat{\varepsilon}_i&=Y_i-a-b\mal x_i&&\forall i\in\set{1,\ldots,n}\\
		\beta&=\begin{pmatrix}
			a\\
			b
		\end{pmatrix},\qquad \hat{\beta}=\begin{pmatrix}
			\hat{a}\\
			\hat{b}
		\end{pmatrix}
	\end{align*}
	Dann gilt:
	\begin{align*}
		S^2=\frac{1}{n-2}\mal\sum\limits_{i=1}^n\klammern{Y_i-\hat{a}-\hat{b}\mal x_i}^2
	\end{align*}
	wobei $\hat{a},\hat{b}$ in Beispiel \ref{beisp3.5einfacheLineareRegression} berechnet wurden:
	\begin{align*}
		\hat{a}&=\overline{Y}-\hat{b}\mal\overline{x},\qquad
		\hat{b}=\frac{\sum\limits_{i=1}^n\big(x_i-\overline{x}\big)\mal\big(Y_i-\overline{Y}\big)}{\sum\limits_{i=1}^n\big(x_i-\overline{x}\big)^2}
	\end{align*}
	
	\begin{bemerkung}
		Den Erwartungswert von $S^2$ zu Fuß auszurechnen, erscheint ziemlich schwierig.
	\end{bemerkung}
\end{beispiel}

\subsection{Schätzung von \texorpdfstring{$\beta$}{Beta} unter linearen Nebenbedingungen} %3.4

Sei $Y=X\mal\beta+\varepsilon$ lineares Modell (LM) mit \eqref{eq:3.2}, \eqref{eq:3.3} und \eqref{eq:3.4} und $\Rg(X)=p\overset{\eqref{eq:3.3}}{<}n$.
Ferner sei $H\in M(q\times p)$ \betone{bekannt} mit $\Rg(H)=q\leq p$.
Sei $c\in\R^q$ ebenfalls \betone{bekannt}.
Der unbekannte Parameter $\beta\in\R^p$ erfülle die Gleichung
\begin{align}\label{eq:3.8}\tag{3.8}
	H\mal\beta=c
\end{align}
Gleichung \eqref{eq:3.8} ist eine lineare Nebenbedingung (NB) für $\beta$.
Gemäß der Methode der kleinsten Quadrate finde $\hat{\beta}_H\in\R^p$ mit
\begin{align*}
	\norm{Y-X\mal\hat{\beta}_H}=\min\set{\norm{Y-X\mal\beta}:\beta\in\R^p\AND   H\mal\beta=c},
\end{align*}
d.h. 
\begin{align*}
	\hat{\beta}_{H,c}:=:\hat{\beta}_H\in\argmin\set{\norm{Y-X\mal\beta}:\beta\in\R^p\AND H\mal\beta=c}
\end{align*}

Zur  ...

\begin{satz}\label{satz3.15}

\end{satz}

\begin{proof}
	To Do
\end{proof}

	
	\appendix
	%\input{anhang}
	
	\newcommand{\pathPrefix}{Loesungen/}
	%% This work is licensed under the Creative Commons
% Attribution-NonCommercial-ShareAlike 4.0 International License. To view a copy
% of this license, visit http://creativecommons.org/licenses/by-nc-sa/4.0/ or
% send a letter to Creative Commons, PO Box 1866, Mountain View, CA 94042, USA.

\chapter{Lösungen der Übungsaufgaben}

\section{Lösung von 
	\texorpdfstring{\hyperref[aufg:1]{Aufgabe 1}}{}
}\label{loes:1}

%TODO

\section{Lösung von 
	\texorpdfstring{\hyperref[aufg:2]{Aufgabe 2}}{}
}\label{loes:2}

%TODO

\section{Lösung von 
	\texorpdfstring{\hyperref[aufg:3]{Aufgabe 3}}{}
}\label{loes:3}

Wir zeigen, dass $\A_0$ ein $\sigma$-Algebra ist.\\
$\Omega\in\A_0$, weil $\Omega\subseteq\A$ und somit $\Omega\cup\emptyset\setminus\emptyset\in\A_0$.\nl
Sei $A_0\in\A_0$. Wir zeigen $A_0^C\in\A_0$.
Da $A_0\in\A_0$ existieren $A\in\A$ und $\P$-Nullmengen so, dass $A_0=(A\cup N)\setminus M\in\A_0$.
\begin{align*}
	A_0^C
	&=\Omega\setminus\big((A\cup N)\setminus M\big)\\
	&=\Omega\setminus\big((A\cup N)\cap M^C\big)\\
	&=\Omega\cap\big((A\cup N)\cap M^C\big)\\
	&=(A\cup N)\cap M^C\\
	\overset{\text{DM}}&=
	(A\cup N)^C\cup M\\
	&=(A^C\cap N^C)\cup M\\
	&=(A^C\cup M)\cap (N^C\cup M)\\
	&=(A^C\cup M)\setminus(N^C\cup M)^C\\
	&=(A^C\cup M)\setminus(N\cap M^C)\\
	&=(\tilde{A}\cap\tilde{N})\setminus\tilde{M}\qquad\mit\tilde{A}:=A^C\in\A,~\tilde{N}:=M,~\tilde{M}:=N\cap M^C~\P\text{-Nullmengen}\\
	&\implies A_0^C\in\A_0
\end{align*}
Sei $(A_{0,n})_{n\in\N}\subseteq\A_0$. %TODO


Wir zeigen nun die Wohldefiniertheit von $\P_0$:
%TODO

\section{Lösung von 
	\texorpdfstring{\hyperref[aufg:4]{Aufgabe 4}}{}
}\label{loes:4}

\section{Lösung von 
	\texorpdfstring{\hyperref[aufg:5]{Aufgabe 5}}{}
}\label{loes:5}

\section{Lösung von 
	\texorpdfstring{\hyperref[aufg:6]{Aufgabe 6}}{}
}\label{loes:6}

\section{Lösung von 
	\texorpdfstring{\hyperref[aufg:7]{Aufgabe 7}}{}
}\label{loes:7}

\section{Lösung von 
	\texorpdfstring{\hyperref[aufg:8]{Aufgabe 8}}{}
}\label{loes:8}

\section{Lösung von 
	\texorpdfstring{\hyperref[aufg:9]{Aufgabe 9}}{}
}\label{loes:9}

\betone{Zeige \ref{item:aufg9(1)}:}\\
Um Stationärität (im weiteren Sinne) zu prüfen, prüfen wir zuerst, dass $\E[Z(t)]$ nicht von $t$ abhängt:
\begin{align}\label{eq:loes9}
	\E\big[Z(t)\big]
	\overset{\Def}&{=}
	\E\left[\sum\limits_{j=1}^n\exp\big(\ii\mal\scaProd{a_j}{t}\big)\mal X_j\right]
	\overset{\Lin}{=}
	\sum\limits_{j=1}^n\exp\big(\ii\mal\scaProd{a_j}{t}\big)\mal\underbrace{\E[X_j]}_{
		\overset{\Vor}{=}0
	}
	=0=:M\quad\forall t\in\R^d
\end{align}

Nun müssen wir noch prüfen, dass die Korrelationsfunktion genau von $t_1-t_2$ ($t_1,t_2\in T$) abhängt:
\begin{align*}
	\E\big[Z(t_1)\mal\overline{Z(t_2)}\big]
	\overset{\Def}&{=}
	\E\left[\klammern{\sum\limits_{j=1}^n\Big(\exp\big(\ii\mal\scaProd{a_j}{t}\big)\mal X_j}\mal\overline{\klammern{\sum\limits_{k=1}^n\exp\big(\ii\mal\scaProd{a_k}{t_2}\big)\mal X_k}}\right]\\
	&=
	\E\left[\klammern{\sum\limits_{j=1}^n\Big(\exp\big(\ii\mal\scaProd{a_j}{t}\big)\mal X_j}\mal\klammern{\sum\limits_{k=1}^n\exp\big(-\ii\mal\scaProd{a_k}{t_2}\big)\mal \overline{X_k}}\right]\\
	&=\E\left[\sum\limits_{j=1}^n\sum\limits_{k=1}^n\exp\Big(\ii\mal\big(\scaProd{a_j}{t_1}-\scaProd{a_k}{t_2}\big)\Big)\mal X_j\mal \overline{X_k}\right]\\
	\overset{\Lin}&{=}
	\sum\limits_{j=1}^n\sum\limits_{k=1}^n\exp\Big(\ii\mal\big(\scaProd{a_j}{t_1}-\scaProd{a_k}{t_2}\big)\Big)\mal\underbrace{\E\big[ X_j\mal \overline{X_k}\big]}_{=0,~\falls j\neq k}\\
	&=\sum\limits_{j=1}^n\exp\big(\ii\mal\scaProd{a_j}{t_1-t_2}\big)\mal\E\big[X_j\mal \overline{X_j}\big]\\
\end{align*}
Somit haben wir die Korrelationsfunktion direkt berechnet:
\begin{align*}
	C\big(t_1,t_2\big)=C\big(t_1-t_2\big)
	&=\sum\limits_{j=1}^n\exp\big(\ii\mal\scaProd{a_j}{t_1-t_2}\big)\mal\E\big[X_j\mal \overline{X_j}\big]
\end{align*}
Nun berechnen wir noch die Kovarianzfunktion:
\begin{align*}
	\sigma\big(t_1,t_2\big)
	\overset{\Def}&{=}
	\E\Big[\big(Z(t_1)-\underbrace{M(t_1)}_{
		\overset{\eqref{eq:loes9}}{=}0
	}\big)\mal\overline{\big(Z(t_2)-\underbrace{M(t_2)}_{
		\overset{\eqref{eq:loes9}}{=}0
	}}\big)\Big]
	=C\big(t_1,t_2\big)
	=C\big(t_1-t_2\big)
\end{align*}

\betone{Zeige \ref{item:aufg9(2)}:}\\
Wir prüfen wieder, wann $\E[Z(t)]$ nicht von $t$ abhängt:
\begin{align*}
	\E\big[Z(t)\big]
	\overset{\Def}&{=}
	\E\big[g(t)\mal X\big]
	\overset{\Lin}{=}
	g(t)\mal\underbrace{\E[X]}_{
		\overset{\Vor}{<}\infty,\text{ konstant}
	}
\end{align*}
Also ist $\E[Z(t)]$ genau dann unabhängig von $t$, wenn $g\colon\R^d\to\C$ eine konstante Funktion ist.\\
In diesem Fall ist auch die zweite Eigenschaft von Stationärität (im weiteren Sinne) erfüllt:
\begin{align*}
	\E\Big[Z(t_1)\mal\overline{T(t_2)}\Big]
	\overset{\Def}&{=}
	\E\Big[g(t_1)\mal X\mal\overline{g(t_2)}\mal \overline{X}\Big]
	\overset{g\equiv\text{konst}}{=}
	\abs{g}^2\mal\E\big[X\mal\overline{X}\big]
\end{align*}
Die Korrelationsfunktion ist also auch konstant.

\section{Lösung von 
	\texorpdfstring{\hyperref[aufg:10]{Aufgabe 10}}{}
}\label{loes:10}

\betone{Zeige \ref{item:Aufg:10(1)}:}\\
Um Stationärität (im weiteren Sinne) zu prüfen, prüfen wir zuerst, dass $\E[X(t)]$ nicht von $t$ abhängt:

\begin{align*}
	\E[X(t)]
	\overset{\Def}&{=}
	\E\big[A\mal\cos(\alpha\mal t)+B\mal\sin(\alpha\mal t)\big]
	\overset{\Lin}{=}
	\underbrace{\E[A]}_{
		\overset{\Vor}{=}0
	}\mal\cos(\alpha\mal t)+\underbrace{\E[B]}_{
		\overset{\Vor}{=}0
	}\mal\sin(\alpha\mal t)
	%=0+0
	=0
\end{align*}

Nun müssen wir noch prüfen, dass die Korrelationsfunktion genau von $t_1-t_2$ ($t_1,t_2\in T$) abhängt:
Seien also $t_1,t_2\in T=\R$ beliebig. 

\begin{align*}
	&\E\big[X(t_1)\mal X(t_2)\big]\\
	\overset{\Def}&{=}
	\E\Big[\big(A\mal\cos(\alpha\mal t)+B\mal\sin(\alpha\mal t)\big)\mal
	\big(A\mal\cos(\alpha\mal t)+B\mal\sin(\alpha\mal t)\big)\Big]\\
	&=\E\Big[\underbrace{A^2}_{
		\overset{\Vor}{=}1
	}\mal\cos(\alpha\mal t_1)\mal\cos(\alpha\mal t_2)+\underbrace{A\mal B}_{
		\overset{\Vor}{=}0
	}\mal\ldots+\underbrace{B\mal A}_{
		\overset{\Vor}{=}0
	}\mal\ldots+\underbrace{B^2}_{
		\overset{\Vor}{=}1
	}\mal\sin(\alpha\mal t_1)\mal\sin(\alpha\mal t_2)\Big]\\
	&=\cos(\alpha\mal t_1)\mal\cos(\alpha\mal t_2)+\sin(\alpha\mal t_1)\mal\sin(t\mal t_2)\\
	\overset{\eqref{eq:Aufg10Additionstheorem}}&{=}
	\cos(\alpha\mal t_1-\alpha\mal t_2)\\
	&=\cos\big(\alpha\mal(t_1-t_2)\big)
\end{align*}

Hierbei wird das Additionstheorem
\begin{align}\label{eq:Aufg10Additionstheorem}
	\cos(x-y)=\cos(x)\mal\cos(y)+\sin(x)\mal\sin(y)\qquad\forall x,y\in\R
\end{align}
verwendet.
Damit ist $X$ stationär (im weiteren Sinne).\nl
\betone{Zeige \ref{item:Aufg:10(2)}:}\\
Nein, $X$ ist nicht stationär im engeren Sinne. 
Zumindest nicht für alle $\alpha\in\R$.
Für $\alpha=0$ ist 
\begin{align*}
	X(t)=A\mal\cos(0\mal t)+B\mal\sin(0\mal t)=A\qquad\forall t\in T
\end{align*}
natürlich stationär im engeren Sinne, da $X$ \betone{nicht} von $t$ abhängt.
%Sei nun also o.B.d.A. $\alpha\neq0$.
Wir zeigen, dass $X$ für $\alpha\neq0$ \betone{nicht} stationär ist:
Setze $t_1:=0$ und $t_2:=\frac{\pi}{2\mal\alpha}$.
Dann besitzen die Zufallsgrößen (= einelementige Zufallsvektoren)
\begin{align*}
	X(t_1)=A\qquad\und\qquad X(t_2)=B
\end{align*}
\betone{nicht} notwendigerweise dieselbe Verteilung, denn es sind keine Voraussetzungen an die Verteilung der Zufallsgrößen $A$ und $B$ gestellt.
So ist z.B. $A\sim\Nor(0,1)$ und $B\sim\Exp(0,1)$ möglich.

\section{Lösung von 
	\texorpdfstring{\hyperref[aufg:11]{Aufgabe 11}}{}
}\label{loes:11}


\section{Lösung von 
	\texorpdfstring{\hyperref[aufg:12]{Aufgabe 12}}{}
}\label{loes:12}

\section{Lösung von 
	\texorpdfstring{\hyperref[aufg:13]{Aufgabe 13}}{}
}\label{loes:13}

\betone{Zeige \ref{item:aufg13_1}:}

\betone{Zeige \ref{item:aufg13_2}:}

\betone{Zeige \ref{item:aufg13_3}:}

\betone{Zeige \ref{item:aufg13_4}:}

\betone{Zeige \ref{item:aufg13_5}:}
Folgt direkt aus \ref{item:aufg13_3} und \ref{item:aufg13_4}.

\section{Lösung von 
	\texorpdfstring{\hyperref[aufg:14]{Aufgabe 14}}{}
}\label{loes:14}

\section{Lösung von 
	\texorpdfstring{\hyperref[aufg:15]{Aufgabe 15}}{}
}\label{loes:15}
 % Falls es Übungsaufgaben gibt, sollte man diese miteinbinden

	\printindex % erstellt Stichwortverzeichnis, muss mit "MakeIndex" gebaut werden!
	\listoffigures 
	%\listoftables
	\nocite{*} % erstellt Literaturverzeichnis selbst wenn nie auf die entsprechende Literatur im Dokument verwiesen wird
	\bibliography{literatur}
\end{document}