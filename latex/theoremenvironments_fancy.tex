% This work is licensed under the Creative Commons
% Attribution-NonCommercial-ShareAlike 4.0 International License. To view a copy
% of this license, visit http://creativecommons.org/licenses/by-nc-sa/4.0/ or
% send a letter to Creative Commons, PO Box 1866, Mountain View, CA 94042, USA.

% THEOREM-ENVIRONMENTS

% Quelle für die Gestaltung der folgenden environments ist:
% https://texblog.org/2015/09/25/fancy-boxes-for-theorem-lemma-and-proof-with-mdframed/

% importing the package
\usepackage[framemethod=TikZ]{mdframed}

% setting counters
\newcounter{satz}[section]\setcounter{satz}{0}
\renewcommand{\thesatz}{\arabic{section}.\arabic{satz}}
%\renewcommand{\thesatz}{\arabic{chapter}.\arabic{section}.\arabic{satz}} % Alternative, falls man 1.1.1 statt 1.1 nummerieren möchte

\newcommand{\newtheoremfancy}[4]{
	% Argument 1: name
	% Argument 2: displayed name
	% Argument 3: counter (yes / no)
	% Argument 4: color
	
	\newenvironment{#1}[1][]{
		% code for counter
		\ifthenelse{\equal{#3}{}}{}{\refstepcounter{satz}}
		
		% box design code
		\ifstrempty{##1}
		% if no title
		{
			\mdfsetup{
				frametitle={
					\tikz[baseline,outer sep=0pt]
					\node[anchor=east,rectangle,fill=#4]
					{\strut #2~\ifthenelse{\equal{#3}{}}{}{\thesatz}};
				}
			}
		}
		% if title
		{
			\mdfsetup{
				frametitle={
					\tikz[baseline,outer sep=0pt]
					\node[anchor=east,rectangle,fill=#4]
					{\strut #2~\ifthenelse{\equal{#3}{}}{}{\thesatz}:~##1};
				}
			}
		}
		% defaults
		\mdfsetup{
			innertopmargin=10pt,
			linecolor=#4,
			linewidth=2pt,
			topline=true,
			frametitleaboveskip=\dimexpr-\ht\strutbox\relax
		}

		% code for environment name
		\begin{mdframed}[]\relax}
		{\end{mdframed}
	}
}

\newtheoremfancy{satz}{Satz}{section}{black!25}
\newtheoremfancy{theorem}{Theorem}{section}{black!25}

\newtheoremfancy{thm}{Theorem}{}{black!25}

%%% unfancy stuff
\newtheoremstyle{mystyle}
  {20pt}   % ABOVESPACE \topsep is default, 20pt looks nice
  {20pt}   % BELOWSPACE \topsep is default, 20pt looks nice
  {\normalfont} % BODYFONT
  {0pt}       % INDENT (empty value is the same as 0pt)
  {\bfseries} % HEADFONT
  {}          % HEADPUNCT (if needed)
  {5pt plus 1pt minus 1pt} % HEADSPACE
	{}          % CUSTOM-HEAD-SPEC
\theoremstyle{mystyle}

% Definitionen der Satz, Lemma... - Umgebungen. Der Zähler von "satz" ist dem "section"-Zähler untergeordnet, alle weiteren Umgebungen bedienen sich des satz-Zählers.
%\newtheorem{satzold}{Satz}[section]
\newtheorem{lemma}[satz]{Lemma}
\newtheorem{korollar}[satz]{Korollar}
\newtheorem{proposition}[satz]{Proposition}
\newtheorem{beispiel}[satz]{Beispiel}
\newtheorem{definition}[satz]{Definition}
\newtheorem{bemerkungnr}[satz]{Bemerkung}
\newtheorem{erinnerungnr}[satz]{Erinnerung}
\newtheorem{vermutung}[satz]{Vermutung}
\newtheorem{folgerung}[satz]{Folgerung}

% Bemerkungen, Erinnerungen und Notationshinweise werden ohne Numerierungen dargestellt.
\newtheorem*{bemerkung}{Bemerkung.}
\newtheorem*{erinnerung}{Erinnerung.}
\newtheorem*{notation}{Notation.}
\newtheorem*{aufgabe}{Aufgabe.}
\newtheorem*{lösung}{Lösung.}
\newtheorem*{beisp}{Beispiel.}
\newtheorem*{defi}{Definition.} 
\newtheorem*{lem}{Lemma.}       
\newtheorem*{konvention}{Konvention.}  
