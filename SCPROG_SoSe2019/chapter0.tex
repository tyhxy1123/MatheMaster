% !TEX root = SCPROG.tex
% This work is licensed under the Creative Commons
% Attribution-NonCommercial-ShareAlike 4.0 International License. To view a copy
% of this license, visit http://creativecommons.org/licenses/by-nc-sa/4.0/ or
% send a letter to Creative Commons, PO Box 1866, Mountain View, CA 94042, USA.

\chapter{Einführung}
Alle Vorlesungsmaterialien findest du hier:\\
\link{https://bildungsportal.sachsen.de/opal/auth/RepositoryEntry/20069122089}\nl
\textbf{Hinweis:} Da viele Vorlesungsmaterialen online stehen werde ich diese hier nicht nocheinmal einpflegen.
Diese Mitschrift bildet eher eine Ergänzung zu den online liegenden Dateien.

\begin{itemize}
	\item Dateiname sollte immer dem Klassennamen entsprechen, z.B. \code{MyClass.java}
	\item Kompilieren: \code{javac MyClass.java}\\
	Der Compiler erzeugt dann die \define{Bytecode}-Datei \code{MyClass.class}
	\item Die Bytecode-Dateien kann man dann in der \undefine{Java virtual machine (JVM)} ausführen via \code{java MyClass} (Dateiendung weglassen (!))
	\item Man kann nur class-Dateien ausführen, die die \code{main}-Hauptroutine enthalten.
	Diese Routine ist der Einstiegspunkt des Programms.
	\item Mit \code{arg[0]},$\ldots$,\code{arg[42]} kann man auf die Kommandozeilenargumente zugreifen.
	\item Einzeilige Kommentare: \code{// comment}. Mehrzeilige Kommentare: \code{/*$\ldots$*/}
	\item String-Konkatenation mit \code{+}
	\item Java ist Case-sensitiv.
	\item Alle Klassen erben von \code{Object}. 
	Es gibt eine Vererbungshierarchie.
	\item \define{Singleton class}: Man kann maximal eine Instanz erzeugen.
	\item Java hat keine expliziten \undefine{Destruktoren}, sondern einen \undefine{Garbage Collector}.
	\item In Java gibt es keine Pointer.
	\item Nützliche Tools (z.B. \code{IOTools.java}):
	\link{https://grundkurs-java.de/}
	\item Für Klassen ohne \code{main}-Methode ist es dennoch sinnvoll, eine \code{main}-Methode zu haben und da Tests auszuführen
	\item Linksshift um $i$: \code{n<\hspace{0.1mm}<$i$} fügt $i$ Nullen rechts ein und schiebt die Bits nach links
	\item Rechtsshift um $i$: \code{n>\hspace{0.1mm}>\hspace{0.1mm}>$i$} analog.
	\item arithmetischer Rechtsshift um $i$: \code{n>\hspace{0.1mm}>$i$} = $\floor{\frac{n}{2}}$
	\item Logische Operatoren werten nur notwendige Ausdrücke aus (sie entsprechen also \code{AndAlso} und \code{OrElse} in VB.NET), Beispiel:\\
	\code{if(data!=null \&\& i>=0 \&\& i<data.length \&\& data[i]>=3)}
	\item Sei $i=5$. Dann ist \code{++i + 3} gleich 9 aber \code{i++ +3} ist gleich 8.
\end{itemize}

%\inputminted{java}{code/MyFirstDate.java}
